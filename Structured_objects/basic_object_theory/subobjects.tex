\section{Subobjects}

A subobject of an object $X$, or to be precise, a sub-$\square\square$ of a $\square\square$ $X$ is a subset of $X$ which is $\square\square$.

\begin{exmp}[Subobject tests]
    The subobject tests for groups, rings, $R$-modules, and $R$-algebras are listed in this example.
    Proving equivalences is left as an exercise.
    \begin{enumerate}
        \item[(a)]
        {
            (Subgroup)
            Let $G$ be a group.
            Then $H$ is a subgroup of $G$ if and only if $ab^{-1}\in H$ for all $a, b\in H$.
        }
        \item[(b)]
        {
            (Subring)
            Let $R$ be a ring.
            Then $S$ is a subring of $R$ if and only if $S$ is a subgroup of $R$ and $S$ is closed under multiplication, i.e., $a-b, ab\in S$ for all $a, b\in S$.
        }
        \item[(c)]
        {
            ($R$-submodule)
            Let $R$ be a ring with identity and $M$ be an $R$-module.
            Then $N$ is an $R$-submodule of $M$ if and only if $N$ is closed under addition and $R$-scalar multiplication, i.e., $a+b, ca\in N$ for all $a, b\in N$ and $c\in R$.
        }
        \item[(d)]
        {
            ($R$-subalgebra)
            Let $R$ be a ring with identity and $A$ be an $R$-algebra.
            Then $B$ is an $R$-subalgebra of $A$ if and only if $B$ is both an $R$-submodule and a subring of $A$, i.e., $B$ is closed under addition, $R$-scalar multiplication, and multiplication.
        }
    \end{enumerate}
\end{exmp}

\begin{prop}
    If $Y_i\leq X$ for each $i$ in the index set $I$, then the intersection of $Y_i$'s is the largest subobject of $X$ contained in each $Y_i$.
\end{prop}

\begin{defi}[Generated subobject]
    Let $X$ be an object $\square\square$ and $E$ be a subset of $X$.
    The sub-$\square\square$ $\genone{E}$ generated by $E$ is defined by the intersection of all sub-$\square\square$'s of $X$ containing $E$.
\end{defi}
\begin{rmk}
    One can easily check that the subobject of $X$ generated by the subset $E$ is the unique smallest subobject of $X$ containing $E$.
\end{rmk}
A subobject generated by a single element is called a cyclic subobject.

\begin{exmp}[Prime subfield]
    Given a field $F$, the prime subfield $E$ of $F$ is the subfield of $F$ generated by the identity.
    If $\ch{F}=0$, then the prime subfield of $F$ is isomorphic to $\bb{Q}$; if $\ch{F}=p>0$, then the prime subfield of $F$ is isomorphic to $\bb{Z}/p\bb{Z}$ (or $\bb{F}_p$).
\end{exmp}