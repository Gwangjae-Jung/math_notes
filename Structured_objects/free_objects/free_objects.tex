\section{Remarks on free objects}

\begin{defi}[Free object]
    For a nonempty set $S$, the pair $(\mc{F}(S), \iota: S\rightarrow\mc{F}(S))$ is called a free $\square\square$ generated by $S$ if it satisfies the following universal property:
    \begin{center}
        For any $\square\square$ $X$ and a map $f: S\rightarrow X$, there is a unique $\square\square$-homomorphism $\widetilde{f}: \mc{F}(S)\rightarrow X$ such that $\widetilde{f}\circ\iota=f$.
        \begin{equation*}
        \begin{tikzcd}[row sep=1.0cm, column sep=1.5cm]
            S\arrow[r, "\iota"]\arrow[dr, "f"']
            &
            \mc{F}(S)\arrow[d, "\widetilde{f}", dashed]\\
            &
            X
        \end{tikzcd}
        \end{equation*}
    \end{center}
\end{defi}

\begin{rmk}
    Note that the map $\iota: S\rightarrow \mc{F}(S)$ in the definition of the free-$\square\square$ isn't mentioned as an injection.
    However, one can easily prove that $\iota$ is necessarily an injection.
    \begin{prop}
        If $(\mc{F}(S), \iota)$ is a free-$\square\square$ generated by $S$, then $\iota$ is an injection, hence we may identify $S$ as a subset of $\mc{F}(S)$.
    \end{prop}
    \begin{proof}
        Fix a point $x\in S$ and define a map $f: S\rightarrow \bb{Z}/2\bb{Z}$ by $f(x)=\ol 1$ and $f(y)=\ol 0$ for all $y\in S\setminus\{x\}$.
        \color{magenta}The main idea of this proof is that the space $\bb{Z}/2\bb{Z}$ is a $\square\square$.\color{black}
        \begin{equation*}
            \begin{tikzcd}[row sep=1.0cm, column sep=1.5cm]
                S\arrow[r, "\iota"]\arrow[dr, "f"']
                &
                \mc{F}(S)\arrow[d, "\widetilde{f}", dashed]\\
                &
                \bb{Z}/2\bb{Z}
            \end{tikzcd}
        \end{equation*}
        Since $f(x)\neq f(y)$ whenever $y\in X\setminus\{x\}$, we have $\iota x\neq\iota y$, as desired.
    \end{proof}
    
    Also, since free objects are defined in terms of a universal property, a free-$\square\square$ generated by $S$ is unique up to $\square\square$-isomorphism.
    \begin{prop}
        If $(\mc{F}_1, \iota_1)$ and $(\mc{F}_2, \iota_2)$ are free-$\square\square$ generated by $S$, then $\mc{F}_1\approx_{\square\square}\mc{F}_2$.
    \end{prop}
    
    Finally, as expected, the set of words with alphabets in $S$ is a free-$\square\square$ generated by $S$.
    \begin{prop}
        $\genone{S}$ is a free-$\square\square$ generated by $S$.
        Hence, we may identify $\genone{S}$ and $\mc{F}(S)$.
    \end{prop}
    In particular, the above proposition implies the existence of a free-$\square\square$ generated by a nonempty set $S$.
\end{rmk}

\begin{prop}
    A $\square\square$ is a homomorphic image of a free-$\square\square$.
\end{prop}
\begin{proof}
    Given a $\square\square$ $X$ generated by a set $S$, let $\mc{F}$ be the free-$\square\square$ generated by $S$ and let $\imath: S\hookrightarrow\mc{F}$ be the natural embedding.
    Then there is a unique $\square\square$-homomorphism $\jmath_*: \mc{F}\rightarrow X$ extending the inclusion map $\jmath: S\hookrightarrow X$, which is necessarily a surjection.
\end{proof}
\begin{rmk}
    Indeed, $X\approx\mc{F}/\ker{\jmath_*}$.
    In other words, $X$ can be obtained from the free-$\square\square$ generated by $X$ by declaring all elements in $\ker{\jmath_*}$ zero.
\end{rmk}

One question may have been in one's mind as soon as we started this chapter:
\begin{center}
    Are $\mc{F}(S)$ and $\mc{F}(T)$ isomorphic if $|S|=|T|$?
\end{center}

\begin{obs}
    Let $S, T$ be nonempty sets, $(\mc{F}(S), \imath)$ be the free-$\square\square$ generated by $S$, and $(\mc{F}(T), \jmath)$ be the free-$\square\square$ generated by $T$.
    Given a map $f: S\rightarrow T$, there is a unique $\square\square$-homomorphism $\widetilde{f}:\mc{F}(S)\rightarrow\mc{F}(T)$ such that $\widetilde{f}\circ\imath=\jmath\circ f$.
    \begin{equation*}
    \begin{tikzcd}[row sep=1.5cm, column sep=1.5cm]
        S
        \arrow[r, "f"]
        \arrow[dr, "\jmath\circ f"{sloped}]
        \arrow[d, "\imath"', hook]
        &
        T
        \arrow[d, "\jmath", hook]
        \\
        \mc{F}(S)
        \arrow[r, "\widetilde{f}"', dashed]
        &
        \mc{F}(T)
    \end{tikzcd}
    \end{equation*}

    Given nonempty sets $S, T, U$, let the free $\square\square$ generated by these sets be denoted by $(\mc{F}(S), \imath)$, $(\mc{F}(T), \jmath)$, $(\mc{F}(U), k)$, respectively.
    \begin{enumerate}
        \item[(a)]
        {
            By letting $S=T$ and $f: S\rightarrow T$ be the identity map and checking that the above diagram commutes if $\widetilde{f}$ is the identity map on $\mc{F}(S)$, it can be deduced that $\widetilde{\id{S}}=\id{\mc{F}(S)}$.
        }
        \item[(b)]
        {
            If $f: S\rightarrow T$ and $g: T\rightarrow U$ are maps, then $\widetilde{g\circ f}=\widetilde{g}\circ \widetilde{f}$.
            \begin{equation*}
            \begin{tikzcd}[row sep=1.0cm, column sep=1.0cm]
                S
                \arrow[d, "\imath"', hook]
                \arrow[r, "f"']
                \arrow[rr, "g\circ f", bend left]
                &
                T
                \arrow[d, "\jmath"', hook]
                \arrow[r, "g"']
                &
                U
                \arrow[d, "k", hook]
                \\
                \mc{F}(S)
                \arrow[r, "\widetilde{f}"]
                \arrow[rr, "\widetilde{g}\circ \widetilde{f}", "\widetilde{g\circ f}"', bend right]
                &
                \mc{F}(T)
                \arrow[r, "\widetilde{g}"]
                &
                \mc{F}(U)
            \end{tikzcd}
            \end{equation*}
        }
    \end{enumerate}

    Now we can answer to the above question.
    Assume as in (b) and let $f: S\rightarrow T$ is a bijection.
    By letting $g=f^{-1}$, because $\widetilde{g}\circ \widetilde{f}=\widetilde{g\circ f}=\id{\mc{F}(S)}$ and $\widetilde{f}\circ\widetilde{g}=\widetilde{f\circ g}=\id{\mc{F}(T)}$, $\widetilde{f}$ denotes a $\square\square$-isomorphism of $\mc{F}(S)$ into $\mc{F}(T)$.
\end{obs}

So far, we have found that the free objects generated by $S$ and $T$ are isomorphic if there is a bijection between $S$ and $T$.
Then how about its converse?
In other words, is there a bijection between $S$ and $T$ if the free objects generated by $S$ and $T$ are isomorphic?
For free groups and modules, this question will be answered when we study linear algebra over principal ideal domains.