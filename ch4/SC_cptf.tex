\section{The Stone-\v{C}ech compactification}

Throughout this section, we assume that $X$ is a completely regular space, unless stated otherwise.

It is known that finding a compactification $Y$ of a completely regular space $X$ to which every continuous map from $X$ into $\bb{R}$ is continuously extended is a basic problem.
Regarding this, if a given real-valued function on $X$ were to be extended, then the function must have been bounded.

The idea to find such a compactification of $X$ is to apply \cref{emb&cptf}.
To be precise, we first find an appropriate embedding of $X$ in terms of all the bounded continuous functions on $X$.
\cref{emb&cptf} then asserts the existence of an extended embedding of a compactification of $X$, which surely is in terms of the ``extended'' bounded continuous functions on $Y$.
This idea is applicable, since we know from the embedding theorem that a completely regular space can be embedded into $[0, 1]^I$ for some $I$, which is a compact space.

Write $I=C_b^0(X, \bb{R})$ for convinience.
For each $\alpha\in I$, let $B_\alpha=[\inf(\alpha), \sup(\alpha)]\subset\bb{R}$, and define
\begin{align*}
    B=\prod_{\alpha\in I}B_\alpha.
\end{align*}
Since $I$ separates points of $X$ from closed subsets of $X$, the map $F: X\rightarrow B$ defined by $F=(f)_{f\in I}$ is an embedding of $X$ into the compact Hausdorff space $B$.
By \cref{emb&cptf}, there is a unique compactification $Y$ of $X$ such that $F$ extends to an embedding $F_*$ of $Y$ into $B$.
($Y$ is the compactification of $X$ induced by the embedding $F$)
The $f$-component $f_*$ of $F_*$ is a desired extension of $f$, since $f_*=\pi_f\circ F_*$ is continuous.
Finally, while one might find an extension of $f$ using another method, the assumption that the codomain is compact implies that the extension of $f$ should be unique.

To sum up, by considering an embedding $F$ with regard to $C_b^0(X, \bb{R})$ and considering the compactification $Y$ of $X$ induced by $F$, we could show that every function in $C_b^0(X, \bb{R})$ extends uniquely to a continuous function on $Y$.
The above observation is summarized as the following proposition:
\begin{prop}\label{SC-cptf_ver1}
    Let $X$ be a completely regular space.
    Then there is a compactification $Y$ of $X$, to which every function $f\in C_0^b(X, \bb{R})$ extends continuously to $Y$ uniquely.
\end{prop}

We wish to consider a compactification of a completely regular space which satisfies a more general property.
Such compactification would be defined in terms of a universal property.

\begin{defi}[The Stone-\v{C}ech compactification of a topological space]\label{SC-cptf}
    Let $X$ be a topological space (not necessarily a completely regular space).
    A pair $(\beta(X), \iota)$ of a compact Hausdorff space $\beta(X)$ and a continuous map $\iota: X\rightarrow \beta(X)$ is called a Stone-\v{C}ech compactification of $X$, if the pair satisfies the following universal property:
    For any compact Hausdorff space $K$ and a continuous map $f: X\rightarrow K$, there is a unique continuous map $\beta(f): \beta(X)\rightarrow K$ such that $\beta(f)\circ\iota = f$.
    \begin{equation*}
        \begin{tikzcd}[row sep=large, column sep=huge]
            X
            \arrow[dr, "f"']
            \arrow[r, "\iota"] &
            \beta(X)
            \arrow[d, "\beta(f)", dashed] \\
            & K
        \end{tikzcd}
    \end{equation*}
\end{defi}

Note that a Stone-\v{C}ech compactification of a topological space $X$ is unique up to homeomorphism, provided that it exists.
While the Stone-\v{C}ech compactification could be considered for topological spaces which are not completely regular, in this note, we only consider the Stone-\v{C}ech compactification of completely regular spaces.
\begin{thm}
    Let $X$ be a completely regular space.
    Then the Stone-\v{C}ech compactification of $X$ exists, which is unique up to equivalence.
\end{thm}

The following proposition states that a compactification $Y$ of $X$ in \cref{SC-cptf_ver1}, together with the inclusion of $X$ into $Y$, is a compactification of $X$ to which every continuous map from $X$ to any compact Hausdorff space extends continuously and uniquely.
\begin{prop}[Existence part]
    Let $X$ be a completely regular space and $Y$ be a compactification of $X$ satisfying the property in \cref{SC-cptf_ver1}.
    Then, for any compact Hausdorff space $K$ and a continuous map $f: X\rightarrow K$, there is a unique continuous map $f_*: Y\rightarrow K$ extending $f$.
\end{prop}
\begin{proof}
    Remark that a compact Hausdorff space is completely regular, so $K$ embeds into $[0, 1]^I$ for some $I$ (let $e: K\hookrightarrow[0, 1]^I$ be such an embedding).
    So we may consider the composition $g:=e\circ f: X\rightarrow e(K)\subset[0, 1]^I$ rather than $f$.
    
    For each $\alpha\in I$, find the unique extension $(g_\alpha)_*: Y\rightarrow[0, 1]$ of $g_\alpha: X\rightarrow[0, 1]$.
    Then, the map $g_*: Y\rightarrow[0, 1]^I$ defined by $g_*=((g_\alpha)_*)_{\alpha\in I}$ is the unique continuous extension of $g: X\rightarrow[0, 1]^I$.
    
    Since we have composited $e$ at the left of $f$ to obtain $g$, to assert that $f_*:=e^{-1}\circ g_*$ is a desired extension of $f$, we need to show that $f_*$ is well-defined.
    Then, it naturally follows that $f_*$ is a unique continuous extension of $f$.
    Because the closure of $X$ in $Y$ is $Y$, we have
    \begin{align*}
        g_*(Y)\subset\overline{g_*(X)}=\overline{g(X)}\subset\overline{e(K)}=e(K).
    \end{align*}
    Therefore, $f_*$ is well-defined, so $f_*$ is a desired extension of $f: X\rightarrow K$.
    The uniqueness follows from the assumption that $K$ is a Hausdorff space.
\end{proof}

\begin{prop}[Uniqueness part]
    A Stone-\v{C}ech compactification of a completely regular space $X$ is unique up to equivalence.
\end{prop}
\begin{proof}
    Let $Y_1$ and $Y_2$ be two Stone-C\v{e}ch compactifications of $X$, and let $\imath_k$ denote the inclusion embedding of $X$ into $Y_k$ for $k=1, 2$.
    \begin{equation*}
        \begin{tikzcd}[row sep=large, column sep=huge]
            & Y_1\arrow[d, "(\imath_2)_*"']\arrow[dd, bend left, "(\imath_1)_*=\textsf{id}_{Y_1}"]\\
            X
            \arrow[ur, "\imath_1", hook]
            \arrow[r, "\imath_2"', hook]
            \arrow[dr, "\imath_1"', hook]
            & Y_2\arrow[d, "(\imath_1)_\star"']\\
            & Y_2
        \end{tikzcd}
        \qquad\qquad
        \begin{tikzcd}[row sep=large, column sep=huge]
            & Y_2\arrow[d, "(\imath_1)_\star"']\arrow[dd, bend left, "(\imath_2)_\star=\textsf{id}_{Y_2}"]\\
            X
            \arrow[ur, "\imath_2", hook]
            \arrow[r, "\imath_1"', hook]
            \arrow[dr, "\imath_2"', hook]
            & Y_1\arrow[d, "(\imath_2)_*"']\\
            & Y_1
        \end{tikzcd}
        \end{equation*}
    Clearly, each embedding is continuous.
    Hence, the universal property of $Y_1$ gives the continuous extension $(\imath_2)_\star: Y_1\rightarrow Y_2$ of $\imath_2$ and the universal property of $Y_2$ gives the continuous extension $(\imath_1)_*: Y_2\rightarrow Y_1$, and these extensions surely extends the identity map on $X$.
    Since the universal property of $Y_1$ gives a unique extension, the extension $(\imath_1)_*: Y_1\rightarrow Y_1$ of $\imath_1$ is forced to be the identity map on $Y_1$.
    Therefore, $(\imath_1)_\star\circ(\imath_2)_*$ is the identity map on $Y_1$.
    Similarly, $(\imath_2)_*\circ(\imath_1)_\star$ is the identity map on $Y_2$.
    Therefore, $(\imath_1)_\star$ and $(\imath_2)_*$ are homeomorphisms, proving that $Y_1$ and $Y_2$ are equivalent.
\end{proof}

So far, we have proved that for a completely regular space the Stone-\v{C}ech compactification exists uniquely.
We end this section with the functorial property of the correspondence $\beta$ given in \cref{SC-cptf}.
\begin{prop}
    The correspondence $\beta$ is a functor from $\textbf{Top}$ (the category of the topological spaces) to $\textbf{CHaus}$ (the category of the compact Hausdorff spaces).
\end{prop}