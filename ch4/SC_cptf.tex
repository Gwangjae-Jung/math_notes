\section{The Stone-\v{C}ech compactification}

Throughout this section, we assume that $X$ is a completely regular space, unless stated otherwise.

Finding a compactification $Y$ of a completely regular space $X$ to which every continuous map from $X$ into $\bb{R}$ is continuously extended is a basic problem.
Regarding this, if a given real-valued function on $X$ were to be extended, then the function must have been bounded.

The idea to find such a compactification of $X$ is to apply \cref{emb&cptf}.
To be precise, we first find an appropriate embedding of $X$ in terms of all the bounded continuous functions on $X$.
\cref{emb&cptf} then asserts the existence of an extended embedding of a compactification of $X$, which surely is in terms of the ``extended'' bounded continuous functions on $Y$.
This idea is applicable, since we know from the embedding theorem that a completely regular space can be embedded into $[0, 1]^I$ for some $I$, which is a compact space.

Write $I=C_b^0(X, \bb{R})$ for convinience.
For each $\alpha\in I$, let $B_\alpha=[\inf(\alpha), \sup(\alpha)]\subset\bb{R}$, and define
\begin{align*}
    B=\prod_{\alpha\in I}B_\alpha.
\end{align*}
Since $I$ separates points of $X$ from closed subsets of $X$, the map $F: X\rightarrow B$ defined by $F=(f)_{f\in I}$ is an embedding of $X$ into the compact Hausdorff space $B$.
By \cref{emb&cptf}, there is a unique compactification $Y$ of $X$ such that $F$ extends to an embedding $F_*$ of $Y$ into $B$; namely, the compactification of $X$ induced by the embedding $F$.
The $f$-component $f_*$ of $F_*$ is a desired extension of $f$, since $f_*=\pi_f\circ F_*$ is continuous.
This proves the existence of a continuous extension of $f$ to a compactification of $X$.
Remark that the uniqueness of a continuous extension of $f$ to a compactification of $X$ is guaranteed by the assumption that the codomain is a Hausdorff space.
\begin{equation*}
\begin{tikzcd}[row sep=large, column sep=huge]
    &
    X
    \arrow[dl, "\textsf{inclusion}"{sloped}, hook]
    \arrow[d, "F"]
    \arrow[dr, "f"]
    &
    \\
    Y
    \arrow[r, "F_*"', hook]
    &
    B
    \arrow[r, "\pi_f"']
    &
    \bb{R}
\end{tikzcd}
\end{equation*}

To sum up, by considering an embedding $F$ with regard to $C_b^0(X, \bb{R})$ and considering the compactification $Y$ of $X$ induced by $F$, we could show that every function in $C_b^0(X, \bb{R})$ extends uniquely to a continuous function on $Y$.
The above observation is summarized as the following proposition:
\begin{prop}\label{SC-cptf_ver1}
    Let $X$ be a completely regular space.
    Then there is a compactification $Y$ of $X$, to which every function $f\in C_0^b(X, \bb{R})$ extends continuously to $Y$ uniquely.
\end{prop}

We wish to consider a compactification of a completely regular space which satisfies a more general property.
Such compactification would be defined in terms of a universal property.

\begin{defi}[The Stone-\v{C}ech compactification of a topological space]\label{SC-cptf}
    Let $X$ be a topological space (not necessarily a completely regular space).
    A pair $(\beta(X), \iota)$ of a compact Hausdorff space $\beta(X)$ and a continuous map $\iota: X\rightarrow \beta(X)$ is called a Stone-\v{C}ech compactification of $X$, if the pair satisfies the following universal property:
    For any compact Hausdorff space $K$ and a continuous map $f: X\rightarrow K$, there is a unique continuous map $f_*: \beta(X)\rightarrow K$ such that $f_*\circ\iota = f$.
    \begin{equation*}
        \begin{tikzcd}[row sep=large, column sep=huge]
            X
            \arrow[dr, "f"']
            \arrow[r, "\iota"] &
            \beta(X)
            \arrow[d, "f_*", dashed] \\
            & K
        \end{tikzcd}
    \end{equation*}
\end{defi}

Note that a Stone-\v{C}ech compactification of a topological space $X$ is unique up to homeomorphism (but may not be unique up to equivalence), provided that it exists.
While the Stone-\v{C}ech compactification could be considered for topological spaces which are not completely regular, in this note, we only consider the Stone-\v{C}ech compactification of completely regular spaces.
\begin{thm}
    Let $X$ be a completely regular space.
    Then a Stone-\v{C}ech compactification of $X$ exists.
    Furthermore, a Stone-\v{C}ech compactification of $X$ is unique up to equivalence.
\end{thm}

The following proposition states that a compactification $Y$ of $X$ in \cref{SC-cptf_ver1}, together with the inclusion of $X$ into $Y$, is a Stone-\v{C}ech compactification of $X$.
\begin{prop}[Existence part]\label{SC-cptf existence for CReg spaces}
    Let $X$ be a completely regular space and $Y$ be a compactification of $X$ satisfying the property in \cref{SC-cptf_ver1}.
    Then, for any compact Hausdorff space $K$ and a continuous map $f: X\rightarrow K$, there is a unique continuous map $f_*: Y\rightarrow K$ extending $f$.
\end{prop}
\begin{proof}
    Remark that a compact Hausdorff space is completely regular, so $K$ embeds into $[0, 1]^I$ for some $I$.
    Let $e: K\hookrightarrow[0, 1]^I$ be such an embedding.
    To apply the result of \cref{SC-cptf_ver1}, we may consider the composition $g:=e\circ f: X\rightarrow e(K)\subset[0, 1]^I$ rather than $f$, because $g$ is into a Euclidean space so every projection of $g$ is into $\bb{R}$.
    
    For each $\alpha\in I$, let $(g_\alpha)_*: Y\rightarrow[0, 1]$ be the unique continuous extension of $g_\alpha: X\rightarrow[0, 1]$.
    Then, the map $g_*: Y\rightarrow[0, 1]^I$ defined by $g_*=((g_\alpha)_*)_{\alpha\in I}$ is the unique continuous extension of $g: X\rightarrow[0, 1]^I$.
    
    To assert that $f_*:=e^{-1}\circ g_*$ is a desired extension of $f$, it suffices to show that $f_*$ is well-defined, by justifying that $g_*(Y)$ is contained in $e(K)$.
    Because the closure of $X$ in $Y$ is $Y$ and $e(K)$ is compact,
    \begin{align*}
        g_*(Y)\subset\overline{g_*(X)}=\overline{g(X)}\subset\overline{e(K)}=e(K).
    \end{align*}
    Therefore, $f_*$ is well-defined and $f_*$ is a desired extension of $f: X\rightarrow K$ to $Y$.
    The uniqueness follows from the assumption that $K$ is a Hausdorff space.
\end{proof}

\begin{prop}[Uniqueness part]
    Let $X$ be a completely regular space.
    \begin{enumerate}
        \item[(a)]
        {
            A Stone-\v{C}ech compactification of $X$ is unique up to homeomorphism.
        }
        \item[(b)]
        {
            A compactification of $X$ which is a Stone-\v{C}ech compactification of $X$ is unique up to equivalence.
        }
    \end{enumerate}
\end{prop}
\begin{proof}
    (a) is clear, since the Stone-\v{C}ech compactification of a topological space is defined by a universal property.
    (b) can be similarly proved, if one applies the assumption that the given maps are inclusions.
\end{proof}

So far, we have proved the existence and the uniqueness of the Stone-\v{C}ech compactification of a completely regular space.
In fact, it is known that the Stone-\v{C}ech compactification exists not only for a completely regular space but also for a topological space.

Here are some basic properties regarding the Stone-\v{C}ech compactification of a completely regular space.
When we constructed the Stone-\v{C}ech compactification of a completely regular space, we adopted a compactification of the space so that the map of the space into the compactification is the inclusion (an embedding, to be general).
The following proposition states that the map from the space into its Stone-\v{C}ech compactification is always an embedding.
\begin{prop}
    Let $X$ be a completely regular space and $(Y, \imath)$ be the Stone-\v{C}ech compactification of $X$.
    Then $\imath$ is an embedding of $X$ into $Y$.
\end{prop}
\begin{proof}
    By \cref{SC-cptf existence for CReg spaces}, there is a compactification $Y$ of $X$ which is a Stone-\v{C}ech compactification of $X$.
    Let $(\beta(X), \imath)$ be a Stone-\v{C}ech compactification of $X$ and $\jmath: X\rightarrow Y$ be the inclusion.
    Let $\jmath_*: Y\rightarrow \beta(X)$ be the unique continuous map such that $\jmath_*\circ\imath=\jmath$, which is induced by the universal property of the Stone-\v{C}ech compactification.
    Since $\beta(X)$ and $Y$ are homeomorphic, it follows that $\jmath_*$ is a homeomorphism, proving that $\jmath$ is an embedding.
\end{proof}
Next, we investigate the maximality of the Stone-\v{C}ech compactification of a completely regular space.
\begin{prop}
    Let $X$ be a completely regular space and $Y$ be any compactification of $X$.
    Then there is a continuous closed surjective map $g: \beta X\rightarrow Y$, where $(\beta X, \imath)$ is the Stone-\v{C}ech compactification of $X$.
    (In fact, if $\jmath: X\hookrightarrow Y$ is the inclusion map, the continuous map $f: \beta X\rightarrow Y$ induced by $\jmath$ is a continuous closed surjective map.)
\end{prop}
\begin{rmk}
    Since a continuous map which is closed and surjective is a quotient map, it follows that every compactification of $X$ is homeomorphic to a quotient of $\beta X$.
\end{rmk}
\begin{proof}
    Let $\imath$ be the embedding of $X$ into $\beta X$ and $\jmath$ be the inclusion of $X$ into $Y$.
    By the universal property of the Stone-\v{C}ech compactification, there is a unique continuous map $f: \beta X\rightarrow Y$ such that $f\circ\imath=\jmath$.
    Since any continuous map from a compact space into a Hausdorff space is a closed map, $f$ is a closed map.
    Because the image of $f$ is a closed subset of $Y$ containing $(f\circ\imath)(X)=\imath(X)=X$ and $X$ is dense in $Y$, we conclude that $f$ is surjective.
\end{proof}

An interesting example is introduced.
\begin{exmp}
    Remark that $\bb{N}$ equips the discrete topology, so every set map from $\bb{N}$ into any topological space is continuous; in particular, $\bb{N}$ is completely regular.
    In this example, we prove that $\card{\beta\bb{N}}\geq\card{I^I}$, where $I=[0, 1]\subset\bb{R}$ and $(\beta X, \iota)$ is the Stone-\v{C}ech compactification of $\bb{N}$.
    
    Let $D=\{d_n\}_{n\in\bb{N}}$ be a countable dense subset of $I^I$ \color{brown}(Why is $I^I$ separable?) \color{black} and let $f: \bb{N}\rightarrow I^I$ be the map defined by $n\mapsto d_n$ for all $n\in\bb{N}$.
    Since $I^I$ is a compact Hausdorff space, by the universal property of the Stone-\v{C}ech compactification, there is a unique continuous map $f_*: \beta\bb{N}\rightarrow I^I$ such that $f_*\circ\iota=f$.
    Since the image of $f_*$ is a closed subset of $I^I$ containing $f(\bb{N})=D$, we find that $\card{\beta\bb{N}}\geq\card{I^I}$.
\end{exmp}

We end this section with defining a functor from $\textbf{Top}$ to $\textbf{CHaus}$, where $\textbf{CHaus}$ is the category of the compact Hausdorff spaces, with morphisms being the continuous maps of compact Hausdorff spaces into compact Hausdorff spaces.
For a topological space $X$, let $\beta(X)$ be the Stone-\v{C}ech compactification of $X$.
For a continuous map $f: X\rightarrow Y$, define $\beta(f): \beta(X)\rightarrow\beta(Y)$ as follows.
Let $(\beta(X), \imath)$ and $(\beta(Y), \jmath)$ be the Stone-\v{C}ech compactifications of $X$ and $Y$, respectively.
Then there is a unique continuous map $(\jmath\circ f)_*: \beta(X)\rightarrow\beta(Y)$ satisfying $(\jmath\circ f)_*\circ\imath=\jmath\circ f$; define $\beta(f)=(\jmath\circ f)_*$.
\begin{equation*}
\begin{tikzcd}[row sep=large, column sep=huge]
    X
    \arrow[r, "f"]
    \arrow[d, "\imath"']
    \arrow[dr, "\jmath\circ f"{sloped}]
    &
    Y
    \arrow[d, "\jmath"]
    \\
    \beta(X)
    \arrow[r, "\beta(f):=(\jmath\circ f)_*"']
    &
    \beta(Y)
\end{tikzcd}
\end{equation*}
Note that the above correspondence is well-defined (when choosing up to homeomorphism), and the universal property of the Stone-\v{C}ech compactification naturally explains the functoriality of $\beta$.
Hence, one may argue as follows.
\begin{prop}
    The map $\beta: \textbf{Top}\rightarrow\textbf{CHaus}$ defined above is a covariant functor.
\end{prop}