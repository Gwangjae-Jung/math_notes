\section{Compactification}

In this section, we deals with a compactification more generally.
Indeed, such compactification covers a one-point compactification.
\begin{defi}[Compactification]
    Given a space $X$, a space $Y$ is called a compactification of $X$ if the following conditions are satisfied:
    \begin{enumerate}
        \item[(\romannumeral 1)]
        {
            $Y$ is a compact Hausdorff space containing $X$ as a subspace.
        }
        \item[(\romannumeral 2)]
        {
            The closure of $X$ in $Y$ is $Y$.
        }
    \end{enumerate}
    Two compactifications $Y_1$ and $Y_2$ of $X$ is said to be equivalent if there is a homeomorphism from $Y_1$ into $Y_2$ extending the identity map on $X$.
\end{defi}
Remark that a topological space which has a compactification is necessarily completely regular.
Hence, it is natural to wonder if every completely regular space has a compactification.
\cref{emb&cptf} suggests that every completely regular space has a compactification; it first suggests that every space which embeds into a compact Hausdorff space has a compactification, and second that if such compactification is required to satisfy an ``extension property'' then such compactification is unique up to equivalence.

\begin{thm}[Extension of an embedding to a compactification]\label{emb&cptf}
    Let $X$ be a space which admits an embedding $h$ of $X$ into a compact Hausdorff space $Z$.
    Then, there is a unique (up to equivalence) compactification $Y$ of $X$ with the following property:
    There is an embedding $H: Y\hookrightarrow Z$ extending $h$.\footnote{Here, $Y$ is unique up to equivalence, but $H$ need not be unique.}
    \begin{equation*}
        \begin{tikzcd}[row sep=large, column sep=huge]
            X \arrow[dr, "h"', hook] \arrow[r, "\textsf{inclusion}", hook] & Y \arrow[d, "H", hook, dashed] \\
            &Z
        \end{tikzcd}
    \end{equation*}
    Such compactification $Y$ is called the compactification of $X$ induced by $h$.
\end{thm}
\begin{proof}
    We first prove the existence of a compactification $Y$ of $X$ induced by $h: X\hookrightarrow Z$, and then we prove its uniqueness.

    \indent\textbf{Step 1. Finding a homeomorphic copy of a compactification of $X$.}\newline
    Let $X_0:=h(X)\approx X$ and $Y_0$ be the closure $\overline{X_0}$ of $X_0$ in $Z$.
    To argue that $Y_0$ is a compactification of $X_0$, we check the axioms of compactificaitions.
    Because $Y_0$ is a closed subset of $Z$, $Y_0$ is a compact Hausdorff space.
    Also, $Y_0$ contains $X_0$ as a subspace; every open subset of $X_0$ is of the form $X_0\cap O=X_0\cap(Y_0\cap O)$, where $O$ is an open subset of $Z$.
    Finally, the closure of $X_0$ in $Y_0$ is clearly $Y_0\cap\overline{X_0}=Y_0$.
    Therefore, $Y_0$ is a compactification of $X_0$.
            
    \indent\textbf{Step 2. Finding a compactification of $X$.}\newline
    To find a compactification $Y$ of $X$, we seek to find a space $Y$ such that $(X, Y)$ and $(X_0, Y_0)$ are homeomorphic.
    Let $A$ be any set disjoint from $X$ which is in bijection with $Y_0\setminus X_0$ (say $k: A\rightarrow Y_0\setminus X_0$ is such a bijection), and define $Y=X\sqcup A$.
    Define a bijective map $H: Y\rightarrow Y_0$ by
    \begin{align*}
        H(x)=\left\{
            \begin{array}{cc}
                h(x)    &   \textsf{(if $x\in X$)}\\
                k(x)    &   \textsf{(if $x\in A$)}
            \end{array}
        \right.
    \end{align*}
    Topologize $Y$ by declaring that $U\subset Y$ is open in $Y$ if and only if $H(U)$ is open in $Y_0$. (Indeed, the collection induced by such declaration is a topology on $Y$.)
    This topologization makes $H$ a homeomorphism.
    It is easy to justify that $Y$ is a compactification of $X$; it is because $H$ extends $h$ and $h: X\rightarrow X_0$ and $H: Y\rightarrow Y_0$ are homeomorphisms.
    To be brief, it is because the following diagram commutes.
    \begin{equation*}
        \begin{tikzcd}[row sep=large, column sep=huge]
            X\arrow[r, "h", "\approx"']\arrow[d, "\imath\,\textsf{(inclusion)}"', hook]
            &
            X_0\arrow[d, "\imath_0\,\textsf{(inclusion)}", hook]\\
            Y\arrow[r, "\approx", "H"'] & Y_0
        \end{tikzcd},
    \end{equation*}
    
    \indent\textbf{Step 3. Justifying the uniqueness of a compactification of $X$}\newline
    Suppose $Y_1$ and $Y_2$ are compactifications of $X$ which extends $h$ to embeddings from $Y_1$ and $Y_2$ into $Z$, respectively.
    Denote such embeddings by $H_1: Y_1\hookrightarrow Z$ and $H_2: Y_2\hookrightarrow Z$, respectively.
    By restricting the codomains of $H_1$ and $H_2$, we find that $H_2\circ H_1^{-1}$ denotes a homeomorphism between $Y_1$ and $Y_2$ extending the identity map on $X$.
    Therefore, $Y_1$ and $Y_2$ are equivalent.
\end{proof}
\begin{cor}
    A topological space $X$ has a compactification if and only if $X$ is completely regular.
    Moreover, in this case, the compactification of $X$ is unique up to equivalence.
\end{cor}
\begin{rmk}
    Since it is already proved that a topological space with a compactification is completely regular, it remains to justify that every completely regular space has a compactification.
    If $X$ is a completely regular space, by the embedding theorem, $X$ embeds into $[0, 1]^I$ for some $I$, which is a compact Hausdorff space; applying \cref{emb&cptf} gives a compactification of $X$.
    The uniqueness of a compactification of $X$ also follows from \cref{emb&cptf}.
\end{rmk}
