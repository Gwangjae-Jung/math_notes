\section{Separable extensions}

\begin{defi}[Separability]
    Given an algebraic field extension $E/F$, an element $\alpha\in E$ is said to be separable over $F$ if its minimal polynomial over $F$ is a separable polynomial.
    In particular, the extension $E/F$ is said to be a separable extension if every element in $E$ is separable over $F$.
    If every algebraic extension over a $F$ is separable, then $F$ is said to be perfect.
\end{defi}
\begin{exmp}
    \begin{enumerate}
        \item[(a)]
        {
            It is easy to check that an irreducible polynomial over a field of characteristic 0 is a separable polynomial.
            It will be proved in the next chapter that an irreducible polynomial over a finite field is also a separable polynomial.
            Hence, fields of characteristic 0 and finite fields are perfect fields.
        }
        \item[(b)]
        {
            Algebraically closed fields are perfect fields, and algebraic extensions of perfect fields are perfect fields.
        }
    \end{enumerate}
\end{exmp}

Some expereice from algebraic extensions helps studying separable extensions.
\begin{qst}\label{qst: separable extn}
    \begin{enumerate}
        \item[(a)]
        {
            Are the sum and multiplication of two separable elements separable?
        }
        \item[(b)]
        {
            Is a separable extension of a separable extension is a separable extension?
        }
    \end{enumerate}
\end{qst}

Our study begins with the degree which can measure the separability.
\begin{obs}
    Suppose that $E/F$ is a field extension and $\alpha\in E$ is algebraic over $F$.
    Then the minimal polynomial $m(t)$ of $\alpha$ over $F$ is irreducible and
    \begin{align*}
        m(t)=((t-\alpha_1)\cdots(t-\alpha_k))^n
    \end{align*}
    for some pairwise distinct elements $\alpha_1, \cdots, \alpha_k\in\ol F$ and $n\in\bb{Z}^{>0}$.
    It is clear from the above expression that $k$ is the number of distinct roots of the minimal polynomial of $\alpha$ over $F$ and that $k$ divides the extension degree of $F(\alpha)/F$.
    
    (Without loss of generality, assume $\alpha_1=\alpha$.)
    After choosing an index $i$, by \cref{iet 1}, there is a unique $F$-embedding $\sigma: F(\alpha)\hookrightarrow\ol F$ such that $\sigma(\alpha)=\alpha_i$, so $k\leq|\emb{F(\alpha)/F}|$.\footnote{When considering $\emb{E/F}$ as a set, it is assumed that an algebraic closure $\ol F$ of $F$ is given and we consider $F$-embeddings of $E$ into $\ol F$. Still, its cardinality does not depend on the choice of an algebraic closure of $\ol F$.}
    Conversely, given $\sigma\in\emb{F(\alpha)/F}$, since $m(t)$ is fixed by the action of $\sigma$, $\sigma(\alpha)$ is a root of $m(t)$; because every $F$-embedding of $F(\alpha)$ into $\ol F$  is determined by its action on $\alpha$, we have $k\geq|\emb{F(\alpha)/F}$.
    Therefore, $k$ also denotes the number of all distinct $F$-embeddings of $F(\alpha)$ into $\ol F$.
\end{obs}
Generalizing the above observation, we can define the separable degree of an algebraic extension as follows:
\begin{defi}\label{old separable degree}
    Let $E/F$ be an algebraic field extension, where $F\leq E\leq \ol F$ with $\ol F$ being the algebraic closure of $F$.
    The separable degree of $E/F$, denoted by $[E:F]_\sep$, is defined by the cardinality of the set of $F$-embeddings of $E$ into $\ol F$.
    In other words,
    \begin{align*}
        [E:F]_\sep:=|\{\tau: E\hookrightarrow\ol F: \textsf{$\tau$ is an $F$-embedding}\}|.
    \end{align*}
    \begin{equation*}
    \begin{tikzcd}[row sep=0.7cm, column sep=1.8cm]
        \ol F
        &
        \ol F
        \\
        E
            \arrow[u, dash]
            \arrow[r, "\tau", "\approx"', dashed]
        &
        \tau(E)
            \arrow[u, dash]
        \\
        F
            \arrow[u, dash]
            \arrow[r, "=", "\id{F}"']
        &
        F
            \arrow[u, dash]
    \end{tikzcd}
    \end{equation*}
\end{defi}

The above definition seems to be poorly defined, since the cardinality of the set of such $F$-embeddings seems to depend on the choice of an algebraic closure of $F$. (This is intuitively (and, in fact, logically) true, for an algebraic closure is unique up to isomorphism.)
Moreover, one may wish to generalize the definition so that $\tau$ extends a field isomorphism, not just the idenity map on $F$.
\begin{obs}
    Let $E/F$ be an algebraic extension and let $\ol F$ and $\widetilde F$ be algebraic extensions of $F$, which contains $E$.
    We will show that there is a bijection
    \begin{align*}
        \{\tau: E\hookrightarrow\ol F: \textsf{$\tau$ is an $F$-embedding}\}
        \longleftrightarrow
        \{\mu: E\hookrightarrow\widetilde F: \textsf{$\mu$ is an $F$-embedding}\},
    \end{align*}
    which explains that the above definition of separable degree does not depend on the choice of an algebraic closure of $F$.

    Consider the following diagram, where an $F$-embedding $\tau: E\hookrightarrow\ol F$ is given.
    \begin{equation*}
    \begin{tikzcd}[row sep=0.7cm, column sep=1.5cm]
        \widetilde F
        &
        &
        \ol F
            \arrow[ll, "\widetilde{\id{F}}"', "\approx"]
        \\
        \star
            \arrow[u, dash]
        &
        E
            \arrow[l, "f(\tau)"', "\approx", dashed]
            \arrow[r, "\tau", "\approx"']
        &
        \tau(E)
            \arrow[u, dash]
        \\
        F
            \arrow[u, dash]
        &
        F
            \arrow[l, "\id{F}", "="']
            \arrow[u, dash]
            \arrow[r, "\id{F}"', "="]
        &
        F
            \arrow[u, dash]
    \end{tikzcd}
    \end{equation*}
    For the diagram to be commutative, it should be satisfied that $f(\tau)=\widetilde{\id{F}}\circ\tau$.
    This establishes a map
    \begin{align*}
        f:
        \{\tau: E\hookrightarrow\ol F: \textsf{$\tau$ is an $F$-embedding}\}
        \rightarrow
        \{\mu: E\hookrightarrow\widetilde F: \textsf{$\mu$ is an $F$-embedding}\},
    \end{align*}
    \color{brown}which is a bijection\color{black}; thus, we may write down $[E: F]_\sep=|\emb{E/F}|$ without confusion.
\end{obs}
\begin{obs}
    As in the preceeding observation, assume that $F\leq E\leq \ol{F}$ is a tower of algebraic extensions, where $\ol F$ is an algebraic closure of $F$.
    And let $\sigma: F\hookrightarrow\ol F$ be a field embedding, i.e., $\sigma: F\rightarrow \sigma(F)$ is a field isomorphism.
    We will justify that
    \begin{align*}
        [E:F]_\sep=|\{\mu: E\hookrightarrow\ol F: \textsf{$\tau$ is a field embedding and $\tau|_F=\sigma$}\}|
    \end{align*}
    (the separable degree of a given algebraic extension does not depend on the base field isomorphism (embedding)) by showing that there is a bijection
    \begin{align*}
        \{\tau: E\hookrightarrow\ol F: \textsf{$\tau$ is an $F$-embedding}\}
        \longleftrightarrow
        \{\mu: E\hookrightarrow\ol F: \textsf{$\tau$ is a field embedding and $\tau|_F=\sigma$}\}.
    \end{align*}

    Consider the following diagram when an $F$-embedding $\tau: E\hookrightarrow\ol F$ is given.
    \begin{equation*}
    \begin{tikzcd}[row sep=0.7cm, column sep=1.5cm]
        \ol F
        &
        &
        \ol F
            \arrow[ll, "\widetilde{\sigma}"', "\approx"]
        \\
        \star
            \arrow[u, dash]
        &
        E
            \arrow[l, "g(\tau)"', "\approx", dashed]
            \arrow[r, "\tau", "\approx"']
        &
        \tau(E)
            \arrow[u, dash]
        \\
        \sigma(F)
            \arrow[u, dash]
        &
        F
            \arrow[l, "\sigma", "\approx"']
            \arrow[u, dash]
            \arrow[r, "\id{F}"', "="]
        &
        F
            \arrow[u, dash]
    \end{tikzcd}
    \end{equation*}
    For the diagram to be commutative, it should be satisfied that $g(\tau)=\widetilde{\sigma}\circ\tau$.
    By defining $g$ so, we have establised a map
    \begin{align*}
        g:
        \{\tau: E\hookrightarrow\ol F: \textsf{$\tau$ is an $F$-embedding}\}
        \rightarrow
        \{\mu: E\hookrightarrow\ol F: \textsf{$\tau$ is a field embedding and $\tau|_F=\sigma$}\}.
    \end{align*}
    \color{brown}which is a bijection\color{black}.
\end{obs}

By the preceeding two observations, we may re-define the separable degree of an algebraic extension as follows.
\begin{defi}[Separable degree]
    Let $E/F$ be an algebraic field extension and let $\ol F$ be the algebraic closure of $F$.
    And let $\sigma: F\hookrightarrow\ol F$ be a field embedding.
    Then the separable degree of $E/F$, denoted by $[E:F]_\sep$, is defined by the cardinality of field embeddings from $E$ into $\ol F$ whose restriction to $F$ is $\sigma$.
    In other words,
    \begin{align*}
        [E:F]_\sep:=|\{\tau: E\hookrightarrow\ol F: \textsf{$\tau$ is a field embedding and $\tau|_F=\sigma$}\}|.
    \end{align*}
    (As observed in the preceeding two observations, this definition is independent of the choice of an algebraic closure of $F$ and a field embedding $\sigma: F\hookrightarrow\ol F$.)
\end{defi}
\begin{rmk}
    Suppose that $\alpha$ is an element which is algebraic over $F$.
    Then $\alpha$ is separable over $F$ if and only if its separable degree and extension degree are the same.
\end{rmk}

In the remaining of this section, we will investigate some properties regarding separable degrees and separability of algebraic field extensions, and the primitive element theorem for finite separable extensions.

\begin{lem}\label{separable extn. multiplicativity}
    Suppose that $K/E$ and $E/F$ are algebraic field extensions.
    \begin{enumerate}
        \item[(a)]
        {
            $[K:F]_\sep=[K:E]_\sep[E:F]_\sep$.
        }
        \item[(b)]
        {
            In particular, if $E/F$ is a finite extension, then the separable degree of $E/F$ divides the extension degree of $E/F$.
        }
    \end{enumerate}
\end{lem}
\begin{proof}
    Fix an algebraic closure $\ol F$ of $F$ containing $K$.
    \begin{enumerate}
        \item[(a)]
        {
            Given $\sigma\in\emb{K/F}$, $\sigma$ is an extension of $\sigma|_E\in\emb{E/F}$, so $[K:F]_\sep\leq[K:E]_\sep[E:F]_\sep$.
            Conversely, there are $[E:F]_\sep$-distinct $F$-embeddings of $E$ into $\ol F$, thus there are at least $[K:E]_\sep[E:F]_\sep$-distinct $F$-embeddings of $K$ into $\ol F$, so $[K:F]_\sep\geq[K:E]_\sep[E:F]_\sep$.
        }
        \item[(b)]
        {
            Assume that $E/F$ is a finite extension and write $E=F(\alpha_1, \cdots, \alpha_n)$ for some elements $\alpha_1, \cdots, \alpha_n\in E$ which are algebraic over $F$.
            Considering the following tower of simple algebraic extensions:
            \begin{align*}
                F\leq F(\alpha_1)\leq F(\alpha_1, \alpha_2)\leq \cdots \leq F(\alpha_1, \cdots, \alpha_n)=E.
            \end{align*}
            In each simple algebraic extension, the separable degree divides the extension degree.
            Since each degree is multiplicative, we find that $[E:F]_\sep$ divides $[E:F]$.
        }
    \end{enumerate}
    This completes the proof.
\end{proof}

\begin{prop}
    Let $E/F$ be a finite extension.
    Then $E/F$ is a separable extension if and only if its separable degree and extension degree are the same.
\end{prop}
\begin{proof}
    Write $E=F(\alpha_1, \cdots, \alpha_r)$ for some $\alpha_1, \cdots, \alpha_r\in E$ which are algebraic over $F$, and consider a tower of simple algebraic extensions.
    If $E/F$ is a separable extension, then each simple extension is separable.
    Hence, the separable degree and the extension degree of each simple extension are the same, and $[E:F]=[E:F]_\sep$.
    Conversely, if $[E:F]=[E:F]_\sep$ and $\alpha\in E$, then $[F(\alpha):F]=[F(\alpha): F]_\sep$, implying that $E/F$ is a sepatable extension.
\end{proof}
\begin{rmk}
    In particular, if $\alpha_1, \cdots, \alpha_r\in\ol F$, then $F(\alpha_1, \cdots, \alpha_r)/F$ is a separable extension if and only if $\alpha_1, \cdots, \alpha_r$ are separable over $F$.
\end{rmk}

\begin{cor}\label{separability equivalence}
    Suppose that $K/E$ and $E/F$ are finite field extensions.
    Then $K/F$ is a separable extension if and only if both $K/E$ and $E/F$ are separable extensions.
\end{cor}
\begin{proof}
    It is clear that both $K/E$ and $E/F$ are separable extensions if $K/F$ is a separable extension.
    Assume conversely that $K/E$ and $E/F$ are separable extensions.
    Since $[K:E]=[K:E]_\sep$ and $[E:F]=[E:F]_\sep$, we have $[K:F]=[K:F]_\sep$, implying that $K/F$ is a separable extension.    
\end{proof}

In fact, \cref{separability equivalence} extends to the following proposition:
\begin{prop}
    Suppose that $K/E$ and $E/F$ are algebraic field extensions.
    Then $K/F$ is a separable extension if and only if both $K/E$ and $E/F$ are separable extensions.
\end{prop}
\begin{proof}
    It is clear that both $K/E$ and $E/F$ are separable extensions if $K/F$ is a separable extension.
    Assume conversely that $K/E$ and $E/F$ are separable extensions.
    Given $\alpha\in K$, there is the separable minimal polynomial $m(t)=t^n+a_{n-1}t^{n-1}+\cdots+a_1t+a_0\in E[t]$ of $\alpha$ over $E$, hence $\alpha$ is a separable element over $F(a_0, a_1, \cdots, a_{n-1})$.
    Since $a_i$ is a separable element over $F$, we can easily find that $F(\alpha, a_0, a_1, \cdots, a_{n-1})/F$ is a separable extension by showing that its extension degree and separable degree are the same.
    Thus, $\alpha$ is a separable element over $F$, and $K/F$ is a separable extension.
\end{proof}
\begin{cor}\label{the field of separable elements}
    Given an algebraic extension $E/F$, let $S_{E/F}$ be the collection of all elements of $E$ which are separable over $F$.
    Then $S_{E/F}$ is an intermediate subfield of $E/F$.
\end{cor}
\begin{proof}
    Given two elements $\alpha, \beta\in E$ which are separable over $F$, $F(\alpha, \beta)/F$ is a separable extension, so $\alpha\pm\beta, \alpha\beta^{\pm 1}$ are separable over $F$.
\end{proof}
\begin{rmk}
    Hence, if $\alpha, \beta\in E$ are algebraic over $F$, then so are $\alpha\pm\beta$ and $\alpha\beta^{\pm 1}$.
\end{rmk}

\begin{prop}
    Suppose that $E_1/F$ and $E_2/F$ are separable extensions, where $E_1, E_2$ are contained in $\ol F$.
    Then $(E_1E_2)/F$ is a separable extension.
\end{prop}
\begin{proof}
    Considering the form of the typical element of $E_1E_2$, it suffices to check that $\alpha\beta\in E_1E_2$ is separable over $F$, where $\alpha\in E_1$ and $\beta\in E_2$; now this is clear by \cref{the field of separable elements}.
\end{proof}

\begin{thm}[Primitive element theorem]\label{primitive element theorem}
    Suppose that $E/F$ is a finite \textit{separable} extension.
    Then there is an element $\alpha\in E$ such that $E=F(\alpha)$.
\end{thm}
\begin{proof}
    \color{teal}This proof is not well-motivated, yet.\color{black}

    Suppose that $F$ is a finite field.
    Because $E/F$ is a finite extension, $E$ is a finite field, hence there is a primitive element for $E/F$.
    Thus, we may assume that $F$ is an infinite field.
    It suffices to show that for any two elements $\beta, \gamma\in \ol F$ there is an element $\alpha\in \ol F$ such that $F(\gamma)=F(\alpha, \beta)$, for $E/F$ is a finite extension.
    Let $\{\beta_1, \cdots, \beta_r\}$ be the set of roots of the minimal polynomial $f(t)$ of $\beta$ over $F$, and let $\{\gamma_1, \cdots, \gamma_s\}$ be the set of roots of the minimal polynomial $g(t)$ of $\gamma$ over $F$, where $\beta_1=\beta$ and $\gamma_1=\gamma$.
    Because $F$ is infinite, there is an element $c\in F$ such that
    \begin{align*}
        c\neq\frac{\beta-\beta_i}{\gamma_j-\gamma}\quad(2\leq i\leq r, 2\leq j\leq s).
    \end{align*}
    Then let $\alpha=\beta+c\gamma$; we will show that $F(\beta, \gamma)=F(\alpha)$.
    Consider the polynomial $h(t)=f(\alpha-ct)\in F(\alpha)[t]$.
    \begin{enumerate}
        \item[(\romannumeral 1)]
        {
            $h(\gamma)=f(\beta)=0$, so the minimal polynomial $m(t)$ of $\gamma$ over $F(\alpha)$ divides $h(t)$.
        }
        \item[(\romannumeral 2)]
        {
            Since $c\gamma-c\gamma_j\neq\beta_i-\beta$ for all $2\leq i\leq r$ and $2\leq j\leq s$, we have $h(\gamma_j)=f(\beta+c\gamma-c\gamma_j)\neq 0$.
        }
    \end{enumerate}
    Because $m(t)$ also divides $g(t)$ and $E/F$ is a separable extension, $m(t)=t-\gamma$.
    So, $\gamma\in F(\alpha)$ and $\beta=\alpha-c\gamma\in F(\alpha)$, hence $F(\beta, \gamma)=F(\alpha)$.
\end{proof}