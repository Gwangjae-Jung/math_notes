\section{Algebraic closures}

\begin{defi}[Algebraic closedness]
    A field $F$ is said to be algebraically closed if every nonconstant polynomial over $F$ has a root in $F$.\footnote{Or equivalently, a field $F$ is said to be algebraically closed if every nonconstant polynomial over $F$ splits completely over $F$, i.e., the only irreducible polynomials over $F$ are the polynomials of degree 1.}
\end{defi}
\begin{obs}\label{obs: alg. cl.}
    For a field $F$, the following are equivalent:
    \begin{enumerate}
        \item[(a)]
        {
            $F$ is algebraically closed.
        }
        \item[(b)]
        {
            If $E/F$ is a field extension and $\alpha\in E$ is algebraic over $F$, then $\alpha\in F$.
        }
        \item[(c)]
        {
            If $E/F$ is an algebraic extension, then $E=F$.
        }
        \item[(d)]
        {
            If $E/F$ is a finite extension, then $E=F$.
        }
    \end{enumerate}
\end{obs}
\begin{proof}
    \hangindent=0.65cm
    \noindent(a)$\Rightarrow$(b):
    Since $F$ is algebraically closed, the minimal polynomial $m_\alpha(t)\in F[t]$ of $\alpha$ over $F$ splits completely over $F$, so $\alpha\in F$.

    \noindent(b)$\Rightarrow$(c)$\Rightarrow$(d):
    Clear.

    \noindent(d)$\Rightarrow$(a):
    Let $f(t)\in F[t]$ be a nonconstant polynomial and let $\alpha$ be a root of $f(t)$.
    Because $F(\alpha)/F$ is a finite extension, $\alpha\in F$.
\end{proof}

\begin{defi}[Algebraic closure]
    An algebraic extension $\ol F$ over $F$ is called an algebraic closure of $F$ if every nonconstant polynomial over $F$ has a root in $\ol F$ (or equivalently, every nonconstant polynomial over $F$ splits completely over $\ol F$).
\end{defi}
\begin{rmk}
    By definition, a field $F$ is algebraically closed if and only if $F$ is an algebraic closure of $F$.
\end{rmk}
The following proposition solves a technical problem when defining algebraic closures.
\begin{prop}
    If $\ol F$ is an algebraic closure of a field $F$, then $\ol F$ is algebraically closed.
    Hence, by \cref{obs: alg. cl.}, an algebraic closure of $\ol F$ is $\ol F$, i.e., there is no strictly larger algebraic extension over $\ol F$.
\end{prop}
\begin{proof}
    Let $f(t)$ be a nonconstant polynomial over $\ol F$ and let $\alpha$ be a root of $f(t)$.
    It suffices to show that $\alpha\in\ol F$.
    Since $\ol{F}(\alpha)/\ol{F}$ and $\ol{F}/F$ are algebraic extensions, $\ol{F}(\alpha)/F$ is an algebraic extension.
    Because the minimal polynomial of $\alpha$ over $F$ splits completely over $\ol{F}$, $\alpha\in\ol{F}$, as desired.
\end{proof}

\begin{thm}
    An algebraic closure of a field exists.
\end{thm}
\begin{proof}[Proof 1]
    We first introduce the proof due to Emil Artin.

    \indent\textbf{Step 1: A remarkable setting.}\newline\noindent
    Let $F$ be a field.
    For every nonconstant polynomial $f=f(x)\in F[x]$, let $x_f$ be an indeterminate and consider the polynomial ring $R:=F[x_f: f\in F[u]]$.
    In this polynomial ring, consider the ideal $I$ generated by the polynomials $f(x_f)$ for $f\in F[u]$.

    We now prove that $I$ is a proper ideal of $R$ by contradiction; assume $I=R$.
    Then we have a relation
    \begin{align*}
        g_1f_1(x_{f_1})+\cdots+g_nf_n(x_{f_n})=1,
    \end{align*}
    where $g_i\in R$ for $i=1, \cdots, n$.
    For simplicity, let $x_i=x_{f_i}$ and let $x_{n+1}, \cdots, x_m$ be the remaining variables occurring in the polynomials $g_j$ for $j=1, \cdots, n$.
    Then the above relation reads
    \begin{align*}
        g_1(x_1, \cdots, x_m)f_1(x_1)+\cdots+g_n(x_1, \cdots, x_m)f_n(x_n)=1.
    \end{align*}
    By Kronecker's thoerem, for each $i=1, \cdots, n$, there is a root $\alpha_i$ of $f_i$; letting $x_i=\alpha_i$ for each $i$, we have $0=1$, a contradiction.

    \textbf{Step 2: Deriving the result.}\newline\noindent
    There is a maximal ideal $M$ of $R$ containing $I$.
    Then $R/M$ is a field which contains an isomorphic copy of $F$.
    Moreover, the image of $x_f$ in $R/M$ is a root of a nonconstant polynomial $f(x)\in F[x]$, since $f(x_f)\in I\subset M$.
    Therefore, every nonconstant polynomial over $F$ has a root in $F_1:=R/M$.
    Continuing the above process on $F_1$ and so on, we obtain an ascending chain $F=F_0\leq F_1\leq F_2\leq \cdots$.

    Define $K:=\bigcup_{n=0}^\infty F_n$.
    Since $(F_n)_{n=0}^\infty$ is monotonically increasing, $K$ is a field.
    If $a(x)\in K[x]$, then $a(x)\in F_N[x]$ for some positive integer $N$, so $a(x)$ has a root in $F_{N+1}$ and in $K$, proving that $K$ is an algebraic closure of $F$.
\end{proof}
\begin{proof}[Proof 2]
    Invoking Zorn's lemma, we may prove the theorem.

    Given a field $F$, let $\mc{X}$ be the collection of all algebraic extensions over $F$.
    Since $F/F$ is algebraic, $\mc{X}$ is nonempty.
    Furthermore, $\mc{X}$ is partially ordered by set inlcusion (or by field extension).

    Given a chain $\mc{C}$ in $\mc{X}$, define
    \begin{align*}
        K=\bigcup_{E\in\mc{C}} E.
    \end{align*}
    \color{brown}It is clear that $K$ is a field and $K/F$ is algebraic, implying that $K\in\mc{X}$ and $K$ is an upper bound of $\mc{C}$. \color{black}

    By Zorn's lemma, $\mc{X}$ has a maximal member $L$.
    Assume that $L$ is not an algebraic closure of $F$.
    Then, there is a nonconstant polynomial with a root $\alpha$ such that $\alpha\notin L$.
    Because $L(\alpha)$ is a proper extension of $L$ and $L(\alpha)$ is an algebraic extension over $F$, the maximality of $L$ is disobeyed.
\end{proof}

In fact, an algebraic closure of a field is unique up to isomorphism.
See \cref{iet 3}.