\section{Basic properties}

\begin{defi}[Polynomial rings]
    Let $R$ be a commutative ring with the nonzero identity.
    The set $R[x]$ is defined as the collection of functions $f: \bb{Z}^{\geq 0}\rightarrow R$ with an integer $n\in\bb{Z}$ such that $f(k)=0$ whenever $k\geq n$.
    Polynomial rings with multiple indeterminates are defined inductively: $R[x_1, \cdots, x_n]:=R[x_1, \cdots, x_{n-1}][x_n]$.
    Imposing the usual operations on $R[x]$, $R[x]$ becomes a commutative ring with the nonzero identity.
\end{defi}
\begin{prop}[A universal property of polynomial rings]
    Let $R$ be a commutative ring with the nonzero identity.
    Let $\phi: R\rightarrow S$ be a ring homomorphism and let $\theta$ be an element of $S$.
    Then there is a unique ring homomorphism $\phi_*: R[x]\rightarrow S$ which extends $\phi$ and maps $x$ to $\theta$.
    \begin{equation*}
    \begin{tikzcd}[row sep=1.0cm, column sep=1.5cm]
        R\arrow[r, "\imath", hook]\arrow[dr, "\phi"']
        &
        R[x]\arrow[d, "\phi_*:\,x\mapsto\theta", dashed]\\
        &
        S
    \end{tikzcd}
    \end{equation*}
\end{prop}
\begin{proof}
    If such $\phi_*$ exists, then it should be satisfied that
    \begin{align*}
        \phi_*\left(\sum_{r=0}^n a_rx^r\right)=\sum_{r=0}^n a_r\theta^r.
    \end{align*}
    \color{brown}Checking details are left to readers.\color{black}
\end{proof}

A simple proposition follows.
\begin{prop}
    Let $I$ be an ideal of $R$ and let $(I)$ be the ideal of $R[x]$ generated by $I$, i.e., $(I)=R[x]I=I[x]$.
    Then $R[x]/(I)\approx(R/I)[x]$.
    In particular, if $I$ is a prime ideal of $R$, then $(I)$ is a prime ideal of $R[x]$.
\end{prop}
\begin{proof}
    Consider the projection map $f: R[x]\rightarrow(R/I)[x]$ defined by
    \begin{align*}
        f\left(\sum_{r=0}^n a_rx^r\right)=\sum_{r=0}^n \ol{a_r}x^r.
    \end{align*}
    \color{brown}Checking details are left to readers.\color{black}
\end{proof}

Another simple, but important, proposition follows.
\begin{thm}[Division algorithm on polynomial rings over fields]
    Let $F$ be a field and impose the following division algorithm on $F[x]$:
    \begin{center}
        Given $a(x), b(x)\in F[x]$ with $b(x)\neq 0$, find $q(x), r(x)\in F[x]$ such that
        \begin{align*}
            a(x)=q(x)b(x)+r(x),
        \end{align*}
        where either $r(x)=0$ or $\deg r(x)<\deg b(x)$.
    \end{center}
    \begin{enumerate}
        \item[(a)]
        {
            For each pair of $a(x)$ and $b(x)$, such $q(x)$ and $r(x)$ exist uniquely, respectively.
        }
        \item[(b)]
        {
            Hence, $F[x]$ is a Euclidean domain.
        }
    \end{enumerate}
\end{thm}
\begin{rmk}
    By the uniqueness part, when $E$ is a field extension of $F$ and $a(x)=Q(x)b(x)+R(x)$ for some $Q(x), R(x)\in E[x]$ with $R(x)=0$ or $\deg R(x)<\deg b(x)$, we have $Q(x)=q(x)$ and $R(x)=r(x)$.
\end{rmk}
\begin{proof}
    We prove the assertion by induction on $\deg a(x)$.
    \color{brown}(By removing the leading term of $a(x)$, the case is reduced to the case which is assumed by the induction hypothesis, proving the existence part.) \color{black}
    To prove the uniqueness part, assume $a(x)=q_1(x)b(x)+r_1(x)=q_2(x)b(x)+r_2(x)$ for some $q_i(x), r_i(x)\in F[x]$ with $r_i(x)=0$ or $\deg r_i(x)<\deg b(x)$ for $i=1, 2$.
    Because $r_1(x)-r_2(x)=(q_2(x)-q_1(x))b(x)$, considering the degrees of the polynomials, the uniqueness part can easily be explained.
\end{proof}

\begin{cor}
    Let $F$ be a field and $f(x)$ be a nonzero and nonunit element of $F[x]$.
    Because $F[x]$ is a Euclidean domain, $(f(x))$ is a maximal ideal of $F[x]$ if and only if $f(x)$ is an irreducible element of $F[x]$, i.e., $f(x)$ is an irreducible polynomial.
    Therefore, $F[x]/(f(x))$ is a field if and only if $f(x)$ is irreducible.
\end{cor}

\begin{cor}
    Let $F$ be a field and $p(x)$ be a polynomial over $F$.
    By the lattice isomorphism theorem, the ideals of $F[x]/(p(x))$ and the ideals of $F[x]$ containing $(p(x))$ are in bijection.
    Also, an ideal of $F[x]$ containing $(p(x))$ is of the form $(a(x))$ for some $a(x)\in F[x]$ with $a(x)|p(x)$ and vice versa.
    Therefore, the ideals of $F[x]/(p(x))$ are of the form $ol{(a(x))}$ with $a(x)|p(x)$ and vice versa.
\end{cor}

\begin{prob}
    Let $F$ be a field and $R=F[x, x^2y, x^3y^2, \cdots, x^{n+1}y^n, \cdots]\nmal F[x, y]$.
    \begin{enumerate}
        \item[(a)]
        {
            Show that the field of fractions of $R$ and $F[x, y]$ are the same. 
        }
        \item[(b)]
        {
            Explain why $R$ contains an ideal which is not finitely generated.
        }
    \end{enumerate}
\end{prob}
\begin{sol}
    \begin{enumerate}
        \item[(a)]
        {
            Clearly, the field $Q_R$ of fractions of $R$ is contained in the field $Q$ of fractions of $F[x, y]$.
            Conversely, if $f(x, y)\in F[x, y]$, for some large integer $N$, we have $x^Nf(x, y)\in R$, thus $Q$ is contained in $Q_R$.
        }
        \item[(b)]
        {
            Consider the following ascending chain $(x)\subsetneq (x, x^2y)\subsetneq (x, x^2y, x^3y^2)\subsetneq\cdots$ of ideals of $R$.
            If $R$ does not contain an infinitely generated ideal, then $R$ does not contain an infinite ascending chain of ideals, being a Noetherian ring.
        }
    \end{enumerate}
\end{sol}

\begin{prob}
    Prove that $(x^i-y^j)$ is a prime ideal of $R[x, y]$, whenever $i$ and $j$ are relatively prime positive integers.
\end{prob}
\begin{sol}
    Note from $R[x, y]=R[x][y]$ that every polynomial in $R[x, y]$ differs by a polynomial in $(x^i-y^j)$ by a polynomial in $R[x, y]$ with degree in $y$ less than $j$.
    In other words, given $a(x, y)\in R[x, y]$, there are $q(x, y), r(x, y)\in R[x, y]$ such that $a(x, y)=(x^i-y^j)q(x, y)+r(x, y)$ with $\deg_y r(x, y)<j$.

    Define a ring homomorphism $f: R[x, y]\rightarrow R[s]$ extending the identity map on $R$ by
    \begin{align*}
        f(x)=s^j,\quad f(y)=s^i.
    \end{align*}
    Then $f(a(x, y))=f(r(x, y))$, so
    \begin{align*}
        f(a(x, y))=k_0(s^j)+s^i k_1(s^j)+\cdots+s^{(j-1)i} k_{j-1}(s^j),
    \end{align*}
    where
    \begin{align*}
        r(x, y)=k_0(x)+k_1(x)y+\cdots+k_{j-1}(x)y^{j-1}\quad(k_0(x),\,k_1(x),\,\cdots,\,k_{j-1}(x)\in R[x]).
    \end{align*}
    Because $i$ and $j$ are relatively prime, the above summation is a partition of $f(r(x, y))$ regarding degree of each monomial in $s$ modulo $j$.
    In other words, all monomials in each summand $s^{mi} k_m(s^j)\,(m=0, 1, \cdots, j-1)$ has the same degree modulo $j$, and any two monomials in two other summands have distinct degree modulo $j$.
    Therefore, $f(a(x, y))=0$ if and only if $k_0(x)=k_1(x)=\cdots=k_{j-1}(x)=0$, i.e., $a(x, y)\in\ker f$ if and only if $a(x, y)\in(x^i-y^j)$.
    By the first isomorphism theorem, we have $R[x, y]/(x^i-y^j)\approx R[s]$, so $(x^i-y^j)$ is a prime ideal of $R[x, y]$.
\end{sol}