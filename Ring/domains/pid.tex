\section{Principal ideal domain}

\begin{defi}[Principal ideal domain]
    An integral domain in which every ideal is principal is called a principal ideal domain.
\end{defi}

\begin{rmk}
    Let $D$ be a principal ideal domain and $a, b\in D$.
    Write $(a, b)=(x)$ and $(a)\cap(b)=(y)$.
    Then, $x$ is a greatest common divisor of $a$ and $b$, and $y$ is a least common multiple of $a$ and $b$, which exist uniquely up to multiplication by units in $D$, respectively.
\end{rmk}

\begin{thm}
    In principal ideal domains, nonzero prime ideals and nonzero maximal ideals coincide.
\end{thm}
\begin{proof}
    It suffices to prove that any nonzero prime ideal of a prinicpal ideal $D$ is a maximal ideal of $D$.
    Let $P=(p)$ be a nonzero prime ideal of $D$, and suppose $P\leq I\nmal D$ with $I=(a)$ for some $a\in D$.
    Since $p\in (a)$, $p=ab$ for some $b\in D$.
    Because $p$ is a prime element of $D$, we have $p|a$ or $p|b$, which, respectively, implies $I=P$ or $a\in D^\times$ so that $I=D$.
    Therefore, every nonzero prime ideal of $D$ is a maximal ideal of $D$.
\end{proof}

We have observed that in any integral domain a prime element is an irreducible element and that nonzero prime ideals and nonzero maximal ideals coincide in principal ideal domains.
The following theorem states some equivalences in principal ideal domains.
\begin{thm}[Equivalences in principal ideal domains]
    Let $D$ be a principal ideal domain and $p$ be a nonzero element of $D$.
    Then the followings are equivalent:
    \begin{enumerate}
        \item[(a)]
        {
            $p$ is a prime element of $D$.
        }
        \item[(b)]
        {
            $p$ is an irreducible element of $D$.
        }
        \item[(c)]
        {
            $(p)$ is a prime ideal of $D$.
        }
        \item[(d)]
        {
            $(p)$ is a maximal ideal of $D$.
        }
    \end{enumerate}
\end{thm}
\begin{proof}
    (a) and (c) are verified to be equivalent when we defined prime elements; we have proved that (c) and (d) are equivalent; we have proved that (a) implies (b).
    Thus, it remains to prove that (b) implies any other statement; here, we will show that (b) implies (d).

    Suppose $(p)\leq I\nmal D$ and write $I=(a)$.
    We can write $p=ab$ for some $b\in D$, thus $a$ or $b$ is a unit in $D$, which, respectively, implies that $I=D$ or $I=P$, implying that $(p)$ is a maximal ideal of $D$.
\end{proof}