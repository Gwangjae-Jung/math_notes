\section{Euclidean domain}

\begin{defi}[Size function and Euclidean domain]
    Let $D$ be an integral domain.
    Any function $N: D\rightarrow\bb{Z}^{>0}$ such that $N(0)=0$ is called a size function on $D$.
    The integral domain $D$ is called a Euclidean domain if it has a size funciton $N$ on $D$ satisfying the following property:
    \begin{center}
        For any $a, b\in D$ with $b\neq 0$, there are $q, r\in D$ such that $a=qb+r$ with either $r=0$ or $N(r)<N(b)$.
    \end{center}
\end{defi}
\begin{exmp}
    Fields, the Gaussian integer ring $\bb{Z}[i]$ are Euclidean domains.
\end{exmp}

\begin{thm}
    Euclidean domains are principal ideal domains.
\end{thm}
\begin{proof}
    Let $D$ be a Euclidean domain and $I$ be an ideal of $D$.
    By the well-ordering principle of $\bb{N}$, there is a nonzero element $\alpha$ of $I$ with the smallest value of a size function on $D$.
    If $x\in I$, there are elements $q, r\in D$ such that $x=q\alpha+r$ with either $r=0$ or $N(r)<N(\alpha)$.
    Because $r=x-q\alpha$, $r$ is an element of $I$, which forces $r=0$ and $x=q\alpha$.
    Therefore, $I=(\alpha)$, proving that $D$ is a principal ideal domain.
\end{proof}