\section{Noetherian ring}

\begin{defi}[Noetherian ring]
    Let $R$ be a ring (not necessarily an integral domain).
    Then the followings are equivalent, and a ring satisfying any of the following property is called a Noetherian ring.
    \begin{enumerate}
        \item[(a)]
        {
            (Finite condition)
            Every ideal of $R$ is finitely generated.
        }
        \item[(b)]
        {
            (Ascending chain condition)
            Every ascending chain of ideals of $R$ is finite.
            To be precise, if $I_1, I_2, \cdots$ are ideals of $R$ such that $I_1\subset I_2\subset\cdots$, then there is an integer $n\in\bb{N}$ such that $I_j=I-k$ for all $j, k\geq n$.
        }
        \item[(c)]
        {
            (Maximal condition)
            Let $S$ be any nonempty collection of ideals of $R$ partially ordered by set inclusion.
            Then $S$ contains a maximal member.
        }
    \end{enumerate}
\end{defi}
\begin{rmk}
    Every principal ideal domain is a Noetherian ring, since it satisfies the finite conditon.
\end{rmk}
\begin{proof}
    We first prove that the finite condition implies the ascending chain condition.
    Let $I_1\subset I_2\subset\cdots$ be an ascending chain of ideals of $R$, and define
    \begin{align*}
        I=\bigcup_{n=1}^\infty I_n.
    \end{align*}
    One can easily check that $I$ is an ideal of $R$.
    By hypothesis, $I=(a_1, \cdots, a_k)$ for some $a_1, \cdots, a_k\in R$; because, for each $i$, $a_i\in I_j$ for some $j\in\bb{N}$, the ascending chain is finite.

    We now prove that the ascending chain condition implies the maximal condition.
    Let $S$ be a nonempty collection of ideals of $R$ and partially order $S$ by set inclusion.
    Choose a member $I_1$ of $S$; if $I_1$ is maximal, the proof is done.
    If $I_1$ is not maximal, there is another member $I_2$ of $S$ strictly containing $I_1$.
    By induction, given a non-maximal member $I_n$ of $S$, there is another member $I_{n+1}$ of $S$ strictly containing $I_n$.
    Because the ascending chain $I_1\subset I_2\subset\cdots$ is finite by hypothesis, there is an integer $n$ such that $I_n$ is maximal, which proves that $S$ contains a maximal member.
    (Hence, in this case, we did not have to apply Zorn's lemma.)

    Finally, we prove that the maximal condition implies the finite condition.
    Define the collection $S$ of ideals of $R$ by
    \begin{align*}
        S:=\{J\nmal R: J\subset I\textsf{ and }J\textsf{ is finitely generated}\}.
    \end{align*}
    By hypothesis, $S$ contains a maximal member $M$.
    We will show that $I=M$ by contradiction.
    Assume $M\subsetneq I$.
    Then there is an element $x\in I\setminus M$, thus $M\subsetneq (M, x)\subset I$.
    Because $(M, x)$ is also finitely generated, $(M, x)\in S$, so $M$ is not a maximal member of $S$, a contradiction.
\end{proof}