\section{Quotient topology}

\subsection{Quotient maps}

\begin{defi}[Quotient map]
    Let $X$ and $Y$ be topological spaces.
    A surjective map $p: X\rightarrow Y$ is called a quotient map if a subset $U$ of $Y$ is open if and only if $p^{-1}(U)$ is open in $X$.
\end{defi}
\begin{rmk}
    Be noted that a quotien map need not be an open map.
\end{rmk}
\begin{exmp}
    Here are some examples of quotient maps.
    \begin{enumerate}
        \item[(a)]
        {
            A homeomorphism is an injective quotient map, and vice versa.
        }
        \item[(b)]
        {
            An open (or closed) continuous surjection is a quotient map.
        }
        \item[(c)]
        {
            A continuous map which has a right inverse is a quotient map.
        }
        \item[(d)]
        {
            A retraction is a quotient map.\footnote{Given a topological space $X$ and its subspace $A$, a continuous map $r: X\rightarrow A$ such that $r(a)=a$ for all $a\in A$ is called a retraction of $X$ onto $A$.}
        }
    \end{enumerate}
\end{exmp}

For topological spaces $X$ and $Y$, we say a subset $A$ of $X$ is saturated with respect to the surjective map $p: X\rightarrow Y$ when $A$ contains every $p^{-1}(\{y\})$ that it intersects, i.e., $A$ is the inverse image of a subset of $Y$.
To say $p$ is a quotient map is equivalent to saying that $p$ is a continuous surjection mapping a saturated open (closed, respcetively) subspaces of $X$ onto an open (closed) subspaces of $Y$.

A restriction of a quotient map to a subspace need not be a quotient map.
One has, however, the following proposition:
\begin{prop}
    Let $p: X\rightarrow Y$ be a quotient map, and let $A$ be a subspace of $X$ that is saturated with respect to $p$.
    Let $q: A\rightarrow p(A)$ be the restriction of $p$ to $A$.
    \begin{enumerate}
        \item[(a)]
        {
            If $A$ is open or closed in $X$, then $q$ is a quotient map.
        }
        \item[(b)]
        {
            If $p$ is an open map or a closed map, then $q$ is a quotient map.
        }
    \end{enumerate}
\end{prop}
\begin{proof}
    We first verify the following two equations:
    \begin{align*}
        q^{-1}(V)=p^{-1}(V)\,(V\subset p(A))\quad\textsf{and}\quad
        p(U\cap A)=p(U)\cap p(A)\,(U\subset X).
    \end{align*}
    To check the first equation, we note that since $V\subset p(A)$ and $A$ is saturated, $p^{-1}(V)$ is contained in $A$, hence $q^{-1}(V)=p^{-1}(V)$.
    To check the second equation, we note the inclusion $p(U\cap A)\subset p(U)\cap p(A)$.
    If $y\in p(U)\cap p(A)$, then $y=p(u)=p(a)$ for some $u\in U$ and $a\in A$.
    Because $A$ is saturated, $p^{-1}(\{y\})$ is contained in $A$, so $u\in A$.
    \begin{enumerate}
        \item[(a)]
        {
            Suppose $A$ is an open subspace of $X$.
            We need to show that a subset $V$ of $p(A)$ is open whenever $q^{-1}(V)$ is open in $A$.
            Because $q^{-1}(V)$ is open in $A$ and $A$ is open in $X$, $q^{-1}(V)$ is open in $X$.
            Because $A$ is saturated and $V\subset p(A)$, $q^{-1}(V)=p^{-1}(V)$, so $V$ is open in $Y$ and $p(A)$.
        }
        \item[(b)]
        {
            Suppose $p$ is an open map.
            Again, we need to show that a subset $V$ of $p(A)$ is open whenever $q^{-1}(V)$ is open in $A$.
            Since $q^{-1}(V)=p^{-1}(V)$ and $q^{-1}(V)$ is open in $A$, $q^{-1}(V)=U\cap A$ for some open subspace $U$ of $X$.
            Hence, $V=q(U\cap A)=p(U\cap A)=p(U)\cap p(A)$, which is open in $p(A)$.
        }
    \end{enumerate}
    The same results hold for closed cases, which can be proved by replacing the words ``open'' by ``closed.''
\end{proof}
\begin{rmk}
    \begin{enumerate}
        \item[(a)]
        {
            It is clear that the composite of quotient maps is again a quotient map.
        }
        \item[(b)]
        {
            Cartesian product of quotient maps need not be a quotient map.
        }
    \end{enumerate}
\end{rmk}

\subsection{Quotient topology}

\begin{defi}[Quotient topology]
    Let $X$ be a topological space and $A$ be a nonempty set.
    Given a surjection $p: X\rightarrow A$, there is a unique topology $\mc{T}$ on $A$ relative to which $p$ is a quotient map, which is given as
    \begin{align}\label{quot_top_check}
        \mc{T}=\{U\subset A: p^{-1}(A)\textsf{ is open in }X\}.
    \end{align}
    This topology is called the quotient topology induced by $p$.
\end{defi}
\begin{prob}
    Check if the collection $\mc{T}$ in \cref{quot_top_check} is a topology on $A$.
\end{prob}
\begin{defi}[Quotient space]
    Let $X$ be a topological space and let $X^*$ be a partition of $X$.
    Let $p: X\rightarrow X^*$ be a surjection that maps each point of $X$ to the element of $X^*$ containing the point.
    In the quotient topology induced by $p$, the space $X^*$ is called a quotient space.
\end{defi}
\begin{rmk}
    The typical open set of $X^*$ is a collection of equivalence classes whose union is an open set of $X$.
\end{rmk}

\begin{thm}
    Let $p: X\rightarrow Y$ be a quotient map.
    Let $Z$ be a topological space and $f: X\rightarrow Z$ be a map which is constant on each set $p^{-1}(\{y\})$ for $y\in Y$.
    Then $f$ induces a map $\ol{f}: Y\rightarrow Z$ such that $\ol{f}\circ p=f$.
    Furthermore,
    \begin{enumerate}
        \item[(a)]
        {
            $\ol{f}$ is continuous if and only if $f$ is continuous.
        }
        \item[(b)]
        {
            $\ol{f}$ is a quotient map if and only if $f$ is a quotient map.
        }
    \end{enumerate}
    \begin{equation*}
    \begin{tikzcd}[row sep=1.0cm, column sep=1.5cm]
        X
        \arrow[d, "p"']
        \arrow[dr, "f"]
        &\\
        Y\arrow[r, "\ol{f}"']
        &
        Z
    \end{tikzcd}
    \end{equation*}
\end{thm}
\begin{proof}
    Well-definedness of $\ol f$ is almost clear.
    To prove (a) and (b), note that $f^{-1}=p^{-1}\circ(\ol f)^{-1}$.
    \begin{enumerate}
        \item[(a)]
        {
            It is clear that $f$ is continuous if $\ol f$ is continuous.
            To show the converse, assume $f$ is continuous and $U$ is an open subspace of $Z$.
            Since $f^{-1}(U)$ is open by continuity and $p$ is a quotient map, $\ol f^{-1}(U)$ is also open, hence $\ol f$ is a continuous map.
        }
        \item[(b)]
        {
            It is clear that $f$ is a quotient map if $\ol f$ is a quotient map.
            To show the converse, assume $f$ is a quotient map.
            By (a), $\ol f$ is a continuous map, and it is easy to check that $\ol f$ is a surjection.
            It, thus, remains to show that $U\subset Z$ is open if $\ol f^{-1}(U)$ is open.
            When $\ol f^{-1}(U)$ is open, $p^{-1}\circ\ol f^{-1}(U)$ is open, and its image under $f$ is open, and the image is given as $(f\circ p^{-1}\circ\ol f)(U)=U$, as desired.
        }
    \end{enumerate}
    Therefore, if $f$ is constant on each set $p^{-1}(\{y\})$ for $y\in Y$, then $\ol f$ is well-defined.
    Furthermore, $\ol f$ is continuous if and only if $f$ is continuous; $\ol f$ is a quotient map if and only if $f$ is a quotient map.
\end{proof}

\begin{cor}
    Let $f: X\rightarrow Z$ be a surjective continuous map.
    Let $X^*$ be the following collection of subsets of $X$:
    \begin{equation*}
        X^*=\{f^{-1}(\{z\}): z\in Z\}.
    \end{equation*}
    Give $X^*$ the quotient topology.
    \begin{enumerate}
        \item[(a)]
        {
            The map $f$ induces a bijective continuous map $f^*: X^*\rightarrow Z$, which is a homeomorphism if and only if $f$ is a quotient map.
        }
        \item[(b)]
        {
            If $Z$ is a Hausdorff space, then so is $X^*$.
        }
    \end{enumerate}
\end{cor}
\begin{proof}
    Letting $p: X\rightarrow X^*$ be the natural projection, $p$ is a quotient map.
    By the preceeding theorem, $f^*$ is a well-defined surjective continuous map.
    By definition of $X^*$, $f^*$ is injective.
    Furthermore, $f^*$ is a quotien map (hence, a homeomorphism) if and only if $f$ is a quotient map.

    Given two distinct points of $X^*$, their images under $f$ are distinct points of $Z$, hence there are disjoint neighborhoods separating those points; their preimage under $f$ are disjoint neighborhoods in $X^*$ separating the given two points.
\end{proof}
