\section{Quotient topology}

\subsection{Quotient maps}

\begin{defi}[Quotient map]
    Let $X$ and $Y$ be topological spaces.
    A surjective map $p: X\rightarrow Y$ is called a quotient map if a subset $U$ of $Y$ is open if and only if $p^{-1}(U)$ is open in $X$.
\end{defi}
\begin{rmk}
    Be noted that a quotient map need not be an open map.
\end{rmk}
\begin{exmp}
    Here are some examples of quotient maps.
    \begin{enumerate}
        \item[(a)]
        {
            A homeomorphism is an injective quotient map, and vice versa.
        }
        \item[(b)]
        {
            An open (or closed) continuous surjection is a quotient map.
        }
        \item[(c)]
        {
            For topological spaces $X$, $Y$, and $Z$, if $p: X\rightarrow Y$ and $q: Y\rightarrow Z$ are quotient maps, then $q\circ p: X\rightarrow Z$ is also a quotient map.
        }
    \end{enumerate}
\end{exmp}

For topological spaces $X$ and $Y$, we say a subset $A$ of $X$ is saturated with respect to a surjective map $p: X\rightarrow Y$ when $A$ contains every $p^{-1}(\{y\})$ that it intersects, i.e., $A$ is the inverse image of a subset of $Y$.
To argue $p$ is a quotient map is equivalent to argue that $p$ is a continuous surjection mapping a saturated open (closed, respcetively) subsets of $X$ onto an open (closed) subsets of $Y$.

A restriction of a quotient map to a subspace need not be a quotient map.
One has, however, the following proposition:
\begin{prop}
    Let $p: X\rightarrow Y$ be a quotient map, and let $A$ be a subspace of $X$ that is saturated with respect to $p$.
    Let $q: A\rightarrow p(A)$ be the restriction of $p$ to $A$.
    \begin{enumerate}
        \item[(a)]
        {
            If $A$ is open or closed in $X$, then $q$ is a quotient map.
        }
        \item[(b)]
        {
            If $p$ is an open map or a closed map, then $q$ is a quotient map.
        }
    \end{enumerate}
\end{prop}
\begin{proof}
    We first verify the following two statements:
    \begin{enumerate}
        \item[(\romannumeral 1)]
        {
            For any subset $V$ of $p(A)$, $q^{-1}(V)=p^{-1}(V)$.
        }
        \item[(\romannumeral 2)]
        {
            Whenever $U\subset X$, $p(U\cap A)=p(U)\cap p(A)$.
        }
    \end{enumerate}
    To verify (\romannumeral 1), we note that since $V\subset p(A)$ and $A$ is saturated, $p^{-1}(V)$ is contained in $A$, hence $q^{-1}(V)=p^{-1}(V)$.
    To verify (\romannumeral 2), we note the inclusion $p(U\cap A)\subset p(U)\cap p(A)$.
    If $y\in p(U)\cap p(A)$, then $y=p(u)=p(a)$ for some $u\in U$ and $a\in A$.
    Because $A$ is saturated, $p^{-1}(\{y\})$ is contained in $A$, so $u\in A$.
    \begin{enumerate}
        \item[(a)]
        {
            Suppose $A$ is an open subset of $X$.
            We need to show that a subset $V$ of $p(A)$ is open in $p(A)$ whenever $q^{-1}(V)$ is open in $A$.
            Because $q^{-1}(V)$ is open in $A$ and $A$ is open in $X$, $q^{-1}(V)$ is open in $X$.
            Because $A$ is saturated and $V\subset p(A)$, $q^{-1}(V)=p^{-1}(V)$, so $V$ is open in $Y$ and $p(A)$.
        }
        \item[(b)]
        {
            Suppose $p$ is an open map.
            Again, we need to show that a subset $V$ of $p(A)$ is open whenever $q^{-1}(V)$ is open in $A$.
            Since $q^{-1}(V)=p^{-1}(V)$ and $q^{-1}(V)$ is open in $A$, $q^{-1}(V)=U\cap A$ for some open subspace $U$ of $X$.
            Hence, $V=q(U\cap A)=p(U\cap A)=p(U)\cap p(A)$, which is open in $p(A)$.
        }
    \end{enumerate}
    The same results hold for closed cases, which can be proved by replacing the words ``open'' by ``closed.''
\end{proof}
\begin{rmk}
    Cartesian product of quotient maps need not be a quotient map.
    See \cref{the product of quotient maps need not be a quotient map}.
\end{rmk}

In practice, it is difficult to check if a given surjective continuous map is a quotient map by merely checking the defining condition.
The following statements will be helpful when justifying that a given surjective continuous map is a quotient map.
\begin{prop}
    \begin{enumerate}
        \item[(a)]
        {
            Let $p: X\rightarrow Y$ be a continuous function.
            If there is a function continuous $f: Y\rightarrow X$ such that $p\circ f=\id{Y}$, then $p$ is a quotient map.
            In short, a continuous map with a continuous right inverse is a quotient map.
        }
        \item[(b)]
        {
            A retraction is a quotient map.\footnote{Given a topological space $X$ and its subspace $A$, a continuous map $r: X\rightarrow A$ such that $r(a)=a$ for all $a\in A$ is called a retraction of $X$ onto $A$.}            
        }
    \end{enumerate}
\end{prop}
\begin{proof}
    \hangindent=0.65cm

    \noindent(a)
    We need to justify that a subset $U$ of $Y$ is open in $Y$ whenever $p^{-1}(U)$ is open in $X$.
    If $U$ be a subset of $Y$ such that $p^{-1}(U)$ is open, then $U=\id{Y}^{-1}(U)=f^{-1}(p^{-1}(U))$, so $U$ is open in $Y$.

    \noindent(b)
    Let $A$ be a subspace of $X$ and $p: X\rightarrow A$ be a retraction of $X$ onto $A$.
    It suffices to check that a subset $U$ of $A$ is open in $A$ whenever $p^{-1}(A)$ is open in $X$.
    Because $U=p(p^{-1}(U))=p^{-1}(U)\cap A$, $U$ is open in $A$.
\end{proof}

\subsection{Quotient topology}

\begin{defi}[Quotient topology]
    Let $X$ be a topological space and $A$ be a nonempty set.
    Given a surjection $p: X\rightarrow A$, there is a unique topology $\mc{T}$ on $A$ relative to which $p$ is a quotient map, which is given as
    \begin{align}\label{quot_top_check}
        \mc{T}=\{U\subset A: p^{-1}(A)\textsf{ is open in }X\}.
    \end{align}
    This topology is called the quotient topology induced by $p$.
\end{defi}
\begin{prob}
    Check if the collection $\mc{T}$ in \cref{quot_top_check} is a topology on $A$.
\end{prob}
\begin{defi}[Quotient space]
    Let $X$ be a topological space and let $X^*$ be a partition of $X$.
    Let $p: X\rightarrow X^*$ be the natural projection, i.e., the surjection that maps each point of $X$ to the member of $X^*$ containing the point.
    In the quotient topology induced by $p$, the space $X^*$ is called a quotient space.
\end{defi}
\begin{rmk}
    The typical open set of $X^*$ is a collection of equivalence classes whose union is an open set of $X$.
\end{rmk}

\begin{thm}
    Let $p: X\rightarrow Y$ be a quotient map.
    Let $Z$ be a topological space and $f: X\rightarrow Z$ be a map which is constant on each set $p^{-1}(\{y\})$ for $y\in Y$.
    Then $f$ induces a unique map $\ol{f}: Y\rightarrow Z$ such that $\ol{f}\circ p=f$.
    Indeed, $f$ is surjective if and only if $\ol{f}$ is surjective.
    Furthermore,
    \begin{enumerate}
        \item[(a)]
        {
            $\ol{f}$ is continuous if and only if $f$ is continuous.
        }
        \item[(b)]
        {
            $\ol{f}$ is a quotient map if and only if $f$ is a quotient map.
        }
    \end{enumerate}
    \begin{equation*}
    \begin{tikzcd}[row sep=1.0cm, column sep=1.5cm]
        X
        \arrow[d, "p"']
        \arrow[dr, "f"]
        &\\
        Y\arrow[r, "\ol{f}"']
        &
        Z
    \end{tikzcd}
    \end{equation*}
\end{thm}
\begin{proof}
    Well-definedness of $\ol f$ and its uniqueness are almost clear.
    Also, it is clear that $f$ is surjective if and only if $\ol{f}$ is surjective.
    Note that $f^{-1}=p^{-1}\circ\ol{f}^{-1}$.

    \hangindent=0.65cm
    \noindent(a)
    It is clear that $f$ is continuous if $\ol f$ is continuous.
    To show the converse, assume $f$ is continuous and $U$ is an open subset of $Z$.
    Since $f^{-1}(U)$ is open in $X$ and $p$ is a quotient map, $\ol{f}^{-1}(U)$ is also open in $Y$.
    Thus, $\ol f$ is a continuous map.

    \noindent(b)
    It is clear that $f$ is a quotient map if $\ol f$ is a quotient map.
    To show the converse, assume $f$ is a quotient map.
    By (a), $\ol f$ is a continuous map, and it is clear that $\ol f$ is a surjection.
    Thus, it remains to show that a subset $U$ of $Z$ is open in $Z$ if (and only if) $\ol{f}^{-1}(U)$ is open in $Y$.
    When $\ol f^{-1}(U)$ is open in $Y$, $f^{-1}(U)=(p^{-1}\circ\ol f^{-1})(U)$ is open in $X$, so $U$ is open in $Z$, as desired.
\end{proof}

\begin{cor}
    Let $f: X\rightarrow Z$ be a surjective continuous map.
    Let $X^*$ be the following collection of subsets of $X$:
    \begin{equation*}
        X^*=\{f^{-1}(\{z\}): z\in Z\}.
    \end{equation*}
    Give $X^*$ the quotient topology induced by the natural projection map $p: X\rightarrow X^*$.
    \begin{enumerate}
        \item[(a)]
        {
            The map $f$ induces a bijective continuous map $f^*: X^*\rightarrow Z$, which is a homeomorphism if and only if $f$ is a quotient map.
        }
        \item[(b)]
        {
            If $Z$ is a Hausdorff space, then so is $X^*$.
        }
    \end{enumerate}
\end{cor}
\begin{proof}
    \hangindent=0.65cm
    \noindent(a)
    Note that the natural projection map $p: X\rightarrow X^*$ is a quotient map.
    By the preceeding theorem, there is a well-defined continuous bijection $f^*: X^*\rightarrow Z$.
    Furthermore, $f^*$ is a quotient map (hence, a homeomorphism) if and only if $f$ is a quotient map.

    \noindent(b)
    Given two distinct points $a^*$ and $b^*$ of $X^*$, $\ol{f}(a^*)$ and $\ol{f}(b^*)$ are distinct points of $Z$, hence there are disjoint neighborhoods separating $\ol{f}(a^*)$ and $\ol{f}(b^*)$; their preimage under $\ol{f}$ are disjoint neighborhoods in $X^*$ separating $a^*$ and $b^*$.
\end{proof}

\subsection{Examples}

\begin{exmp}
    Define an equivalence relation $\sim$ on the plane $X=\bb{R}^2$ as follows:
    \begin{center}
        $(x_1, y_1)\sim(x_2, y_2)$ if and only if $x_1+y_1^2=x_2+y_2^2$.
    \end{center}
    \color{brown}(It is left as an exercise to check that $\sim$ is indeed an equivalence relation on $X$.) \color{black}
    Let $X^*$ be the corresponding quotient space.
    In fact, $X^*$ is the set which consists of the collections of the form $f^{-1}(\{c\})$ for $c\in\bb{R}$, where $f(x, y)=x+y^2$.
    Because $f$ is a continuous surjection onto $\bb{R}$, there is a unique continuous bijection $f^*$ such that $f^*\circ p=f$, where $p: X\rightarrow X^*$ is the natural projection.
    Furthermore, the map $\iota: \bb{R}\rightarrow\bb{R}^2$ defined by $\iota(x)=(x, 0)$ for all $x\in\bb{R}$ is a continuous right inverse of $f$, so $f$ is a quotient map.
    Therefore, $X^*$ and $\bb{R}$ are homeomorphic.

    Let $\sim$ denote the equivalence relation on $X$ defined as follows:
    \begin{center}
        $(x_1, y_1)\sim(x_2, y_2)$ if and only if $x_1^2+y_1^2=x_2^2+y_2^2$.
    \end{center}
    \color{brown}(It is left as an exercise to check that $\sim$ is indeed an equivalence relation on $X$.) \color{black}
    Now, $X^*$ is the collection of the collections of the form $g^{-1}(\{c\})$ for $c\geq 0$, where $g(x, y)=x^2+y^2$.
    Because $g$ is surjective and continuous, there is a unique bijective continuous map $f^*: X^*\rightarrow[0, \infty)$.
    Because the map $r: [0, \infty)\rightarrow\bb{R}^2$ defined by $r(x)=(\sqrt{x}, 0)$ for all $x\in[0, \infty)$ is a continuous right inverse of $g$, $g$ is a quotient map.
    Therefore, $X^*$ and $[0, \infty)$ (equipped with the order topology) are homeomorphic.
\end{exmp}
\begin{exmp}
    We will justify that $D^2$ and $S^3$ are homeomorphic.
    Define a map $f: D^2\rightarrow S^3$ by
    \begin{align*}
        f
        \begin{pmatrix}
            r\cos(\theta)\\
            r\sin(\theta)
        \end{pmatrix}
        =
        \begin{pmatrix}
            \sin(\pi r)\cos(\theta)\\
            \sin(\pi r)\sin(\theta)\\
            \cos(\pi r)
        \end{pmatrix},
    \end{align*}
    And let $X^*$ be the partition of $X$ defined by the collection of $f^{-1}(\{p\})$ for $p\in S^3$.
    Since $f$ is surjective and continuous, the naturally induced map $f^*: X^*\rightarrow S^3$ is also surjective and continuous.

    To show that $f^*$ is a homeomorphism, it suffices to show that $f^*$ is a quotient map, which can be justified by showing that $f$ is a quotient map.
    Generally, we show that $f$ is a quotient map by finding a continuous right inverse of $f$, which is quite difficult in this case.
    However, considering the nature of $f$, it is intuitively clear that a subset $U$ of $S^3$ is open in $S^3$ whenever its preimage $f^{-1}(U)$ is open in $D^2$.
\end{exmp}
\begin{exmp}
    Let $X^*$ be the quotient space $(S^1\times S^1)/\sim$, where $(z, w)\sim(u, v)$ if and only if $zw=uv$.
    Our goal is to justify that $X^*\approx S^1$.
    Consider the function $f: S^1\times S^1\rightarrow S^1$ defined by $f(u, v)=uv$.
    Then, the collection of all preimages of singletons in $S^1$ under $f$ is precisely the collection of all equivalence classes with respect to $\sim$.
    Because $f$ is continuous and surjective, there is a unique bijective continuous map $f^*: X^*\rightarrow S^1$.
    To show that $f^*$ is a homeomorphism, it suffies to show that $f$ is a quotient map.
    Considering the map $\iota: S^1\rightarrow S^1\times S^1$ defined by $\iota(z)=(z, 1)$, we can easily observe that $\iota$ is a continuous right inverse of $f$.
    Hence, $f$ is a quotient map, and it follows that $f^*$ is a homeomorphism.
\end{exmp}

\begin{prob}\label{the product of quotient maps need not be a quotient map}
    Let $\bb{R}_K$ be the set $\bb{R}$ equipped with the $K$-topology, and let $Y$ be the quotient space obtained from $\bb{R}_K$ by collapsing the set $K$ to a point.
    And let $p: \bb{R}_K\rightarrow Y$ be the quotient map.
    \begin{enumerate}
        \item[(a)]
        {
            Show that $Y$ satisfies the $T_1$ axiom but $Y$ is not a Hausdorff space.
        }
        \item[(b)]
        {
            Show that $p\times p: \bb{R}_K\times\bb{R}_K\rightarrow Y\times Y$ is not a quotient map.
        }
    \end{enumerate}
\end{prob}
\begin{sol}
    \begin{enumerate}
        \item[(a)]
        {
            Remark that the $K$-topology on $\bb{R}$ is generated as a basis by $\{(a, b): \textsf{$a, b\in\bb{R}$ and $a<b$}\}\cup\{(a, b)-K: \textsf{$a, b\in\bb{R}$ and $a<b$}\}$.

            Every singleton in $Y$ is of the form $\{a\}$ for $a\in\bb{R}\setminus K$ or $\{K\}$.
            With respect to the natural projection $p: \bb{R}_K\rightarrow Y$, the preimages of both types of singletons are closed in $\bb{R}_K$, so $Y$ satisfies the $T_1$ axiom.

            To argue that $Y$ is not a Hausdorff space, observe $0^*$ and $k^*$, where $k\in K$ and note that a typical open set in $Y$ is the collection of equivalence classes belonging to $Y$ whose union is an open set in $\bb{R}_K$.
            Given a neighborhood of $0^*$, there is a basis member $(a, b)-K$ of $\bb{R}_K$ containing $0$.
            If $k_1$ is a point of $K$ such that $0<k_1<b$, then $k^*=k_1^*$, so a neighborhood of $k^*$ in $Y$ necessarily contains the projection of a neighborhood of $k_1$ in $\bb{R}_K$.
            This justifies that there is no pair of disjoint neighborhoods of $0^*$ and $k^*$ in $Y$.
        }
        \item[(b)]
        {
            Because $Y$ is not a Hausdorff space, the diagonal of $Y$ is not closed in $Y\times Y$.
            The preimage of the diagonal of $Y$ under $p\times p$ is
            \begin{align*}
                \{(x, x): x\in\bb{R}\setminus K\}\cup (K\times K)=\{(x, x): x\in\bb{R}\}\cup (K\times K).
            \end{align*}
            Since the diagonal of $\bb{R}$ is closed in $\bb{R}\times\bb{R}$, it is also closed in a finer space $\bb{R}_K\times\bb{R}_K$; $K\times K$ is obviously closed in $\bb{R}_K\times\bb{R}_K$.
            Hence, the preimage of the diagonal of $Y$ under $p\times p$ is closed under $\bb{R}_K\times\bb{R}_K$.
            This is invalid, if $p\times p$ is a quotient map.
        }
    \end{enumerate}
\end{sol}