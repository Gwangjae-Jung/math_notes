\section{Basic observation regarding Galois extensions}

Remark that an algebraic field extension which is both separable and normal is called a Galois extension.
When $E/F$ is a Galois extension, we write $\aut{E/F}=\gal{E/F}$; because $E/F$ is a normal extension, we have $\aut{E/F}=\gal{E/F}=\emb{E/F}$ and we call $\gal{E/F}$ the Galois group of $E/F$ (note that $\aut{E/F}$ is a group with the multiplication being function composition.)
\begin{rmk}
    \begin{enumerate}
        \item[(a)]
        {
            (A review of \cref{equiv_normal_extensions})
            An algebraic field extension $E/F$ is a normal extension if and only if $\emb{E/F}=\aut{E/F}$.
            Hence, in particular, if $E/F$ is a finite extension, then $E/F$ is a normal extension if and only if $[E:F]_\sep=|\aut{E/F}|$.
        }
        \item[(b)]
        {
            (Finite Galois extension)
            A finite field extension $E/F$ is a Galois extension if and only if $[E:F]=|\aut{E/F}|$.
            \color{brown}(It is left as an exercise to check the equivalence.)\color{black}
        }
        \item[(c)]
        {
            Except for \Cref{infty_gal_extn}, all Galois extensions are assumed to be finite extensions.
        }
    \end{enumerate}
\end{rmk}

Note that if $K/F$ is a Galois extension and $E$ is an intermediate subfield of $K/F$, then $K/E$ is a Galois extension and $\gal{K/E}\leq\gal{K/F}$.
Conversely, given a subgroup $H$ of $\gal{K/F}$, we define the fixed field $K^H$ of $H$ in $K$ by
\begin{align*}
    K^H:=\{x\in K: \textsf{$\sigma x=x$ for all $\sigma\in H$}\}.
\end{align*}
It is easy to check that $F\leq K^H\leq K$.

In the following section, the fundamental theorem of finite Galois' extensions, also known as Galois' main theorem, is introduced, in which we are interested in a one-to-one bijection between the collection of the intermediate subfields of a finite Galois extension and the collection of the subgroups of the Galois group of the extension.
The following propositions, which could be introduced before proving Galois' main theorem, are moved to this section, due to its generality.

\begin{prop}
    Suppose that $E/F$ is a separable extension.
    If there is a positive integer $n$ such that $[F(\alpha): F]\leq n$ for all $\alpha\in E$, then ($E/F$ is a finite extension and) $[E:F]\leq n$. 
\end{prop}
\begin{proof}
    By assumption, there is an element $\beta\in E$ for which $[F(\beta): F]$ is the greatest.
    \begin{center}
        Goal: To show that $E=F(\beta)$.
    \end{center}
    It is easily deduced from the primitive element theorem for finite separable extensions.
    If $F(\beta)<E$, then there is an element $\gamma\in E\setminus F(\beta)$ and $F(\beta)<F(\beta, \gamma)$; because $F(\beta, \gamma)/F$ is a finite separable extension, there is an element $\alpha\in F(\beta, \gamma)$ such that $F(\beta, \gamma)=F(\alpha)$, which contradicts the maximality of $\beta$.
\end{proof}
\begin{lem}[Artin's theorem]
    Let $K$ be a field and $H$ be a finite subgroup of $\aut{K}$.
    Then $K/K^H$ is a Galois extension and $\gal{K/K^H}=H$.\footnote{If $H$ is an infinite subgroup of $\aut{K}$, Artin's theorem is not valid, in general. Hence, the map $E\mapsto\gal{K/E}$ is not a surjection, in general.}
\end{lem}
\begin{proof}
    We first show that $K/K^H$ is a Galois extension.
    For this, we show that for any $\alpha\in K$ the minimal polynomial $m(t)$ of $\alpha$ over $K^H$ is separable and has all roots in $K$.
    
    \color{magenta}Let $\{\sigma_1, \cdots, \sigma_r\}$ be a maximal subset of $H$ such that $\sigma_1\alpha, \cdots, \sigma_r\alpha$ are pairwise distinct. \color{black}
    Letting $\tau_i=\sigma_1^{-1}\circ\sigma_i$ for each integer $i=1, \cdots, r$, then $\tau_1\alpha, \cdots, \tau_r\alpha$ are pairwise distinct.
    If they are not maximally pairwise distinct, then there is another field automorphism $\tau_{r+1}$ of $K$ such that $\tau_1\alpha, \cdots, \tau_r\alpha, \tau_{r+1}\alpha$ are pairwise distinct; then $\sigma_1\alpha, \cdots, \sigma_r\alpha, \sigma_{r+1}\alpha$ are also pairwise distinct, which contradicts the maximality of $\{\sigma_1, \cdots, \sigma_r\}$.
    Hence, $\{\tau_1=\id{K}, \cdots, \tau_r\}$ is also a maximal subset of $H$ such that $\tau_1\alpha, \cdots, \tau_r\alpha$ are pairwise distinct; thus, we may assume that $\sigma_1=\id{K}$.

    Define a polynomial
    \begin{align*}
        f(t):=(t-\sigma_1\alpha)\cdots(t-\sigma_r\alpha),
    \end{align*}
    which is satisfied by $\alpha$.
    Given $\tau\in H$, by the maximality of $\{\sigma_1, \cdots, \sigma_r\}$, we have $(\tau\circ\sigma_i)(\alpha)\in\{\sigma_1\alpha, \cdots, \sigma_r\alpha\}$ for all $i$; $\tau\in H$ \textit{permutes} $\{\sigma_1\alpha, \cdots, \sigma_r\alpha\}$.
    Therefore, $f^\tau(t)=f(t)$ and $f(t)\in K^H[t]$, for every coefficient of $f(t)$ is fixed by every field automorphism in $H$.
    Because $f(t)$ is a multiple of $m(t)$ and $f(t)$ is separable, $m(t)$ is separable and $K/K^H$ is a separable extension.
    Moreover, a root of $m(t)$ is of the form $\sigma_i\alpha$ for some $i$, which is contained in $K$, so $K/K^H$ is a normal extension.
    Because $[F(\alpha): F]\leq r\leq |H|<\infty$ for all $\alpha\in K$, we conclude that $K/K^H$ is a finite Galois extension with $[K:K^H]\leq |H|$.

    We finally show that $\gal{K/K^H}=H$.
    It is clear that $H\leq\gal{K/K^H}$, thus it follows from
    \begin{align*}
        |H|\leq|\gal{K/K^H}|=[K:K^H]\leq|H|
    \end{align*}
    that $\gal{K/K^H}=H$.
\end{proof}