\section{Abelian extensions and solvable extensions}

\begin{defi}
    Let $K/F$ be a Galois extension.
    \begin{enumerate}
        \item[(a)]
        {
            (Abelian extension)
            $K/F$ is called an abelian extension if $\gal{K/F}$ is an abelian group.
        }
        \item[(b)]
        {
            (Solvable extension)
            $K/F$ is called a solvable extension if $\gal{K/F}$ is a solvable group.
        }
    \end{enumerate}
\end{defi}

Throughout this section, in order for Galois's theorem to be valid, we assume that all extensions are finite extensions.
We introduce some applications of Galois's theorem in some kinds of finite extensions.

\begin{obs}
    If $K/F$ is a finite cyclic(abelian) extension and $F\leq E\leq K$, then $K/E$ and $E/F$ are also cyclic(abelian) extensions.
    The converse is not true, in general.
\end{obs}
\begin{prop}
    Suppose that $F\leq E\leq K$ and $K/F$ is a finite Galois extension.
    If $E/F$ and $K/E$ are solvable extensions, then $K/F$ is also a solvable extension.
\end{prop}
\begin{proof}
    $\gal{E/F}\approx{\gal{K/F}}/{\gal{K/E}}$ and $\gal{K/E}$ are solvable, so $\gal{K/F}$ is solvable.
\end{proof}
\begin{prop}
    Suppose that $F\leq E\leq K$ and $K/F$ is a solvable extension.
    \begin{enumerate}
        \item[(a)]
        {
            $K/E$ is a solvable extension.
        }
        \item[(b)]
        {
            If $E/F$ is a Galois extension, then $E/F$ is a solvable extension.
        }
    \end{enumerate}
\end{prop}
\begin{proof}
    (a) is clear, because a subgroup of a solvable group is solvable.
    (b) is clear, because $\gal{E/F}\approx{\gal{K/F}}/{\gal{K/E}}$ is a quotient of a solvable group by its normal subgroup.
\end{proof}
\begin{prop}
    Suppose that $K/F$ is a finite Galois extension.
    Then the followings are equivalent:
    \begin{enumerate}
        \item[(a)]
        {
            $K/F$ is a (finite) solvable extension.
        }
        \item[(b)]
        {
            $K/F$ has an abelian tower; there are fields $F_1, \cdots F_k$ such that
            \begin{align*}
                F=F_0\leq F_1\leq \cdots\leq F_{k-1}\leq F_k=K
            \end{align*}
            and $F_i/F_{i-1}$ is an abelian extension for $i=1, \cdots, k$.
        }
    \end{enumerate}
\end{prop}
\begin{proof}
    (This equivalence is almost clear by Galois's theorem.)
    Assume first that $K/F$ is a finite solvable extension, and let $\{\id{K}\}=H_0\nmal H_1\nmal \cdots\nmal H_k=\gal{K/F}$ be a `solvability chain' of $\gal{K/F}$.
    It is easy to check that $K=K^{H_0}\geq K^{H_1}\geq \cdots\geq K^{H_k}=F$ and $K^{H_{i-1}}/K^H$ is an abelian extension.
    Assuming that $K/F$ has an abelian tower, Galois's theorem establishes a corresponding tower for $\gal{K/F}$, and it is clear that $\gal{F_i/F_{i-1}}\approx\gal{K/F_{i-1}}/\gal{K/F_i}$ is abelian.
\end{proof}

Some topics regarding abelian extensions will be introduced later when studying cyclotomic extensions.
In studying solvability of polynomial equations, we will justify that the equation is solvable by radicals if and only if its Galois group is a solvable group.