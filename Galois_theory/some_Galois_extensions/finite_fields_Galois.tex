\section{Galois extensions over finite fields}

Throughout this section, $p$ is a positive prime number and $q=p^n$ for some positive integer $n$.

\begin{thm}
    $\gal{\bb{F}_{p^n}/\bb{F}_p}=\genone{\sigma_p}\approx \mu_n$, where $\sigma_p: \bb{F}_{p^n}\rightarrow\bb{F}_{p^n}$ is the $\bb{F}_p$-automorphism defined by $\sigma_p(x)=x^p$ for $x\in\bb{F}_{p^n}$.
\end{thm}
\begin{proof}
    Of course, one could directly show that $\gal{\bb{F}_{p^n}/\bb{F}_p}$ and the cyclic group generated by $\sigma_p$ are the same.
    Its rigorous justification, however, seems to have technical difficulty.
    Thus, we prove the thoerem by justifying that the subgroup $\genone{\sigma_p}$ of the Galois group has the same order of the Galois group.

    Note that $|\gal{\bb{F}_{p^n}/\bb{F}_p}|=[\bb{F}_{p^n}: \bb{F}_p]=n$ and $\sigma_p\in \gal{\bb{F}_{p^n}/\bb{F}_p}$.
    Since $\sigma_p^r(x)=x^{p^r}$ for all integer $r$ and $x\in \bb{F}_{p^n}$, the order of $\sigma_p$ is $n$, as desired.
\end{proof}
\begin{obs}
    \begin{enumerate}
        \item[(a)]
        {
            Considering the subgroup lattice of $\mu_n$, there is a unique subgroup of index $d$, where $d$ is a positive divisor of $n$.
            By Galois' thoerem, it is equivalent to the statement that there is a unique intermediate subfield of $\bb{F}_{p^n}/\bb{F}_p$ whose extension degree over $\bb{F}_p$ is $d$.
            In fact, $\bb{F}_{p^d}$ is such a field, so $\bb{F}_{p^d}$ is a unique interediate subfield of $\bb{F}_{p^n}/\bb{F}_p$ whose extension degree over $\bb{F}_p$ is $d$.
            Because $\mu_n$ is abelian, $\bb{F}_{p^d}/\bb{F}_p$ is clearly a (finite) Galois extension, and we have
            \begin{align*}
                \gal{\bb{F}_{p^d}/\bb{F}_p}\approx\frac{\gal{\bb{F}_{p^n}/\bb{F}_p}}{\gal{\bb{F}_{p^n}/\bb{F}_{p^d}}}.
            \end{align*}        
        }
        \item[(b)]
        {
            Let $x$ be an element of $\bb{F}_{p^n}$ and $d$ be a positive divisor of $n$.
            By Galois' theorem, $x\in\bb{F}_{p^d}$ if and only if $\sigma_p^{n/d} x=x$.
            Writing $\bb{F}_{p^n}=\bb{F}_p(\alpha)$ for some $\alpha\in\bb{F}_{p^n}$, we can write $x=\alpha^k$ for some positive integer $k$.
        }
    \end{enumerate}
\end{obs}

\color{red}
\begin{prop}
    The polynomial $t^{p^n}-t$ is precisely the product of all the distinct irreducible polynomials over $\bb{F}_p$ of degree $d$, where $d$ runs through all positive divisors of $n$.
\end{prop}
\begin{proof}
    
\end{proof}
\color{black}

We have observed that $t^p-t\in\bb{F}_p[t]$ is a \textit{reducible} separable polynomial, whose roots are exactly the elements of $\bb{F}_p$.
If one adds a nonzero constant to the polynomial, one can get a irreducible separable polynomial, as illustrated in the following proposition.
\begin{prop}[Artin-Schreier extension]
    Let $p$ be a positive prime number and $a$ be a nonzero element of $\bb{F}_p$.
    \begin{enumerate}
        \item[(a)]
        {
            The polynomial $f(t)=t^p-t+a\in\bb{F}_p[t]$ is irreducible and separable over $\bb{F}_p$.
        }
        \item[(b)]
        {
            The splitting field $K$ for $f(t)$ over $\bb{F}_p$ is $\bb{F}_{p^p}$.
            Writing $K=\bb{F}_p(\alpha)$, $\gal{K/\bb{F}_p}=\genone{\sigma_p}$, where $\sigma_p: K\rightarrow K$ is the $\bb{F}_p$-automorphism defined by $\sigma_p\alpha=\alpha+1$.
        }
    \end{enumerate}
\end{prop}
\begin{proof}
    Because $f'(t)=-1\neq 0$, $f(t)$ is separable.
    To show the irreducibility of $f(t)$, observe that $\beta+1$ is a root of $f(t)$ if $\beta\in K$ is a root of $f(t)$.
    So, (after writing $\beta_i=\beta+(i-1)$ for each integer $1\leq i\leq p$) we may write
    \begin{align*}
        f(t)=(t-\beta_1)(t-\beta_2)\cdots(t-\beta_p).
    \end{align*}
    Letting $m_i(t)$ be the minimal polynomial of $\beta_i$ over $\bb{F}_p$, we have $m_{i+k}(t)=m_i(t-k)$ for all allowed indices $i, k$.
    Hence, if $f(t)$ is not irreducible, then $\deg m_1(t)=\cdots=\deg m_p(t)<p$, so $\beta_i\in\bb{F}_p$ for all $1\leq i\leq p$; then, $f(t)=t^p-t$ and $a=0$, a contradiction.
    Therefore, when $a\in\bb{F}_p$ is nonzero, $f(t)$ is an irreducible and separable polynomial over $\bb{F}_p$.

    Since $\deg f(t)=p$, the splitting field $K$ for $f(t)$ over $\bb{F}_p$ is isomorphic to the finite field $\bb{F}_{p^p}$ and $\gal{K/\bb{F}_p}\approx\mu_p$.
    Thus, we may write $K=\bb{F}_p(\alpha)$ for some $\alpha\in K$.
    To complete the proof, observe that the map $\tau: K\rightarrow K$ defined by $\tau\alpha=\alpha^p=\alpha-a$ is an $\bb{F}_p$-automorphism of $K$ of order $p$.
    (In addition, note that $\genone{\tau}=\genone{\sigma_p}\approx Z_p$.)
\end{proof}
