\section{Lagrange resolvent}\label{Lagrange_resolvent}

In this section, we assume the following:
\begin{enumerate}
    \item[(\romannumeral 1)]
    {
        $F$ is a field containing a primitive $n$-th root of unity, where $n$ is not divisible by $\ch{F}$.
    }
    \item[(\romannumeral 2)]
    {
        $K/F$ is a cyclic extension of degree $n$.
    }
\end{enumerate}
And let $\sigma$ be a generator of $\gal{K/F}$.

\begin{defi}[Lagrange resolvent]
    For $\alpha\in K$ and any $n$-th root of unity $\zeta$, define the Lagrange resolvent $\mc{L}_\sigma(\alpha, \zeta)\in K$ by
    \begin{align*}
        \mc{L}_\sigma(\alpha, \zeta)=\alpha+\zeta\sigma(\alpha)+\zeta^2\sigma^2(\alpha)+\cdots+\zeta^{n-1}\sigma^{n-1}(\alpha).
    \end{align*}
\end{defi}
\begin{obs}\label{the proof of cyclic->radical}
    Let $\zeta$ be any $n$-th root of unity.
    \begin{enumerate}
        \item[(a)]
        {
            By \cref{linear_independence_of_charaaters}, $\{\id{K}, \sigma, \cdots, \sigma^{n-1}\}$ is $K$-linearly independent.
            Hence, in particular, there is an element $\alpha\in K$ such that $\mc{L}_\sigma(\alpha, \zeta)\neq 0$.
        }
        \item[(b)]
        {
            One can easily find that $\sigma^k\mc{L}=\zeta^{-k}\mc{L}$ for all integer $k$.
            Thus, when $\zeta$ is a primitive $n$-th root of unity and $\alpha$ is given as in (a), $\id{K}\in\gal{K/F}$ is the unique automorphism fixing $\mc{L}$.
            Hence, $\mc{L}$ is contained in $K$ but not in proper subfield of $K$ containing $F$.
            This implies that $K=F(\mc{L})$.
        }
        \item[(c)]
        {
            Furthermore, since $\sigma\mc{L}=\zeta^{-1}\mc{L}$, we have $\sigma(\mc{L}^n)=(\zeta^{-1}\mc{L})^n=\mc{L}^n$.
            By Galois' theorem, we have $\mc{L}^n\in F$.
        }
    \end{enumerate}
    To sum up, if $\zeta$ is a primitive $n$-th root of unity and $\alpha$ is an element of $K$ such that $\mc{L}_\sigma(\alpha, \zeta)\neq 0$, then $K=F(\mc{L})$ and $\mc{L}^n$ belongs to $F$.
\end{obs}