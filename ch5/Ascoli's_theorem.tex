\section{Ascoli's theorem}

Throughout this section, we assume that $X$ is a topological space and $(Y, d)$ is a metric space, unless stated otherwise.
In this section, we study a generalized Ascoli's theorem, which characterizes the compact subspaces of $C(X, Y)$ in the topology of compact convergence.
Also, some particular versions of Ascoli's theorem will be introduced.

\subsection{Generalized Ascoli's theorem}

\begin{thm}[Ascoli's theorem]
    Let $X$ be a topological space and $(Y, d)$ be a metric space.
    Give $C(X, Y)$ the topology of compact convergence, and let $\mc{F}$ be a subset of $C(X, Y)$.
    \begin{enumerate}
        \item[(a)]
        {
            Suppose that $\mc{F}$ is pointwise equicontinuous under $d$ and pointwise relatively compact.
            Then $\mc{F}$ is contained in a compact subspace of $C(X, Y)$; to say equivalently, $\mc{F}$ is relatively compact.
        }
        \item[(b)]
        {
            The converse of (a) is valid if $X$ is a locally compact Hausdorff space.
        }
    \end{enumerate}
\end{thm}
\begin{rmk}
    For a Hausdorff space $A$, a subspace $B$ of $A$ is contained in a compact subspace of $A$ if and only if $A$ is relatively compact.
    Since if part is clear, we assume that $B$ is contained in a compact subpsace $K$ of $A$.
    Note that the closure $\ol{B}$ of $B$ in $X$ is contained in the closure $\ol{K}$ of $K$ in $X$.
    Since $A$ is a Hausdorff space, every compact subspace of $A$ is closed, so $\ol{K}=K$.
    Hence, $\ol{A}\subset K$, implying that $\ol{A}$ is a closed subspace of $K$.
    Therefore, $A$ is relatively compact.
\end{rmk}
\begin{proof}[Proof of (a)]
    Impose $Y^X$ the product topology, which equals to the topology of pointwise convergence.
    Note that $Y^X$ is a Hausdorff space and that $C(X, Y)$ (which equips to topology of compact convergence) is not a subspace of $Y^X$.
    And let $\mc{G}$ denote the closure of $\mc{F}$ in $Y^X$. (It will be clear in Step 1 why we consider the closure of $\mc{C}$ in $Y^X$, not in $C(X, Y)$.)

    Our proof then consists of the following four steps:
    \begin{itemize}
        \item
        {
            Step 1: Showing that $\mc{G}$ is a compact subspace of $Y^X$.
        }
        \item
        {
            Step 2: Showing that each member of $\mc{G}$ is continuous, and $\mc{G}$ is pointwise equicontinuous under $d$.
        }
        \item
        {
            Step 3: The topology on $Y^X$ and the topology on $C(X, Y)$ coincide on $\mc{G}$.
        }
        \item
        {
            Step 4: Concluding the proof of (a).
        }
    \end{itemize}

    \textbf{Step 1.}
    For each point $a\in X$, let $C_a$ denote the closure of $\mc{F}(a)$ in $Y$.
    By assumption, each $C_a$ is compact, hence the product $C=\prod_{a\in X} C_x$ is also compact.
    Because $\mc{F}\subset C\subset Y^X$, the closure of $\mc{F}$ in $Y^X$ is also contained in $C$; because $Y^X$ is a Hausdorff space, $\mc{G}$ is a compact subspace of $Y^X$.

    \textbf{Step 2.}
    It suffices to check that $\mc{G}$ is pointwise equicontinuous under $d$.
    Given $p\in X$, let $V_p$ be a neighborhood of $p$ in $X$ such that $f\in\mc{F}$ and $x\in V_p$ implies $d(f(x), f(p))<\epsilon$.
    Given $g\in\mc{G}$ and $x\in V_p$, we can find $h\in\mc{F}$ such that $h\in\pi_p^{-1}(B_d(g(p), \epsilon))\cap\pi_x^{-1}(B_d(g(x), \epsilon))$, i.e., $d(h(p), g(p))<\epsilon$ and $d(h(x), h(x))<\epsilon$.
    In this case,
    \begin{align*}
        d(g(x), g(p))\leq d(g(x), h(x))+d(h(x), h(p))+d(h(p), g(p))<3\epsilon,
    \end{align*}
    which proves that $\mc{G}$ is pointwise equicontinuous.

    \textbf{Step 3.}
    It is clear that the topology on $Y^X$ restricted to $\mc{G}$ is coarser than the topology on $C(X, Y)$ restricted to $\mc{G}$.
    Thus, it remains to show that the converse inclusion; for this, given a basis member $B_K(f, \epsilon)\cap \mc{G}$ of the latter topology (assume $f\in\mc{G}$), we need to find a basis member $B$ of the pointwise convergence topology on $Y^X$ such that $f\in B\cap\mc{G}\subset B_K(f, \epsilon)\cap\mc{G}$. (How could we skip some procedures?)

    Using pointwise equicontinuity of $\mc{G}$, for each point $x\in X$, let $U_x$ be a neighborhood of $x$ in $X$ such that $t\in U_x$ and $g\in\mc{G}$ implies $d(g(t), g(x))<\epsilon/4$; using compactness of $K$, cover $K$ with finitely many neighborhoods $U_{p_1}, \cdots, U_{p_n}$.
    Then, choose
    \begin{align*}
        B=\bigcap_{i=1}^n \pi_{p_i}^{-1}(B_d(f(p_i), \epsilon/4))
    \end{align*}
    so that whenever $g\in B\cap\mc{G}$ we have $d(g(p_i), f(p_i))<\epsilon/4$ for all $i$.
    If $g\in B\cap\mc{G}$ and $x\in K$, there is an index $i$ such that $x\in U_{p_i}$, so
    \begin{align*}
        d(g(x), f(x))\leq d(g(x), g(p_i))+d(g(p_i), f(p_i))+d(f(p_i), f(x))<\dfrac{3}{4}\epsilon.
    \end{align*}
    This implies that $g\in B_K(f, \epsilon)\cap\mc{G}$, as desired.

    \textbf{Step 4.}
    Using the results from Step 1 to Step 3, we derive that $\mc{F}$ is contained in a compact subspace of $C(X, Y)$.
    By Step 1 and Step 2, the closure $\mc{G}$ of $\mc{F}$ in $Y^X$ is a compact subspace of $Y^X$ (in the product topology) which is contained in $C(X, Y)$; by Step 3, $\mc{G}$ is also a subspace of $C(X, Y)$ (in the topology of compact convergence).
    Therefore, $\mc{F}$ is contained in a compact subspace $\mc{G}$ of $C(X, Y)$.
\end{proof}
\begin{proof}[Proof of (b)]
    Assume that $X$ is a locally compact Hausdorff space, and let $\mc{H}$ be a compact subspace of $C(X, Y)$ which contains $\mc{F}$.
    We show that $\mc{H}$ is pointwise equicontinuous under $d$ and pointwise (relatively) compact, from which it follows that $\mc{F}$ is pointwise equicontinuous under $d$ and pointwise relatively compact.

    We first show that $\mc{H}$ is pointwise compact.
    For this, we fix a point $a\in X$, and consider the following composition:
    \begin{equation*}
    \begin{tikzcd}[row sep=1.0cm, column sep=1.0cm]
        C(X, Y)\arrow[r, "\imath_a", hook]
        &
        X\times C(X, Y)\arrow[r, "\textsf{ev}"]
        &
        Y
        \\
        H\arrow[rr, "\jmath"']\arrow[u, hook]
        &
        &
        H(a)\arrow[u, hook]
    \end{tikzcd},
    \end{equation*}
    where $\imath_a$ maps $f\in C(X, Y)$ to $(a, f)\in X\times C(X, Y)$.
    Because $\imath_a$ is continuous for all $a\in X$ and $\textsf{ev}$ is continuous (see \cref{ev is continuous if the domain is LCH}, and note that the compact convergence topology on $C(X, Y)$ and the compact-open topology on the space coincide).
    Hence, $\jmath$ maps the compact subspace $\mc{H}$ of $C(X, Y)$ onto a compact $\mc{H}(a)$ of $Y$.

    We now show that $\mc{H}$ is pointwise equicontinuous under $d$; for this, we fix a point $a\in X$ and let $A$ be a compact subspace of $X$ containing a neighborhood of $a$ in $X$.
    Then it suffices to check that the following restriction
    \begin{align*}
        \mc{R}:=\{f|_A: f\in\mc{H}\}
    \end{align*}
    is equicontinuous at $a$.
    Give $C(A, Y)$ the compact convergence topology.
    Because $A$ is compact, the compact convergence topology on $C(A, Y)$ and the uniform topology on $C(A, Y)$ coincide.
    Also, it can be easily verified that the restriction map $r: C(X, Y)\rightarrow C(A, Y)$ is continuous (see the following problem), which implies that $\mc{R}=r(\mc{H})$ is a compact subspace of $C(A, Y)$.
    Hence, $\mc{R}$ is totally bounded under the uniform metric on $C(A, Y)$ induced by $d$, so $\mc{R}$ is pointwise equicontinuous under $d$.
\end{proof}
\begin{prob}
    Show that the restriction map $r: C(X, Y)\rightarrow C(A, Y)$ in the proof of (b) above is continuous.
\end{prob}
\begin{sol}
    Given $f\in C(X, Y)$ and a basis member $B_2:=B_K(f, \epsilon)$ of the topology on $C(A, Y)$, because $K$ is a compact subspace of $A$, $K$ is a compact subspace of $X$.
    Hence, one can consider the basis member $B_1:=B_K(f, \epsilon)$ of the topology on $C(X, Y)$, and this basis member is mapped onto $B_2$ by the restriction map.
\end{sol}

\subsection{Some particular versions of Ascoli's theorem}

We first introduce the following classical version of Ascoli's theorem.
The following remark plays as a technically essential tool in this subsection, and it worths considering the remark as a lemma:
\begin{rmk}
    When $(Y, d)$ is a complete metric space, relative compactness and total boundedness for subspaces of $Y$ coincide.
\end{rmk}
\begin{thm}[Classical version of Ascoli's theorem]
    Let $X$ be a compact space; let $(\bb{R}^n, d)$ denote the Euclidean space in either the square metric or the Euclidean metric; give $C(X, \bb{R}^n)$ the corresponding uniform topology (which coincides the compact convergence topology).
    For a subset $\mc{F}$ of $C(X, \bb{R}^n)$, the following are equivalent:
    \begin{enumerate}
        \item[(a)]
        {
            $\mc{F}$ is pointwise equicontinuous under $d$ and pointwise bounded.
        }
        \item[(b)]
        {
            $\mc{F}$ is relatively compact in $C(X, \bb{R}^n)$.
        }
    \end{enumerate}
\end{thm}
\begin{proof}
    Since boundedness and total boundedness coincide in $\bb{R}$, it is clear that (a) implies (b).
    To prove the converse implication, assume that $\mc{F}$ is relatively compact in $C(X, \bb{R}^n)$ and let $\mc{G}$ be the closure of $\mc{F}$ in $C(X, \bb{R}^n)$.
    What we want to show is the following:
    \begin{center}
        $\mc{G}$ is pointwise equicontinuous under $d$ and pointwise bounded.
    \end{center}
    We first prove that $\mc{G}$ is pointwise bounded.
    Since $\mc{G}$ is a compact subspace of $C(X, Y)$, given $\epsilon>0$, there are finitely many maps $f_1, \cdots, f_n\in\mc{F}$ such that $\mc{G}\subset\bigcup_{i=1}^n B_{\ol\rho}(f_i, \epsilon)$, thus $\mc{G}$ is pointwise bounded.
    Now we prove pointwise equicontinuity.
    Suppose $g\in\mc{G}$ and let $f\in\mc{F}$ be a point such that $\ol\rho(g, f)<\epsilon/3$; given a point $p\in X$, let $V$ be a neighborhood of $p$ in $X$ such that $x\in V$ and $h\in\mc{F}$ implied $d(h(x), h(p))<\epsilon/3$.
    We then have
    \begin{align*}
        d(g(x), g(p))\leq d(g(x), f(x))+d(f(x), f(p))+d(f(p), g(p))<\epsilon
    \end{align*}
    for all $x\in V$, proving pointwise equicontinuity of $\mc{G}$.
\end{proof}
\begin{cor}
    Let $X$ be a compact space; let $d$ denote either the square metric or the Euclidean metric on $\bb{R}^n$; give $C(X, \bb{R}^n)$ the corresponding uniform topology (which coincides the compact convergence topology).
    For a subset $\mc{F}$ of $C(X, \bb{R}^n)$, the following are equivalent:
    \begin{enumerate}
        \item[(a)]
        {
            $\mc{F}$ is compact.
        }
        \item[(b)]
        {
            $\mc{F}$ is closed in $C(X, \bb{R}^n)$, pointwise equicontinuous under $d$, and bounded under the sup metric $\rho$.
        }
    \end{enumerate}
\end{cor}
\begin{proof}
    Note the following observations:
    \begin{itemize}
        \item
        {
            By Ascoli's theorem, $\mc{F}$ is relatively compact if and only if $\mc{F}$ is pointwise equicontinuous under $d$ and pointwise totally bounded.
        }
        \item
        {
            By Heine-Borel theorem, $\mc{F}$ is compact if and only if $\mc{F}$ is complete and totally bounded under the sup metric.
        }
        \item
        {
            Because $C(X, Y)$ is complete under the sup metric, closedness and completeness coincide for subspaces of $C(X, Y)$.
        }
    \end{itemize}
    Hence, (a) implies (b).
    To show that (b) implies (a), it suffices to check that boundedness of $\mc{F}$ implies pointwise total boundedness of $\mc{F}$.
    Since $\mc{F}\subset B_{\ol\rho}(0, R)$ for some real number $R>0$, $\mc{F}(a)\subset[-R, R]$ for all $a\in X$.
    Being a subspace of a totally bounded space, $\mc{F}(a)$ is also totally bounded for each $a\in X$, as desired.
\end{proof}

As the second type, we assume that $X$ is a compact space and $(Y, d)$ is a complete metric spcae.
\begin{thm}
    Let $X$ be a compact space, $(Y, d)$ be a complete metric space, and $\mc{F}$ be a subset of $C(X, Y)$.
    Give $C(X, Y)$ the compact convergence topology (which coincides the uniform topology).
    \begin{enumerate}
        \item[(a)]
        {
            If $\mc{F}$ is pointwise equicontinuous and pointwise totally bounded, then $\mc{F}$ is relatively compact.
        }
        \item[(b)]
        {
            The converse of (a) holds if $X$ is a Hausdorff space.
        }
    \end{enumerate}
\end{thm}
\begin{rmk}
    Suppose $X$ is a compact Hausdorff space.
    Then $\mc{F}$ is relatively compact if and only if $\mc{F}$ is pointwise equicontinuous and pointwise totally bounded.
\end{rmk}
\begin{proof}
    The results follow from Ascoli's theorem and the first remark.
\end{proof}

As the third type, we assume that $X$ is a locally compact Hausdorff space and $(Y, d)$ is a complete metric space.
\begin{thm}
    Suppose $X$ is a locally compact Hausdorff space and $(Y, d)$ is a complete metric space.
    Give $C(X, Y)$ the compact convergence topology.
    Then a subset $\mc{F}$ of $C(X, Y)$ is relatively compact if and only if $\mc{F}$ is pointwise totally bounded and pointwise equicontinuous under $d$.
\end{thm}
\begin{proof}
    Note again that pointwise total boundedness and pointwise relative compactness coincide, since $(Y, d)$ is complete.
\end{proof}

\begin{cor}[Generalized Arzel\`{a}'s theorem]
    Let $X$ be a $\sigma$-compact Hausdorff space and $(Y, d)$ be a complete metric space.
    Give $C(X, Y)$ the compact convergence topology.
    If a sequence $(f_n)_{n\in\bb{N}}\subset C(X, Y)$ is pointwise equicontinuous and pointwise totally bounded under $d$, then the sequence $(f_n)_{n\in\bb{N}}$ has a subsequence which converges compactly in $C(X, Y)$.
\end{cor}
\begin{proof}
    \color{red}
    It is not solved, yet.
    \color{black}
\end{proof}
\begin{cor}[Arzel\`{a}'s theorem of undergraduate mathematical analysis]
    Let $X$ be a compact Hausdorff space and $(Y, d)$ be a complete metric space.
    Give $C(X, Y)$ the compact convergence topology (which coincides the uniform topology).
    If a sequence $(f_n)_{n\in\bb{N}}\subset C(X, Y)$ is pointwise equicontinuous and pointwise totally bounded under $d$, then the sequence $(f_n)_{n\in\bb{N}}$ has a subsequence which converges compactly (or uniformly, since $X$ is assumed to be compact) in $C(X, Y)$.
\end{cor}
\begin{proof}
    By the preceeding theorem, $(f_n)_{n\in\bb{N}}$ is relatively compact in $C(X, Y)$.
    Because $C(X, Y)$ is a Hausdorff space, the closure of $(f_n)_{n\in\bb{N}}$ is closed, which justifies the assertion.
\end{proof}
\begin{rmk}
    When $Y$ is given as $\bb{R}$ or $\bb{C}$, total boundedness reduces to boundedness, since they are equivalent on those metric spaces.
\end{rmk}
