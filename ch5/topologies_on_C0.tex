\section{Topologies on the space of continuous functions}

In this section, we first impose topologies on $C(X, Y)$ to which we can correspond pointwise convergence and uniform convergence.
Throughout this section, unless stated otherwise, $X$ and $Y$ are assumed to be topological spaces.

\subsection{Topology of pointwise convergence and uniform topology}

\begin{defi}[Topology of pointwise convergence]
    The topology on $Y^X$ generated by the collection of subsets
    \begin{align*}
        S(x, U):=\{f\in Y^X: f(x)\in U\}\quad(x\in X, U\textsf{ is open in }X)
    \end{align*}
    is called the topology of pointwise convergence, or the point-open topology.
\end{defi}
From now on, we say a sequence $(f_n)_{n\in\bb{N}}\subset Y^X$ converges to $f\in Y^X$ pointwise when the sequence converges to $f$ in the topology of pointwise convergence.
\begin{rmk}
    \begin{enumerate}
        \item[(a)]
        {
            Indeed, the topology on $Y^X$ of pointwise convergence and the product topology on $Y^X$ coincide; note that $S(x, U)=\pi_x^{-1}(U)$ for all $x\in X$ and open subspace $U$ of $X$.
        }
        \item[(b)]
        {
            The pointwise convergence coincides our old concept of pointwise convergence.
            To be specific, when $(f_n)_{n\in\bb{N}}\subset Y^X$, $f_n\rightarrow f$ pointwise if and only if $f_n(x)\rightarrow f(x)$ for each $x\in X$.
        }
    \end{enumerate}
\end{rmk}

When considering uniform convergence, the codomain must be given as a metric space.
\begin{defi}[Uniform topology]
    Assume $Y$ is a metric space with the metric $d$ on $Y$.
    The uniform topology on $Y$ is the topology on $Y^X$ induced by the uniform metric $\ol\rho$ induced by $d$.
\end{defi}
From now on, we say a sequence $(f_n)_{n\in\bb{N}}\subset Y^X$ converges to $f\in Y^X$ uniformly when the sequence converges to $f$ in the uniform topology.
\begin{rmk}
    \begin{enumerate}
        \item[(a)]
        {
            The uniform convergence coincides our old concept of uniform convergence.
        }
        \item[(b)]
        {
            (Review of \cref{Complete spaces of functions})
            Suppose $(f_n)_{n\in\bb{N}}\subset C(X, Y)$, where $Y$ is a metric space.
            If $f_n\rightarrow f$ uniformly, then $f\in C(X, Y)$; it is because $C(X, Y)$ is a closed subspace of $Y^X$, when $Y^X$ is assumed to equip the uniform metric.
        }
        \item[(c)]
        {
            (Review of \cref{families of totally bounded continuous maps are equicontinuous})
            If a subset of $C(X, Y)$ is totally bounded under the uniform metric, then the subset is pointwise equicontinuous under $d$.
        }
    \end{enumerate}
\end{rmk}
Another remark: When $X$ is a topological space and $Y$ is a metric space, the uniform topology on $Y^X$ is finer than the topology on $Y^X$ of pointwise convergence.

\subsection{Topology of compact convergence}
Still, we assume that $Y$ is a metric space.
\begin{defi}[Topology of compact convergence]
    The topology on $Y^X$ generated by the collection of the subsets
    \begin{align}\label{basis of the topology of compact convergence}
        B_C(f, \epsilon):=\left\{g\in Y^X: \sup_{x\in C} d(f(x), g(x))<\epsilon\right\}\quad
        \begin{pmatrix}
            \textsf{$C$ is a compact subspace of $X$,}\\
            f\in Y^X,\, \epsilon>0
        \end{pmatrix}
    \end{align}
    as a basis is called the topology of compact convergence or the topology of uniform convergence on compact sets.\footnote{The supremum of $d(f(x), g(x))$ for $x\in C$ is finite since $C$ is compact.}
\end{defi}
\begin{rmk}
    The collection of sets in \cref{basis of the topology of compact convergence} is indeed a basis of the topology on $Y^X$ of compact convergence, and its justification is given here.
    \begin{enumerate}
        \item[(1)]
        {
            It is clear that the collection covers $Y^X$.
        }
        \item[(2)]
        {
            Given two basis members $B_C(f, a)$ and $B_K(g, b)$, where $C$ and $K$ are compact subspaces of $X$ and $a, b>0$ and a point $u\in B_C(f, a)\cap B_K(g, b)$, we wish to find a basis member $B$ such that $u\in B\subset B_C(f, a)\cap B_K(g, b)$.
            \begin{center}
                For a point $u\in B_C(f, a)$, if $\delta=a-\sup\{d(f(x), u(x)): x\in C\}>0$, then $B_C(u, \delta)\subset B_C(f, a)$.
            \end{center}
            Using the above observation, we can find a small positive real numbera $\alpha$ such that $B_C(u, \alpha)\subset B_C(f, a)$ and $B_K(u, \alpha)\subset B_K(g, b)$, and $B_{C\cup K}(u, \alpha)$ is a desired basis member.
        }
    \end{enumerate}
\end{rmk}
From now on, we say a sequence $(f_n)_{n\in\bb{N}}\subset Y^X$ converges to $f\in Y^X$ compactly when the sequence converges to $f$ in the topology of compact convergence, i.e., $f_n\rightarrow f$ uniformly on every compact subspace of $X$.

\begin{thm}[Inclusions regarding topologies on $Y^X$]
    Assume $(Y, d)$ is a metric space.
    Then,
    \begin{align*}
        (\textsf{uniform topology})\supset(\textsf{topology of compact convergence})\supset(\textsf{topology of pointwise convergence}).
    \end{align*}
    Furthermore, if $X$ is compact, then the first two topologies coincide; if $X$ is discrete, then the last two coincide.
\end{thm}
\begin{proof}
    We first show the inclusion.
    \begin{enumerate}
        \item[(1)]
        {
            (The uniform topology is finer than the topology of compact convergence)
            Given a basis member $B_C(f, \epsilon)$ of the topology of compact convergence and its point $g$, we can find a positive real number $r<1$ such that $B_C(g, r)\subset B_C(f, \epsilon)$.
            It is clear that $g\in B_{\ol\rho}(g, r)\subset B_C(g, r)\subset B_C(f, \epsilon)$.
        }
        \item[(2)]
        {
            (The topology of compact convergence is finer than the topology of pointwise convergence)
            Given a basis member $S(x, B_d(p, \epsilon))$ of the topology of pointwise convergence and its point $g$, because $d(g(x), p)<\epsilon$, there is a positive real number $r$ such that $B_d(g(x), r)\subset B_d(p, \epsilon)$.
            Because $\{x\}$ is compact, it is clear that $g\in B_{\{x\}}(g, r/2)\subset S(x, B_d(p, \epsilon))$.
        }
    \end{enumerate}

    We now show the coincidence under each case.
    When $X$ is compact, then $B_\rho(f, \epsilon)$ is a basis member of the topology of compact convergence as well as a basis member of the uniform topology, proving the first coincidence.
    When $X$ is discrete and $B_C(f, \epsilon)$ is a basis member of the topology of compact convergence, the compact subspace $C$ of $X$ is necessarily a finite subset of $X$; hence, writing $C=\{x_1, \cdots, x_n\}$, we have $B_C(f, \epsilon)=\bigcap_{i=1}^n S(x_i, B_d(f(x_i), \epsilon))$.
\end{proof}

Before studying further properties of the topology of compact convergence, we introduce a topological space in which a subset is open if and only if its restriction to any compact subspace is open in the compact subspace.
\begin{defi}[Compactly generated space]
    A topological space $X$ is said to be compactly generated if a subspace $A$ of $X$ is open when the following property is satisfied:
    \begin{center}
        $A\cap C$ is open in $C$ whenever $C$ is a compact subspace of $X$.
    \end{center}
\end{defi}
\begin{rmk}
    We may replace the word ``open'' by ``closed,'' by considering set complements.
\end{rmk}
\begin{exmp}
    Some examples of compactly generated spaces are introduced here.
    \begin{enumerate}
        \item[(a)]
        {
            (Locally compact spaces)
            Suppose $X$ is a locally compact space and let $A$ be a subset of $X$ such that $A\cap C$ is open in $C$ whenever $C$ is a compact subspace of $X$.
            We wish to justify that $A$ is an open subspace of $X$.

            Given $x\in A$, choose a neighborhood $U$ of $x$ in $X$ that lies in a compact subspace $C$ of $X$.
            Since $A\cap C$ is open in $C$, $A\cap U=(A\cap C)\cap U$ is open in $U$, and hence in $X$.
            Then $A\cap U$ is a neighborhood of $x$ in $X$ contained in $A$, so $A$ is open in $X$.
        }
        \item[(b)]
        {
            (First-countable spaces)
            Suppose $X$ is a first-countable space and let $B$ be a subset of $X$ such that $B\cap C$ is closed in $C$ whenever $C$ is a compact subspace of $X$.
            We wish to justify that $B$ is a closed subset of $X$.

            Let $x$ be a point of $\ol{B}$, where the overline notation means the closure in $X$.
            Since $X$ has a countable base at $x$, there is a sequence $(x_n)_{n\in\bb{N}}$ of points in $B$ converging to $x$.
            The subspace $C=\{x\}\cup\{x_n: n\in\bb{N}\}$ is compact, so $B\cap C$ is closed in $C$.
            Because $B\cap C$ contains all $x_n$, it also contains $x$, so $x\in B$.
            Therefore, $B$ is closed.
        }
    \end{enumerate}
\end{exmp}

For pointwise convergence, the limit map of continuous maps need not be continuous; this is valid for uniform convergence.
For compact convergence, the assertion is valid when the domain is compactly generated, which is justified by the below theorem after a lemma.
\begin{lem}
    If $X$ is compactly generated, then a map $f: X\rightarrow Y$ is continuous if and only if for each compact subspace $C$ of $X$, the restriction $f|_C$ is continuous.
\end{lem}
\begin{proof}
    Only if part is obvious, and if part is also clear if one remarks that $(f|_C)^{-1}(U)=C\cap f^{-1}(U)$ whenever $U\subset Y$.
\end{proof}
\begin{thm}
    Let $X$ be a topological space, $(Y, d)$ be a metric space.
    Then $C(X, Y)$ is closed in $Y^X$ in the topology of compact convergence.
    (Hence, if a sequence $(f_n)_{n\in\bb{N}}\subset C(X, Y)$ converges to $f$ compactly, then $f$ is continuous.)
\end{thm}
\begin{proof}
    We need to show that a limit point $f$ of $C(X, Y)$ in $Y^X$ belongs to $C(X, Y)$.
    By the preceeding lemma, it suffices to show that $f|_C$ is continuous whenever $C$ is a compact subspace of $X$.
    Given a compact subspace $C$ of $X$, by considering the neighborhoods $B_C(f, 1/n)$ of $f$ in $Y^X$, we can find a sequence $(g_n^C)_{n\in\bb{N}}$ of points in $C(X, Y)$ such that $g_n^C\in B_C(f, 1/n)$.
    In this case, $g_n^C|_C\rightarrow f|_C$ uniformly, so $f|_C$ is continuous.
\end{proof}
\begin{rmk}
    The space $C(X, Y)$ ($X$ is a topological space and $(Y, d)$ is a metric space) is closed in $Y^X$ in the uniform topology, or in the topology of compact convergence when it is further assumed that $X$ is compactly generated.
\end{rmk}

\subsection{Compact-open topology}

When we introduced the uniform topology and the topology of compact convergence, we had to assume that $Y$ is a metric space (and the topologies seem to depend on the choice of the metric on $Y$).
It is natural to ask whether either of these topologies can be extended to the case where $Y$ is an arbitrary topological space.
It is known that there is no satisfactory answer to this question for the space $Y^X$; fortunately, one can prove something for its subspace $C(X, Y)$.

To study such case, in this subsection we assume $X$ and $Y$ are topological spaces.
\begin{defi}[Compact-open topology]
    The topology on $C(X, Y)$ generated by the collection of the subsets of the form
    \begin{align*}
        S(C, U):=\{f\in C(X, Y): f(C)\subset U\}
        \quad
        (\textsf{$C$ is a compact subspace of $X$ and $U$ is open in $X$})
    \end{align*}
    as a subbasis is called the compact-open topology.
\end{defi}
\begin{rmk}
    \begin{enumerate}
        \item[(a)]
        {
            The above definition of compact-open topology naturally extends to $Y^X$, which is not of our interest.
        }
        \item[(b)]
        {
            Clearly, the compact-open topology is finer than the topology of pointwise convergence (the point-open topology).
        }
    \end{enumerate}
\end{rmk}
\begin{obs}\label{compact-open and compact cnvg}
    When $(Y, d)$ is a metric space, the compact-open topology on $C(X, Y)$ and the topology of compact convergence on $C(X, Y)$ coincide.
\end{obs}
\begin{proof}
    The following result helps in this proof:
    \begin{quotation}
        Suppose $A$ is a compact subspace of $X$ and $V$ is an open subspace of $X$ containing $A$.
        Then, there is $\epsilon>0$ such that $V$ contains the $\epsilon$-neighborhood of $A$.
    \end{quotation}

    We first prove that the topology of compact convergence is finer than the compact-open topology.
    Given a subbasis member $S(C, U)$ of the compact-open topology ($C$ is compact and $U$ is open in $X$) and its point $f$, note that $f$ is continuous so that $f(C)$ is compact.
    Because $f(C)\subset U$, there is $\epsilon>0$ such that the $\epsilon$-neighborhood of $f(C)$ is contained in $U$.
    Therefore, $f\in B_C(f, \epsilon)\subset S(C, U)$.

    Now we prove that the compact-open topology is finer than the topology of compact convergence.
    Given a basis member $B_C(f, \epsilon)$ of the topology of compact convergence and its point $g$, without loss of generality, it suffices to find a member $A$ of the topology of compact convergence such that $f\in A\subset B_C(f, \epsilon)$.\footnote{There is no loss of generality, since we can find a positive real number $r$ such that $B_C(g, r)\subset B_C(f, \epsilon)$.}
    For each point $x\in C$, there is a neighborhood $U_x$ of $x$ in $X$ such that
    \begin{center}
        $t\in\ol{U_x}$ implies $d(f(t), f(x))<\epsilon/3$.\footnote{By letting $U_x$ be a neighborhood of $x$ in $X$ such that $t\in U_x$ implies $d(f(t), f(x))<\epsilon/4$, the result follows from the inclusion $f(\ol{U_x})\subset\ol{f(U_x)}$.}
    \end{center}
    Using the compactness of $C$, choose finitely many $V_i=U_{x_i}\,(i=1, \cdots, n)$ covering $C$.
    If we write $C_i=\ol{V_i}\cap C$ for each $i$, the finite intersection $\bigcap_{i=1}^n S(C_i, B_d(f(x_i), \epsilon/3))$ contains $f$ and is contained in $B_C(f, \epsilon)$.
\end{proof}
In the beginning of this subsection, it is mentioned as a statement that the compact convergence topology (as well as the uniform topology) seems to depend on the metric on the codomain.
By \cref{compact-open and compact cnvg}, we can obtain the following coincidences:
\begin{cor}\label{compact convergence topology is independent}
    If $Y$ is a metric space (so that the compact convergence topology and the compact-open topology coincide), the compact convergence topology on $C(X, Y)$ is independent of the metric on $Y$.
    Hence, if it is further assumed that $X$ is compact (so that the uniform topology and the compact convergence topology coincide), the uniform topology on $C(X, Y)$ is independent of the metric on $Y$.
\end{cor}

\subsubsection{Continuity of evaluation maps under the compact-open topology on $C(X, Y)$}

We end this section with some theory which will be helpful when studying Ascoli's theorem in the following section.

\begin{defi}[Evaluation maps]
    Let $A$ be a nonempty set and $P$ be a collection of maps from $A$ into a nonempty set $B$.
    The map $\textsf{ev}_P: A\times P\rightarrow B$ defined by $\textsf{ev}_P(a, f)=f(a)$ for $a\in A, f\in P$ is called the evaluation map with maps in $P$.
    If the context is clear, we may omit the subscript.
\end{defi}

\begin{exmp}
    Let $X$ and $Y$ be topological spaces and consider the evaluation map $\textsf{ev}: X\times C(X, Y)\rightarrow Y$.
    Suppose further that there is a metric $d$ on $Y$ inducing the topology on $Y$.
    Show that, under the compact-open topology on $C(X, Y)$, that the evaluation map is continuous.
\end{exmp}

\begin{thm}\label{ev is continuous if the domain is LCH}
    Let $X$ be a locally compact Hausdorff space and suppose $C(X, Y)$ equips the compact-open topology.
    Then the evaluation map $\textsf{ev}: X\times C(X, Y)\rightarrow Y$ is continuous.
\end{thm}
\begin{proof}
    To show the continuity of the evaluation map, it suffices to show that there is a neighborhood of $(a, f)\in X\times C(X, Y)$ which is mapped into an arbitrarily given neighborhood $W$ of $f(a)$ in $Y$.
    Using the continuity of $f$, choose a neighborhood $U$ of $a$ in $X$ such that $f(U)\subset W$.
    By hypothesis, there is a neighborhood $V$ of $a$ whose closure in $X$ is a compact subspace of $U$.
    The box $U\times S(\ol{U}, W)$ is a desired neighborhood of $(a, f)$.
\end{proof}

A consequence of this theorem is the theorem that follows.
\begin{defi}
    Given a map $f: X\times Z\rightarrow Y$, there is a corresponding map $F: Z\rightarrow Y^X$, defined by $(F(z))(x)=f(x, z)$ for all $x\in X$ and $z\in Z$.
    Conversely, given $F: Z\rightarrow Y^X$, the above equation defines a corresponding map $f: X\times Z\rightarrow Y$.
    We say that $F$ is the map of $Z$ into $C(X, Y)$ which is induced by $f$.
\end{defi}
\begin{thm}
    Let $X$ and $Y$ be topological spaces and give $C(X, Y)$ the compact-open topology.
    If $f: X\times Z\rightarrow Y$ is continuous, then so is its induced map $F: Z\rightarrow C(X, Y)$.
    The converse holds if $X$ is a locally compact Hausdorff space.
\end{thm}
\begin{proof}
    We first prove that $f$ is continuous when $X$ is a locally compact Hausdorff space and $F$ is continuous.
    Note that if $\alpha_i: A_i\rightarrow B_i$ denotes a continuous map for each $i=1, \cdots, n$ then $\alpha_1\times\cdot\alpha_n$ is also continuous, and remark the following commutative diagram.
    \begin{equation*}
    \begin{tikzcd}[column sep=huge]
        X\times Z
        \arrow[rr, bend left, "f"]
        \arrow[r, "\textsf{id}_Z\times F"']
        &
        C(X, Y)\arrow[r, "\textsf{ev}"']
        &
        Y
    \end{tikzcd}
    \end{equation*}

    We now prove that $F$ is continuous if $f$ is continuous.
    For this, we fix a point $z_0\in Z$ and seek to find a neighborhood $W$ of $z_0$ in $Z$ such that $F(W)\subset S(C, U)$, where $S(C, U)$ is any subbasis member of the compact-open topology on $C(X, Y)$ which contains $F(z_0)$, i.e., $f(C\times\{z_0\})\subset U$.
    By definition, for each $c\in C$, $f(c, z_0)\in U$; by continuity, there is an open box $A_c\times B_c$ in $X\times Z$ containing $(c, z_0)$ which is mapped into $U$ under $f$.
    Since $C\times\{z_0\}$ is compact, this slice can be covered by finitely many boxes of the form $A_c\times B_c$; write $C\times\{z_0\}\subset\bigcup_{i=1}^n (A_{c_i}\times B_{c_i})$ and $B=\bigcap_{i=1}^n B_{c_i}$.
    Clearly, $B$ is a neighborhood of $z_0$ in $Z$ and $f(C\times B)\subset U$, i.e., $F(B)\in S(C, U)$, as desired.
\end{proof}
