\section{Complete metric spaces}

\subsection{Complete metric spaces and uniform metrics}

\begin{defi}[Complete metric space]
    Let $(X, d)$ be a metric space.
    A sequence $(x_n)_{n\in\bb{N}}$ of points in $X$ is called a Cauchy sequence if the following condition is satisfied: For any positive real number $\epsilon$, there is a positive integer $N$ such that $n, m\geq N$ implies $d(x_n, x_m)<\epsilon$.
    The metric space $(X, d)$ is called a complete metric space if every Cauchy sequence in $X$ is convergent.
\end{defi}
\begin{rmk}
    The completeness of a metric space is a metric property, rather than a topological property.
    Hence, the completeness may not be preserved under homeomorphisms.
\end{rmk}

Some basic properties regarding the completeness of metric spaces are given as follows.
Proving them is left as an exercise.
See \cref{completeness_basic_sol}.
\begin{prop}\label{completeness_basic}
    Let $(X, d)$ and $(X_n, d_n)$ be complete metric spaces for $n\in\bb{N}$.
    \begin{enumerate}
        \item[(a)]
        {
            (Reduction of criterion)
            $X$ is complete if and only if every Cauchy sequence in $X$ has a convergent subsequence.
        }
        \item[(b)]
        {
            (Completeness and closedness)
            Let $Y$ be any nonempty subset of $X$ and let $e=d|_{Y\times Y}$.
            Then $(Y, e)$ is complete if and only if $Y$ is closed in $X$.
        }
        \item[(c)]
        {
            (Completeness and countable products)
            The product space $Z=\prod_{n\in\bb{N}} X_n$ is complete under the metric $D: Z\times Z\rightarrow [0, \infty)$, which is defined by
            \begin{align*}
                D(x, y)=\sup_{n\in\bb{N}}\left\{\frac{1}{n}\ol{d_n}(x_n, y_n)\right\}
                \quad
                (x, y\in Z),
            \end{align*}
            where $\ol{d_n}$ is the standard bounded metric induced by $d_n$ for each $n\in\bb{N}$.
            In fact, defining a metric $\kappa: Z\times Z\rightarrow [0, \infty)$ by $\kappa(x, y)=\sum_{n\in\bb{N}} {2^{-n} \ol{d_n}(x_n, y_n)}$ for all $x, y\in Z$, $Z$ is also complete under $\kappa$.
        }
    \end{enumerate}
\end{prop}

Let $(X, d)$ be a metric space (not necessarily being complete) and $I$ be a nonempty index set.
In \cref{uniform topology}, we have defined the uniform metric $\ol\rho$ on $X^I$ correspoinding to $d$, which is defined by $\ol\rho(f, g)=\sup\left\{\ol{d}(f(a), g(a)): a\in I\right\}$.
Remark that the metric topology induced by the uniform metric is finer than the product topology and coarser than the box topology.

In (c) of \cref{completeness_basic}, the completeness of the product of complete metric spaces is ensured only for countable products.
The following theorem ensures the completeness for uncountable products, when the the uniform topology is considered.
\begin{thm}\label{uniform topology and completeness}
    Let $X$ be a nonempty set.
    If $(Y, d)$ is a complete metric space, then $(Y^X, \ol\rho)$ is complete.
\end{thm}
\begin{proof}
    Let $(f_n)_{n\in\bb{N}}$ be a Cauchy sequence in $Y^X$.
    Because $\ol\rho(f_n, f_m)\rightarrow 0$ as $n, m\rightarrow\infty$, for each $x\in X$, $(f_n(x))_{n\in\bb{N}}$ is a Cauchy sequence in $Y$.
    Hence, $(f_n)_{n\in\bb{N}}$ is pointwise convergent.
    To prove that $(f_n)_{n\in\bb{N}}$ is convergent in $Y^X$, first observe that $d(f_n(t), f(t))\leq d(f_n(t), f_p(t))+d(f_p(t), f(t))$ for any point $t$ of $X$ and any positive integer $p$.
    Let $N$ be an integer such that $n, m\geq N$ implies $\ol\rho(f_n, f_m)<\epsilon/2$; given a point $x$ of $X$, let $M$ be an integer such that $d(f_p(x), f(x))<\epsilon/2$ whenever $p\geq M$.
    Then, whenever $p\geq M, N$, we have $d(f_n(x), f(x))<\epsilon/2+\epsilon/2=\epsilon$.
    Therefore, $(f_n)_{n\in\bb{N}}$ converges to $f$ in $Y^X$.
\end{proof}

\begin{thm}[Complete spaces of functions]\label{Complete spaces of functions}
    Let $X$ be a topological space, $(Y, d)$ be a complete metric space, and assume that $Y^X$ equips the uniform metric $\ol\rho$ corresponding to $d$.
    \begin{enumerate}
        \item[(a)]
        {
            $C(X, Y)$ and $B(X, Y)$ are closed subspaces of $Y^X$.
        }
        \item[(b)]
        {
            Hence, if $Y$ is complete, then $C(X, Y)$ and $B(X, Y)$ are complete.
        }
    \end{enumerate}
\end{thm}
\begin{proof}
    Because (b) is immediate from \cref{uniform topology and completeness} and (a), it suffices to prove (a).
    Assume first that $f\in Y^X$ is a limit point of $C(X, Y)$ in $Y^X$.
    Then, there is a sequence $(f_n)_{n\in\bb{N}}$ in $C(X, Y)$ satisfying $\ol\rho(f_n, f)\rightarrow 0$ as $n\rightarrow\infty$.
    Being a uniform limit of a sequence of continuous maps, $f$ is also continuous.
    \color{brown}The proof for $B(X, Y)$ is also easy, which is left as an exercise. \color{black}
\end{proof}

For $B(X, Y)$, one may impose the metric $\rho: B(X, Y)\times B(X, Y)\rightarrow[0, \infty)$ defined by
\begin{align*}
    \rho(f, g)=\sup\left\{d(f(x), g(x)): x\in X\right\}.
\end{align*}
This metric is called the supremum (or 'sup') metric corresponding to $d$.
In particular, when $X$ is compact, since $C(X, Y)\subset B(X, Y)$, one may make use of the supremum metric, rather than the uniform metric.
In fact, when $X$ is compact, the supremum metric on $C(X, Y)$ is the standard bounded metric corresponding to the uniform metric on $C(X, Y)$.
In other words, $\ol\rho(f, g)=\min\{1, \rho(f, g)\}$ for all $f, g\in C(X, Y)$.
\color{brown}Its justification is left as an exercise.\color{black}

\subsection{Isometric embedding of a metric space}

Recall that a map $h$ from a metric space $(X, d_X)$ into $(Y, d_Y)$ satisfying $d_Y(h(a), h(b))=d_X(a, b)$ for all $a, b\in X$ is called an isometric embedding of $X$ into $Y$.
If $h: X\rightarrow Y$ is an isometric embedding and $Y$ is complete, the closure of $h(X)$ in $Y$ is also complete.

When trying to define objects, one should be interested in their existence, uniqueness, and exactness.

\begin{thm}[Existence of a completion of a metric space]\label{existence of a completion}
    Let $(X, d)$ is a metric space, and consider the map $\phi: X\rightarrow B(X, \bb{R})$ defined by
    \begin{align*}
        \phi(x)=l_x\textsf{ for }x\in X,
    \end{align*}
    where $l_x: X\rightarrow\bb{R}$ maps $t\in X$ to $d(t, x)-d(t, a)$.
    (Here, $a$ is a given point of $X$, so $|l_x(t)|\leq d(x, a)$ for all $t\in X$)
    Then $\phi$ is an isometric embedding of $X$ into $B(X, \bb{R})$, where $B(X, \bb{R})$ equips the sup metric.
    Because $\bb{R}$ is complete, the closure of $\phi(X)$ in $B(X, \bb{R})$ is also complete.
    Therefore, every metric space has a completion.
\end{thm}
\begin{thm}[Uniqueness of a completion of a metric space]\label{uniqueness of a completion}
    Let $h_1: X\hookrightarrow Y_1$ and $h_2: X\hookrightarrow Y_2$ be isometric embeddings of $X$ into complete metric spaces $(Y_1, d_1)$ and $(Y_2, d_2)$.
    Denote the closure of $h_i(X)$ in $Y_i$ by $\ol{h_i(X)}$ ($i=1, 2$).
    Then, there is an isometric homeomorphism $H: \ol{h_1(X)}\rightarrow \ol{h_2(X)}$ which equals $h_2\circ h_1^{-1}$ on $h_1(X)$.
    Hence, the completion of $X$ is unique up to an isometric homeomorphism.
    % A line is intentionally broken.

    \begin{equation*}
    \begin{tikzcd}[row sep=large, column sep=huge]
        h_1(X)\arrow[d, hook]
        \arrow[rr, bend left, "h_2\circ h_1^{-1}", "\textsf{isometric}"']
        &
        X
        \arrow[l, "\textsf{isometric}", "h_1"']
        \arrow[r, "h_2", "\textsf{isometric}"']
        &
        h_2(X)\arrow[d, hook]
        \\
        \ol{h_1(X)}
        \arrow[d, hook]
        \arrow[rr, "H", "\textsf{isometric}"']
        &&
        \ol{h_2(X)}
        \arrow[d, hook]
        \\
        Y_1 && Y_2
    \end{tikzcd}
    \end{equation*}
\end{thm}
\begin{proof}[Proof of \cref{existence of a completion}]
    Since it is checked that $l_x\in B(X, \bb{R})$, the map $\phi$ is well defined.
    Thus, it remains to check that $\phi$ is an isometric embedding of $X$ into $B(X, \bb{R})$.
    For this, it suffices to show that $\ol\rho(l_x, l_y)=d(x, y)$ for all $x, y\in X$, where $\rho$ is the uniform metric on $B(X, \bb{R})$.
    Observing that $l_x(t)-l_y(t)=d(t, x)-d(t, y)$, we can find that $\rho(l_x, l_y)\leq d(x, y)$ and $l_x(x)-l_y(x)=-d(x, y)$, proving that $\rho(l_x, l_y)=d(x, y)$.
\end{proof}
\begin{proof}[Proof of \cref{uniqueness of a completion}]
    Remark that $\ol{h_i(X)}$ is a complete metric space for $i=1, 2$.
    Define a map $H: \ol{h_1(X)}\rightarrow\ol{h_2(X)}$ as follows:
    \begin{enumerate}
        \item[(1)]
        {
            When $x\in h_1(X)$, let $H(x):=h_2(h_1^{-1}(x))$.
            Since $h_1$ is a bijection between $X$ and $h_1(X)$, the above setting is not ambiguous.
        }
        \item[(2)]
        {
            When $x\in \ol{h_1(X)}\setminus h_1(X)$, let $(x_n)_{n\in\bb{N}}$ be a sequence in $h_1(X)$ which converges to $x$, and let $H(x):=\lim H(x_n)$.

            This setting seems to be ambiguous, because the resulting $H(x)$ might differ by the choice of $(x_n)_{n\in\bb{N}}$ (also, even the existence of the limit could be inquired).
            \begin{itemize}
                \item
                {
                    Because $d_2(H(x_n), H(x_m))=d_1(x_n, x_m)$ for all $n, m\in\bb{N}$, $(H(x_n))_{n\in\bb{N}}$ is a Cauchy sequence in $\ol{h_2(X)}$.
                    Since $\ol{h_2(X)}$ is complete, $(H(x_n))_{n\in\bb{N}}$ is convergent, with the limit in $\ol{h_2(X)}$.
                }
                \item
                {
                    Suppose $(a_j)_{j\in\bb{N}}$ is another sequence in $h_1(X)$ converging to $x$.
                    Letting $X=\lim H(x_n)$ and $A=\lim H(a_j)$, we find that
                    \begin{eqnarray*}
                        d_2(X, A)&\leq&d_2(X, H(x_n))+d_2(H(x_n), H(a_j))+d_2(H(a_j), A)\\
                        &\leq&d_2(X, H(x_n))+d_1(x_n, a_j)+d_2(H(a_j), A)\xrightarrow{n, j\rightarrow\infty}0,
                    \end{eqnarray*}
                    so $X=A$ and the above setting of $H(x)$ is not ambiguous.
                }
            \end{itemize}
        }
    \end{enumerate}
    We just established a map $H$ from $\ol{h_1(X)}$ into $\ol{h_2(X)}$.
    It remains to show that $H$ is an isometry.
    For this, it suffices to show that $H$ is an isometric homeomorphism, and $H$ coincides $h_2\circ h_1^{-1}$ on $h_1(X)$.\footnote{Since an isometric embedding is an open map, it suffices to show that $H$ is an isometric bijection.}
    \begin{enumerate}
        \item[(a)]
        {
            Given $a, b\in\ol{h_1(X)}$, let $(a_n)_{n\in\bb{N}}$ and $(b_n)_{n\in\bb{N}}$ be seqences in $h_1(X)$ converging to $a$ and $b$, respectively.
            Then $d_2(H(a), H(b))=\lim d_2(H(a_n), H(b_n))=\lim d_1(a_n, b_n)=d_1(a, b)$, so $H$ is isometric and injective.
            To check the surjectivity, let $y$ be a point of $\ol{h_2(X)}$ and let $(y_n)_{n\in\bb{N}}$ be a sequence in $h_2(X)$ converging to $y$.
            Letting $x_n=h_1(h_2^{-1}(y_n))$ for each $n\in\bb{N}$, the limit of $(x_n)_{n\in\bb{N}}$ exists (say $x\in\ol{h_1(X)}$ is the limit), and we have $H(x)=\lim H(x_n)=\lim y_n=y$.
        }
        \item[(b)]
        {
            Clearly, $H$ extends $h_2\circ h_1^{-1}$.
        }
    \end{enumerate}
    Therefore, there is an isometric homeomorphism between two completions of $X$.
    In other words, there is a unique completion of a metric space up to isometric homeomorphism.
\end{proof}

To emphasize the result of \cref{existence of a completion,uniqueness of a completion}, we leave the definition of the completion of a metric space.
\begin{defi}[The completion of a metric space]
    Let $(X, d)$ be a metric space, and let $h: (X, d)\rightarrow (Y, e)$ be an isometric embedding, where $(Y, e)$ is a complete metric space.
    The closure of the image of $h$ in $Y$ is called the completion of $X$.
    The completion of a metric space always exists, and it is unique up to isometric homeomorphism.
\end{defi}


\subsection{Problems}

\begin{prob}\label{completeness_basic_sol}
    Prove \cref{completeness_basic}.
\end{prob}
\begin{sol}
    \begin{enumerate}
        \item[(a)]
        {
            Easy.
        }
        \item[(b)]
        {
            If $Y$ is a nonempty closed supset of $X$ and $(y_n)_{n\in\bb{N}}$ is a Cauchy sequence in $Y$, then it converges, due to the completeness of $X$; and the limit belongs to $Y$, due to the closedness of $Y$ in $X$, which implies the completeness of $Y$.
            Conversely, if $(Y, e)$ is complete and $y$ is a limit point of $Y$ in $X$, then for each $n\in\bb{N}$, there is a point $y_n\in B_d(y, 1/n)$; since the sequence $(y_n)_{n\in\bb{N}}$ converges to $y$ and $Y$ is complete, $y\in Y$, so $Y$ is closed in $X$.
        }
        \item[(c)]
        {
            Remark that $D$ and $\rho$ induces the product topology on $Z$.
            If $(z_n)_{n\in\bb{N}}$ is a Cauchy sequence in $Z$, it easily follows from the definition of $D$ or $\rho$ that $(\pi_k(z_n))_{n\in\bb{N}}$ is a Cauchy sequence in $X_k$ for each $k\in\bb{N}$.
            Since each $X_k$ is a complete metric space, each sequence $(\pi_k(z_n))_{n\in\bb{N}}$ in $X_k$ is convergent, hence the sequence $(z_n)_{n\in\bb{N}}$ is convergent in $Z$.
        }
    \end{enumerate}
\end{sol}