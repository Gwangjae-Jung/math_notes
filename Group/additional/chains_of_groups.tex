\section{Chains of groups}

There are two widely used chains for groups.
One is generally called the composition series, and the other will be called the solvability chain (or series) in this note.

We first introduce the composition series, after introducing its motivation.
\begin{defi}[Maximal (normal) subgroup]
    A subgroup $M$ of $G$ is called a maximal (normal) subgroup of $G$ if
    \begin{enumerate}
        \item[(a)]
        {
            $H<G$ ($H\triangleleft G$) and
        }
        \item[(b)]
        {
            whenever $H\leq K\leq G$ ($H\leq K\nmal G$), either $K=H$ or $K=G$.
        }
    \end{enumerate}
\end{defi}
\begin{defi}[Simple group]
    A group is called a simple group if there are only two normal subgroups: the trivial subgroup and the group itself.
    In other words, a group is simple if and only if the trivial subgroup is the maximal normal subgroup.
\end{defi}

From definition and our intuition, we first impose a straightforward result and its outstanding corollary, which states that every finite group has a \textit{simple normal chain}, which is a composition series.
\begin{prop}
    Let $G$ be a nontrivial group.
    Then the followings are equivalent:
    \begin{enumerate}
        \item[(a)]
        {
            $N$ is a maximal normal subgroup of $G$.
        }
        \item[(b)]
        {
            $N\triangleleft G$ and $G/N$ is simple.
        }
    \end{enumerate}
\end{prop}
\begin{proof}
    Use the lattice isomorphism theorem.
\end{proof}
\begin{defi}[Composition series]
    Let $G$ be a group, and suppose there are finitely many subgroups $G_1, \cdots, G_n$ such that
    \begin{enumerate}
        \item[(a)]
        {
            $\{1_G\}=G_n\triangleleft G_{n-1}\triangleleft \cdots \triangleleft G_1 \triangleleft G_0=G$ and
        }
        \item[(b)]
        {
            each $G_i/G_{i+1}$ ($i=0, 1, \cdots, n-1$) is a simple group.
        }
    \end{enumerate}
    To be short, a composition series is a simple normal chain.
\end{defi}
\begin{cor}
    Every finite group has a composition series.
\end{cor}
\begin{proof}
    Since $G$ is finite, $G$ has a maximal normal subgroup $G_1$, and $G_1$ also has a maximal normal subgroup.
    By induction, we can establish a finite normal chain.
    Simplicity follows from normality.
\end{proof}

Now we study solvability chains of groups.
\begin{defi}[Solvability chain]
    Let $G$ be a group, and suppose there are finitely many subgroups $G_1, \cdots, G_n$ such that
    \begin{enumerate}
        \item[(a)]
        {
            $\{1_G\}=G_n\triangleleft G_{n-1}\triangleleft \cdots \triangleleft G_1 \triangleleft G_0=G$ and
        }
        \item[(b)]
        {
            each $G_i/G_{i+1}$ ($i=0, 1, \cdots, n-1$) is an abelian group.
        }
    \end{enumerate}
    To be short, a solvability chain is an abelian normal chain.
    In addition, a group which has a solvability chain is called a solvable group.
\end{defi}
As every finite group has a composition series, every abelian group (even if it is infinite) has a solvability chain.

Solvable groups behave well under subgroups and homomorphic images.
\begin{prop}
    Suppose $G$ is a solvable group.
    Then subgroups of $G$ are solvable, and homomorphic images of $G$ are also solvable.\footnote{Because the proposition holds when the words `solvable' are replaced by `abelian' or `cyclic,' solvable groups can be seen as a generalization of abelian groups.}
\end{prop}
\begin{proof}
    To prove the first part of the proposition, let $H$ be a subgroup of $G$ and let $\{1_G\}=G_n\triangleleft G_{n-1}\triangleleft \cdots \triangleleft G_1 \triangleleft G_0=G$ be a solvability chain of $G$.
    A good idea for the proof is to consider restrictions as follows:
    \begin{align*}
        \{1_G\}=G_n\cap H\triangleleft G_{n-1}\cap H\triangleleft \cdots \triangleleft G_1\cap H \triangleleft G_0\cap H=H.
    \end{align*}
    For each $i=0, 1, \cdots, n$, define $H_i=G_i\cap H$.
    It remains to show that $H_i/H_{i+1}$ is abelian for $i=0, 1, \cdots, n-1$, and for this, we decide to prove that $H_i/H_{i+1}$ embeds into $G_i/G_{i+1}$.
    Because the natural embedding $\iota_i: H_i\hookrightarrow G_i$ satisfies $\iota_i^{-1}(G_{i+1})=H_{i+1}$, $\iota_i$ induces a group monomorphism $\ol{\iota_i}: H_i/H_{i+1}\hookrightarrow G_i/G_{i+1}$.

    Before proving the second part of the proposition, note that a homomorphic image of $G$ is isomorphisc to $G/N$ for some normal subgroup $N$ of $G$.
    Consider the canonical projection map $\pi: G\rightarrow G/N$ and the following chain:
    \begin{align*}
        \{\ol{1_G}\}=\pi(G_n)\triangleleft \pi(G_{n-1})\triangleleft \cdots \triangleleft \pi(G_1) \triangleleft \pi(G_0)=\pi(G).
    \end{align*}
    Letting $\pi_i: G_i\rightarrow G_i/N$ for each $i=0, 1, \cdots, n$, because $G_{i+1}\leq\pi_i^{-1}(G_{i+1}/N)$, $\pi_i$ induces the following group homomorphism:
    \begin{align*}
        \ol{\pi_i}: \frac{G_i}{G_{i+1}}\rightarrow\frac{G_i/N}{G_{i+1}/N}.
    \end{align*}
    Because $\ol{\pi_i}$ is surjective, it is derived that $\ol{G_i}/\ol{G_{i+1}}$ is abelian for each $i=0, 1, \cdots, n-1$, as desired.
\end{proof}
To review, the preceeding proposition states that solvable groups behave well under subgroups and homomorphic images (or equivalently, quotients by normal subgroups).
The following proposition can be considered a converse of the preceeding proposition.
\begin{prop}
    Let $G$ be a group and $N$ be a normal subgroup of $G$.
    If both $N$ and $G/N$ are solvable, then $G$ is also solvable.
\end{prop}
\begin{proof}
    Use the lattice isomorphism theorem and concatenate solvability chains.
\end{proof}
Another well-behaviour of solvable groups is given as the following proposition:
\begin{prop}
    If $G_1$ and $G_2$ are solvable groups, then so is $G_1\times G_2$.
\end{prop}
\begin{proof}
    Find solvability chains of $G_1$ and $G_2$, and enlarge subgroups of $G_1\times G_2$ term by term and step by step. \color{brown}Detailed proof is left as an exercise, because the proof is easy.\color{black}
\end{proof}

We now introduce some results derived when composition series and solvability chain meet each other.
\begin{prop}
    \begin{enumerate}
        \item[(a)]
        {
            A finite abelian simple group is a cyclic group of a prime order.
        }
        \item[(b)]
        {
            A finite solvable simple group is a cyclic group of a prime order.
        }
        \item[(c)]
        {
            Thus, a finite nonabelian simple group is never solvable.
        }
    \end{enumerate}
\end{prop}
\begin{proof}
    (a) easily follows, because every subgroup of an abelian group is a normal subgroup.
    To prove (b), note that a solvable simple group is necessarily abelian, and (a) forces such a group to be a cyclic group of a prime order.
    (c) follows easily when one assumes that there is a finite nonabelian simple group which is solvable; by (b), such a group is cyclic, which contradicts (b).
\end{proof}

We end this section with an alternative definition of `finite' solvable groups.
\begin{prop}
    If $G$ is a finite group, then $G$ is solvable if and only if there are finitely many subgroups $G_1, \cdots, G_n$ of $G$ such that
    \begin{enumerate}
        \item[(a)]
        {
            $\{1_g\}=G_n\triangleleft G_{n-1}\triangleleft \cdots\triangleleft G_1\triangleleft G_0=G$ and
        }
        \item[(b)]
        {
            $G_i/G_{i+1}$ is a cyclic group of a prime order for each $i=0, 1, \cdots, n-1$.
        }
    \end{enumerate}
\end{prop}
\begin{proof}
    If part is clear, so it remains to prove only if part.
    Since $G$ is finite, a composition series $\{1_g\}=G_n\triangleleft G_{n-1}\triangleleft \cdots\triangleleft G_1\triangleleft G_0=G$ of $G$ exists.
    Because each $G_i/G_{i+1}$ is finite, simple, and solvable, each quotient needs to be a cyclic group of a prime order, as desired.
\end{proof}