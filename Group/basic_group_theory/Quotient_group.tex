\section{Quotient group}

Imposing the quotient by a subgroup $H$ of $G$ gives a coset space, and the coset space is indeed a partition of $G$.
\begin{thm}[Lagrange's theorem]
    Suppose $G$ is a finite group and let $H$ be a subgroup of $G$.
    Then $|H|$ divides $|G|$ and $|G|=|G/H||H|$.
\end{thm}
\begin{proof}
    Imposing a quotient by a subgroup gives a partition of the group.
\end{proof}
\begin{rmk}
    Clearly, partitions of a partition give a partition, from which some other results can be derived.
    For example, if $H\leq K\leq G$ and $G/K$ and $K/H$ are finite, then $[G: H]=[G:K][K:H]$.
\end{rmk}

Motivation of normal subgroups is introduced in the first part of this note.
Thus, in the remaining of this section, further properties of normal subgroups are introduced.

We first start with kind of trivial statements, regarding the structures of quotient groups.
Proving the following propositions is left as an exercise.
\begin{prop}
    Let $G$ be an abelian group.
    Then every subgroup of $G$ is naturally a normal subgroup and every quotient group is abelian.
    Furthermore, if $G$ is cyclic, then every quotient group is cyclic.
\end{prop}
\begin{prop}[Restriction of normality]
    Suppose $H\leq K\leq G$.
    If $H$ is a normal subgroup of $G$, then $H$ is normal in $K$.
\end{prop}
\begin{prop}[Normality of restriction]
    Let $K$ be a subgroup of a group $G$.
    If $N$ is a normal subgroup of $G$, then the restriction of $N$ to $K$, i.e., $N\cap K$, is normal in $K$.
\end{prop}
\begin{prop}
    Let $G$ be a group.
    \begin{enumerate}
        \item[(a)]
        {
            The center $Z(G)$ of $G$ is a normal subgroup of $G$.
        }
        \item[(b)]
        {
            If $\phi\in\aut(G)$, then $\phi(Z(G))=Z(G)$.
        }
    \end{enumerate}
\end{prop}

The following lemma is not about a normal subgroup, but it helps in some other situations.
\begin{lem}
    If $H$ and $K$ are finite subgroups of $G$, then
    \begin{equation*}
        |HK|=\frac{|H||K|}{|H\cap K|}.
    \end{equation*}
\end{lem}
\begin{proof}
    Note that $HK=\bigcup_{h\in H}{hK}$ so $|HK|$ is divisible by $|K|$.
    Given $a\in H$ and $b\in K$, assume $hk=ab$ for some $h\in H, k\in K$.
    Then $a^{-1}h=bk^{-1}\in H\cap K$ and vice versa, and there are precisely $|H\cap K|$ choices for $(h, k)$.
\end{proof}

We now introduce a criterion for determining if the composite of two subgroups is a subgroup.
\begin{prop}
    Let $H$ and $K$ be subgroups of $G$.
    Then $HK$ is a subgroup of $G$ if and only if $H$ and $K$ commute, i.e., $HK=KH$.
    Hence, in particular, if $H\leq N_G(K)$, i.e., one normalizes the other, then $HK$ is a subgroup of $G$.
\end{prop}
\begin{proof}
    Assume $HK$ is a subgroup of $G$.
    We need to show that $KH\subset HK$ and $HK\subset KH$.
    Because $HK$ is a subgroup of $G$, for any $hk\in HK$ with $h\in H$ and $k\in K$, we have $k^{-1}h^{-1}=(hk)^{-1}\in HK$, from which it is derived that $KH\subset HK$.
    We could also derive from the same equation that $HK\subset KH$, since $k^{-1}h^{-1}\in KH$.

    Now assume conversely that $HK=KH$.
    Given $ab, hk\in HK$ with $a, h\in H$ and $b, k\in K$, we have $(ab)(hk)^{-1}=a(bk^{-1}h^{-1})$.
    Because $bk^{-1}h^{-1}\in KH=HK$, $bk^{-1}h^{-1}=uv$ for some $u\in H$ and $v\in K$, and $(ab)(hk)^{-1}=(au)v\in HK$, as desired.
\end{proof}