\section{Basic theory of group action}

\begin{defi}[Group action]
    Let $G$ be a group and $A$ be a nonempty set.
    A function $\sigma: G\times A\rightarrow A$ is called a group action of $G$ on $A$, if $\sigma$ satisfies the following axioms: simply writing $\sigma(g, a)=g\cdot a$ for all $g\in G$ and $a\in A$,
    \begin{enumerate}
        \item[(a)]
        {
            $1_G\cdot a=a$ for all $a\in A$.
        }
        \item[(b)]
        {
            $x\cdot(y\cdot a)=xy\cdot a$ for all $x, y\in G$ and $a\in A$.
        }
    \end{enumerate}
\end{defi}

\begin{defi}
    Suppose that a group $G$ acts on a nonempty set $A$.
    \begin{enumerate}
        \item[(a)]
        {
            (Stabilizer)
            Given $a\in A$, the \color{brown}subgroup $G_a:=\{g\in G: g\cdot a=a\}$ of $G$ \color{black}is called the stabilizer of $a\in A$.
        }
        \item[(b)]
        {
            (Orbit)
            Given $a\in A$, the subset $G\cdot a=\{g\cdot a: g\in G\}$ is called the orbit of $a$.
        }
        \item[(c)]
        {

        }
        \item[(a)]
        {

        }
    \end{enumerate}
\end{defi}

\begin{obs}
    Suppose that a group $G$ acts on a nonempty set $A$.
    Then the collection of all orbits in $A$ form a partition of $A$.
    Hence, we have the following equation, when $A$ is a finite set:
    \begin{align*}
        |A|=|A^G|+\sum_{i=1}^r|G\cdot a_i|,
    \end{align*}
    where $A^G$ is the collection of elements of $A$ whose orbit has exactly one element and $a_1, \cdots, a_r$ are representatives of pairwise disjoint orbits with more than one elements.
\end{obs}