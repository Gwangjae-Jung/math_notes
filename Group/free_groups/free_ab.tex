\section{Remarks on free abelian groups}

\begin{defi}[Free abelian group]
    A free $\bb{Z}$-module is called a free abelian group.
\end{defi}

Note that the above definition is quite natural in the sense that an abelian group can be considered a $\bb{Z}$-module and vice versa.
Other than the above somewhat formal definition, there is an intuitive method of considering free abelian groups.
As noted in the chapter of free objects, any given structured object $X$ can be considered the free object $\mc{F}(S)$ with some relations imposed.
The free abelian group $\mc{F}(S)$ (here, $S$ is a nonempty set) can be considered free groups ``in which every pair of elements commute,'' and the normal subgroup playing such a role is exactly $[\mc{F}(S), \mc{F}(S)]$.
Therefore, one might suggest that $\mc{F}(S)/[\mc{F}(S), \mc{F}(S)]$ is the free abelian group generated by $S$.
\begin{thm}
    Let $(\mc{F}_\textsf{gp}(S), \imath)$ be the free group generated by a nonempty set $S$, and let $(\mc{F}_\textsf{ab}(S), \jmath)$ be the free abelian group generated by $S$.
    Then $\mc{F}_\textsf{gp}(S)/[\mc{F}_\textsf{gp}(S), \mc{F}_\textsf{gp}(S)]\approx \mc{F}_\textsf{ab}(S)$.
\end{thm}
\begin{proof}
    For convinience, let $\mc{F}=\mc{F}_\textsf{gp}$, $\mc{A}=\mc{F}_\textsf{ab}$, and $\ol{\mc{F}}=\mc{F}/[\mc{F}, \mc{F}]$.
    Consider the following commutative diagrams:
    \begin{equation*}
    \begin{tikzcd}[row sep=1.5cm, column sep=1.5cm]
        S
        \arrow[r, "\imath"', hook]
        \arrow[rr, "\phi:=\pi\circ\imath", bend left]
        \arrow[dr, "\jmath"', hook]
        &
        \mc{F}
        \arrow[r, "\pi"', ->>]
        &
        \ol{\mc{F}}
        \\
        &
        \mc{A}
        \arrow[ur, "\widetilde\phi"']
        &
    \end{tikzcd}
    \quad\quad
    \begin{tikzcd}[row sep=1.5cm, column sep=1.5cm]
        S
        \arrow[r, "\imath", hook]
        \arrow[dr, "\jmath"', hook]
        &
        \mc{F}
        \arrow[r, "\pi", ->>]
        \arrow[d, "\widetilde\jmath"]
        &
        \ol{\mc{F}}
        \arrow[dl, "\ol{\widetilde\jmath}"]
        \\
        &
        \mc{A}
        &
    \end{tikzcd},
    \end{equation*}
    where $\ol{\jmath_*}$ is a well-defined group homomorphism, since $[\mc{F}, \mc{F}]\leq\jmath_*^{-1}(\{0\})$.
    The above two commutative diagrams yields the following commutative diagrams.
    \begin{equation*}
    \begin{tikzcd}[row sep=0.7cm, column sep=0.4cm]
        S
        \arrow[rrrr, "\jmath", hook]
        \arrow[drrr, "\phi"]
        \arrow[ddrr, "\jmath"', hook]
        &&&&
        \mc{A}
        \arrow[dl, "\widetilde\phi"']
        \arrow[ddll, "\id{\mc{A}}", bend left]
        \\
        &&&
        \ol{\mc{F}}
        \arrow[dl, "\ol{\widetilde\jmath}"']
        &
        \\
        &&
        \mc{A}
        &&
    \end{tikzcd}
    \quad\quad
    \begin{tikzcd}[row sep=0.7cm, column sep=0.4cm]
        S
        \arrow[rrrr, "\phi"]
        \arrow[drrr, "\jmath", hook]
        \arrow[ddrr, "\phi"']
        &&&&
        \ol{\mc{F}}
        \arrow[dl, "\ol{\widetilde\jmath}"']
        \\
        &&&
        \mc{A}
        \arrow[dl, "\widetilde\phi"']
        &
        \\
        &&
        \ol{\mc{F}}
        &&
    \end{tikzcd}
    \end{equation*}
    From the left diagram, we can find that $\ol{\widetilde\jmath}\circ\widetilde\phi=\id{\mc{A}}$, by using a universal property of $\mc{A}$.
    From the right diagram, one can notice that $\phi=(\widetilde\phi\circ\widetilde\jmath)\circ\phi$, from which it can be deduced that $\ol s=(\widetilde\phi\circ\widetilde\jmath)(\ol s)$ whenever $s\in S$.
    Because $\ol{\mc{F}}$ is generated by $\ol s$ for $s\in S$ and $\widetilde\phi\circ\widetilde\jmath$ is a group homomorphism extending $\phi$, we have $\widetilde\phi\circ\widetilde\jmath=\id{\mc{\ol{F}}}$.
\end{proof}