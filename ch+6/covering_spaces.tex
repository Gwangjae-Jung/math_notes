\section{Covering spaces}

\begin{defi}[Covering space]
    Let $p: E\rightarrow B$ be a surjective continuous map.
    An open subset $U$ of $B$ is said to be evenly covered by $p$ if
    \begin{enumerate}
        \item[(\romannumeral 1)]
        {
            there is a partition $\{V_\alpha\}_{\alpha\in\mc{A}}$ of $p^{-1}(U)$ into open subsets of $E$, and
        }
        \item[(\romannumeral 2)]
        {
            the restriction of $p$ to $V_\alpha$ denotes a homeomorphism from $V_\alpha$ onto $U$ for each $\alpha\in\mc{A}$.
        }
    \end{enumerate}
    A continuous surjection $p: E\rightarrow B$ is called a covering map and $E$ is called a covering space of $B$ if every point of $B$ admits a neighborhood of $B$ which is evenly covered by $p$.
\end{defi}

In the following proposition, some basic properties of covering maps are introduced.
\begin{prop}
    Let $p: E\rightarrow B$ be a covering map.
    \begin{enumerate}
        \item[(a)]
        {
            An open subset of an open subset of $B$ which is evenly covered by $p$ is also evenly covered by $p$.
        }
        \item[(b)]
        {
            For each point $b$ of $B$, $p^{-1}(\{b\})$ has the discrete topology.
        }
        \item[(c)]
        {
            $p$ is an open map.
        }
    \end{enumerate}
\end{prop}

\begin{exmp}
    \begin{enumerate}
        \item[(a)]
        {
            (Projection)
            Let $\pi_X: X\times Y\rightarrow X$ be the canonical projection map.
            If $Y$ has the discrete topology, then $\pi_X$ is a covering map.
        }
        \item[(b)]
        {
            Let $p: E\rightarrow B$ be a continuous surjection and $U$ be a nonempty subset of $B$ which is evenly covered by $p$.
            If $U$ is connected, then the partition of $p^{-1}(U)$ into slices is unique, because each slice is a connected component of $p^{-1}(U)$.
        }
        \item[(c)]
        {
            Let $p: E\rightarrow B$ be a covering map.
            If $B_0$ is a subspace of $B$ and $E_0=p^{-1}(B_0)$, then $p_0:=p|_{E_0}: E_0\rightarrow B_0$ is a covering map.
            \begin{equation*}
            \begin{tikzcd}[row sep=1.0cm, column sep=1.8cm]
                E\arrow[r, "p", "\textsf{covering}"']
                \arrow[d, dash]
                &
                B
                \arrow[d, dash]
                \\
                E_0\arrow[r, "p_0"', "\textsf{covering}"]
                &
                B_0
            \end{tikzcd}
            \end{equation*}
        }
        \item[(d)]
        {
            Let $p_k: E_k\rightarrow B_k$ be a covering map for $k=1, 2$.
            Then $p_1\times p_2: E_1\times E_2\rightarrow B_1\times B_2$ is a covering map.
        }
    \end{enumerate}
\end{exmp}

\color{red}
\begin{prob}
    Let $q: X\times Y$ and $r: Y\times Z$ be covering maps, and let $p=r\circ q$.
    Show that $p$ is a covering map, provided that every fiber of $r$ is finite.
\end{prob}
\begin{sol}
    
\end{sol}

\begin{prob}
    Let $p: E\rightarrow B$ be a covering map.
    \begin{enumerate}
        \item[(a)]
        {
            Show that $E$ is a Hausdorff, regular, completely regular, or locally compact Hausdorff space, provided that $B$ is so.
        }
        \item[(b)]
        {
            Show that $E$ is compact if $B$ is compact and every fiber of $p$ is finite.
        }
    \end{enumerate}
\end{prob}
\begin{sol}
    
\end{sol}
\color{black}