\section{Measurable functions}

\begin{defi}[Measurable function]
    Suppose $(X, \mc{M})$ and $(Y, \mc{N})$ are measurable spaces.
    A map $f: X\rightarrow Y$ is said to be $(\mc{M}, \mc{N})$-measurable if $f^{-1}(E)\in\mc{M}$ whenever $E\in\mc{N}$. (When the $\sigma$-algebras on $X$ and $Y$ are understood, $f$ is just called a measurable function.)
\end{defi}

It is easy to check that the composition of two measurable functions is also measurable, as it held for continuous functions.
Remarking that we could reduce to basis members for checking the continuity of a given map, one might wish to establish the counterpart in the theory of measurable functions.
\begin{lem}[Reduction to generators]
    Suppose as in the above definition, and assume $\mc{}$ is generated by $\mc{F}$.
    Then a map $f: X\rightarrow Y$ is measurable if and only if $f^{-1}(E)\in\mc{M}$ for all $E\in\mc{F}$.
\end{lem}
\begin{proof}
    We only need to show the if part; we want $f^{-1}(N)\in\mc{M}$ for all $N\in\mc{N}$.
    Consider the collection
    \begin{align*}
        \{E\in\mc{N}: f^{-1}(E)\in\mc{M}\}.
    \end{align*}
    Because $\mc{M}$ is a $\sigma$-algebra (on $Y$) and the above collection contains $\mc{F}$, it contains $\mc{N}$.
\end{proof}
\begin{cor}
    Suppose $X$ and $Y$ are topological spaces and $f: X\rightarrow Y$ is a continuous map.
    Then $f$ is a $(\borel{X}, \borel{Y})$-measurable function.
\end{cor}
\begin{proof}
    The topology on $Y$ generates the Borel $\sigma$-algebra on $Y$.
\end{proof}

\begin{defi}
    Let $(X, \mc{M})$ and $(Y, \mc{N})$ be a measurable space.
    \begin{enumerate}
        \item[(a)]
        {
            A real or complex-valued map $f$ defined on $X$ is called a $\mc{M}$-measurable map if it is $(\mc{M}, \borel{\bb{R}})$ or $(\mc{M}, \borel{\bb{C}})$-measurable.
            In particular, when $X=\bb{R}$, $f$ is said to be Lebesgue (or Borel) measurable if $f$ is $\mc{L}$ (or $\borel{\bb{R}}$)-measurable.\footnote{Of course, one can extend this definition for a metric (or metrizable) space as the codomain. Nevertheless, in this note, a measurable function will denote a real or complex-valued function only.}
        }
        \item[(b)]
        {
            Suppose $E\in\mc{M}$.
            We say a map $f: X\rightarrow Y$ is measurable on $E$ if $f|_E$ is measurable, where $E$ equips the restriction of $\mc{M}$ to $E$ as the $\sigma$-algebra.
        }
    \end{enumerate}
\end{defi}
\begin{rmk}
    \begin{enumerate}
        \item[(a)]
        {
            It must be noted that Borel measurability is preserved under compositions while Lebesgue meaurability is not.
        }
        \item[(b)]
        {
            Let $f$ be a function from $X$ into $\ol{\bb{R}}$ and let $A=f^{-1}(\bb{R})$.
            Then $f$ is measurable if and only if $f$ is measurable on $\bb{R}$ and both $f^{-1}(\{\infty\})$ and $f^{-1}(\{-\infty\})$ are measurable sets in $X$.
        }
    \end{enumerate}
\end{rmk}

\begin{prop}
    Let $(X, \mc{M})$ and $(Y_\alpha, \mc{N}_\alpha)\,(\alpha\in A)$ be measurable spaces, and let
    \begin{align*}
        Y=\prod_{\alpha\in A}Y_\alpha,\quad\mc{N}=\bigotimes_{\alpha\in A}\mc{N}_\alpha.
    \end{align*}
    Let $f_\alpha: X\rightarrow Y_\alpha$ be a map for each $\alpha\in A$ and $f: X\rightarrow Y$ be the map such that $\pi_\alpha\circ f=f_\alpha$ for all $\alpha$.
    Then $f$ is $(\mc{M}, \mc{N})$-measurable if and only if $f_\alpha$ is $(\mc{M}, \mc{N}_\alpha)$-measurable for each $\alpha\in A$.
\end{prop}
\begin{proof}
    Because $\pi_\alpha$ is $(\mc{N}, \mc{N}_\alpha)$-measurable, $f_\alpha$ is $(\mc{M}, \mc{N}_\alpha)$-measurable for each $\alpha\in A$, provided that $f$ is $(\mc{M}, \mc{N})$-measurable.
    To show the converse, note that $f^{-1}(\pi_\alpha^{-1}(E_\alpha))=f_\alpha^{-1}(E_\alpha)$ is a member of $\mc{M}$ whenever $\alpha\in A$ and $E_\alpha\in\mc{N}_\alpha$.
\end{proof}

\begin{cor}
    A complex-valued function $f$ on $X$ is $\mc{M}$ measurable if and only if $\real{f}$ and $\imag{f}$ are $\mc{M}$-measurable.
    (Here, $(X, \mc{M})$ is a measurable space.)
\end{cor}
\begin{proof}
    It follows from $\borel{\bb{C}}=\borel{\bb{R}^2}=\borel{\bb{R}}\otimes\borel{\bb{R}}$.
\end{proof}

We now prove that the complex-valued measurable functions (with a given domain) form a $\bb{C}$-algebra (when $\bb{R}$ is given as the codomain, then the collection is an $\bb{R}$-algebra).
\begin{lem}
    Suppose $(X, \mc{M})$ is a measurable space and $f,\,g: X\rightarrow F$ are $\mc{M}$-measurable functions, where $F=\bb{R}$ or $F=\bb{C}$.
    Then $f+g,\, fg,\, cf\, (c\in\bb{R})$ are $\mc{M}$-measurable.
\end{lem}
\begin{proof}
    The usual addition and multiplication in $\bb{R}$ and $\bb{C}$ are continuous.
\end{proof}

\begin{prop}
    If $(f_n)_{n\in\bb{N}}$ is a sequence of $\ol{\bb{R}}$-valued measurable functions on $(X, \mc{M})$, then the functions
    \begin{eqnarray*}
        &g_1(x)=\sup_n f_n(x),\quad g_2(x)=\inf_n f_n(x)&\\
        &g_3(x)=\limsup_n f_n(x),\quad g_4(x)=\liminf_n f_n(x)&
    \end{eqnarray*}
    are all measurable.
    If $\lim_n f_n(x)$ exists for every $x\in X$, then $f=\lim_n f_n$ is also measurable.
\end{prop}
\begin{proof}
    Remark that $g_1^{-1}(a, \infty]=\bigcup_n f_n^{-1}(a, \infty]$ and $g_2^{-1}[-\infty, a)=\bigcup_n f_n^{-1}[-\infty, a)$ whenever $a\in\ol{\bb{R}}$.
\end{proof}

Now we introduce the definition of a simple function and its significance in the integral theory.
\begin{defi}[Simple function]
    Let $(X, \mc{M})$ be a measurable function.
    A function $s: X\rightarrow\bb{C}$ is called a simple function if $s$ is measurable and has a finite range.
    Indeed, $s$ is a simple function if and only if the range of $s$ is $\{a_1, \cdots, a_k\}$ for some complex numbers $a_1, \cdots, a_k$ and $s=\sum_{k=1}^n a_k\chi_{E_k}$, where $E_k=f^{-1}(E_k)$ is measurable.
\end{defi}
We now introduce a remarkable observation that an $\ol{\bb{R}}$-valued (or complex-valued) measurable function $f$ defined on a measurable space $(X, \mc{M})$ can be approximated by a simple function.
This observation, especially together with the monotone convergence theorem which will be introduced in the following section, is essential in establishing the integral theory.
\begin{thm}\label{measurable functions as limits of simple functions}
    Let $(X, \mc{M})$ be a measurable space.
    \begin{enumerate}
        \item[(a)]
        {
            If $f: X\rightarrow[0, \infty]$ is a measurable function, then there is a sequence $(\phi_n)_{n\in\bb{N}}$ of simple functions on $X$ such that
            \begin{enumerate}
                \item[(\romannumeral 1)]
                {
                    $0\leq\phi_1\leq\phi_2\leq\cdots\leq f$,
                }
                \item[(\romannumeral 2)]
                {
                    $\phi_n\rightarrow f$ pointwise on $X$,
                }
                \item[(\romannumeral 3)]
                {
                    and $\phi_n\rightarrow f$ uniformly on any set on which $f$ is bounded.
                }
            \end{enumerate}
        }
        \item[(b)]
        {
            If $f: X\rightarrow\bb{C}$ is a measurable function, then there is a sequence $(\phi_n)_{n\in\bb{N}}$ of simple functions such that
            \begin{enumerate}
                \item[(\romannumeral 1)]
                {
                    $0\leq|\phi_1|\leq|\phi_2|\leq\cdots\leq|f|$,
                }
                \item[(\romannumeral 2)]
                {
                    $\phi_n\rightarrow f$ pointwise on $X$,
                }
                \item[(\romannumeral 3)]
                {
                    and $\phi_n\rightarrow f$ uniformly on any set on which $f$ is bounded.
                }
            \end{enumerate}
        }
    \end{enumerate}
\end{thm}
\begin{proof}
    In proving (a), for each $n\in\bb{N}$ and $0\leq k\leq 2^{2n}-1$, define
    \begin{align*}
        E_n^k:=f^{-1}([k\cdot 2^{-n}, (k+1)\cdot 2^{-n})),\quad F_n:=f^{-1}([2^n, \infty]).
    \end{align*}
    And let
    \begin{align*}
        s_n:=\sum_{k=1}^{2^{2n}-1}\frac{k}{2^n}\chi_{E_n^k}+2^n\chi_{F_n}.
    \end{align*}
    Clearly, $0\leq s_1\leq s_2\leq\cdots\leq f$ and $s_n\rightarrow f$ pointwise.
    Moreover, given a subset of $X$ over which $|f|<2^j$ for some positive integer $j$, $|f-s_m|\leq 2^{-m}$ whenever $m\geq j$; this proves the desired uniform convergence.
    In proving (b), decompose $f$ into the real and the imaginary parts and decompose each part into the nonnegative and the negative parts, and then use the result of (a).
\end{proof}

If $\mu$ is a measure on a measurable space $(X, \mc{M})$, then one may wish to except $\mu$-null sets from consideration.
In this respect, it is simpler when $\mu$ is complete.
\begin{prop}
    Let $(X, \mc{M})$ and $(Y, \mc{N})$ be measurable spaces.
    The following statements each are valid if and only if $\mu$ is complete:
    \begin{enumerate}
        \item[(a)]
        {
            If $f: X\rightarrow Y$ is a measurable function and $g: X\rightarrow Y$ is a function such that $f=g$ $\mu$-a.e., then $g$ is measurable.
        }
        \item[(b)]
        {
            Let $Y$ be a topological space and $\mc{N}$ be the Borel $\sigma$-algebra on $Y$.
            If $(f_n)_{n\in\bb{N}}$ is a sequence of measurable functions from $X$ into $Y$ and $f_n\rightarrow f$ $\mu$-a.e., then $f$ is measurable.
        }
    \end{enumerate}
\end{prop}
\begin{proof}
    \begin{enumerate}
        \item[(a)]
        {
            Assume first that $\mu$ is complete and let $g$ be a function on $X$ such that $f=g$ $\mu$-a.e.
            Letting $D=\{x\in X: f(x)\neq g(x)\}$, then $D$ is a $\mu$-null set.
            Hence, whenever $E\in\mc{N}$, $g^{-1}(E)$ and $f^{-1}(E)$ differ by at most $D$, i.e., $f^{-1}(E)\setminus D\subset g^{-1}(E)\subset f^{-1}(E)\cup D$.
            Therefore, $g^{-1}(E)\in\mc{M}$ and $g$ is measurable.

            Assume conversely and let $N$ be a subset of a $\mu$-null set.
            Letting $f=0$ and $g=\chi_N$, by hypothesis, $g$ is measurable, hence $N\in\mc{M}$ and $\mu$ is complete.
        }
        \item[(b)]
        {
            Assume first that $\mu$ is complete and suppose $f_n\rightarrow f$ $\mu$-a.e.
            Letting $D$ be the set of a point $x$ of $X$ such that $f_n(x)\nrightarrow f(x)$.
            Then $(X\setminus D)\cap f^{-1}(V)=(X\setminus D)\cap\liminf_{n\rightarrow\infty} f_n^{-1}(V)$, thus
            \begin{align*}
                (\liminf_{n\rightarrow\infty} f_n^{-1}(V))\setminus D\subset f^{-1}(V)\subset(\liminf_{n\rightarrow\infty} f_n^{-1}(V))\cup D
            \end{align*}
            and $f^{-1}(V)\in\mc{M}$.
            This proves that $f$ is measurable.

            Assume conversely and let $N$ be a subset of a $\mu$-null set.
            Then $0\rightarrow\chi_N$ $\mu$-a.e., so $N$ is measurable by hypothesis.
        }
    \end{enumerate}
    This completes the proof of the equivalences.
\end{proof}

On the other hand, the following result shows that one is unlikely to commit any serious blunders by forgetting to worry about completeness of the measure.
Later in this chapter, the following lemma will be applied to identify $L(\mu)$ and $L(\mu)$, where $(X, \ol{\mc{M}}, \ol\mu)$ is the completion of $(X, \mc{M}, \mu)$.
\begin{lem}\label{nullities are removable}
    Let $(X, \mc{M}, \mu)$ be a measure space and let $(X, \ol{\mc{M}}, \ol{\mu})$ be its completion.
    If $f$ is an $\ol{\mc{M}}$-measurable function on $X$, there is an $\mc{M}$-measurable function $g$ such that $f=g$ $\ol{\mu}$-a.e..
\end{lem}
\begin{rmk}
    Roughly speaking, according to this lemma, we may discard $\ol\mu$-null sets of $X$ to find an $\mc{M}$-measurable function which coincides $f$ $\ol\mu$-almost everywhere.
\end{rmk}
\begin{proof}
    We first check the statement for simple functions.

    \textbf{Step 1: The proof for simple functions.}\newline\noindent
    Suppose $f$ attains $a\in\bb{C}$ only.
    Because $f$ is $\ol{\mc{M}}$-measurable, $f=a\chi_E$ for some $E\in\ol{\mc{M}}$.
    By definition, $E=M\cup N$ for some $M\in\mc{M}$ and a subset $N$ of a $\mu$-null set.
    Letting $g=a\chi_M$, we have $f=g$ $\ol{\mu}$-almost everywhere.
    The case for simple functions is now obvious.

    \textbf{Step 2: Completing the proof.}\newline\noindent
    Assume $f$ is $\ol{\mc{M}}$-measurable, and let $(\phi_n)_{n\in\bb{N}}$ be a sequence of simple functions given as in \cref{measurable functions as limits of simple functions}.
    For each $n\in\bb{N}$, using the result of Step 1, let $\varphi_n$ be an $\mc{M}$-measurable simple function on $X$ such that $\phi_n=\varphi_n$ $\ol{\mu}$-almost everywhere.
    Let $D_n$ be the subset of a point $x$ of $X$ such that $\phi_n(x)\neq\varphi_n(x)$, and let $D$ be the union of $D_n$ for all $n$.
    Then $\ol{\mu}(D)=0$ and $\phi_n=\varphi_n$ on $X\setminus D$ for all $n$, so $f=\lim_n\varphi_n$ on $X\setminus D$.
    If $N$ is a subset of $X$ which belongs to $\mc{M}$ such that $D\subset N$ and one defines $g=\chi_{X\setminus N}\lim_n\phi_n$, we find that $g$ is a measurable function which coincides $f$ $\ol\mu$-almost everywhere.
\end{proof}

\subsection*{Problems}

\begin{prob}[Exercise 2.3]\label{exercise 2.3}
    Let $(f_n)_{n\in\bb{N}}$ be a sequence of measurable functions on a measurable space $(X, \mc{M})$.
    Show that the collection $E:=\{x\in X:\lim_n f_n(x)\textsf{ exists}\}$ is a measurable set.
\end{prob}
\begin{sol}
    Suppose first that every $f_n$ is real-valued.
    Then $E$ is the collection of points $x$ in $X$ such that $\liminf_n f_n(x)=\limsup_n f_n(x)$.
    Letting $s:=\limsup_n f_n$ and $i=\liminf_n f_n$, we have $E=(s-i)^{-1}(\{0\})$.
    Because $s$ and $i$ are measurable functions, $E$ is a measurable function.

    Now, assume that at least one $f_n$ is complex-valued function.
    Then $x\in E$ if and only if both $\real{f_n}$ and $\imag{f_n}$ are convergent at $x$.
    Using the preceeding result, we can easily derive the property.
\end{sol}

\begin{prob}[Exercise 2.5]\label{exercise 2.5}
    Assume that $(X, \mc{M})$ is a measurable space and $X=A\cup B$ where $A,\,B\in\mc{M}$.
    Show that a function $f$ on $X$ is measurable if and only if $f$ is measurable on $A$ and $B$.
\end{prob}
\begin{sol}
    Easy.
\end{sol}

\begin{rmk}[Forming a measurable limit function]
    Suppose $(f_n)_n$ is a sequence of measurable functions, and let $E$ be the subset of the domain on which $(f_n)_n$ is convergent.
    Let $f$ be a map defined on $X$ such that $f(x)=\lim_n f_n(x)$ for all $x\in E$.
    By \cref{exercise 2.3}, $E$ is a measurable set, so $f|_E$ is measurable on $E$.
    To find a `measurable' map $f$ on the entire domain, we just let $f(x)=0$ for all $x\in X\setminus E$.
    Because $f$ is then measurable on $E$ and $X\setminus E$, by \cref{exercise 2.5}, such map $f$ is measurable on the entire domain.
    Such construction of ``almost convergent'' limit funciton will be done in later sections.
\end{rmk}

\begin{prob}[Exercise 2.6]
    Show that the supremum of an uncountable family of measurable $\ol{\bb{R}}$-valued functions on $X$ can fail to be measurable (unless the $\sigma$-algebra on $X$ is very special).
\end{prob}
\begin{sol}
    We use the set we found when we proved the existence of a subset of $\bb{R}$ which is not Lebesgue measurable.
    Impose an equivalence relation $\sim$ on $[0, 1]$ by declaring $x\sim y$ if and only if $x-y\in\bb{Q}$, and let $N$ be the collection of all representatives.
    Then $N$ is not Lebesgue measurable (so it is necessarily uncountable).
    For each $\alpha\in N$, let $f_\alpha=\chi_{\{\alpha\}}$.
    Then the supremum of $\{f_\alpha\}_{\alpha\in N}$ is $\chi_N$, which is not measurable if the domain equips the Borel $\sigma$-algebra (if the domain equips the discrete $\sigma$-topology, then any function on the domain is measurable).
\end{sol}

\begin{prob}[Exercise 2.8]
    Let $f: \bb{R}\rightarrow\bb{R}$ be a monotonic function.
    Show that $f$ is measurable.
\end{prob}
\begin{sol}
    Remark that the Borel $\sigma$-algebra on $\bb{R}$ is generated by closed rays in $\bb{R}$ and that $f^{-1}([a, \infty))$ is a closed ray in $\bb{R}$.
\end{sol}