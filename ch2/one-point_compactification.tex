\section{Locally compact spaces and one-point compactification}

Some of the properties which are most desired for a topological space to have are the space being metrizable or being a compact Hausdorff space.
In this section, we impose a situation in which a topological space embeds into a compact Hausdorff space.

\begin{defi}[Local compactness]
    A space $X$ is said to be locally compact at a point $a\in X$ if there is a compact subspace $C$ of $X$ containing a neighborhood of $a$.\footnote{To prevent a confusion, suppose the neighborhood of $a$ is larger than the compact subspace $C$. In this case, we may choose $C=\{a\}$ for any topological space, making the notion of local compactness meaningless.}
    If $X$ is locally compact at every point of $X$, then $X$ is said to be locally compact.
\end{defi}

A simple observation follows:
\begin{obs}
    Let $\{X_\alpha\}_\alpha$ be an indexed family of nonempty spaces.
    \begin{enumerate}
        \item[(a)]
        {
            If $\prod_\alpha X_\alpha$ is locally compact, then each $X_\alpha$ is locally compact and $X_\alpha$ is compact for all but finitely many values of $\alpha$.
            If the product space is compact, then each $X_\alpha$ is compact.
        }
        \item[(b)]
        {
            The converse of the first statement in (a) is also true.
        }
    \end{enumerate}
\end{obs}
\begin{proof}
    Write $X=\prod_\alpha X_\alpha$.

    \hangindent=0.65cm
    \noindent(a)
    Given a point $x\in X$, there is a compact subspace $C$ of $X$ and a basis member $\prod_\alpha B_\alpha$ such that $x\in\prod_\alpha B_\alpha\subset C$.
    Because $\pi_\alpha(C)$ is a compact subspace of $X_\alpha$ containing the neighborhood $B_\alpha$ of $x_\alpha\in X_\alpha$ for each $\alpha$, each $X_\alpha$ is locally compact.
    Moreover, because $B_\alpha=X_\alpha$ for all but finitely many values of $\alpha$, $X_\alpha$ is compact for all but finitely many values of $\alpha$.
    In particular, if $X$ is compact, then $X_\alpha=\pi_\alpha(X)$ is compact for all $\alpha$.
    
    \noindent(b)
    Given a point $x\in X$, for each $\alpha$, use (local) compactness to find a compact subspace $C_\alpha$ of $X_\alpha$ containing a neighborhood $B_\alpha$ of $x_\alpha$ in $X_\alpha$.
    When finding such subspaces, choose $B_\alpha=C_\alpha=X_\alpha$ whenever $X_\alpha$ is compact.
    Then, the product of $B_\alpha$ for all $\alpha$ is a basis member of $X$ containing $x$; the product of $C_\alpha$ for all $\alpha$ is a compact subspace of $X$.
\end{proof}

\begin{thm}[Existence and uniqueness of a one-point compactification]
    Let $X$ be a space.
    Then $X$ is locally compact and Hausdorff if and only if there is a space $Y$ with the following properties:
    \begin{enumerate}
        \item[(a)]
        {
            $Y$ is a compact and Hausdorff space containing $X$ as a subspace, i.e., $Y$ is a compactification of $X$.
        }
        \item[(b)]
        {
            $Y\setminus X$ is a one-point set.
        }
    \end{enumerate}
    Moreover, if $Y_1$ and $Y_2$ are such spaces, then $Y_1$ and $Y_2$ coincide on $X$ and are homeomorphic, i.e., $Y$ is uniquely determined up to equivalence.\footnote{Equivalence of compactness is introduced in \Cref{compactification}.}
\end{thm}
\begin{proof}
    Since it is easier to check the uniqueness, we first explain the uniqueness (up to equivalence).

    \textbf{Step 1: Proving the uniqueness part.}\newline\noindent
    Suppose $Y_1$ and $Y_2$ are compact Hausdorff spaces satisfying (a), (b), and (c).
    Let $p$ and $q$ denote the unique point of $Y_1\setminus X$ and $Y_2\setminus X$, respectively.
    Define a map $h: Y_1\rightarrow Y_2$ by
    \begin{align*}
        h(x)=\left\{
        \begin{array}{cc}
            x   &   \textsf{(if $x\in X$)}\\
            q   &   \textsf{(otherwise, i.e., $x=p$)}
        \end{array}
        \right..
    \end{align*}
    We show $h$ is a homeomorphism extending the identity map on $X$; and for this, it suffices to verify that $h$ is a continuous map, because the openness of $h$ will follow by symmetry.
    If $U$ is an open subset of $Y_2$ contained in $X$, its preimage is $U$; because $U$ is open in $X$ and $X$ is open in $Y_1$, $U$ is open in $Y_1$.
    Assume $U$ is an open subset of $Y_2$ containing $q$.
    Then the subset $C=Y_2\setminus U$ is closed in $Y_2$, so $C$ is compact and is contained in $X$.
    It follows that $h^{-1}(C)=C$ is a compact subspace of $Y_1$ and that $h^{-1}(C)$ is closed in $Y_1$.
    Therefore, $h$ is a continuous map.

    \textbf{Step 2: Proving the existence part.}\newline\noindent
    Suppose first that $X$ is a locally compact Hausdorff space.
    Let $p$ be any element not in $X$, and let $Y=X\sqcup\{p\}$.
    And impose a topology on $Y$ by declaring the following subsets to be open in $Y$:
    \begin{enumerate}
        \item[(T1)]
        {
            Subsets which are open in $X$.
        }
        \item[(T2)]
        {
            Subsets of the form $Y\setminus C$, where $C$ is a compact subspace of $X$. (See \cref{check 1p cptf topology}.)
        }
    \end{enumerate}
    It is left as an exercise to check that the above collection is a topology on $Y$. (See \cref{check 1p cptf topology}.)

    We first show that $Y$ contains $X$ as a subspace.
    (Clearly, the topology on $X$ is coarser than the subspace topology on $X$ inherited from $Y$.)
    By merely intersecting any subset of $Y$ of either type with $X$, one can easily observe that those two topologies are equal.

    To show that $Y$ is compact, let $\mc{A}$ be any open cover of $Y$.
    Then there is a member $Y\setminus C\in\mc{A}$ of the second type which contains $p$.
    Since $C$ is a compact subspace of $X$, finitely many members of $\mc{A}$ cover $C$.
    These members, together with $Y\setminus C$, cover $Y$, as desired.

    Finally, we show that $Y$ is a Hausdorff space, in which it suffices to show that $p$ and any point $a$ in $X$ can be separated by disjoint open subsets of $Y$.
    Since $X$ is locally compact, there is a compact subspace $C$ of $X$ containing a neighborhood $U$ of $a$ in $X$, and $(U, Y\setminus C)$ is a desired pair.

    \textbf{Step 3: Proving the converse.}\newline\noindent
    Suppose such space $Y$ exists for a space $X$.
    Being a subspace of the Hausdorff space $Y$, $X$ is a Hausdorff space.
    Given a point $a\in X$, because $Y$ is a Hausdorff space, there are neighborhoods $U$ and $V$ of $a$ and $p$ in $Y$ which are disjoint.
    Since $Y\setminus V$ is a closed subspace of $Y$ contained in $X$, $Y\setminus V$ is a compact subspace of $X$ containing $U$.
    Therefore, $X$ is locally compact.
\end{proof}
\begin{prob}\label{check 1p cptf topology}
    \begin{enumerate}
        \item[(a)]
        {
            Show that the collection imposed in Step 2 of the above proof is a topology on $Y$.
        }
        \item[(b)]
        {
            Explain why we assume in the second type that $C$ is compact, rather than closed in $Y$.
        }
    \end{enumerate}
\end{prob}
\begin{sol}
    \begin{enumerate}
        \item[(a)]
        {
            What we need to do is to check the axioms of topology.
            
            Clearly, $\varnothing$ is of the first type and $Y$ is of the second type, so they belong to the collection.

            A union of sets of the first type is an open subset of $X$, hence the union is of the first type.
            If $\{V_\beta=Y\setminus C_\beta\}_\beta$ is a collection of sets of the second type (each $C_\beta$ is a compact subspace of $X$), their union is of the second type; because the union is $Y\setminus K$ (where $K=\bigcap_\beta C_\beta$) and $K$ is a closed subspace of every $C_\beta$, $K$ is compact.
            The union of a set $U$ of the first type and a set $V=Y\setminus C$ of the second type ($C$ is a compact subset of $X$) is of the second type, since
            \begin{align*}
                U\cup (Y\setminus C)=Y\setminus (C\setminus U)
            \end{align*}
            and $C\setminus U$ is compact (because $C\setminus U$ is a closed subset of the compact subset $C$).

            A finite intersection of sets of the first type is an open subset of $X$, hende the intersection is of the first type.
            If $(V_\beta=Y\setminus C_k)_{k=1}^n$ is a collection of sets of the second type (each $C_k$ is a compact subspace of $X$), their intersection is of the second type, because the union is $Y\setminus K$, where $K=\bigcup_{k=1}^n C_k$ is compact.
            The intersection of a set $U$ of the first type and a set $V=Y\setminus C$ of the second type ($C$ is a compact subspace of $X$) is of the first type, since
            \begin{align*}
                U\cap (Y\setminus C)=U\setminus C
            \end{align*}
            is open in $X$.
        }
        \item[(b)]
        {
            If $C$ in the second type is assumed to be closed subsets of $X$ rather than being compact, then $Y$ may not be compact, for a closed subset $C$ of $X$ may not be compact.
            (Remark that a compact subset of a Hausdorff space is closed.)
        }
    \end{enumerate}
\end{sol}

Let $X$ be a locally compact Hausdorff space, and let $Y$ be a space (which is unique up to equivalence) constructed as above.
\begin{itemize}
    \item
    {
        If $X$ is compact, then $X$ is a closed subspace of $Y$, so the closure of $X$ in $Y$ is $X$, which is a proper subset of $Y$.
    }
    \item
    {
        If $X$ is not compact, then $X$ is not closed in $Y$ \color{brown}(why?)\color{black}, so the closure of $X$ in $Y$ is $Y$.
    }
\end{itemize}
If $Y$ is such a space and the closure of $X$ in $Y$ is $Y$, $Y$ is called the one-point compactification (or the Alexandroff compactification) of $X$.
\begin{rmk}
    The above observation states that a locally compact Hausdorff space $X$ has a one-point compactification if and only if $X$ is not compact.
\end{rmk}
\begin{rmk}
    In \Cref{compactification}, we will consider the general concept of compactification (before studying the Stone-\v{C}ech compactification) over completely regular spaces.
    Since one-point compactifications are also compactifications, it will be better if it is explained that a locally compact Hausdorff space $X$ is completely regular.
    If the space $X$ is compact, then $X$ is normal, so $X$ is completely regular; if $X$ is not compact, then $X$ is a subspace of its one-point compactification, so $X$ is completely regular.
\end{rmk}

For Hausdorff spaces, local compactness can be intuitively considered as follows:
\begin{prop}
    Suppose $X$ is a Hausdorff space.
    Then $X$ is locally compact if and only if given $x\in X$ and its neighborhood $U$ in $X$, there is a relatively compact neighborhood $V$ of $a$ in $X$ whose closure in $X$ is contained in $U$.\footnote{By relatively compact we mean that the closure of the space is compact.}
\end{prop}
\begin{proof}
    Suppose $X$ is a locally compact Hausdorff space, and suppose further that a point $x\in X$ together with its neighborhood $U$ in $X$ is given.
    Since $X$ has a compactification $Y$ such that $Y\setminus X$ is a singleton, the subset $C:=Y\setminus U$ is a closed (of course, compact) subspace of $Y$.
    By regularity, there are disjoint neighborhoods $V$ of $x$ and $W$ of $C$ in $Y$.
    Then, $V$ is a neighborhood of $x$ in $X$ contained in $Y\setminus W\subset U$.
    Furthermore, if the overline notation denotes the closure of in $Y$, we have $\ol V\subset \ol{Y\setminus W}=Y\setminus W\subset U$.
    So, the closure of $V$ in $X$ satisfies $X\cap\overline{V}=\overline{V}\subset Y\setminus W\subset U$.
    Because $\ol V$ is closed in $Y$, it follows that $V$ is relatively compact in $X$.

    The converse is clear, for we may let $C$ be the closure of $V$ in $X$.
\end{proof}

As a homeomorphism implies the topologically equivalence of two topological spaces, one might expect the one-point compactifications of two homeomorphic spaces to be homeomorphic.
\begin{thm}\label{homeomorphism extends to 1p cptf.s}
    If $f: A\rightarrow B$ is a homeomorphism of locally compact Hausdorff spaces $A$ and $B$, then $f$ extends to a homeomorphism of the one-point compactifications of $A$ and $B$.
\end{thm}
\begin{proof}
    Assume that $A$ and $B$ are not compact (otherwise, there is nothing to prove), and let $X$ and $Y$ be the one-point compactifications of $A$ and $B$, respectively.
    If $f$ were to be extended to $X$, such an extension (as a set map) should be given as the bijection $\widetilde{f}: X\rightarrow Y$, which is defined by
    \begin{align*}
        \widetilde{f}(x)=\left\{\begin{array}{cc}
            f(x)    &   \textsf{(if $x\in A$)}\\
            q       &   \textsf{(if $x=p$)}
        \end{array}\right.,
    \end{align*}
    where $p$ and $q$ are the unique elements of $X\setminus A$ and $Y\setminus B$, respectively.
    To verify that $\widetilde{f}$ is a continuous map, let $V$ be an open subset of $Y$.
    If $V$ is contained in $B$, $\widetilde{f}^{-1}(V)=f^{-1}(V)$ is open in $A$, hence in $X$.
    If $V$ is not contained in $B$, $Y\setminus V$ is closed in $Y$ (hence, compact), implying that $\widetilde{f}^{-1}(Y\setminus V)=f^{-1}(Y\setminus V)$ is compact (hence, closed in $X$), i.e., $f^{-1}(V)$ is open in $X$.
    By symmetry, it follows that $\widetilde{f}$ has an open map.
    This completes the proof.
\end{proof}
\begin{exmp}
    We introduce some applications of \cref{homeomorphism extends to 1p cptf.s}.
    \begin{enumerate}
        \item[(a)]
        {
            Since $\bb{R}\approx S^1\setminus\{1\}$, the one-point compactification of $\bb{R}$ is homeomorphic to $S^1$.
            Similarly, because $\bb{C}\approx\bb{R}^2\approx S^2\setminus\{n\}$ (where $n$ is any point in $S^2$), the one-point compactification of $\bb{C}$ or $\bb{R}^2$ is homeomorphic to $S^2$.
            In addition, the one-point compactification of $\bb{C}^\times$ can be identified by gluing two distinct points of $S^2$.
        }
        \item[(b)]
        {
            Since $\bb{N}\approx K$, where $K=\left\{1/n: n\in\bb{N}\right\}$, the one-point compactification of $\bb{N}$ is homeomorphic to $K\sqcup\{0\}$.
        }
    \end{enumerate}
\end{exmp}