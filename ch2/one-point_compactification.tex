\section{Locally compact spaces and one-point compactification}

Some of the properties which are most desired for a topological space to have are the space being metrizable or being a compact Hausdorff space.
In this section, we impose a situation in which a topological space embeds into a compact Hausdorff space.

\begin{defi}[Local compactness]
    A space $X$ is said to be locally compact at the point $a\in X$ if there is a compact subspace $C$ of $X$ containing a neighborhood of $a$.
    If $X$ is locally compact at every point of $X$, then $X$ is said to be locally compact.
\end{defi}

One property regarding local compactness:
\begin{prop}
    Let $\{X_\alpha\}_\alpha$ be an indexed family of nonempty spaces.
    \begin{enumerate}
        \item[(a)]
        {
            If $\prod_\alpha X_\alpha$ is locally compact, then each $X_\alpha$ is locally compact and $X_\alpha$ is compact for all but finitely many values of $\alpha$.
            If the product space is compact, then each $X_\alpha$ is compact.
        }
        \item[(b)]
        {
            The converse of the first statement in (a) is also true.
        }
    \end{enumerate}
\end{prop}
\begin{proof}
    Remark that projection maps are continuous.
    Let $X=\prod_\alpha X_\alpha$, and we first prove (a).
    Given a point $x\in X$, there is a compact subspace $C$ of $X$ containing a basis member $\prod_\alpha B_\alpha$ that contains $x$.
    For each $\alpha$, $\pi_\alpha(C)$ is a compact subspace of $X_\alpha$, which contains the neighborhood $B_\alpha$ of $x_\alpha\in X_\alpha$.
    Becuase for all but finitely many values of $\alpha$ it is satisfied that $B_\alpha=X_\alpha$, it is proved that each $X_\alpha$ is locally compact and all but finitely many of $X_\alpha$'s are compact.
    In particular, if $X$ is compact, then each $X_\alpha=\pi_\alpha(X)$ is compact.
    
    Given a point $x\in X$, for each $\alpha$, use local compactness and compactness to find a compact subspace $C_\alpha\subset X_\alpha$ such that $C_\alpha$ contains a neighborhood $B_\alpha$ of $x_\alpha$ in $X_\alpha$.
    When finding such subspaces, we choose $B_\alpha=C_\alpha=X_\alpha$ whenever $X_\alpha$ is compact.
    Then, the product of $B_\alpha$'s is a basis member of $X$ containing $x$; the product of $C_\alpha$'s is a compact subspace of $X$.
\end{proof}

\begin{thm}[Existence and uniqueness of a one-point compactification]
    Let $X$ be a space.
    Then $X$ is locally compact and Hausdorff if and only if there is a space $Y$ with the following properties:
    \begin{enumerate}
        \item[(a)]
        {
            $Y$ is a compact and Hausdorff space containing $X$ as a subspace, i.e., $Y$ is a compactification of $X$.
        }
        \item[(b)]
        {
            $Y\setminus X$ is a one-point set.
        }
    \end{enumerate}
    Moreover, if $Y_1$ and $Y_2$ are such spaces, then they coincides on $X$ and they are homeomorphic, i.e., $Y$ is uniquely determined up to equivalence.\footnote{Equivalence of compactness is introduced in Chapter 4.}
\end{thm}
\begin{proof}
    Since it is easier to check the uniqueness, we first explain the uniqueness (up to equivalence).

    \textbf{Step 1: Proving the uniqueness part.}\newline\noindent
    Suppose $Y_1$ and $Y_2$ are compact Hausdorff spaces satisfying (a), (b), and (c).
    Let $p$ and $q$ denote the unique point of $Y_1\setminus X$ and $Y_2\setminus X$, respectively.
    Define a map $h: Y_1\rightarrow Y_2$ by
    \begin{center}
        $h(x)=x$ if $x\in X$, and $h(p)=q$.
    \end{center}
    We show $h$ is a homeomorphism extending the identity map on $X$; and for this, it remains to verify that $h$ is a continuous map, because the openness of $h$ will follow from symmetry.
    Suppose first that $U\subset Y_2$ is an open subset contained in $X$.
    Clearly, its preimage is $U$, and because $U$ is open in $X$ and $X$ is open in $Y_1$, $U$ is open in $Y_1$.
    Now, suppose that $U\subset Y_2$ is an open subset containing $q$.
    Let $C=Y_2\setminus U$ be the complement of $U$ in $Y_2$, which is contained in $X$.
    The preimage of $C$ is $C$; because $C$ is closed and compact in $Y_2$, $C$ is a compact subspace of $X$, a compact subspace of $Y_1$, so $C$ is closed in $Y_1$, too.
    Thus, $h$ is a continuous map.

    \textbf{Step 2: Proving the existence part.}\newline\noindent
    Suppose first that $X$ is a locally compact Hausdorff space.
    Let $p$ be any element not in $X$, and let $Y=X\sqcup\{p\}$.
    And impose a topology on $Y$ by declaring the following sets to be open in $Y$:
    \begin{enumerate}
        \item[(T1)]
        {
            Subsets which are open in $X$.
        }
        \item[(T2)]
        {
            Subsets of the form $Y\setminus C$, where $C$ is a compact subspace of $X$.
        }
    \end{enumerate}
    It is left as an exercise to check that the above collection is a topology on $Y$. (See \cref{check 1p cptf topology})

    We first show that $Y$ contains $X$ as a subspace.
    Clearly, the topology on $X$ is coarser than the subspace topology on $X$ inherited from $Y$.
    Conversely, the subspace topology on $X$ inherited from $Y$ is also coarser than the topology on $X$, which can be easily checked by intersecting $X$ with the sets of either type.

    To show that $Y$ is a compact space, let $\mc{A}$ be any open cover of $Y$.
    There is a member $A\in\mc{A}$ containing $p$, and the susspace $Y\setminus A$ is closed in $Y$.
    Since $Y\setminus A$ is contained in the Hausdorff space $X$, $Y\setminus A$ is compact.
    This proves the compactness of $Y$.

    Finally, we show that $Y$ is a Hausdorff space, in which it remains to show that $p$ and any point $a$ of $X$ can be separated by disjoint sets open in $Y$.
    Since $X$ is locally compact, there is a compact subspace $C$ of $X$ containing a neighborhood $U$ of $a$ in $X$.
    $(U, Y\setminus C)$ is a desired pair.

    \textbf{Step 3: Proving the converse.}\newline\noindent
    Suppose such space $Y$ exists for a space $X$.
    Being a subspace of the Hausdorff space $Y$, $X$ is also a Hausdorff space.
    Given a point $a\in X$, because $Y$ is a Hausdorff space, there are neighborhoods $U$ and $V$ of $a$ and the unique point $p\in Y\setminus X$ in $Y$ which are disjoint.
    Since $Y\setminus V$ is a closed subspace of $Y$ contained in $X$, $Y\setminus V$ is a compact subspace of $X$ containing the neighborhood $U$ of $a$.
    Therefore, $X$ is locally compact.
\end{proof}

\begin{prob}\label{check 1p cptf topology}
    Show that the collection imposed in the proof of the existence part is a topology on $Y$.
\end{prob}
\begin{sol}
    \textbf{Step 1. Cheking the axiom of covering.}\newline\noindent
    Clearly, $\varnothing$ is of the first type and $Y$ is of the second type, so they belong to the collection.

    \textbf{Step 2. Checking the axiom of union.}\newline\noindent
    A union of sets of the first type is an open subset of $X$, hence the union is of the first type.
    If $(V_\beta=Y\setminus C_\beta)_\beta$ is a collection of sets of the second type (each $C_\beta$ is a compact subspace of $X$), their union is of the second type, because the union is $Y\setminus K$, where $\ds{K=\bigcap_\beta C_\beta}$, and $K$ is a closed subspace of $C_\beta$'s, so $K$ is compact.
    The union of a set $U$ of the first type and a set $V=Y\setminus C$ of the second type ($C$ is a compact subspace of $X$) is of the second type, since
    \begin{align*}
        U\cup (Y\setminus C)=Y\setminus (C\setminus U)
    \end{align*}
    and $C\setminus U$ is compact \color{brown}(why?)\color{black}.

    \textbf{Step 3. Checking the axiom of intersection.}\newline\noindent
    A finite intersection of sets of the first type is an open subset of $X$, hende the intersection is of the first type.
    If $(V_\beta=Y\setminus C_k)_{k=1}^n$ is a collection of sets of the second type (each $C_k$ is a compact subspace of $X$), their intersection is of the second type, because the union is $Y\setminus K$, where $\ds{K=\bigcup_{k=1^n} C_k}$ is compact.
    The intersection of a set $U$ of the first type and a set $V=Y\setminus C$ of the second type ($C$ is a compact subspace of $X$) is of the first type, since
    \begin{align*}
        U\cap (Y\setminus C)=U\setminus C
    \end{align*}
    is open in $X$.
\end{sol}

Let $X$ be a locally compact Hausdorff space, and let $Y$ be a space (which is unique up to equivalence) constructed as above.
If $X$ is compact, then $X$ is a closed subspace of $Y$, so the closure of $X$ in $Y$ is $X$, a proper subset of $Y$.
On the other hand, if $X$ is not compact, as $X$ is a subspace of a compact Hausdorff space $Y$, $X$ is not closed ini $Y$, hence the closure of $X$ in $Y$ is $Y$.
If $Y$ is such a space and the closure of $X$ in $Y$ is $Y$, $Y$ is called the one-point compactification (or the Alexandroff compactification) of $X$.
\begin{rmk}
    The above observation states that a locally compact Hausdorff space $X$ has a one-point compactification if and only if $X$ is not compact.
\end{rmk}
\begin{rmk}
    In Chapter 4, we will consider the general concept of compactification (before studying the Stone-\v{C}ech compactification) over completely regular spaces.
    Since one-point compactifications are also compactifications, here we briefly explain that the locally compact Hausdorff space $X$ is completely regular.
    If the space $X$ is compact, then $X$ is normal, so $X$ is completely regular; if $X$ is not compact, then $X$ is a subspace of its one-point compactification, so $X$ is completely regular.
\end{rmk}

When the given space is a Hausdorff, the following property can be used as an alternative definition of local compactness.
\begin{prop}
    Suppose $X$ is a Hausdorff space.
    Then $X$ is locally compact if and only if given $x\in X$ and its neighborhood $U$ in $X$, there is a precompact neighborhood $V$ of $a$ in $X$ whose closure in $X$ is contained in $U$.
\end{prop}
\begin{proof}
    Suppose $X$ is a locally compact Hausdorff space, and suppose further that a point $x\in X$ together with its neighborhood $U$ in $X$ is given.
    Since $X$ has a compactification $Y$ such that $Y\setminus X$ has the unique point $p$, the subspace $C:=Y\setminus U$ is a closed (hence, compact) subspace of $Y$ containing $p$.
    Using regularity, one can find a neighborhood $V$ of $x$ and a neighborhood $W$ of $C$ which are disjoint.
    Then $V$ is a neighborhood of $x$ in $X$ contained in $Y\setminus W\subset U$.
    Furthermore, denoting the closure of $V$ in $Y$ by $\overline{V}$, the closure of $V$ in $X$ satisfies $X\cap\overline{V}\subset\overline{V}\subset Y\setminus W\subset U$.
    It is clear that the closure of $V$ in $X$ is compact, since $X$ is a Hausdorff space.

    The converse of the statement is clear, because the set $C=\overline{V}$ is a desired compact subspace of $X$ for $X$ to be locally compact.
\end{proof}

As homeomorphism implies that the given two spaces are ``topologically equivalent,'' one might expect one-point compactifications of two homeomorphic spaces to be homeomorphic.
\begin{thm}\label{homeomorphism extends to 1p cptf.s}
    If $f: A\rightarrow B$ denotes a homeomorphism of locally compact Hausdorff spaces, then $f$ extends to a homeomorphism of their one-point compactifications.
\end{thm}
\begin{proof}
    Let $X$ and $Y$ be the one-point compactifications of $A$ and $B$, respectively.
    Define $\tilde{f}: X\rightarrow Y$ by
    \begin{center}
        $\tilde{f}(a)=f(a)$ if $a\in A$, and $\tilde{f}(p)=q$,
    \end{center}
    where $p$ and $q$ are the unique elements of $X\setminus A$ and $Y\setminus B$, respectively.
    Clearly, $\tilde{f}$ is a bijection.
    Also, $\tilde{f}$ is a continuous map; if $V\subset B$ is an open subset of $B$, then $\tilde{f}^{-1}(V)=f^{-1}(V)$ is open in $A$, hence $\tilde{f}^{-1}(V)$ is open in $X$; if $C\subset B$ is a compact subset of $B$, then $\tilde{f}^{-1}(Y\setminus C)=X\setminus f^{-1}(C)$ is open in $X$ (because $f^{-1}(C)$ is a compact subspace of $A$).
    It is easy to show that $\tilde{f}$ has a continuous inverse, which completes the proof.
\end{proof}
\begin{exmp}
    We introduce some applications of \cref{homeomorphism extends to 1p cptf.s}.
    \begin{enumerate}
        \item[(a)]
        {
            Since $\bb{R}\approx S^1\setminus\{1\}$, the one-point compactification of $\bb{R}$ is homeomorphic to $S^1$.
        }
        \item[(b)]
        {
            Since $\bb{N}\approx K$, where $K=\left\{\dfrac{1}{n}: n\in\bb{N}\right\}$, the one-point compactification of $\bb{N}$ is homeomorphic to $K\sqcup\{0\}$.
        }
    \end{enumerate}
\end{exmp}