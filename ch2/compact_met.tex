\section{Compact metrizable spaces}

We first introduce some other types of compactness:
\begin{defi}
    Let $X$ be a space.
    \begin{enumerate}
        \item[(a)]
        {
            $X$ is said to be limit point compact, if every infinite subset of $X$ has a limit point in $X$.
        }
        \item[(b)]
        {
            $X$ is said to be sequentially compact, if every (infinite) sequence of points of $X$ has a convergent subsequence.
        }
    \end{enumerate}
\end{defi}
\begin{rmk}
    \begin{enumerate}
        \item[(a)]
        {
            Compactness implies limit point compactness.   
        }
        \item[(b)]
        {
            Let $X$ be a metric space.
            If $X$ is sequentially compact, then $X$ is complete and totally bounded.\footnote{Remark that both completeness and total boundedness are properties of metric spaces, for they can be discussed only when a metric on a metrizable space is given.}
        }
    \end{enumerate}
\end{rmk}

\begin{thm}
    Suppose $X$ is a metrizable space.
    Then the following are equivalent:
    \begin{enumerate}
        \item[(a)]
        {
            $X$ is compact.
        }
        \item[(b)]
        {
            $X$ is limit point compact.
        }
        \item[(c)]
        {
            $X$ is sequentially compact.
        }
    \end{enumerate}
\end{thm}
\begin{proof}
    We already proved that (a) implies (b) and it is easy to show that (b) implies (c).
    Thus, it remains to prove (a) under (c), so we assume that $X$ is sequentially compact.
    In this proof, we use the strategy by imposing $X$ a metric $d$ to understand $X$ as a metric space, not just a metrizable space.
    
    \textbf{Step 1: Showing the existence of a Lebesgue number.}\newline\noindent
    We wish to prove that for any open cover $\mc{A}$ of $X$ there is a real number $\delta>0$ such that every open set in $X$ with diameter less than $\delta$ is contained in some member of $\mc{A}$.
    If there is no such $\delta$, for each $n\in\bb{N}$, there is an open set $C_n$ of diameter less than $1/n$ which is not contained in any member of an open cover $\mc{A}$ of $X$.
    Choosing a point $x_n\in C_n$ for each $n\in\bb{N}$, by the sequential compactness of $X$, there is a convergent subsequence $(x_{n_k})_k$ of $(x_n)_n$ (with the limit denoted by $x$).\footnote{It is natural to consider countably many objects when one should consider sequences.}
    Let $A$ be a member of $\mc{A}$ containing $x$, and let $r>0$ be a real number such that $B_d(x, r)\subset A$.
    If a positive integer $i$ is large enough so that
    \begin{align*}
        d(x, x_{n_i})<\frac{r}{2}
        \quad\textsf{and}\quad
        \frac{1}{n_i}<\frac{r}{2},
    \end{align*}
    then $C_{n_i}\subset A$, a contradiction.
    
    \textbf{Step 2: Deriving that $X$ is compact.}\newline\noindent
    Let $\mc{A}$ be an open cover of $X$ and let $\delta>0$ be a Lebesgue number for $\mc{A}$.
    Because a sequentially compact metric space is totally bounded, finitely many balls in $X$ of radius $\delta/3$ cover $X$.
    So there is a finite subcollection of $\mc{A}$ which covers $X$.
\end{proof}
\begin{rmk}
    The result of Step 1 in the above proof is also known as the Lebesgue number lemma.
\end{rmk}

For metric spaces, the above equivalence reduces to the Heine-Borel theorem.
\begin{thm}[Heine-Borel theorem]
    Let $(X, d)$ be a metric space.
    Then the following are equivalent:
    \begin{enumerate}
        \item[(a)]
        {
            $X$ is compact.
        }
        \item[(b)]
        {
            $X$ is limit point compact.
        }
        \item[(c)]
        {
            $X$ is complete and totally bounded.
        }
    \end{enumerate}
\end{thm}
\begin{proof}
    By the preceeding theorem, it suffices to show that (c) is equivalent to compactness.
    Since it is already observed that sequential compactness implies (c), it suffices to show that (c) implies any of compactnesses.
    For this, we will show that (c) implies sequential compactness.

    Let $(x_n)_n\subset X$ be a sequence.
    Since $X$ is sequentially compact, there is a ball $B_1$ in $X$ of radius $2^{-1}$ which contains $x_n$'s for infinitely many $n\in N_1\subset\bb{N}$.
    Because $X\cap B_1$ is also totally bounded, a ball $B_2$ with the center in $X\cap B_1$ of radius $2^{-2}$ contains $x_n$'s for infinitely many $n\in N_2\subset N_1$.
    Continuing inductively, we can find $n_i\in N_i$ for each $i\in\bb{N}$ with $n_1<n_2<\cdots$.
    Then $d(x_{n_i}, x_{n_k})\leq 2^{1-i}$ if $i<k$.
    Because $X$ is complete, $(x_n)_n$ has a convergent subsequence.
    Therefore, (c) implies sequential compactness.
\end{proof}

\begin{prob}
    Let $U$ be an open subset of $\bb{C}$ which contains $\bb{D}$.
    Show that there is a positive real number $r>1$ such that $B(0, r)\subset U$.
\end{prob}
\begin{sol}
    Suppose that there is no such $r>1$.
    Then, we can choose $a_n\in B(0, 1+1/n)\setminus U$ for each $n\in\bb{N}$.
    Since $(a_n)_{n\in\bb{N}}\subset\ol{B(0, 2)}$, $(a_n)_n$ contains a convergent subsequence $(a_{n_k})_{k\in\bb{N}}$.
    Letting $\alpha$ be the limit of $(a_{n_k})_k$, we find that $\alpha$ is a limit point of $\bb{C}\setminus U$.
    Because $\bb{C}\setminus U$ is closed, $\alpha\notin U$.
    On the other hand, $\alpha\in\partial\bb{D}$ \color{brown}(why?)\color{black}, we have $\alpha\in\partial\bb{D}\subset U$, a contradiction.
    Therefore, $B(0, r)\subset U$ for some real number $r>1$.
\end{sol}
\begin{sol}[Alternative solution]
    Note that the collection of `polar rectangles' is a basis for the topology on $\bb{C}$, where a polar rectangle is of the form
    \begin{align*}
        a\leq \rho\leq b,\quad s\leq \theta\leq t,
    \end{align*}
    where $a, b, s, t$ are real numbers such that $a\leq b$.
    Because $\partial\bb{D}$ is compact, for each point $e^{ix}\in\partial\bb{D}\,(0\leq x<2\pi)$ there is a polar rectangle $P_x: a(x)\leq \rho\leq b(x), s(x)\leq \theta\leq t(x)$ such that $e^{ix}\in P_x\subset U$.
    Choosing finitely many members among $P_x$'s and letting $r$ the smallest $b(x)$, we find that $B(0, r)\subset U$.
\end{sol}