\section{Compact spaces}

\begin{defi}[Compact space]
    A topological space $X$ is said to be compact if every covering of $X$ by sets open in $X$ has a finite subcover.
    If every such covering has a countable subcover, then $X$ is said to be a Lindel\"{o}f space.
\end{defi}
\begin{rmk}
    In fact, we can impose an alternative definition of compactness as follows:
    \begin{quotation}
        The topological space $X$ is said to be compact, if for every collection $\mc{C}$ of closed sets in $X$ having the finite intersection property, the intersection of the members of $\mc{C}$ is nonempty. 
    \end{quotation}
\end{rmk}
\begin{proof}
    Establish some contrapositions to check the above two statements are equivalent.
    The proof is left as an exercise.
\end{proof}

As connectedness, compactness of subspaces can also be argued in larger spaces.
\begin{thm}[Compactness of subspaces]
    Let $A$ be a subspace of $X$.
    Then $A$ is compact if and only if every covering of $A$ by sets open in $X$ has a finite subcover.
    (This statement is valid if the words `compact' are replaced by `Lindel\"{o}f.')
\end{thm}
\begin{proof}
    Suppose $A$ is compact, and $\mathcal{A}=\{A_\alpha\}_\alpha$ is a covering of $A$ by sets open in $X$.
    Then the naturally induced collection $\{A_\alpha\cap A\}$ is a covering of $A$ by sets open in $A$, and we can find a finite subcover.
    The corresponding finite subcover of $\mathcal{A}$ covers $A$.

    Suppose conversely that every covering of $A$ by sets open in $X$ has a finite subcover, and let $\mathcal{A}=\{A_\alpha\}_\alpha$ be a covering of $A$ by sets in $A$.
    Since each $A_\alpha$ can be written as $A\cap O_\alpha$ for some subset $O_\alpha$ open in $X$, the collection $\{O_\alpha\}_\alpha$ has a finite subcover.
    The corresponding finite subcover of $\mathcal{A}$ covers $A$.
\end{proof}

\begin{thm}
    Regarding closedness and compactness of subspaces, the following statements holds:
    \begin{enumerate}
        \item[(a)]
        {
            Every closed subspace of a compact space is compact.
        }
        \item[(b)]
        {
            Every Hausdorff space is compactly normal.
            In other words, any two disjoint compact subspaces can be separated by (disjoint) neighborhoods.
        }
        \item[(c)]
        {
            Every compact subspace of a Hausdorff space is closed. (Hence, every compact Hausdorff space is normal.)
        }
    \end{enumerate}
\end{thm}
\begin{sol}
    \begin{enumerate}
        \item[(a)]
        {
            Let $X$ be a compact space and $Y$ be a closed subspace of $X$.
            Let $\mc{A}=\{A_\alpha\}_\alpha$ be a covering of $Y$ by sets open in $X$.
            Because the collection $\mc{A}\cup\{X\setminus Y\}$ is also an open covering of $X$, the collection has a finite subcover.
            Among them, excluding $X\setminus Y$ gives a finite subcover of $\mc{A}$ which covers $Y$.
        }
        \item[(b)]
        {
            We first prove that a Hausdorff space $X$ is compactly regular.
            Let $p$ be a point of $X$ and $A$ be a compact subspace of $X$ not containing $p$.
            For each $a\in A$, let $U_a$ and $V_a$ be disjoint neighborhoods of $p$ and $a$, respectively.
            Since $A$ can be covered by finitely many $V_a$'s, the intersection of the corresponding $U_a$'s is a neighborhood of $p$.
            Thus, $X$ is compactly regular.

            To show that $X$ is compactly normal, let $A, B$ be disjoint compact subspaces of $X$.
            For each $a\in A$, let $U_a$ and $V_a$ be disjoint neighborhoods of $a$ and $B$, respectively.
            Since $A$ can be covered by finitely many $U_a$'s, the union of the corresponding $U_a$'s and the intersection of the corresponding $V_a$'s are disjoint neighborhoods of $A$ and $B$, respectively.
            Therefore, $X$ is compactly normal.
        }
        \item[(c)]
        {
            If $A$ is a compact subspace of a Hausdorff space $X$, then it follows from (b) that $X\setminus A$ is open, hence $A$ is closed.
            In (a), we have shown that a closed subspace of a compact space is compact.
            Thus, compactness and closedness of a subspace of a compact Hausdorff space coincide.
            Hence, every compact Hausdorff space is normal.
        }
    \end{enumerate}
\end{sol}

\begin{prop}
    A continuous image of a compact space is compact.
\end{prop}
\begin{proof}
    Because this proposition is easy to prove, the proof will be left as an exercise.
\end{proof}

\begin{prop}
    Let $f: X\rightarrow Y$ be a bijective continuous map.
    If $X$ is compact and $Y$ is Hausdorff, then $f$ is a homeomorphism.
\end{prop}
\begin{proof}
    If $A$ is a closed subspace of $X$, then $A$ is compact, and $f(A)$ is a compact subspace of $Y$, hence $f(A)$ is a closed subspace of $Y$.
    Therefore, $f$ is a homeomorphism.
\end{proof}

\begin{thm}\label{products of compact spaces}
    The product of finitely many compact spaces is compact.
    (In fact, this theorem extends to arbitrary products of compact spaces, which is called the Tychonoff's theorem.
    The proof of the Tychonoff's theorem will not be introduced in this note; read your textbook.)
\end{thm}
When proving the above theorem, the first version of the tube lemma will be used.
\begin{lem}\label{tube lemmas}
    \begin{enumerate}
        \item[(a)]
        {
            (The tube lemma)
            Consider the product space $X\times Y$, where $Y$ is a compact space.
            If $N$ is an open subspace of $X\times Y$ containing a slice $x_0\times Y$ for some $\{x_0\}\in X$, then $N$ contains a tube $W\times Y$ about $x_0\times Y$, where $W$ is a neighborhood of $x_0$ in $X$.
        }
        \item[(b)]
        {
            (A generalization of the tube lemma)
            Let $A$ and $B$ be compact subspaces of $X$ and $Y$, respectively, and let $N$ be an open subspace of $X\times Y$ containing $A\times B$.
            Then, there are neighborhoods $U$ and $V$ of $A$ in $X$ and $B$ in $Y$, respectively, such that $A\times B\subset U\times V\subset N$.
            (Our first version comes when $A=\{x_0\}$ and $B=Y$.)
        }
    \end{enumerate}
\end{lem}
\begin{proof}[Proof of \cref{tube lemmas}]
    \begin{enumerate}
        \item[(a)]
            Since $N$ is open, for each point $(x_0, y)\in\{x_0\}\times Y$, there is a neighborhood $A_y\times B_y$ of the point $(x_0, y)$ contained in $N$, where $A_y$ and $B_y$ are open in $X$ and $Y$, respectively.
            Because $\{x_0\}\times Y$ is homeomorphic to $Y$, hence $\{x_0\}\times Y$ is also compact.
            Hence, the slice can be covered by finitely many above basis members.
            For convinience, write such members as $A_i\times B_i$ for $i=1, \cdots, n$.
            Write $W=\bigcap_{i=1}^n A_i$, then $\{x_0\}\times Y\subset W\times Y\subset N$.
        \item[(b)]
            Since $N$ is open, for each point $(a, t)\in\{a\}\times B$, there is a neighborhood $A_t^a\times B_t^a$ of the point $(a, t)$ contained in $N$, where $A_t^a$ and $B_t^a$ are open in $X$ and $Y$, respectively.
            Because the slice $\{a\}\times B$ is compact for each $a\in A$, the slice can be covered by finitely many neighborhoods; write it as $A_i^a\times B_i^a$ for $i=1, \cdots, n^a$.
            Finally, define
            \begin{align*}
                U^a:=\bigcap_{i=1}^{n^a}{A_i^a},\quad V^a:=\bigcup_{i=1}^{n^a}{B_i^a}.
            \end{align*}
            Then, $U^a\times V^a$ is an open set contained in $N$ containing $\{a\}\times Y$.
            Since $A$ is compact, finitely many $U^a$'s cover $A$; write them as $U^1, \cdots, U^k$.
            Finally, define
            \begin{align*}
                U:=\bigcup_{i=1}^k U^k,\quad V:=\bigcap_{i=1}^k V^k.
            \end{align*}
            Then, $U\times V$ is an open set contained in $N$ containing $A\times B$.
    \end{enumerate}
    This completes the proof of tube lemmas.
\end{proof}
\begin{proof}[Proof of \cref{products of compact spaces}]
    Let $X, Y$ be compact spaces, and $\mc{A}$ be an open cover of $X\times Y$.
    Given $x_0\in X$, because the slice $\{x_0\}\times Y$ is compact, there are finitely many members of $\mc{A}$ covering the slice $\{x_0\}\times Y$, and such members cover a tube $U(x_0)\times Y$ about $\{x_0\}\times Y$.
    Because finitely many $U(x_0)$'s cover $X$, the compactness of $X\times Y$ is justified.
    The general result can be obtained with help of mathematical induction.
\end{proof}

Before introducing some problems, we introduce some properties of compact ordered sets and a relevant theorem.
\begin{rmk}[Compact ordered sets]
    Suppose $X$ is an ordered set in the order topology, and assume $X$ is compact.
    \begin{enumerate}
        \item[(a)] $X$ has the greatest and the least element; otherwise, one can construct an open cover of $X$ with no finite subcover.
        \item[(b)] (Extreme value theorem) Let $f: K\rightarrow Y$ be a continuous map, where $K$ is a compact space and $Y$ is an ordered set in the order topology. Then $f$ attains the maximum and the minimum. This is because the subspace $f(K)$ of $Y$ is compact.
        \item[(c)] If $X$ satisfies the least upper bound property, then every closed interval in $X$ is compact.
    \end{enumerate}
\end{rmk}

The first problem deals with the distance between a point and a subspace and neighborhoods of subspaces in a metric space.
In particular, (c) implies that the original definition of the $\epsilon$-neighborhood of a subspace and our intuition coincide.
\begin{prob}
    Let $(X, d)$ be a metric space and $A$ be a nonempty subset of $A$.
    \begin{enumerate}
        \item[(a)] Show that $\dist{d}{x}{A}=0$ if and only if $x\in\overline{A}$.
        \item[(b)] Show that if $A$ is compact, then $\dist{d}{x}{A}=d(x, a)$ for some $a\in A$, whenever $x\in X$.
        \item[(c)] Define the $\epsilon$-neighborhood of $A$ in $X$ to be the set
            \begin{align*}
                U(A, \epsilon):=\{x\in X:\dist{d}{x}{A}<\epsilon\}.
            \end{align*}
            Show that $U(A, \epsilon)$ equals the union of the open balls $B_d(a, \epsilon)$ for $a\in A$.
        \item[(d)] Assume that $A$ is compact, and let $U$ be an open set in $X$ containing $A$.
            Show that $U$ contains an $\epsilon$-neighborhood of $A$ for some $\epsilon>0$.
        \item[(e)] Show that the result in (d) need not hold if $A$ is closed but not compact.
    \end{enumerate}
\end{prob}
\begin{sol}
    \begin{enumerate}
        \item[(a)]
        {    
            Almost clear.
            If $x\in\overline{A}$, then whenever $k>0$, there is a point $a\in A$ such that $d(x, a)<k$.
            Hence, $\dist{d}{x}{A}=0$.
            Assuming conversely, whenever $k>0$, there is a point $a\in A$ such that $d(x, a)<k$, so $B_d(x, k)$ intersects $A$.
            Hence, $x\in\overline{A}$.
        }
        \item[(b)]
        {
            Given a point $x\in X$, define a function $f_x: A\rightarrow[0, \infty)$ by $f_x(a)=d(x, a)$ for $a\in A$.
            Since $|f_x(a)-f_x(b)|\leq d(a, b)$ whenever $a, b\in A$, $f_x$ is (uniformly) continuous.
            Because the domain $A$ of $f_x$ is a compact space, $f_x$ attains the minimum.
            Furthermore, $\dist{d}{x}{A}=\inf\{d(x, a): a\in A\}=\min\{f_x(a): a\in A\}$, so $\dist{d}{x}{A}=d(x, a)$ for some $a\in A$.
        }
        \item[(c)]
        {
            It is clear that every $\epsilon$-ball with the center in $A$ is contained in the $\epsilon$-neighborhood of $A$, so one inclusion is obvious.
            Suppose $u\in U(A, \epsilon)$ so that $\dist{d}{u}{A}<\epsilon$.
            It implies that the $\epsilon$-ball centered at $u$ intersects $A$ at a point $a\in A$, so the $\epsilon$-ball centered at $a$ contains $u$.
            Therefore, $U(A, \epsilon)$ is contained in the union of the $\epsilon$-balls with centers in $A$.
        }
        \item[(d)]
        {    
            Let $f: A\rightarrow \bb{R}$ be a function defined by $f(a)=\dist{d}{a}{X\setminus U}$.
            \begin{itemize}
                \item
                {
                    $f$ is continuous; for $a, b\in A$, we have $f(a)\leq d(a, x)\leq d(a, b)+d(b, x)$ for all $x\in X\setminus U$, so $f(a)\leq d(a, b)+f(b)$, from which, by symmetry, we obtain $|f(a)-f(b)|\leq d(a, b)$.
                }
                \item
                {
                    $f(a)>0$ for all $a\in A$; otherwise, if $f(p)=0$ for some $p\in A$, then $p\in\ol{X\setminus U}=X\setminus U$, a contradiction.
                }
            \end{itemize}
            Because $A$ is compact, $f$ attains the minimum $\delta>0$.
            Hence, for example, if $\epsilon=\delta/2$, then $U(A, \epsilon)$ is contained in $U$.
        }
        \item[(e)]
        {
            The $x$-axis in the $xy$-plane is closed but not compact, and the following open region contains the $x$-axis:
            \begin{align*}
                R:=\{(x, y)\in\bb{R}^2: y<e^{-|x|}\}.
            \end{align*}
            No neighborhood of the $x$-axis is contained in $R$.
        }
    \end{enumerate}
\end{sol}

\begin{prob}
    Let $X$ be a compact Hausdorff space.
    Let $\mc{A}$ be a collection of closed connected subsets of $X$ that is simply ordered by proper inclusion.
    Then $Y:=\bigcap_{A\in\mc{A}} A$ is connected.
\end{prob}
\begin{sol}
    Suppose $Y$ is not connected, and let $(C, D)$ be a separation of the intersection $Y=\bigcap_{A\in\mc{A}}A$.
    Because $C$ and $D$ are closed in $Y$ and $Y$ is closed in $X$, $C$ and $D$ are closed in $X$.
    Since $X$ is normal, there are disjoint neighborhoods $U$ and $V$ of $C$ and $D$, respectively.
    Suppose $A\setminus (U\sqcup V)$ is empty for some $A\in\mc{A}$.
    Since $A$ contains $Y$, $A\subset U\sqcup V$, $A$ has a separation $(A\cap U, A\cap V)$, a contradiction.
    Therefore, $A\setminus(U\sqcup V)$ is nonempty for all $A\in\mc{A}$, and the collection $\mc{C}$ of $A\setminus(U\sqcup V)$ is a simply ordered collection of sets closed in $X$.
    Therefore, $\mc{C}$ satisfies the finite intersection property, and the intersection of members in $\mc{C}$ is nonempty, because $X$ is compact.
    Thus, $Y$ is not contained in $U\sqcup V$, a contradiction.
    Therefore, $Y$ is connected.
\end{sol}