\section{Connected spaces}

\begin{defi}[Separation of a topological space]
    Let $X$ be a topological space.
    A pair $(A, B)$ is called a separation of $X$ if $\{A, B\}$ is a partition of $X$ by nonempty open subsets of $X$.
    If $X$ has no separation, then $X$ is called a connected space.
\end{defi}
\begin{rmk}
    \begin{enumerate}
        \item[(a)]
        {
            Connectedness is a topological property.
            Hence, connectedness is preserved under homeomorphisms.
        }
        \item[(b)]
        {
            The topological space $X$ is connected if and only if the only subsets of $X$ which are both open and closed in $X$ are $\varnothing$ and $X$.
        }
    \end{enumerate}
\end{rmk}

When discussing connectedness of a subspace $A$ of $X$ according to the above definition, subsets of $A$ shall be considered.
The following theorem lets us argue the connectedness of a subspace $A$ of $X$ \textit{in terms of the subsets of $X$}.
\begin{thm}[Connected subspace]
    Suppose $X$ is a topological space and $A$ is a subspace of $X$.
    Then $(U, V)$ is a separation of $A$ if and only if
    \begin{enumerate}
        \item[(\romannumeral 1)]
        {
            $\{U, V\}$ is a partition of $A$ by nonempty subsets of $A$ (here, $U$ and $V$ need not be open or closed in $A$ or $X$)
        }
        \item[(\romannumeral 2)]
        {
            and neither $U$ nor $V$ contains a limit point (in $X$) of the other, i.e., $\overline{U}\cap V=U\cap\overline{V}=\varnothing$. (The overline notations stand for the closure in $X$.)
        }
    \end{enumerate}
\end{thm}
\begin{proof}
    Suppose $(U, V)$ is a separation of $Y$.
    Because $U$ is both open and closed in $Y$, $U=\overline{U}\cap Y$.
    Since $U\cap V=\varnothing$, we have $\overline{U}\cap V=\varnothing$.
    For the same reason, $U\cap\ol V=\varnothing$.

    Conversely, assuming (\romannumeral 1) and (\romannumeral 2), we have $\overline{U}\cap A=\overline{U}\cap(U\sqcup V)=U$, so $U$ is closed in $A$.
    For the same reason, $V$ is also closed in $A$, so $(U, V)$ is a separation of $A$.
\end{proof}

We introduce a useful lemma regarding connected spaces, implying that a connected subspace lies entirely in a `separation component.'
\begin{lem}
    If $(C, D)$ is a separation of $X$ and if $Y$ is a connected subspace of $X$, then $Y$ lies entirely within $C$ or $D$.
\end{lem}
\begin{proof}
    Write $A=Y\cap C$ and $B=Y\cap D$.
    Since $(C, D)$ is a separation of $X$, $C$ and $D$ are open (and closed) in $X$ and in $Y$.
    Because $Y$ is connected, either $C$ or $D$ is empty, as desired.
\end{proof}

\begin{prop}\label{nonempty intersection keeps connectedness}
    Suppose $\{U_\alpha\}_{\alpha\in I}$ is a collection of connected subspaces, and assume all the members have a common point.
    Then the union of the members of the collection is connected.
\end{prop}
\begin{proof}
    Suppose the union $A$ of $A_\alpha$ for $\alpha\in I$ is not connected.
    Then there is a separation $(U, V)$ of $A$, and each $A_\alpha$ resides entirely in either $U$ or $V$.
    Without loss of generality, assume $A_{\alpha_0}$ is in $U$ for some $\alpha_0\in I$.
    Since a common point is in $U$ and not in $V$, $A_\alpha\subset U$ for all $\alpha\in I$, so $A\subset U$, a contradiction.
\end{proof}

\begin{prop}\label{adding limit points keeps connectedness}
    If $A$ is a connected subspace of $X$, then adding some of its limit points keeps the space connected.
    To be precise, if $A\subset B\subset\overline{A}$, then $B$ is connected.
\end{prop}
\begin{proof}
    Suppose $B$ is not connected and let $(U, V)$ be a separation of $B$.
    Since $A$ is a connected subspace of $B$, without loss of generality, $A\subset U$, so $\ol{A}\subset \overline{U}$.
    Because $\overline{U}$ and $V$ are disjoint, $\ol{A}\cap V=\varnothing$, a contradiction.
\end{proof}

\begin{thm}\label{continuity keeps connectedness}
    A continuous image of a connected space is connected.
\end{thm}
\begin{proof}
    Let $X$ be a connected space and $f: X\rightarrow Y$ be a continuous map.
    Suppose $f(X)$ is not connected and let $(U, V)$ be a separation of $f(X)$.
    Since $U$ and $V$ are open in $f(X)$, $f^{-1}(U)$ and $f^{-1}(V)$ are open in $X$.
    Thus, $(f^{-1}(U), f^{-1}(V))$ is a separation of $X$, which contradicts the connectedness of $X$.
\end{proof}

The idea of the proof of the following proposition is remarkable.
\begin{thm}
    A finite product of connected spaces is connected in the product topology.
\end{thm}
\begin{proof}
    It suffices to prove for the product of two connected spaces, for the desired result can be obtained by induction.
    Let $X_1$ and $X_2$ be connected spaces.
    Given a point $(a, b)\in X_1\times X_2$, define the `cross' $C(a, b)$ at $(a, b)$ by $(\{a\}\times X_2)\cup(X_1\times\{b\})$.
    Because $X_1\times\{b\}\approx X_1$ and $\{a\}\times X_2\approx X_2$ are connected, by \cref{nonempty intersection keeps connectedness}, $C(a, b)$ is connected.
    Because
    \begin{align*}
        X_1\times X_2=\bigcup_{a\in X_1}C(a, b)
    \end{align*}
    for any point $b$ of $X_2$ and the intersection of $C(a, b)$ for $a\in X_1$ is nonempty, by \cref{nonempty intersection keeps connectedness}, $X_1\times X_2$ is connected.
\end{proof}

In fact, the above proposition extends to an arbitrary product.
\begin{thm}
    Let $\{X_\alpha\}_{\alpha\in I}$ be a family of connected spaces.
    Then the product $X=\prod_{\alpha\in I}X_\alpha$ is connected in the product topology.
\end{thm}
\begin{proof}
    We first fix a point $a=(a_\alpha)_{\alpha\in I}\in X$, and let
    \begin{align*}
        X_K=\{x\in X
        :
        x_\alpha=a_\alpha\textsf{ whenever }\alpha\in I\setminus K
        \}
    \end{align*}
    for each finite subset $K$ of $I$.
    In other words, $X_K$ is the subset of $X$ which consists of all points permitting all values only at each index in $K$.

    \textbf{Step 1: Showing that the union of $X_K$'s is connected.}\newline\noindent
    $X_K\approx\prod_{\alpha\in K}X_\alpha$ is connected by the preceeding theorem.
    Since each $X_K$ contains $a$, the union $Y$ of all $X_K$ is connected.

    \textbf{Step 2: Showing that the closure of $Y$ in $X$ is $X$.}\newline\noindent
    Suppose $p\in X$ and let $W$ be a neighborhood of $p$.
    There is a basis member $B=\prod_{\alpha\in I} B_\alpha$ such that $p\in B\subset W$ (clearly, $J:=\{\alpha\in I: B_\alpha\neq X_\alpha\}=\{\alpha_1, \cdots, \alpha_n\}$).
    Because $B$ intersects $X_J$, $B$ intersects $Y$, so $p\in\overline{Y}$ and $X=\ol Y$.

    Therefore, the product space $X$ is connected.
\end{proof}

We give a problem in the textbook, with a solution using the cross we constructed in an earlier proof.
\begin{prob}
    Let $X, Y$ be connected spaces and $A, B$ are nonempty proper subsets of $X, Y$, respectively.
    Show that $(X\setminus A)\times(Y\setminus B)$ is connected.
\end{prob}
\begin{sol}
    Let $(p, q)$ be a point of $(X\setminus A)\times(Y\setminus B)$, and define
    \begin{align*}
        M:=\bigcup_{x\in X\setminus A}C(x, q),\quad N:=\bigcup_{y\in Y\setminus B}C(p, y).
    \end{align*}
    It is easy to check that $M, N$ are nonempty and contain $(p, q)$, and $M\cup N=(X\setminus A)\times(Y\setminus B)$.
    Therefore, $(X\setminus A)\times(Y\setminus B)$ is connected.
\end{sol}

\begin{exmp}
    Since $\bb{R}$ is connected in the standard topology, $\bb{R}^\bb{N}$ is connected in the product topology.
    However, $\bb{R}^\bb{N}$ is not connected \textit{in the uniform topology} (and in the box topology \color{brown}(why?)\color{black}).
    To justify this assertion, let $A$ be the set of all bounded sequences in $\bb{R}^\bb{N}$, and let $B$ be the set of all unbounded sequences in $\bb{R}^\bb{N}$.
    It is clear by definition that $\bb{R}^\bb{N}=A\sqcup B$ and $A, B$ are nonempty, and one can easily check that $A$ and $B$ are open in $\bb{R}^\bb{N}$.
    Thus, $\bb{R}^\bb{N}$ is not connected.
\end{exmp}

\begin{thm}[Intermediate value property]
    Let $X$ be a connected space, $Y$ be an ordered set in the order topology, and let $f: X\rightarrow Y$ be a continuous map.
    If $a, b\in X$ and $r$ is a point in $Y$ between $f(a)$ and $f(b)$, there is a point $p\in X$ such that $f(p)=r$.
\end{thm}
\begin{proof}
    Suppose there is an intermediate value $r$ such that $r\notin f(X)$.
    Then $f(X)=(f(X)\cap (-\infty, r))\sqcup(f(X)\cap(r, \infty))$, forming a separation of $f(X)$.
    Here arises a contradiction, because the connectedness of $X$ implies the connectedness of $f(X)$.
\end{proof}

\begin{prob}\label{Particular uncountable space - connected "metrizable" spaces}
    Show that a connected metrizable space with more than one point is uncountable.\footnote{The result of this problem will be generalized later. See \cref{nonempty intersection keeps connectedness}.}
\end{prob}
\begin{sol}
    Let $X$ be a connected metrizable space with more than one point.
    Let $d$ be a metric on $X$ inducing the topology on $X$, and define a function $f: X\rightarrow\bb{R}$ by $f(x)=d(x, b)$ for $x\in X$, where $b\in X$ is given.
    Note that $f(X)$ is connected because $X$ is connected and $f$ is continuous.
    Thus, $f(X)$ and $X$ are uncountable, for $f(a)\neq f(b)$.
\end{sol}