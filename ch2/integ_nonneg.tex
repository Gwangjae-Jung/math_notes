\section{Integration of nonnegative functions}

Throughout this section, we fix a measure space $(X, \mc{M}, \mu)$, and we define
\begin{defi}[The class of nonnegative measurable functions]
    We define
    \begin{align*}
        L^+:=\textsf{(the space of all measurable functions from $X$ to $[0, \infty]$)}.
    \end{align*}
\end{defi}

\subsection{Integration of nonnegative measurable simple functions}

If $\phi\in L^+$ is a simple function with the standard representation $\phi=\sum_{j=1}^n a_j\chi_{E_j}$, we define the integral of $\phi$ with respect to $\mu$ by
\begin{align*}
    \int\phi\,d\mu:=\sum_{j=1}^n a_j\mu(E_j),
\end{align*}
where $d\mu$ would be omitted if the context is clear.
In addition, if $A\in\mc{M}$, we define $\int_A\phi$ to be $\int \phi\chi_A$.

Some basic, and seemingly obvious, statements regarding integrations of simple functions now follow.
\begin{prop}
    Let $\phi$ and $\varphi$ be simple functions in $L^+$.
    \begin{enumerate}\label{properties for simple L+}
        \item[(a)]
        {
            If $c\geq 0$, then $\int c\phi=c\int\phi$.
        }
        \item[(b)]
        {
            $\int(\phi+\varphi)=\int\phi+\int\varphi$.
        }
        \item[(c)]
        {
            If $\phi\leq\varphi$, then $\int\phi\leq\int\varphi$.
        }
        \item[(d)]
        {
            The map $A\mapsto\int_A\phi$ defined for all $A\in\mc{M}$ is a measure on $\mc{M}$.
        }
    \end{enumerate}
\end{prop}
\begin{proof}
    Straightforward.
\end{proof}

\subsection{Integration of nonnegative measurable functions}

Given a function $f\in L^+$, we define the integral of $f$ by
\begin{align*}
    \int f:=\sup\left\{
        \int s: \textsf{$s$ is simple and $0\leq s\leq f$}
    \right\}.
\end{align*}

\begin{thm}[Monotone convergence theorem]
    Suppose $(f_n)_{n\in\bb{N}}\subset L^+$ is monotonically increasing.
    If $f:=\lim_n f_n(=\sup_n f_n)$, then $\int f=\lim_n\int f_n$.
\end{thm}
\begin{proof}
    Remark that $f=\sup_{n\in\bb{N}} f_n\in L^+$ and $\int f\geq\lim_n\int f_n$.
    To prove the equality, fix a constant $0<\rho<1$ and set $E_n:=\{x\in X: f_n(x)\geq\rho f(x)\}$ for each $n\in\bb{N}$.
    Note that the sequence $\{E_n\}_{n\in\bb{N}}$ is increasing and $\bigcup_{n\in\bb{N}} E_n=X$. \color{brown}(Why?) \color{black}
    We also have the following inequality:
    \begin{align*}
        \int f_n\geq\int_{E_n}f_n\geq\rho\int_{E_n}f,
        \quad\textsf{so}\quad
        \lim_n\int f_n\geq\rho\lim_n\int_{E_n}f\geq\rho\lim_n\int_{E_n}s,
    \end{align*}
    where $s$ is a simple function in $L^+$ such that $0\leq s\leq f$.
    By (d) of \cref{properties for simple L+}, we have $\lim_n\int_{E_n}s=\int s$, thus $\lim_n\int f_n\geq\rho\int s$ and $\lim_n\int f_n\geq\rho\int f$.
    This proves the theorem.
\end{proof}
\begin{rmk}
    In the last section, we studied that for a (real or complex-valued) measurable function $f$ on $X$, there is a monotonically increaisng sequence of (simple) functions in $L^+$ which converges to $f$ pointwise on $X$ (and uniformly on any set on which $f$ is bounded).
    Thus, we may freely use the monotone convergence theorem for functions in $L^+$.
\end{rmk}

Using the monotone convergence theorem, we can prove that the properties in \cref{properties for simple L+} holds accordingly for nonsimple functions in $L^+$.
Proving them is left as an exercise.

We introduce some basic but helpful properties.
Justitifications are left as exercises.
\begin{exmp}
    \begin{enumerate}
    {
        \item[(a)]
        {
            Suppose $f\in L^+$.
            Then $\int f=0$ if and only if $f=0$ $\mu$-almost everywhere.
        }
        \item[(b)]
        {
            Suppose $(f_n)_{n\in\bb{N}}\subset L^+$ and $f\in L^+$.
            If $f_n(x)$ increases to $f(x)$ for $\mu$-almost every $x\in X$, then $\int f=\lim_{n\rightarrow\infty}\int f_n$.
        }
    }
    \end{enumerate}
\end{exmp}
\begin{prop}[Fatou's lemma]
    If $(f_n)_{n\in\bb{N}}$ is any sequence in $L^+$, then
    \begin{align*}
        \int\left(\liminf_{n\rightarrow\infty} f_n\right)\leq\liminf_{n\rightarrow\infty}\int f_n.
    \end{align*}
\end{prop}
\begin{exmp}
    \begin{enumerate}
        \item[(a)]
        {
            If $(f_n)_n$ is any sequence in $L^+$, $f\in L^+$, and $f_n\rightarrow f$ $\mu$-almost everywhere, then $\int f\leq\liminf_n\int f_n$.
        }
        \item[(b)]
        {
            If $f\in L^+$ and $\int f<\infty$, then $f^{-1}(\{\infty\})$ is $\mu$-null and $f^{-1}([0, \infty))$ is a $\sigma$-finite for $\mu$.
        }
    \end{enumerate}
\end{exmp}

\subsection*{Problems}

\begin{prob}
    Let $(X, \mc{M}, \mu)$ be a measure space.
    Suppose that $(f_n)_{n\in\bb{N}}$ is a sequence in $L^+$ and $f_n\rightarrow f$ pointwise, and $\int f=\lim\int f_n<\infty$.
    Show that $\int_E f=\lim\int_E f_n$ whenever $E\in\mc{M}$.
    Explain why this equation may fail if $\int f=\lim\int f_n=\infty$.
\end{prob}
\begin{sol}
    Whenever $E\in\mc{M}$, by Fatou's lemma, we have
    \begin{align*}
        \int_E f\leq\liminf\int_E f_n
        \quad\textsf{and}\quad
        \int_{X\setminus E}f\leq\liminf\int_{X\setminus E} f_n
    \end{align*}
    so $\int f\leq\liminf\int_E f_n+\liminf\int_{X\setminus E} f_n=\liminf \int f_n$ and $\int_E f=\lim\int_E f_n$.

    Let $f_n=\chi_{(0, \infty)}+n^2\chi_{(-1/n, 0)}$ for each $n\in\bb{N}$.
    Then $\int_{(-1, 0)} f=0$ but $\lim\int_{(-1, 0)} f_n=\infty$.
\end{sol}

\color{orange}
\begin{prob}
    Let $(X, \mc{M}, \mu)$ be a measure space and suppose $f\in L^+$, and define $\lambda: \mc{M}\rightarrow [0, \infty]$ by $\lambda(E)=\int_E f\,d\mu$ for all $E\in\mc{M}$.
    \begin{enumerate}
        \item[(a)]
        {
            Show that $\lambda$ is a measure on $\mc{M}$.
        }
        \item[(b)]
        {
            Prove that $\int f\,d\lambda=\int fg\,d\mu$ for all $g\in L^+$.
        }
    \end{enumerate}
\end{prob}
\color{black}