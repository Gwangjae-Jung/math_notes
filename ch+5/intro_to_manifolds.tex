\section{Introduction to manifolds}

\begin{defi}
    \begin{enumerate}
        \item[(a)]
        {
            (Locally Euclidean space)
            A topological space $X$ is called locally Euclidean if there is a non-negative integer $n$ such that every point in $X$ has a neighborhood which is homeomorphic to an open subset of $\bb{R}^n$.
        }
        \item[(b)]
        {
            ($m$-manifold)
            A second countable Hausdorff space in which every point has a neighborhood which is homeomorphic to an open subset of $\bb{R}^m$ is called an $m$-manifold.
        }
        \item[(c)]
        {
            (Support of a real-valued function)
            Given a real valued function $\phi: X\rightarrow\bb{R}$, the support of $\phi$ is defined to be the closure of $\phi^{-1}(\bb{R}\setminus\{0\})$ in $X$.
        }
    \end{enumerate}
\end{defi}
\begin{rmk}[Chart, atlas]
    In theory of differential manifold, a second countable Hausdorff space which is a locally Euclidean space is called a manifold.
    For this definition, we first observe the notions which follow, after assuming that $X$ is a topological space.
    \begin{enumerate}
        \item[(a)]
        {
            (Chart)
            A pair $(U, \varphi)$ of an open subset $U$ of $X$ and a homeomorphism $\varphi$ from $U$ onto an open subset of a Euclidean space is called a chart.\footnote{A chart is also called a coordinate chart, coordinate patch, coordinate map}
        }
        \item[(b)]
        {
            (Atlas)
            An indexed family $\{(U_\alpha, \varphi_\alpha)\}_{\alpha\in I}$ is called an atlas for $X$ if it covers $X$.
        }
    \end{enumerate}
    One should remark that a topological space $X$ is a locally Euclidean space if and only if $X$ admits an atlas of which the codomain of each chart is $\bb{R}^n$ for some non-negative integer $n$.
\end{rmk}
\begin{rmk}[The separability of manifolds]
    We wish to call, for example, a 1-manifold a curve, a 2-manifold a surface, and so on.
    Consider the line with two origins, which is second countable, locally Euclidean, but not a Hausdorff space.
    Drawing the line with two origins, nobody would wish to call it a curve, so the condition of being a Hausdorff space is essential in defining manifolds.
\end{rmk}

\begin{prop}
    Every manifold is completely regular, hence, metrizable.
\end{prop}
\begin{proof}
    Let $M$ be an $m$-manifold for some positive integer $m$.
    It suffices to show that $M$ is locally compact; then $M$ is a locally compact Hausdorff space, a completely regular space, and the metrizability is immediate from the Urysohn metrization theorem.
    Let $x$ be a point of $M$ and $A$ be a neighborhood of $x$ in $M$ which embeds into an open subset $W$ of $\bb{R}^m$; let $f: A\rightarrow W$ be such homeomorphism.
    Because $\bb{R}^m$ is a locally compact Hausdorff space, there is a relatively compact neighborhood $V$ of $f(x)$ in $\bb{R}^m$ whose closure in $\bb{R}^m$ is contained in $W$.
    Then $f^{-1}(V)$ is open in $A$, hence, in $M$.
    Because $f^{-1}(\ol{V})$ is a compact subspace of $A$, it is a compact subspace of $M$ which contains a neighborhood of $x$, namely, $f^{-1}(V)$.
    This proves that $M$ is locally compact.
\end{proof}

In dealing with the support of a function $f: X\rightarrow \bb{R}$, because the support $K$ of $f$ is defined to be closed in $X$, the argument that $X\setminus K$ will be frequently used.
In particular, whenever $x$ is a point in $X\setminus K$, there is a neighborhood $U$ of $x$ on which $f$ vanishes.

In the beginning of this section, we proved that an $m$-manifold is completely regular, hence metrizable.
In fact, because an $m$-manifold $X$ is second countable and completely regular, $X$ embeds into $\bb{R}^\bb{N}$ or $[0, 1]^\bb{N}$, which is infinite dimensional.
We wish to reduce the dimension to a finite number, which is possible when $X$ is a compact manifold.
\begin{thm}
    If $M$ is a compact $m$-manifold with a positive integer $m$, then $X$ embeds into $\bb{R}^N$ for some positive integer $N$.
\end{thm}
\begin{proof}
    Because $M$ is a compact $m$-manifold, there is a finite open cover $\{U_1, \cdots, U_n\}$ of $X$ such that $U_i$ is homeomorphic to an open subset of $\bb{R}^m$ for $i=1, \cdots, n$; let $g_i: U_i\hookrightarrow\bb{R}^m$ denote such homeomorphism for $i=1, \cdots, n$.
    Remark that $M$ is a compact Hausdorff space, so $M$ is normal.
    Hence, there is a partition $\{\phi_1, \cdots, \phi_n\}$ of unity dominated by $\{U_1, \cdots, U_n\}$.
    Letting $A_i$ be the support of $\phi_i$ for $i=1, \cdots, n$, define a map $h_i: X\rightarrow\bb{R}^m$ for each $i=1, \cdots, n$ by
    \begin{align*}
        h_i(x)=\left\{
        \begin{array}{cc}
            \phi_i(x)\cdot g_i(x)   &   \textsf{for $x\in U_i$}\\
                    0               &   \textsf{for $x\in X\setminus A_i$}
        \end{array}\right..
    \end{align*}
    Remark that $h_i$ is well-defined and continuous, for ${h_i}|_{U_i}$ and ${h_i}|_{X\setminus A_i}$ are continuous.

    Define a map $F: X\rightarrow\bb{R}^{(m+1)n}$ by
    \begin{align*}
        F(x)=(\phi_1(x), \cdots, \phi_n(x), h_1(x), \cdots, h_n(x)).
    \end{align*}
    It is clear that $F$ is continuous, so it remains to show that $F$ is injective to show $F$ is an embedding; because $X$ is compact and the codomain is a Hausdorff space, $F$ is a closed map.
    Suppose $F(x)=F(y)$ for some points $x$ and $y$ in $X$.
    Then $\phi_i(x)=\phi_i(y)$ for $i=1, \cdots, n$, so $\phi_i(x)=\phi_i(y)$ for some $i$ and $x, y\in U_i$.
    Then $\phi_i(x)\cdot g_i(x)=h_i(x)=h_i(y)=\phi_i(y)\cdot g_i(y)$, so $g_i(x)=g_i(y)$.
    Because $g_i$ is injective, it follows that $x=y$, as desired.
\end{proof}