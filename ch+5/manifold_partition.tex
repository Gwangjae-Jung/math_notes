\section{Manifolds and partition of unity}

\begin{defi}
    \begin{enumerate}
        \item[(a)]
        {
            (Locally Euclidean space)
            A topological space $X$ is called locally Euclidean if there is a non-negative integer $n$ such that every point in $X$ has a neighborhood which is homeomorphic to an open subset of $\bb{R}^n$.
        }
        \item[(b)]
        {
            ($m$-manifold)
            A second countable Hausdorff space in which every point has a neighborhood which is homeomorphic to an open subset of $\bb{R}^m$ is called an $m$-manifold.
        }
        \item[(c)]
        {
            (Support of a real-valued function)
            Given a real valued function $\phi: X\rightarrow\bb{R}$, the support of $\phi$ is defined to be the closure of $\phi^{-1}(\bb{R}\setminus\{0\})$ in $X$.
        }
    \end{enumerate}
\end{defi}
\begin{rmk}[Chart, atlas]
    In theory of differential manifold, a second countable Hausdorff space which is a locally Euclidean space is called a manifold.
    For this definition, we first observe the notions which follow, after assuming that $X$ is a topological space.
    \begin{enumerate}
        \item[(a)]
        {
            (Chart)
            A pair $(U, \varphi)$ of an open subset $U$ of $X$ and a homeomorphism $\varphi$ from $U$ onto an open subset of a Euclidean space is called a chart.\footnote{A chart is also called a coordinate chart, coordinate patch, coordinate map}
        }
        \item[(b)]
        {
            (Atlas)
            An indexed family $\{(U_\alpha, \varphi_\alpha)\}_{\alpha\in I}$ is called an atlas for $X$ if it covers $X$.
        }
    \end{enumerate}
    One should remark that a topological space $X$ is a locally Euclidean space if and only if $X$ admits an atlas of which the codomain of each chart is $\bb{R}^n$ for some non-negative integer $n$.
\end{rmk}
\begin{rmk}[The separability of manifolds]
    We wish to call, for example, a 1-manifold a curve, a 2-manifold a surface, and so on.
    Consider the line with two origins, which is second countable, locally Euclidean, but not a Hausdorff space.
    Drawing the line with two origins, nobody would wish to call it a curve, so the condition of being a Hausdorff space is essential in defining manifolds.
\end{rmk}

In dealing with the support of a function $f: X\rightarrow \bb{R}$, because the support $K$ of $f$ is defined to be closed in $X$, the argument that $X\setminus K$ will be frequently used.
In particular, whenever $x$ is a point in $X\setminus K$, there is a neighborhood $U$ of $x$ on which $f$ vanishes.

\begin{defi}[Partition of unity]
    Let $\{U_1, \cdots, U_n\}$ be a finite open covering of a topological space $X$.
    The collection of continuous functions
    \begin{center}
        $\phi_i: X\rightarrow [0, 1]$ for $i=1, \cdots, n$
    \end{center}
    is called a partition of unity dominated by $\{U_1, \cdots, U_n\}$ if both of the following conditions are satisfied:
    \begin{enumerate}
        \item[(\romannumeral 1)]
        {
            The support of each $\phi_i$ is contained in $U_i$ for $i=1, \cdots, n$.
        }
        \item[(\romannumeral 2)]
        {
            $\sum_{i=1}^n \phi_i(x)=1$ for all $x\in X$.\footnote{This condition can be alleviated when $\Phi:=\sum_{i=1}^n \phi_i>0$, since we may consider $\phi_i/\Phi$ in place of $\phi_i$.}
        }
    \end{enumerate}
\end{defi}
\begin{rmk}
    Some assumptions in the above definition is due to technical problems.
    \begin{enumerate}
        \item[(a)]
        {
            The condition that $\{U_1, \cdots, U_n\}$ should cover $X$ is obviously essential for defining a partition of unity.
            Otherwise, for any point $x\in X\setminus\bigcup_{i=1}^n U_i$, we have $\phi_i(x)=0$ for $i=1, \cdots, n$.
        }
        \item[(b)]
        {
            The openness of each $U_i$ is essential; otherwise, for example, when $U_1=X=[0, 1]$, it is allowed by assumption that the support of $\phi_1$ to be $X$, such as $\phi_1=\id{X}$, and this setting is problematic at the point 0 in $X$.
            Moreover, when the space $X$ is assumed to be normal; in this case, there is a neighborhood of the support of $\phi_i$ in $X$ which is contained in $U_i$, or one may apply the Urysohn lemma for the support of $\phi_i$ and $X\setminus U_i$.
        }
    \end{enumerate}
\end{rmk}

\begin{thm}[Existence of finite partitions of unity]
    Let $\{U_1, \cdots, U_n\}$ be a finite open covering of the normal space $X$.
    Then there is a partition of unity dominated by $\{U_1, \cdots, U_n\}$.
\end{thm}
\begin{proof}
    The main idea of the proof is to find an open cover $\{V_1, \cdots, V_n\}$ such that $\ol{V_i}\subset U_i$ for $i=1, \cdots, n$, where, as usual, the overline is used to denote the closure in $X$.
    If this is proved, we may make use of another open over $\{W_1, \cdots, W_n\}$ such that $\ol{W_i}\subset V_i$ for $i=1, \cdots, n$.
    Then, for each $i=1, \cdots, n$, by the Urysohn lemma, there is a continuous map $\phi_i: X\rightarrow [0, 1]$ such that $\phi_i(\ol{W_i})=\{1\}$ and $\phi_i(X\setminus V_i)=\{0\}$; in this case, the support of $\phi_i$ is contained in the closure of $V_i$, which is contained in $U_i$, by construction.
    Because $\{W_1, \cdots, W_n\}$ covers $X$, the function $\Phi: X\rightarrow[0, \infty)$ defined by $\Phi=\sum_{i=1}^n \phi_i$ is positive, which proves the existence of finite partitions of unity.

    To complete the proof, it remains to justify the shrinking process can be done.
    The idea is to observe the portion of $X$ which can be covered by only one member of $\{U_1, \cdots, U_n\}$.
    Let $A=X\setminus(U_2\cup\cdots\cup U_n)$.
    Then $A_1$ is a closed subset of $X$ which is contained in $U_1$.
    Hence, by normality, there is a neighborhood $V_1$ of $A_1$ whose closure in $X$ is contained in $U_1$, and we have $X=V_1\cup U_2\cup\cdots\cup U_n$.
    In general, for an open cover $\{V_1, \cdots, V_{k-1}, U_k, U_{k+1}, \cdots, U_n\}$, define $A_k=X\setminus(V_1\cup\cdots\cup V_{k-1})\setminus(U_{k+1}\cup\cdots\cup U_n)$.
    Then $A_k$ is a closed subset of $X$ contained in $U_k$.
    Hence, by normality, there is a neighborhood $V_k$ of $A_k$ whose closure in $X$ is contained in $U_k$, and we have $X=V_1\cup\cdots V_k\cup U_{k+1}\cup\cdots U_n$.
    By induction, the shrinking process is justified.
\end{proof}