\section{The embedding theorem}

In this section, we introduce the embedding theorem, stating that every completely regular space embeds into a (possibly uncountable dimensional) Euclidean space $\bb{R}^I$ for some index set $I$.
As an application of the embedding theorem, we will prove the Urysohn metrization theorem, which states that every second-countable regular space is metrizable.
When proving the Urysohn metrization theorem, the fact that a countable dimensional Euclidean space $\bb{R}^\bb{N}$ is metrizable will be applied.

\begin{thm}[The embedding theorem]
    Let $X$ be a completely regular space and let $\left\{f_\alpha\right\}_{\alpha\in I}$ be a collection of continuous functions from $X$ into $\bb{R}$ separating points and closed subsets of $X$.
    Define the function $F: X\rightarrow\bb{R}^I$ by $F=(f_\alpha)_{\alpha\in I}$.
    Then $F$ denotes an embedding of $X$ into $\bb{R}^I$.
    Furthermore, if each $f_\alpha$ maps $X$ into $[0, 1]$, then $F$ denotes an embedding of $X$ into $[0, 1]^I$.
\end{thm}
\begin{proof}
    Because $\bb{R}^I$ (or $[0, 1]^I$) equips the product topology, $F$ is continuous.
    If $a$ and $b$ are distinct points of $X$, there is an index $i\in I$ such that $f_i(a)\neq 0$ and $f_i(b)=0$, so $F$ is injective.
    Thus, to show that $F$ is an embedding of $X$, it suffices to show that $F$ maps an open subset of $X$ onto an open subset of $F(X)$.
    
    Assume $U$ is an open subset of $X$, and let $z$ be a point of $F(U)$, and let $x$ be the point of $U$ such that $F(x)=z$.
    The goal is to find a neighborhood of $z$ in $F(X)$ which is contained in $F(U)$.
    Let $i\in I$ be an index such that $f_i(x)\neq 0$ and $f_i(X\setminus U)=\{0\}$, and set $W=\pi_i^{-1}(\bb{R}\setminus\{0\})\cap F(X)$.
    \begin{enumerate}
        \item[(\romannumeral 1)]
        {
            Clearly, $W$ is open in $F(X)$ and $W$ contains $z$.
        }
        \item[(\romannumeral 2)]
        {
            For a point $y\in W$, let $a$ be a point in $X$ such that $F(a)=y$.
            Since $y\in\pi_i^{-1}(\bb{R}\setminus\{0\})$, we have $f_i(a)=\pi_i(y)\neq 0$, so $a\notin X\setminus U$, implying $y\in F(U)$.
            In other words, $W\subset F(U)$.
        }
    \end{enumerate}
    So $F(U)$ is open in $F(X)$, which proves that $F$ is an embedding of $X$ into $\bb{R}^I$ (or into $[0, 1]^I)$.
\end{proof}

We now introduce the Urysohn metrization theorem, whose result is stronger than the result of the Urysohn lemma.
\begin{thm}[Urysohn metrization theorem]\label{Urysohn met thm}
    Every second countable regular space is metrizable.
\end{thm}
To prove this theorem, we first prove the following countability lemma.
\begin{lem}\label{countable separating family}
    Let $X$ be a second countable regular space.
    Then there is a countable family of continuous functions from $X$ into $[0, 1]$ separating points of $X$ from closed subsets of $X$.
\end{lem}
\begin{proof}[Proof of \cref{countable separating family}]
    Let $\{B_n\}_{n\in\bb{N}}$ be a countable basis of the topology on $X$.
    For each pair $(m, n)$ of positive integers for which $\ol{B_m}\subset B_n$, by the Urysohn lemma, there is a continuous function $f_{m, n}$ from $X$ into $[0, 1]$ such that $f(\ol{B_m})=\{1\}$ and $f(X\setminus B_n)=\{0\}$.
    The collection of such continuous function is countable, and it separates points of $X$ from closed subsets of $X$. (Remark how we obtained alternative definitions of some separabilities of topological spaces.)
\end{proof}
\begin{proof}[Proof of \cref{Urysohn met thm}]
    By \cref{countable separating family} and the embedding theorem, $X$ embeds into $\bb{R}^\bb{N}$, which is metrizable.
\end{proof}

\begin{exmp}
    Suppose $X$ is a compact Hausdorff space.
    If $X$ is metrizable, one can easily deduce that $X$ is second countable, using the compactness of $X$.
    The converse implication that $X$ is metrizable when $X$ is second countable is an immediate result of the Urysohn metrization theorem, because a compact Hausdorff space is normal.
\end{exmp}

In the following chapter, the embedding theorem is applied when studying the Stone-\v{C}ech compactification of a completely regular space, to which every continuous function into a compact Hausdorff space extends continuously (and uniquely\footnote{The uniqueness follows from the assumption that the codomain is a Hausdorff space.}).