\section{The embedding theorem}

In this section, we introduce the embedding theorem, stating that every completely regular space can be embedded into a (possibly uncountable dimensional) Euclidean space $\bb{R}^I$ for some $I$.
As an application of the embedding theorem, we will prove Urysohn metrization theorem, stating that every second-countable regular space is metrizable, using the fact that a countable dimensional Euclidean space $\bb{R}^\bb{N}$ is metrizable.

\begin{thm}[The embedding theorem]
    Let $X$ be a completely regular space and let $\left\{f_\alpha\right\}_{\alpha\in I}$ be a collection of continuous functions from $X$ into $\bb{R}$ separating points and closed subspaces of $X$.
    Define the function $F: X\rightarrow\bb{R}^I$ by $F=(f_\alpha)_{\alpha\in I}$.
    Then $F$ denotes an embedding of $X$ into $\bb{R}^I$.
    Furthermore, if each $f_\alpha$ maps $X$ into $[0, 1]$, then $F$ denotes an embedding of $X$ into $[0, 1]^I$.
\end{thm}
\begin{proof}
    It is clear that $F$ is continuous, since $\bb{R}^I$ (or $[0, 1]^I$) equips the product topology.
    If $a$ and $b$ are disjoint points of $X$, there are indices $i, j\in I$ such that $f_i(a)=1,\, f_i(b)=0$ and $f_j(a)=0,\, f_j(b)=1$, hence $F$ is injective.
    Therefore, to show that $F$ is an embedding of $X$, it remains to show that $F$ is a homeomorphism from $X$ into its image $F(X)$.
    For this, we need to show that $F$ maps an open set in $X$ onto an open set in $F(X)$.
    
    Suppose $U$ is an open subset of $X$, and let $z$ be a point of $F(U)$, and let $x$ be a point of $U$ such that $F(x)=z$.
    We seek to find a neighborhood of $z$ in $F(X)$ contained in $F(U)$.
    Let $i\in I$ be an index such that $f_i(x)>0$ and $f_i(X\setminus U)=\{0\}$, and set $W=\pi_i^{-1}((0, \infty))\cap F(X)$.
    \begin{enumerate}
        \item[(a)] Clearly, $W$ is open in $F(X)$ and $W$ contains $z$.
        \item[(b)] If $y\in W$, there is a point $a\in X$ such that $F(a)=y$.
        Since $y\in\pi_i^{-1}((0, \infty))$, we have $f_i(a)=\pi_i(y)>0$, so $a\notin X\setminus U$, implying $y\in F(U)$.
        Thus, $W$ is a neighborhood of $z$ in $F(X)$ which is contained in $F(U)$.
    \end{enumerate}
    Therefore, $F$ denotes an embedding of $X$ into $\bb{R}^I$ (or into $[0, 1]^I)$.
\end{proof}

Here is Urysohn metrization theorem.
\begin{thm}[Urysohn metrization theorem]\label{Urysohn met thm}
    Every second countable regular space is metrizable.
\end{thm}

To prove this theorem, we first prove the following countability lemma.
\begin{lem}\label{countable separating family}
    Let $X$ be a second countable regular space.
    Then there is a countable family of continuous functions from $X$ into $[0, 1]$ separating points of $X$ from closed subsets of $X$.
\end{lem}
\begin{proof}[Proof of \cref{countable separating family}]
    Let $\{B_n\}$ be a countable basis of the topology on $X$.
    Whenever possible, let $f_{m, n}$ be a continuous function from $X$ into $[0, 1]$ such that $f(\overline{B_m})=\{1\}$ and $f(X\setminus B_n)=\{0\}$, where $\overline{B_m}\subset B_n$ (this could be done by applying Urysohn lemma).
    The collection of such continuous function will separate points of $X$ from closed subsets of $X$, and the collection is countable.
\end{proof}

It is time to prove Urysohn metrization theorem.
\begin{proof}[Proof of \cref{Urysohn met thm}]
    By the embedding theorem, $X$ can be embedded into $\bb{R}^\bb{N}$.
    Since $\bb{R}^\bb{N}$ is a countable product of metrizable spaces, $\bb{R}^\bb{N}$ is metrizable, and its subspaces are also metrizable.
\end{proof}

In the following chapter, the embedding theorem is applied when studying the Stone-\v{C}ech compactification of a completely regular space, to which evert continuous function into a compact Hausdorff space extends continuously (and uniquely\footnote{The uniqueness follows from the assumption that the codomain is a Hausdorff space.}).