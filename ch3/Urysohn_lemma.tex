\section{The Urysohn lemma}
\begin{thm}[Urysohn lemma]\label{Urysohn lemma}
    If $X$ is a normal space, then any two (nonempty) disjoint closed subspaces of $X$ can be separated by a continuous function on $X$.
    To be precise, if $A$ and $B$ are (nonempty) disjoint closed subspaces of $X$, there is a continuous function $f: X\rightarrow[0, 1]$ such that $f(A)=\{0\}$ and $f(B)=\{1\}$.
\end{thm}
\begin{rmk}
    The Urysohn lemma implies that if every pair of closed sets in $X$ can be separated by disjoint open sets, i.e., $X$ is normal, then each such pair can be separated by a continuous function.
    The converse is obvious. \color{brown}(Why?)\color{black}
\end{rmk}
\begin{proof}
    Let $A$ and $B$ be nonempty and disjoint closed subsets of $X$.

    \textbf{Step 1.}
    Let $P$ be the set of rational numbers in $[0, 1]$ and write $P=\{x_1, x_2, x_3, \cdots\}$ with $x_1=1, x_2=0$.
    Now define an open subset $U_p$ for $p\in P$ as follows:
    \begin{enumerate}
        \item[(1)]
        {
            Fitst, let $U_1:=X\setminus B$, which contains $A$.
            Second, by normality, we can find a neighborhood $U_0$ of $A$ in $X$ whose closure is in $U_1$.
        }
        \item[(2)]
        {
            Suppose we have constructed $U_p$ for $p\in P_n$, where $P_n=\{x_1, \cdots, x_n\} $ for $n\geq 3$ such that $p, q\in P_n$ with $p<q$ implies
            \begin{align}\label{rule}
                \overline{U_p}\subset U_q.
            \end{align}
            For convinience, denote $r=p_{n+1}$.
            We want $U_r$ to satisfy \cref{rule} with all $U_p$ with indices in $P_{n+1}$.
            For this, it suffices to care the immediate successor $s$ and predecessor $p$ of $r$ in $P_{n+1}$; if $U_r$ satisfies \cref{rule} with $U_p$ and $U_s$, then $U_r$ satisfies \cref{rule} with all the other $U_p$ with $p\in P_{n+1}$.

            Since $U_p$ and $U_s$ satisfies $\overline{U_p}\subset U_s$, by normality, we can find a neighborhoof $U_r$ of the closed subspace $\overline{U_p}$ in $X$ such that $\overline{U_r}\subset U_s$.
            For such $U_r$, \cref{rule} is satisfied for every pair of $U_p$ with $p\in P_{n+1}$.
        }
    \end{enumerate}
    By induction, we have $U_p$ defined for all $p\in P$.

    \textbf{Step 2.}
    We now define $U_p$ for all rational numbers $p$ as follows: just let $U_p=\varnothing$ whenever $p<0$ and $U_p=X$ whenever $p>1$.
    For such $U_p$ with $p\in\bb{Q}$, \cref{rule} is satisfied.
    
    \textbf{Step 3.}
    For each point $x\in X$, define the set
    \begin{align*}
        C(x):=\{p\in\bb{Q}:x\in U_p\}.
    \end{align*}
    For each $x$, the set $C(x)$ is bounded (below by a nonnegative real number, and above by 1).
    Thus, we may define the function $f: X\rightarrow[0, 1]$ by $f(x)=\inf C(x)$ for $x\in X$.

    \textbf{Step 4.}
    We now show that the function $f$ is a continuous map mapping $A$ onto $\{0\}$ and $B$ onto $\{1\}$.
    First, since every point $a\in A$ belongs to $U_0$, $f(a)=0$; since every point $b\in B$ does not belong to $U_1$, $f(b)=1$.
    Before proving continuity, we note the following lemma:
    \begin{center}
        $x\in\overline{U_r}$ implies $f(x)\leq r$, and $x\notin U_r$ implies $f(x)\geq r$.
    \end{center}
    To check the continuity, fix a point $x\in X$ and let $(a, b)$ be a neighborhood of $f(x)$ in $\bb{R}$.
    We want to find a neighborhood $W$ of $x$ in $X$ whose image under $f$ is contained in $(a, b)$.\footnote{When trying to prove continuuity with the original definition of continuity, you might encounter some technical problem.}
    Using the density of $\bb{Q}$, we can find rational numbers $c, d$ such that $a<c<f(x)<d<b$.
    If we can find a neighborhood of $x$ in $X$ whose image under $f$ is a subset of $[c, d]$, the proof ends.
    By the above lemma, if $c<f(x)<d$, then $x\in U_d\setminus\overline{U_c}$.
    Hence, $W:=U_d\setminus\overline{U_c}$ is a neighborhood of $x$ in $X$.
    Furthermore, if $y\in W$, then $c\leq f(y)\leq d$, hence $W$ is a neighborhood of $x$ in $X$ with the image in $(a, b)$.
    Therefore, $f$ is a continuous function from $X$ into $[0, 1]$ mapping $A$ into $\{0\}$ and $B$ onto $\{1\}$.
\end{proof}
\begin{prob}
    Prove the lemma introduced in the fourth step of the proof of \cref{Urysohn lemma}.
\end{prob}
\begin{sol}
    Suppose first that $x\in\overline{U_r}$.
    Whenever $r<p$, since $x\in\overline{U_r}\subset U_p$, $C(x)$ contains $p\in\bb{Q}$ with $p>r$; $f(x)=\inf C(x)\leq r$.
    Conversely, if $x\notin U_r$, whenever $p<r$, since $\overline{U_p}\subset U_r$, $p\notin C(x)$ for all $p<r$ and it implies that $f(x)=\inf C(x)\geq r$.
\end{sol}

The following problem deals with a property of completely regular spaces, and its solution does not require the Urysohn lemma.
\begin{prob}
    Let $X$ be a completely regular space and $A, B$ be disjoint closed subspaces of $X$.
    Show that if $A$ is compact, then there is a continuous function $f: X\rightarrow[0, 1]$ such that $f(A)=\{0\}$ and $f(B)=\{1\}$.
\end{prob}
\begin{sol}
    For each point $a\in A$, let $f_a: X\rightarrow[0, 1]$ be a continuous function such that $f_a(a)=0$ and $f_a(B)=\{1\}$.
    Note that the collection $\{f_a^{-1}([0, r)):a\in A\}$ is an open cover of $A$ by sets open in $X$, where $r$ is a real number such that $0<r<1/2$.
    Thus, $A$ can be covered by finitely many sets $f_{a_i}^{-1}([0, r))$ for $i=1, \cdots, n$.
    Define the function $f: X\rightarrow[0, 1]$ by $f(x)=f_1(x)\times\cdots\times f_n(x)$ for $x\in X$.
    Then $0\leq f(a)<r$ for all $a\in A$ and $f(B)=\{1\}$.
    If $g:[0, 1]\rightarrow[0, 1]$ is the function defined by
    \begin{eqnarray*}
        g(x)=\left\{
        \begin{matrix}
            0 & \textsf{if }0\leq x<r\\
            \dfrac{x-r}{1-r} & \textsf{otherwise}
        \end{matrix}\right.,
    \end{eqnarray*}
    then the composite $g\circ f$ is a desired continuous map from $X$ into $[0, 1]$.
\end{sol}

In \cref{Particular uncountable space - connected "metrizable" spaces}, we have proved that every connected metrizable space having more than one point is uncountable.
As we have studied in the preceeding section that every metrizable space is normal, one might want to generalize the result to a broader category of spaces.
In the following problem, we will prove that every connected regular space with more than one point is uncountable.
\begin{prob}[Nontrivial connected regular spaces are uncountable]\label{nontrivial connected regular spaces are uncountable}
    Prove that a connected normal space having more than one point is uncountable.
    And deduce that a connected regular space having more than one point is uncountable.
\end{prob}
\begin{sol}
    Suppose $X$ is a connected normal space having more than one point.
    Since every singletone is closed, there is a continuous map $f: X\rightarrow[0, 1]$ such that $f(a)=0$ and $f(b)=1$, where $a, b$ are disjoint points of $X$.
    Since $X$ is connected, $f(X)$ is connected, so $f(X)=[0, 1]$.
    Therefore, $X$ is necessarily uncountable.
            
    Assume there is a connected regular space $X$ with more than one point which is countable.
    It is clear that $X$ is Lindel\"{o}f, so $X$ is a Lindel\"{o}f regular space; a normal space.
    It contradicts to the former result.
\end{sol}