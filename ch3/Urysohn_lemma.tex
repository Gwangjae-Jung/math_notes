\section{The Urysohn lemma}
\begin{thm}[Urysohn lemma]\label{Urysohn lemma}
    If $X$ is a normal space, then any two nonempty disjoint closed subsets of $X$ can be separated by a continuous function on $X$.
    To be precise, if $A$ and $B$ are nonempty disjoint closed subsets of $X$, there is a continuous function $f: X\rightarrow[0, 1]$ such that $f(A)=\{0\}$ and $f(B)=\{1\}$.
\end{thm}
\begin{rmk}
    By the Urysohn lemma, every pair of closed subsets of $X$ can be separated by disjoint open subsets of $X$ (in short, $X$ is normal) if and only if each such pair can be separated by a continuous function.
    (Here, if part is obvious.)
\end{rmk}
\begin{proof}
    Let $A$ and $B$ be nonempty and disjoint closed subsets of $X$.

    \textbf{Step 1.}
    Let $P$ be the set of rational numbers in $[0, 1]$ and write $P=\{x_1, x_2, x_3, \cdots\}$ with $x_1=1$ and $x_2=0$.
    Define an open subset $U_p$ for each $p\in P$ as follows:
    \begin{enumerate}
        \item[(1)]
        {
            First, let $U_1:=X\setminus B$, which contains $A$.
            Second, by normality, we can find a neighborhood $U_0$ of $A$ in $X$ whose closure in $X$ is contained in $U_1$.
        }
        \item[(2)]
        {
            Suppose, for some positive integer $n\geq 2$, that we have constructed $U_p$ for all $p\in P_n=\{x_1, \cdots, x_n\}$ satisfying the following rule, which is satisfied when $n=2$:
            \begin{align}\label{rule}
                \textsf{Whenever $p,\,q\in P_n$ and $p<q$, $\overline{U_p}\subset U_q$}.
            \end{align}
            Write $r=p_{n+1}$.
            We want $U_r$ to satisfy \cref{rule} with all $U_p$ with indices in $P_{n+1}$.
            In this case, it suffices to care the immediate successor $s$ and predecessor $p$ of $r$ in $P_{n+1}$; if $U_r$ satisfies \cref{rule} with $U_p$ and $U_s$, then $U_r$ satisfies \cref{rule} with all the other $U_p$ with $p\in P_{n+1}$.
            Since $U_p$ and $U_s$ satisfies $\overline{U_p}\subset U_s$, by normality, there is a neighborhoof $U_r$ of $\overline{U_p}$ in $X$ such that $\overline{U_r}\subset U_s$.
            For such $U_r$, \cref{rule} is satisfied for $P_{n+1}$.
        }
    \end{enumerate}
    By induction, $U_p$ can be defined for all $p\in P$ with \cref{rule} being satisfied.

    \textbf{Step 2.}
    We now define $U_p$ for all rational numbers $p$ as follows: Just let $U_p=\varnothing$ whenever $p<0$, and $U_p=X$ whenever $p>1$.
    Even in this case, \cref{rule} is satisfied.
    
    \textbf{Step 3.}
    For each point $x\in X$, define the set
    \begin{align*}
        C(x):=\{p\in\bb{Q}:x\in U_p\}.
    \end{align*}
    For each $x$, the set $C(x)$ is bounded below by a nonnegative real number and it contains 1.
    Thus, the function $f: X\rightarrow[0, 1]$ defined by $f(x)=\inf C(x)$ for all $x\in X$ is well-defined.

    \textbf{Step 4.}
    We now show that the function $f$ is a continuous map mapping $A$ onto $\{0\}$ and $B$ onto $\{1\}$.
    First, since every point $a\in A$ belongs to $U_0$, $f(a)=0$; since every point $b\in B$ does not belong to $U_1$, $f(b)=1$.
    Before proving continuity, we note the following lemma:
    \begin{align}\label{lemma in proving the Urysohn lemma}
        \textsf{$x\in\overline{U_r}$ implies $f(x)\leq r$, and $x\notin U_r$ implies $f(x)\geq r$.}
    \end{align}
    To check the continuity, fix a point $x\in X$ and let $(a, b)$ be a neighborhood of $f(x)$ in $\bb{R}$.
    We want to find a neighborhood $W$ of $x$ in $X$ whose image under $f$ is contained in $(a, b)$.\footnote{When trying to prove continuity with the original definition of continuity, you might encounter some technical problem.}
    Using the density of $\bb{Q}$, we can find rational numbers $c, d$ such that $a<c<f(x)<d<b$.
    Now, it suffices to find a neighborhood of $x$ in $X$ whose image under $f$ is a subset of $[c, d]$.
    By the above lemma, if $c<f(x)<d$, then $x\in U_d\setminus\overline{U_c}$.
    Hence, $U_d\setminus\overline{U_c}$ is a neighborhood of $x$ in $X$, and $f(U_d\setminus\ol{U_c})\subset[c, d]$.
    Therefore, $f$ is a continuous function from $X$ into $[0, 1]$ mapping $A$ into $\{0\}$ and $B$ onto $\{1\}$.
\end{proof}
\begin{proof}[Proof of \cref{lemma in proving the Urysohn lemma}]
    If $x\in \ol{U_r}$, then $x\in U_s$ (i.e., $s\in C(x)$) for all $s\in\bb{Q}$ satisfying $r<s$, so $f(x)\leq r$.
    If $x\notin U_r$, then $x\notin U_p$ (i.e., $p\notin C(x)$) for all $p\in\bb{Q}$ satisfying $p\leq r$, so $f(x)\geq r$.
\end{proof}

In \cref{Particular uncountable space - connected "metrizable" spaces}, we have proved that every connected metrizable space having more than one point is uncountable.
As we have studied in the preceeding section that every metrizable space is normal, one might want to generalize the result to a broader category of spaces.
In the following problem, we will prove that every connected regular space with more than one point is uncountable.
\begin{prob}[Nontrivial connected regular spaces are uncountable]\label{nontrivial connected regular spaces are uncountable}
    Prove that a connected normal space having more than one point is uncountable.
    And deduce that a connected regular space having more than one point is uncountable.
\end{prob}
\begin{sol}
    Assume $X$ is a connected normal space having more than one point.
    By the Urysohn lemma, there is a continuous map $f: X\rightarrow[0, 1]$ such that $f(A)=\{0\}$ and $f(B)=\{1\}$, where $A$ and $B$ are nonempty disjoint closed subsets of $X$.
    Since $X$ is connected, $f(X)$ is connected so $f(X)=[0, 1]$, implying that $X$ is uncountable.
            
    Suppose there is a connected regular space $X$ which is countable.
    It is obvious that $X$ is a Lindel\"of space, so $X$ is a Lindel\"of regular space; a normal space.
    It contradicts to the former result.
\end{sol}