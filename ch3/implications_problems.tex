\section{Countability and separation axioms (Problems)}

\begin{prob}
    Show that the suggested alternative definition for regular and normal spaces are equivalent to the original definitions.
\end{prob}
\begin{sol}
    Suppose $X$ is a regular space.
    We first show part (a).
    Given a point $a\in X$ with its neighborhood $U$ in $X$, consider the closed subspace $B:=X\setminus U$ of $X$.
    Using the regularity of $X$, one can find a neighborhood $V$ of $a$ in $X$ and a neighborhood $W$ of $B$ in $X$ which are disjoint.
    Then $\overline{V}\subset X\setminus W\subset X\setminus B=U$, proving the only if part.
    To show the if part, suppose whenever a point $a$ of $X$ with its neighborhood $U$ in $X$ is given, there is a neighborhood of $a$ in $X$ whose closure in $X$ is contained in $U$ exists.
    Given a point $x\in X$ and a closed subspace $B$ of $X$ not containing $x$, consider the neighborhood $U:=X\setminus B$ of $x$ in $X$.
    Let $V$ be a neighborhood of $x$ in $X$ whose closure is contained in $U$.
    Then $V$ and $X\setminus\overline{V}$ are disjoint open subsets of $X$ separating $x$ and $B$, proving the if part.

    Part (b) is almost clear, and part (c) follows if we replace points of $X$ with closed subspaces of $X$.
\end{sol}

\color{red}
\begin{prob}[Tube lemma for Lindel\"{o}f spaces]
    
\end{prob}
\begin{sol}
    
\end{sol}
\color{black}

\begin{prob}
    Prove \cref{converse inheritance} for complete regular spaces.
\end{prob}
\begin{enumerate}
    \item[(a)]{
        Let $X$ be a completely regular space and $Y$ be a subspace of $X$.
        If $a\in Y$ and $B$ is a closed subspace of $Y$ not containing $a$, then $B=Y\cap\overline{B}$, so $a\notin\overline{B}$.
        Therefore, there is a continuous function $f: X\rightarrow[0, 1]$ such that $f(a)=0$ and $f(\overline{B})=\{1\}$.
        The restriction of $f$ on $Y$ separates $a$ and $B$.
    }
    \item[(b)]{
        Suppose $\{X_\alpha\}_{\alpha\in A}$ is a collection of completely regular spaces and write $X=\prod_{\alpha\in A}X_\alpha$.
        Let $a$ be a point of $X$ and $B$ be a closed subspace of $X$ not containing $a$.
        Then $a$ is contained in the open subspace $X\setminus B$, so there is a basis member $\prod_{\alpha\in A} O_\alpha$ about $a$ in $X$ contained in $X\setminus B$, with $O_\alpha\neq X_\alpha$ if and only if $\alpha=\alpha_1, \cdots, \alpha_n$.
        For each $\alpha_i\,(i=1, \cdots, n)$, let $f_i: X_{\alpha_i}\rightarrow[0, 1]$ be a continuous function such that $f_i(a_{\alpha_i})=1$ and $f_i(X_{\alpha_i}\setminus O_{\alpha_i})=\{0\}$.
        Then, define the function $f: X\rightarrow[0, 1]$ by $f(x)=f_1(x_{\alpha_1})\times\cdots\times f_n(x_{\alpha_n})$ for $x\in X$.
        The function $f$ is continuous and $f(a)=1$, and $f$ maps $B$ onto $\{0\}$.
    }
\end{enumerate}

\begin{prob}
    Prove part (d) of \cref{converse inheritance}.
\end{prob}
\begin{sol}
    Let $\{X_\alpha\}_{\alpha\in A}$ be a collection of $T_1$-spaces such that the product space $\ds{X:=\prod_{\alpha\in A} X_\alpha}$ is a Hausdorff (regular, normal) space.
    We show that each $X_\alpha$ is also a Hausedorff (regular, normal) space.
    \begin{enumerate}
        \item[(1)]
        {
            When $X$ is a Hausdorff space and $p, q$ are disjoint points of $X$, let $U_\beta$ and $V_\beta$ be disjoint open sets in $X_\beta$ separating $p_\beta$ and $q_\beta$, where $\beta\in A$.
            Then $\prod_{\alpha\in A} U_\alpha$ and $\prod_{\alpha\in A} V_\alpha$ are desired sets, where $U_\alpha=X_\alpha$ and $V_\alpha=X_\alpha$ when $\alpha\neq\beta$.
        }
        \item[(2)]
        {
            Suppose $X$ is a regular space, and let $p$ be a point of $X$.
            Given $\beta\in A$, let $U_\beta$ be a neighborhood of $p_\beta$ in $X_\beta$ and $U_\alpha=X_\alpha$ for all $\alpha\neq\beta$.
            Since $\prod_{\alpha\in A}U_\alpha$ is a neighborhood of $p$ in $X$, by normality, there is a basis member $\prod_{\alpha\in A}V_\alpha$ containing $p$ with the closure in $X$ contained in $\prod_{\alpha\in A}U_\alpha$.
            Projecting onto $X_\beta$, we find that $V_\beta$ is a neighborhood of $p_\beta$ whose closure is contained in $U_\beta$.
            Therefore, if the product space is regular, then so is each $X_\alpha$.
        }
        \item[(3)]
        {
            Given $\beta\in A$, let $B_\beta$ be a closed subspace of $X_\beta$ and let $B_\alpha=X_\alpha$ for all $\alpha\neq\beta$.
            Also, let $U_\beta$ be a neighborhood of $B_\beta$ in $X_\beta$ and let $U_\alpha=X_\alpha$ for all $\alpha\neq\beta$.
            Then, the subspace $B:=\prod_{\alpha\in A}B_\alpha$ is a closed subspace of $X$ contained in the open subspace $U:=\prod_{\alpha\in A} U_\alpha$ of $X$.
            By normality, there is a neighborhood $V$ of $B$ in $X$ whose closure in $X$ is contained in $U$.

            Define $F_\beta=\prod_{\alpha\in A} Z_\alpha$, where $Z_\beta=X_\beta$ and $Z_\alpha=\{x_\alpha\}$ for some $x_\alpha\in X_\alpha$ when $\alpha\neq\beta$.
            Then, the map $\imath: X_\beta\rightarrow F_\beta$ defined by
            \begin{eqnarray*}
                (\pi_\alpha\circ\imath)(p)=\left\{\begin{matrix}
                    p & \textsf{if }\alpha=\beta\\
                    x_\alpha & \textsf{otherwise}
                \end{matrix}\right.
            \end{eqnarray*}
            for all $p\in X_\beta$ is a homeomorphism.
            Note that $F_\beta\cap V$ is open in $F_\beta$ and contains $F_\beta\cap B$ (which is closed in $F_\beta$) and that the closure of $F_\beta\cap V$ in $F_\beta$ is contained in $F_\beta\cap U$ (which is open in $F_\beta$).
            Because $\imath^{-1}(F_\beta\cap B)=B_\beta$ and $\imath^{-1}(F_\beta\cap V)=V_\beta$, the open subspace $\imath^{-1}(F_\beta\cap V)$ of $X_\beta$ is a neighborhood of $B_\beta$ in $X_\beta$ whose closure in $X_\beta$ is contained in $U_\beta$.
            Therefore, $X_\beta$ is a normal space.
        }
    \end{enumerate}
\end{sol}

\color{red}
\begin{prob}
    Let $f, g: X\rightarrow Y$ be continuous maps, where $Y$ is a Hausdorff space.
    Show that $\{x\in X: f(x)=g(x)\}$ is a closed subspace of $X$.
\end{prob}
\begin{sol}
    
\end{sol}
\color{black}