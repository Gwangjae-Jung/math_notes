\section{Countability and separation axioms (Problems)}

\begin{prob}\label{separability equivalences}
    Show that the suggested alternative definition for regular, completely regular, and normal spaces are equivalent to the original definitions.
\end{prob}
\begin{sol}
    We use the overline notation to denote the closure in $X$.
    \begin{enumerate}
        \item[(a)]
        {
            Assume $B=\{x\}$ for a given point $x\in X$ in the proof of (c).
        }
        \item[(b)]
        {
            (In this case, the proof is quite simple, for we do not have to consider larger subsets.)
            Let $X$ be a completely regular space and suppose that a point $x$ of $X$ and its neighborhood $U$ in $X$ is given.
            Then $B=X\setminus U$ is closed in $X$, so there is a continuous map $f: X\rightarrow [0, 1]$ such that $f(x)=1$ and $f(B)=\{0\}$.

            Assume conversely, and let $x$ be a point of $X$ and $B$ be a closed subset of $X$ not containing $x$.
            Then $U=X\setminus B$ is a neighborhood of $x$ in $X$, so there is a continuous map $f: X\rightarrow [0, 1]$ such that $f(x)=1$ and $f(X\setminus U)=\{0\}$.
        }
        \item[(c)]
        {
            Let $X$ be a normal space and suppose that a closed subset $B$ of $X$ and its neighborhood $U$ in $X$ are given.
            Then $C=X\setminus U$ is closed in $X$, so there are disjoint neighborhoods $V$ of $B$ and $W$ of $C$ in $X$.
            Then $B\subset V\subset X\setminus W$ and $\ol{V}\subset\ol{X\setminus W}=X\setminus W\subset X\setminus C=U$.

            Assume conversely, and let $B$ and $C$ be disjoint closed subsets of $X$.
            If $V$ is a neighborhood of $B$ in $X$ such that $\ol V\subset X\setminus C$, then $V$ and $X\setminus\ol V$ are disjoint neighborhood of $B$ and $C$ in $X$.
        }
    \end{enumerate}   
\end{sol}

\begin{prob}\label{inheritance: countabilities}
    Prove \cref{inheritance of countabilities}.
\end{prob}
\begin{sol}
    \begin{enumerate}
        \item[(a)]
        {
            Let $Y$ be a subspace of $X$.
            If $\{B_n\}_{n\in\bb{N}}$ is a countable base at $p\in Y$ in $X$, then $\{B_n\cap Y\}_{n\in\bb{N}}$ is a countable base at $p$ in $Y$; if $\{B_n\}_{n\in\bb{N}}$ is a countable basis of the topology on $X$, then $\{B_n\cap Y\}_{n\in\bb{N}}$ is a countable basis of the topology on $Y$.

            For spaces $X_1, X_2, \cdots$, let $\mc{B}_n=\{B_{n, k}\}_{k\in\bb{N}}$ be a countable base at $x_n\in X_n$ in $X_n$.
            Then
            \begin{align*}
                \left\{
                    \prod_{j\in\bb{N}} U_j:
                    \begin{array}{c}
                        \textsf{$U_j\in \mc{B}_j$ for each $j\in\bb{N}$ and}\\
                        \textsf{$U_j=X_j$ for all but finitely many values of $j$}
                    \end{array}
                \right\}
            \end{align*}
            is a countable base at $x\in \prod_{n\in\bb{N}} X_n$, where $\pi_n(x)=x_n$ for all $n\in\bb{N}$.
            When each $\mc{B}_n$ is a countable basis of the topolgy on $X_n$, then the above collection is a countable basis of the topology on $\prod_{n\in\bb{N}} X_n$.
        }
        \item[(b)]
        {
            Let $Y$ be a closed subspace of a Lindel\"of space $X$, and let $\{A_\alpha\}_\alpha$ be a covering of $Y$ by sets open in $Y$ (then, for each $\alpha$, $A_\alpha=Y\cap O_\alpha$ for an open subset $O_\alpha$ of $X$).
            Since $X\setminus Y$ is open in $X$, countably many members $A_\alpha$, together with $X\setminus Y$, cover $X$, as desired.
        }
        \item[(c)]
        {
            Let $X$ be a separable space and $Y$ be an open subspace of $X$.
            And let $D$ be a countable dense subset of $X$, and use the overline notation to denote the closure in $X$.
            We wish to show that $D\cap Y$ is a (countable) dense subset of $Y$, i.e., $\ol{D\cap Y}\cap Y=Y$.
            For this, it suffices to show $Y\subset\ol{D\cap Y}$.
            In fact, if $p\in Y$ and whenever $V$ is a neighborhood of $p$ in $X$ contained in $Y$, then $V$ contains a point of $D\cap Y$.
            Hence, every point of $Y$ belongs to $\ol{D\cap Y}$, as desired.
        }
    \end{enumerate}
\end{sol}

\begin{prob}\label{inheritance: Hausdorff}
    Prove \cref{inheritance of separabilities} for Hausdorff spaces.
\end{prob}
\begin{sol}
    All (a), (b), and (c) are obvious.
\end{sol}

\begin{prob}\label{inheritance: regular}
    Prove \cref{inheritance of separabilities} for regular spaces.
\end{prob}
\begin{sol}
    \begin{enumerate}
        \item[(a)]
        {
            Let $X$ be a regular space and $Y$ be a subspace of $X$.
            Assume that $p\in Y$ and $C$ is a closed subset of $Y$ not containing $p$.
            Because $C=Y\cap A$ for some subset $A$ closed in $X$ and $A$ does not contain $p$, by regularity of $X$, there are disjoint neighborhoods $U$ of $p$ and $V$ of $A$ in $X$.
            Then $Y\cap U$ and $Y\cap V$ are disjoint neighborhoods of $p$ and $C$ in $Y$.
        }
        \item[(b)]
        {
            Let $X_\alpha$ be a regular space for each $\alpha\in\mc{A}$ (here, $\mc{A}$ is an index set), and let $X=\prod_{\alpha\in\mc{A}} X_\alpha$.
            To justify the regularity of $X$, we make use of the alternative definition.
            Let $x$ be a point of $X$ and $U$ be a neighborhood of $x$ in $X$.
            Here, we may assume that $U=\prod_{\alpha\in\mc{A}} B_\alpha$, where $B_\alpha$ is open in $X_\alpha$ for each $\alpha\in\mc{A}$ and $B_\alpha=X_\alpha$ for all but finitely many values of $\alpha\in\mc{A}$.
            For each $\alpha\in\mc{A}$, use the regularity of $X_\alpha$ to find a neighborhood $V_\alpha$ of $x_\alpha$ in $X_\alpha$ whose closure in $X_\alpha$ is contained in $B_\alpha$ (when $B_\alpha=X_\alpha$, choose $V_\alpha=X_\alpha$).
            Then $V=\prod_{\alpha\in\mc{A}} V_\alpha$ is a neighborhood of $x$ in $X$ whose closure is contained in $U$.
        }
        \item[(c)]
        {
            Given $\alpha\in\mc{A}$, let $p_\alpha$ be a point of $X_\alpha$ and $C_\alpha$ be a closed subset of $X_\alpha$ not containing $p_\alpha$.
            Let $x$ be any point of $X$ whose $\alpha$-component is $p_\alpha$ and let $K=\pi_\alpha^{-1} (C_\alpha)$.
            Because $K$ is closed in $X$ and does not contain $x$, by the regularity of $X$, there are disjoint neighborhoods $U$ of $x$ and $V$ of $K$ in $X$.
            Applying the result of the following remark (together with the homeomorphism), we can conclude that $X_\alpha$ is regular.
        }
    \end{enumerate}
\end{sol}
\begin{rmk}[An embedding of a space into a product space]
    Let $\{X_\alpha\}_{\alpha\in\mc{A}}$ be a collection of topological spaces, and assume that $X_\alpha$ is a singleton whenever $\alpha\neq\beta$.
    Then $X_\beta$ is homeomorphic to $X=\prod_{\alpha\in\mc{A}} X_\alpha$.
    Indeed, one can impose an explicit homeomorphism.
    Let $s_\alpha$ be the unique element of $S_\alpha$ for each $\alpha\neq\beta$, and let $\imath: X_\beta\rightarrow X$ be the map defined by
    \begin{align*}
        (\pi_\alpha\circ\imath)(x)=\left\{\begin{array}{cc}
            x   &   \textsf{(if $\alpha=\beta$)}\\
            s_\alpha    &   \textsf{(otherwise)}
        \end{array}\right.
        \quad
        \textsf{for all $x\in X_\beta$}.
    \end{align*}
    Then $\imath$ is a desired homeomorphism.
    As an example, $\bb{R}^2$ and the plane $\{(x, y, z)\in\bb{R}^3: y=17\}\approx\bb{R}^2\times\{17\}$ are homeomorphic.
\end{rmk}

\begin{prob}\label{inheritance: completely regular}
    Prove \cref{inheritance of separabilities} for completely regular spaces.
\end{prob}
\begin{sol}
    \begin{enumerate}
        \item[(a)]
        {
            Let $Y$ be a subspace of a completely regular space $X$, and let $p$ be a point of $Y$ and $A$ be a closed subset of $Y$.
            Then $A=Y\cap B$ for some closed subset $B$ of $X$, and $B$ does not contain $p$.
            By the complete regularity of $X$, there is a continuous map $f: X\rightarrow [0, 1]$ separating $p$ and $B$.
            The restriction of $f$ to $Y$ separates $p$ and $A$.
        }
        \item[(b)]
        {
            Let $X_\alpha$ be a completely regular space for each $\alpha\in\mc{A}$ (here, $\mc{A}$ is an index set), and let $X=\prod_{\alpha\in\mc{A}} X_\alpha$.
            Let $p$ be a point of $X$ and $U$ be a neighborhood of $p$ in $X$.
            We may assume that $U=\prod_{\alpha\in\mc{A}} B_\alpha$, with $B_\alpha$ being an open subset of $X_\alpha$ for each $\alpha\in\mc{A}$ and $B_\alpha=X_\alpha$ for all but $\alpha_1, \cdots, \alpha_n\in\mc{A}$.
            For each $i=1, \cdots, n$, let $f_i: X_{\alpha_i}\rightarrow [0, 1]$ be a continuous map such that $f_i(p_{\alpha_i})=1$ and $f_i(X_{\alpha_i}\setminus B_{\alpha_i})=\{0\}$.
            Then, the map $f: X\rightarrow [0, 1]$ defined by $f(x)=f_1(x_{\alpha_1})\times\cdots\times f_n(x_{\alpha_n})$ for $x\in X$ is a continuous map with the properties that $f(p)=1$ and $f(X\setminus U)=\{0\}$.
        }
    \end{enumerate}
\end{sol}

\begin{prob}\label{inheritance: normal}
    Prove \cref{inheritance of separabilities} for normal spaces.
\end{prob}
\begin{sol}
    \begin{enumerate}
        \item[(a)]
        {    
            Suppose $Y$ is a closed subspace of a normal space $X$.
            If $A$ and $B$ are disjoint closed subspaces of $Y$, then they are closed in $X$, too.
            By normality, there are disjoint open subspaces $U$ and $V$ in $X$ containing $A$ and $B$, respectively.
            Then $U\cap Y$ and $V\cap Y$ are disjoint open subsets of $Y$ containing $A$ and $B$.
        }
        \item[(c)]
        {
            Given $\alpha\in\mc{A}$, let $B_\alpha$ and $C_\alpha$ be disjoint closed subsets of $X_\alpha$, and let $B=\pi_\alpha^{-1}(B_\alpha)$ and $C=\pi_\alpha^{-1}(C_\alpha)$.
            Because $B\cap C=\varnothing$, by the regularity of $X$, there are disjoint neighborhoods $U$ of $B$ and $V$ of $B$ in $X$.
            Again, applying the natural homeomorphism given in the previous remark, one can conclude that $X_\alpha$ is normal.
        }
    \end{enumerate}
\end{sol}

\begin{prob}
    Let $X$ be a completely regular space and $A,\,B$ be disjoint closed subsets of $X$.
    Show that there is a continuous function $f: X\rightarrow[0, 1]$ such that $f(A)=\{0\}$ and $f(B)=\{1\}$, if $A$ is compact.
\end{prob}
\begin{sol}
    For each point $a\in A$, let $f_a: X\rightarrow[0, 1]$ be a continuous function such that $f_a(a)=0$ and $f_a(B)=\{1\}$.
    Note that the collection $\{f_a^{-1}([0, r)):a\in A\}$ is an open cover of $A$ by sets open in $X$, where $r$ is any real number satisfying $0<r<1/2$.
    Thus, there are finitely many points $a_1, \cdots, a_n$ of $A$ such that $\bigcup_{j=1}^n f_{a_i}^{-1}([0, r))$ covers $A$.
    Define the function $f: X\rightarrow[0, 1]$ by $f(x)=f_1(x)\times\cdots\times f_n(x)$ for $x\in X$.
    Then $0\leq f(a)<r$ for all $a\in A$ and $f(B)=\{1\}$.
    If $g:[0, 1]\rightarrow[0, 1]$ is the function defined by
    \begin{eqnarray*}
        g(x)=\left\{
        \begin{matrix}
            0 & \textsf{if }0\leq x<r\\
            \dfrac{x-r}{1-r} & \textsf{otherwise}
        \end{matrix}\right.,
    \end{eqnarray*}
    then the composite $g\circ f$ is a desired continuous map from $X$ into $[0, 1]$.
\end{sol}

\begin{prob}
    Let $f,\,g: X\rightarrow Y$ be continuous maps, where $Y$ is a Hausdorff space.
    Show that $\{x\in X: f(x)=g(x)\}$ is a closed subspace of $X$.
\end{prob}
\begin{sol}
    We prove the closedness of the given subset by proving that $A:=\{x\in X: f(x)\neq g(x)\}$ is open in $X$.
    Note that $A=\bigcup_{a, b\in Y,\, a\neq b}(f^{-1}(\{a\})\cap g^{-1}(\{b\}))$.
    If, for each pair $(a, b)$ of distinct points of $Y$, $U_a$ and $V_b$ are disjoint neighborhoods of $a$ and $b$ in $Y$, we have
    \begin{align*}
        A=\bigcup_{a, b\in Y,\, a\neq b} (f^{-1}(U_a)\cap g^{-1}(V_b)),
    \end{align*}
    because $f^{-1}(U_a)\cap g^{-1}(V_b)=\bigcup_{s\in U_a}\left(f^{-1}(\{s\})\cap g^{-1}(V_b)\right)=\bigcup_{s\in U_a,\, t\in V_b}(f^{-1}(\{s\})\cap g^{-1}(\{t\}))$ for each pair $(a, b)$.
    It easily follows that $A$ is open in $X$.
\end{sol}
\begin{rmk}
    Let $X$ be a Hausdorff space and a continuous map $r: X\rightarrow A$ be a retraction.
    Because $A=\{x\in X: r(x)=\id{X}(x)\}$, so $A$ is closed in $X$.
\end{rmk}

\begin{prob}
    For a topological space $X$ and its subset $A$, we say that $A$ is \textit{discrete} if every point of $A$ is an isolated point of $A$ in $X$.
    Show that an uncountable subset of a second countable space is indiscrete, i.e., every uncountable subset of a second countable space has a limit point.
\end{prob}
\begin{sol}
    Let $\mc{B}$ be a countable basis of the second countable space $X$, and suppose $A$ is an uncountable discrete subset of $X$.
    For each point $x$ in $A$, let $B_x\in\mc{B}$ be chosen so that $B_x\cap A=\{x\}$.
    Since the collection of $B_x$ for $x\in A$ is countable, there are at most countably many distinct singletons $\{x\}$ for $x\in A$, a contradiction.
\end{sol}