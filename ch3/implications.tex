\section{Countability and separation axioms}

\subsection{Definitions regarding countability and separation axioms}

\begin{defi}[Countability axioms]
    Let $X$ be a topological space.
    \begin{enumerate}
        \item[(a)]
        {
            $X$ is said to have a countable base at a point $x\in X$ if there is a countable collection $\mc{B}$ of neighborhoods of $x$ in $X$ such that every neighborhood of $x$ contains at least one member of $\mc{B}$.
        }
        \item[(b)]
        {
            $X$ is said to be a first-countable space if every point of $X$ has a countable base.
        }
        \item[(c)]
        {
            $X$ is said to be a second-countable space if $X$ has a countable basis.
        }
    \end{enumerate}
    Remark that the second countability implies the first countability.
\end{defi}

For a second-countable space, countability does not remain only on the existence of countable bases.
In fact, countability presents in every basis of the topology on a second-countable space.
\begin{prop}
    If a space $X$ is second-countable and $\mc{C}$ is any basis of $X$, then $\mc{C}$ has a countable subcollection which is a basis of $X$.
\end{prop}
\begin{proof}
    Let $\mc{B}=\{B_n\}_{n\in\bb{N}}$ be a countable basis of $X$.
    For each $m, n\in\bb{N}$, whenever it is possible, choose a member $C_m^n\in\mc{C}$ such that $B_m\subset C_m^n\subset B_n$.
    \color{magenta}To be precise, for each $n\in\bb{N}$ and a point $p\in B_n$, find all possible members $C\in\mc{C}$ such that $p\in C\subset B_n$.
    For each $C$, find a member $B_m\in\mc{B}$ such that $p\in B_m\subset C\subset B_n$.
    If a member of $\mc{C}$ satisfying the last inclusion is already found, discard the newly found member of $\mc{C}$; otherwise, let such $C$ be denoted by $C_m^n$.\color{black}
    
    We first show that the collection $\mc{C}^*:=\{C_m^n\}_{m, n}$ is a countable basis of $X$.
    The countability of $\mc{C}^*$ is obvious, and the construction of $\mc{C}^*$ asserts that $\mc{C}^*$ covers $X$.
    Assume that a point $p\in X$ belongs to two members $C_{m_1}^{n_1}$ and $C_{m_2}^{n_2}$ of $\mc{C}^*$.
    Let $C_0$ be a basis member of $\mc{C}$ such that $p\in C_0\subset C_{m_1}^{n_1}\cap C_{m_2}^{n_2}$, and let $l\in\bb{N}$ be an index such that $p\in B_l\subset C_0$.
    \color{magenta}By the above construction\color{black}, for some integer $i\in\bb{N}$, we have $p\in B_i\subset C_i^l\subset B_l$, so $\mc{C}^*$ is a countable basis for a topology on $X$.

    Because $\mc{C}^*$ is contained in $\mc{C}$, it suffices to prove that the topology generated by $\mc{C}^*$ is finer than the topology on $X$, which is clear \color{magenta}by the above construction\color{black}: Given a point $x$ of $X$ with a basis member $B_n\in\mc{B}$ containing $x$, there is a basis member $C_l^n\in\mc{C}$ containing $x$ for some $l\in\bb{N}$.
\end{proof}

\begin{defi}[Separation axioms]
    A topological space $X$ is called a Fr\'echet space (or a $T_1$ space) if every finite subset of $X$ is closed in $X$.
    In the rest definitions, assume $X$ is a Fr\'echet space.
    \begin{enumerate}
        \item[(a)]
        {
            $X$ is called a Hausdorff space (or a $T_2$ space) if given two distinct points $a$ and $b$ in $X$, there are disjoint neighborhoods of $a$ and $b$.
        }
        \item[(b)]
        {
            $X$ is called a regular space (or a $T_3$ space) if given a point $a\in X$ and a nonempty closed subset $B\subset X$ not containing $a$, there are disjoint neighborhoods of $a$ and $B$.
        }
        \item[(c)]
        {
            $X$ is called a completely regular space (or a $T_{3\frac{1}{2}}$ space) if, given a point $a\in X$ and a nonempty closed subset $B\subset X$ not containing $a$, there is a continuous function $f: X\rightarrow[0, 1]$ such that $f(a)=1$ and $f(B)=\{0\}$.
        }
        \item[(d)]
        {
            $X$ is called a normal space (or a $T_4$ space) if, given two nonempty disjoint subsets in $X$ there are disjoint neighborhoods of those closed subsets.
        }
    \end{enumerate}
\end{defi}
\begin{rmk}[Alternative definitions for some separabilities]
    Assume that $X$ is a $T_1$ space.
    \begin{enumerate}
        \item[(a)]
        {
            $X$ is a regular space if and only if given a point $a\in X$ with its neighborhood $U$ in $X$, there is a neighborhood of $a$ in $X$ whose closure in $X$ is contained in $U$.
        }
        \item[(b)]
        {
            $X$ is a completely regular space if and only if given a point $p$ of $X$ and its neighborhood $U$ in $X$, there is a continuous function $f: X\rightarrow[0, 1]$ such that $f(p)=1$ and $f(X\setminus U)=\{0\}$.
        }
        \item[(c)]
        {
            $X$ is a normal space if and only if given a nonempty closed subset $B\subset X$ with its neighborhood $U$ in $X$, there is a neighborhood of $B$ in $X$ whose closure in $X$ is contained in $U$.
        }
    \end{enumerate}
    For their proof, see \cref{separability equivalences}.
\end{rmk}

\subsection{Basic properties of countabilities and separabilities}
As one can expect from the fact that a basis of the topology plays a major role in theory, second-countability is a strong condition.
\begin{prop}\label{2nd ctbl implies}
    Suppose $X$ is a second-countable space.
    \begin{enumerate}
        \item[(a)]
        {
            $X$ is a Lindel\"of space, i.e., every open cover of $X$ contains a countable subcover.
        }
        \item[(b)]
        {
            $X$ is a separable space, i.e., $X$ has a countable dense subset.
        }
    \end{enumerate}
\end{prop}
\begin{proof}
    \color{brown}Left as an exercise.\color{black}
\end{proof}
\begin{rmk}
    A space $X$ is said to be hereditarily Lindel\"of space if every subspace of $X$ is a Lindel\"of space.
\end{rmk}
The converse of \cref{2nd ctbl implies} are valid when the space is metrizable.
\begin{prop}
    For metrizable spaces, second countability, separability, and being a Lindel\"{o}f space coincide.
    In other words,
    \begin{enumerate}
        \item[(a)]
        {
            every metrizable Lindel\"{o}f space is second countable.
        }
        \item[(b)]
        {
            every metrizable separable space is second countable.
        }
    \end{enumerate}
\end{prop}
\begin{proof}
    \hangindent=0.65cm
    Let $X$ be a metrizable space, and let $d$ be a metric on $X$ which induces the topology on $X$.

    \noindent(a)
    For each $n\in\bb{N}$, the open cover $\{B_d\left(x, n^{-1}\right): x\in X\}$ has a countable subcover; let each member of a countable subcover be denoted by $B_k^n=B_d(x_k^n, n^{-1})$ with $k\in\bb{N}$.
    We want to show that the collection $\mc{B}:=\{B_k^n\}_{n, k\in\bb{N}}$ is a countable basis of the topology on $X$.
    Clearly, $\mc{B}$ is an open cover of $X$.
    If a point $p\in X$ is contained in two members $B_{k_1}^{n_1}$ and $B_{k_2}^{n_2}$, let $r$ be the positive number which is the minimum among the following four positive numbers:
    \begin{align*}
        d(p, x_{k_1}^{n_1}),\quad\dfrac{1}{n_1}-d(p, x_{k_1}^{n_1}),\quad d(p, x_{k_2}^{n_2}),\quad\dfrac{1}{n_2}-d(p, x_{k_2}^{n_2}).
    \end{align*}
    Suppose $n$ is large so that $2/n<r$, and let $j$ be an index such that $B_j^n$ contains $p$.
    Then $B_j^n$ is a member of $\mc{B}$ such that $p\in B_j^n\subset B_{k_1}^{n_1}\cap B_{k_2}^{n_2}$.
    Thus, $\mc{B}$ is a basis of a topology on $X$, and it is easy to show that $\mc{B}$ generates the metric topology on $X$ induced by $d$.

    \noindent(b)
    If $D=\{a_n\}_{n\in\bb{N}}$ is a countable dense subset of $X$, then $\{B_d(a_n, 1/k): k\in\bb{N}\}$ is a basis of the topology on $X$.
\end{proof}

We now introduce some inheritance properties.
\begin{prop}[Inheritance of countabilities]\label{inheritance of countabilities}
    Some countabilities are preserved as follow:
    \begin{enumerate}
        \item[(a)]
        {
            First and second countabilities are inherited to subspaces and countable product spaces.
        }
        \item[(b)]
        {
            Every closed subspace of a Lindel\"{o}f space is a Lindel\"{o}f space.
        }
        \item[(c)]
        {
            Every open subspace of a separable space is a separable space.
        }
    \end{enumerate}
\end{prop}
\begin{proof}
    See \cref{inheritance: countabilities}.
\end{proof}

\begin{prop}[Inheritance of separabilities]\label{inheritance of separabilities}
    Some separabilities are preserved as follow:
    \begin{enumerate}
        \item[(a)]
        {
            A subspace of a Hausdorff (regular, completely regular) space is a Hausdorff (regular, completely regular) spaces.
            (For a normal space $X$, a \textit{closed} subspace of $X$ is normal.)
        }
        \item[(b)]
        {
            A product of Hausdorff (regular, completely regular) spaces is a Hausdorff (regular, completely regular) spaces.
        }
        \item[(c)]
        {
            If the product space $\prod_{\alpha\in \mc{A}} X_\alpha$ is a Hausdorff (regular, normal) space, then so is each $X_\alpha$ for $\alpha\in \mc{A}$.
        }
    \end{enumerate}
\end{prop}
\begin{proof}
    See \cref{inheritance: Hausdorff,inheritance: regular,inheritance: completely regular,inheritance: normal}.
\end{proof}

\subsection{Examples of regular spaces}
\begin{exmp}[Ordered spaces are regular]
    Let $X$ be an ordered space, and let $U=(a, b)$ be a basis member in $X$ containing $p\in X$.
    \begin{enumerate}
        \item[(\romannumeral 1)]
        {
            Suppose $U$ consists of only one point $p$.
            Then, $U$ is both open and closed in $X$, so we may choose $V=U$.
        }
        \item[(\romannumeral 2)]
        {
            Suppose that $(a, p)$ or $(p, b)$ is empty and the other is nonempty.
            Without loss of generality, we assume that $(p, b)$ is empty and $(a, p)$ is nonempty.
            Choose a point $a'\in(a, p)$ and let $V=(a', b)=(a', p]$.
            Any point $x\in X$ with $x>p$ does not belong to the closure of $V$ in $X$; $(p, \infty)$ does not intersect $U$.
            Any point $x\in X$ with $x<a'$ does not belong to the closure of $V$ in $X$; $(-\infty, a')$ does not intersect $V$.
            Hence, the closure of $V$ in $X$ is contained in $[a', p]\subset U$.
        }
        \item[(\romannumeral 3)]
        {
            Suppose $(a, p)$ and $(p, b)$ are nonempty.
            Let $a'$ and $b'$ be points of $X$ in $(a, p)$ and $(p, b)$, respectively, and let $V=(a', b')$.
            Clearly, $V$ is a neighborhood of $p$ in $X$, and it is easy to check that the closure of $V$ in $X$ is contained in $U$.
        }
    \end{enumerate}
    Therefore, every ordered space is a regular space.
    In fact, it is known that every ordered space is a normal space, whose proof will not be introduced in this note.
\end{exmp}

\begin{exmp}[Locally compact Hausdorff spaces are regular]
    Suppose $X$ is a locally compact Hausdorff space.
    Without loss of generality, we may assume $X$ is not compact. (You will see why.)
    Let $Y$ be the one-point compactification of $X$.
    Since $Y$ is a compact Hausdorff space, $Y$ is normal and regular, as illustrated in the following subsection.
    Hence, $X$, a subspace of $Y$, is also regular.
\end{exmp}

\subsection{Examples of normal spaces}
\begin{thm}
    Lindel\"{o}f regular spaces are normal.
\end{thm}
\begin{proof}
    Let $A$ and $B$ be disjoint closed subspaces of $X$.
    For each $a\in A$, let $S_a$ be a neighborhood of $a$ in $X$ contained in $X\setminus B$.
    Using regularity, let $U_a$ be a neighborhood of $a$ in $X$ whose closure in $X$ is contained in $S_a$.
    And construct $V_b$ for each $b\in B$ as we constructed $U_a$ for each $a\in A$.

    Even if $\{U_a\}_{a\in A}$ and $\{V_b\}_{b\in B}$ are open covers of $A$ and $B$, respectively, the unions of the members in each collection need not be disjoint.
    (After drawing pictures on a sketchbook) one may wish to set
    \begin{align*}
        F_n:=U_n\setminus\bigcup_{k=1}^n\overline{V_k},\quad G_n:=V_n\setminus\bigcup_{k=1}^n\overline{U_k}
    \end{align*}
    for each $n\in\bb{N}$, and
    \begin{align*}
        F:=\bigcup_{n=1}^\infty F_n,\quad G:=\bigcup_{n=1}^\infty G_n.
    \end{align*}
    We wish $F$ and $G$ to be disjoint neighborhoods of $A$ and $B$, respectively.
    Clearly, $F$ and $G$ are neighborhoods of $A$ and $B$, since
    \begin{align*}
        F=\bigcup_{n=1}^\infty U_n\setminus\bigcup_{n=1}^\infty\overline{V_n}\supset A,\quad G=\bigcup_{n=1}^\infty V_n\setminus\bigcup_{n=1}^\infty\overline{U_n}\supset B.
    \end{align*}
    Also, $F\cap G=\varnothing$; otherwise, $F_n\cap G_k\neq\varnothing$ for some $n, l\in\bb{N}$.
    Without loss of generality, we may assume $n\leq k$.
    If $p\in F_n\cap G_k$, then $p\in F_n\subset U_n$; on the other hand, $G_k$ does not intersect $\overline{U_n}$, hence $p\notin G_k$.
    
    Therefore, $F$ and $G$ separate $A$ and $B$ and $X$ is a normal space.
\end{proof}

\begin{thm}
    Metrizable spaces are normal.
\end{thm}
\begin{proof}
    Let $X$ be a metrizable space and $d$ be a metric on $X$ inducing the topology on $X$.
    Let $A$ and $B$ be nonempty and disjoint closed subspaces of $X$.
    For each $a\in A$ and $b\in B$, let $s_a$ and $t_b$ be positive real numbers such that $B_d(a, s_a)\subset X\setminus B$ and $B_d(b, t_b)\subset X\setminus A$.
    Define
    \begin{align*}
        U:=\bigcup_{a\in A}B_d(a, s_a/3),\quad V:=\bigcup_{b\in B}B_d(b, t_b/3).
    \end{align*}
    Then $U$ and $V$ are neighborhoods of $A$ and $B$ in $X$.
    If $U\cap V$ is nonempty, then $B_d(a, s_a/3)\cap B_d(b, t_b/3)$ contains a point $z\in X$.
    We then have the following inequality:
    \begin{align*}
        d(a, b)\leq d(a, z)+d(z, b)<\dfrac{s_a}{3}+\dfrac{t_b}{3}<\max\left\{s_a, t_b\right\},
    \end{align*}
    so $b\in B_d(a, s_a)$ or $a\in B_d(b, t_b)$, a contradiction.
    Therefore, $U\cap V$ is empty, proving that $X$ is normal.
\end{proof}

\begin{thm}
    Well-ordered sets are normal. (In fact, every ordered space is normal.)
\end{thm}
\begin{proof}
    Let $X$ be a well-orderd set.

    \textbf{Step 1: Proving that every subspace of $X$ of the form $(x, y]$ is open in $X$.}\newline\noindent
    Let $A=(x, y]$ be a subset of $X$.
    Then $X\setminus A=(-\infty, x]\sqcup(y, \infty)$; because $(y, \infty)$ has the least element $y'\in X$, $X\setminus A=(-\infty, x]\sqcup[y', \infty)$ is closed in $X$.

    \textbf{Step 2: Proving that $X$ is normal.}\newline\noindent
    Let $A$ and $B$ be nonempty disjoint closed subspace of $X$.
    \begin{enumerate}
        \item[(\romannumeral 1)]
        {
            Assume that neither of them contains the least element $m$ of $X$.
            For each $a\in A$, there is a neighborhood $(x_a, a]$ of $a$ in $X$ not intersecting $B$; for each $b\in B$, there is a neighborhood $(y_b, b]$ of $b$ in $X$ not intersecting $A$.
            Define
            \begin{align*}
                U:=\bigcup_{a\in A}(x_a, a],\quad V:=\bigcup_{b\in B}(y_b, b].
            \end{align*}
            They are open in $X$ and $U$ covers $A$, and $V$ covers $B$.
            If $U$ and $V$ are not disjoint, $(x_a, a]\cap(y_b, b]$ is nonempty for some $a\in A$ and $b\in B$,
            Without loss of generality, we may assume $a<b$; then we have $a\in(y_b, b]$, a contradicton.
        }
        \item[(\romannumeral 2)]
        {
            Suppose, without loss of generality, $A$ contains $m$.
            Because the singletone $\{m\}=[m, m]=(-\infty, m]$ is not only closed but also open in $X$, $A\setminus\{m\}$ is also closed in $X$.
            By the preceeding part, there are disjoint neighborhood $U$ and $V$ of $A\setminus\{m\}$ and $B$, respectively.
            Then, $U\cup\{m\}$ and $V\setminus\{m\}$ are disjoint neighborhoods of $A$ and $B$, respectively.
        }
    \end{enumerate}
    Therefore, every well-ordered space is normal.
\end{proof}

\begin{exmp}[$\bb{R}_l$ is normal]
    Suppose $A$ and $B$ are disjoint closed subspaces of $\bb{R}_l$.
    For each $a\in A$, let $[a, x_a)$ be a neighborhood of $a$ in $\bb{R}_l$ contained in $\bb{R}_l\setminus B$; for each $b\in B$, let $[b, y_b)$ be a neighborhood of $b$ in $\bb{R}_l$ contained in $\bb{R}_l\setminus A$.
    Now set
    \begin{align*}
        U:=\bigcup_{a\in A}[a, x_a),\quad V:=\bigcup_{b\in B}[b, y_b).
    \end{align*}
    To show normality, it remains to show that $U$ and $V$ are disjoint.
    If $U\cap V\neq\varnothing$, then $[a, x_a)\cap[b, y_b)$ is nonempty for some $a\in A$ and $b\in B$.
    Since $a\neq b$, we may assume $a<b$; then $a<b<x_a<y_b$, and this implies that the point $b$ of $B$ is contained in $[a, x_a)$, a contradiction.
    Therefore, $\bb{R}_l$ is a normal space.
\end{exmp}

\begin{exmp}[Compact Hausdorff spaces are normal]
    Because closedness and compactness coincide in compact Hausdorff spaces and compact Hausdorff spaces are compactly normal, compact Hausdorff spaces are normal.
\end{exmp}