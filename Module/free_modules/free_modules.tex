\section{Remarks on free modules}

\begin{obs}[A free $R$-module has an $R$-basis]\label{free module iff there is a basis}
    Let $R$ be a ring with the nonzero identity, and assume that there is a free $R$-module $\mc{F}_R(S)$ genereated by a set $S$.
    Being a free object, we have $\mc{F}_R(S)=\genone{S}$, hence $\mc{F}_R(S)$ is the collection of all $R$-linear combinations of $S$.
    Moreover, since free objects are free from relations,
    \begin{center}
        any two distinct $R$-linear combinations should be distinct.
    \end{center}
    In other words, if the two finite sums $\sum_{s\in S}(a_s\cdot s)$ and $\sum_{s\in S}(a_s\cdot s)$ (where $a_s, b_s\in R$ for each $s\in S$) are equal, then $a_s=b_s$ for each $s\in S$.

    To sum up, if there is a free $R$-module generated by $S$, then $S$ is an $R$-basis of $\mc{F}_R(S)$ and $\mc{F}_R(S)\approx\genone{S}\approx\bigoplus_{s\in S}R$.
\end{obs}

\begin{thm}[Existence of free modules]\label{existence of free modules}
    Let $R$ be a ring with the nonzero identity and $S$ be a set.
    Then $\bigoplus_{s\in S} R$ is a free $R$-module.
    Therefore, $\mc{F}_R(S)\approx\bigoplus_{s\in S} R$.
\end{thm}
\begin{proof}
    One needs to check if $\bigoplus_{s\in S} R$ satisfies a universal property of free $R$-modules.
    For this, it suffices to find an $R$-basis of $\bigoplus_{s\in S}$; one can easily verify that the collection $\{e_s: s\in S\}$ is an $R$-basis of $\bigoplus_{s\in S}R$, where $e_s\in\bigoplus_{s\in S}R$ is the element defined by $e_s(t)=\delta_{st}$ for all $s, t\in S$.
    Then, together with the injection $\imath: S\hookrightarrow\bigoplus_{s\in S} R$ defined by $\imath(s)=e_s$, one can easily check that $\bigoplus_{s\in S} R$ satisfies the universal property.
\end{proof}

If we state the preceeding theorem with its proof in other words, we can obtain the following result.
\begin{thm}
    The followings are equivalent:
    \begin{enumerate}
        \item[(a)]
        {
            $\mc{F}$ is a free $R$-module.
        }
        \item[(b)]
        {
            An $R$-module $\mc{F}$ has an $R$-basis.
        }
    \end{enumerate}
\end{thm}
\begin{proof}
    The proof of [(a) implies (b)] is given in \cref{free module iff there is a basis}, and the proof of [(b) implies of (a)] can easily be obtained by modifying the proof of \cref{existence of free modules}: If $\mc{B}$ is an $R$-basis of an $R$-module $\mc{F}$, then $(\mc{F}, \imath)$ is a free $R$-module generated by $\mc{B}$, where $\imath: \mc{B}\hookrightarrow\mc{F}$ is the inclusion map.
\end{proof}
