\section{Finitely generated module over a principal ideal domain}

\begin{rmk}[Dimension theorem for vector spaces over fields]
    Suppose that $F$ is a field and $V, W$ are vector spaces over $F$.
    And let $L: V\rightarrow W$ be an $F$-linear map.
    Assuming that $\{w_i:i\in I\}$ is an $F$-basis of $\range(L)$, let $v_i$ be a vector in $V$ for each $i\in I$ such that $L(v_i)=w_i$ and define $U=\sum_{i\in I}Fv_i$.
    \begin{enumerate}
        \item[(a)]
        {
            $\{v_i: i\in I\}$ is an $F$-basis of $U$ and $U\approx W$.
        }
        \item[(b)]
        {
            $V=\ker L\oplus U$.
        }
    \end{enumerate}
    In particular, if both $V$ and $W$ are finite dimensional, then $\dim_F V=\dim_F(\ker L)+\dim_F(\range L)$.
\end{rmk}

In fact, the above dimension theorem extends to the following theorem.
\begin{thm}[Decomposition theorem for modules over integral domains]\label{domain-kernel-free_image}
    Assume that $D$ is an integral domain.
    Let $M$ and $N$ be $D$-modules and $\phi: M\rightarrow N$ be a $D$-module homomorphism \textit{with a free image}, i.e., $\range\phi$ is a free $D$-submodule of $N$.
    Then there is a free $D$-submodule $\mc{F}$ of $M$ such that
    \begin{center}
        $M=\ker\phi\oplus\mc{F}$\quad and\quad $\mc{F}\approx \range\phi$.
    \end{center}
\end{thm}
\begin{proof}
    \textbf{Step 1. Setting a $D$-submodule}\newline\indent
    Since $\range\phi$ is a free $D$-module, there is a $D$-basis $\mc{B}=\{v_i: i\in I\}$ of $\range\phi$.
    For each $i\in I$, choose $u_i\in M$ such that $\phi(u_i)=v_i$, and set $K=\sum_{i\in I} Du_i$.

    \textbf{Step 2. Justifying that the constructed submodule is isomorphic to the image of $\phi$}\newline\indent
    Clearly, $K$ is mapped onto $\range\phi$ by the restriction of $\phi$ to $K$.
    Assuming that $\phi|_K(\sum_{i\in I}a_iu_i)=0$, where the sum is finite, we have $a_i=0$ for all $i\in I$, for $\sum_{i\in I}a_iv_i=0$.
    Hence, $\phi|_K$ is a $D$-module isomorphism and $K\approx\range\phi$.

    \textbf{Step 3. Deriving a desired result}\newline\indent
    Therefore, $M=K+\ker\phi$.
    To show the sum is direct, assume that $x\in K\cap\ker\phi$.
    Since $x\in K$, we may write $x=\sum_{i\in I}a_iu_i$, where the sum is finite; since $x\in\ker\phi$, we have $0=\phi(x)=\sum_{i\in I}a_iv_i$, implying that $a_i=0$ for all $i\in I$.
    Thus, $K\cap\ker\phi=0$ and the sum $M=K+\ker\phi$ is direct.
\end{proof}

Before studying further theory, we introduce some lemmas.
Among the following lemmas, the first two lemmas could have been introduced in the preceeding sections, since they assume the ring to be an integral domain, rather than a PID.
\begin{lem}\label{free modules over domains are torsion-free}
    Free modules over integral domains are torsion-free.
\end{lem}
\begin{proof}
    Let $D$ be an integral domain and $M$ be a free $D$-module.
    And let $\{x_i: i\in I\}$ be a $D$-basis of $M$, and suppose that $x$ is a torsion element of $M$ which is annihilated by a nonzero scalar $r\in D$.
    Writing $x=\sum_{i\in I}a_ix_i$ (the sum is assumed to be finite), from $rx=0$ we have $ra_i=0$ for all $i\in I$.
    Since $D$ is an integral domain and $r\neq 0$, we have $a_i=0$ for all $i\in I$.
    Therefore, $x=0$ and $M$ is torsion-free.
\end{proof}
\begin{lem}\label{integral domain is isomorphic to the rank 1 free module over itself}
    Let $D$ be an integral domain and $M$ be a free $D$-module.
    If $x$ is a nonzero element of $M$, then $D\approx Dx$ as $D$-modules.
\end{lem}
\begin{proof}
    Define the map $\alpha: D\rightarrow Dx$ by $\alpha(t)=tx$ for $t\in D$.
    One can easily check that $\alpha$ is a $D$-module epimorphism.
    If $\alpha(t)=0$, because $D$ is an integral domain, we have $t=0$, hence $\alpha$ is a $D$-module isomorphism.
\end{proof}
In the remaining of this chapter, $D$ is assumed to be a PID.
\begin{lem}\label{pid is isomorphic to its ideals}
    Let $D$ be a PID and $M$ be a nonzero free $D$-module.
    If $I$ is a nonzero ideal of $D$, then $I\approx D$ as $D$-modules.
    In other words, every ideal of the PID $D$ is a free $D$-module of rank 1.
\end{lem}
\begin{proof}
    We can write $I=(a)=aD$ for some $a\in D\setminus\{0\}$.
    And define the map $\alpha: D\rightarrow I$ by $\alpha(x)=ax$ for $x\in D$.
    One can easily check that $\alpha$ is a $D$-module epimorphism.
    If $\alpha(x)=0$, by the law of cancellation, we have $x=0$.
    Therefore, $\alpha$ is a $D$-module isomorphism.
\end{proof}

The following theorem states a result what we have expected, but its proof is not simple.
\begin{thm}\label{submodules of a finite-rank free module over a pid are free}
    Let $D$ be a PID, and let $M$ be a \textit{finitely generated} free $D$-module.
    Then a nonzero $D$-submodule $N$ of $M$ is also a free $D$-module, and $\rank_D(N)\leq\rank_D(M)$.
\end{thm}
\begin{proof}
    Let $r=\rank_D(M)$ and $\mc{B}=\{x_1, \cdots, x_r\}$ is a $D$-basis of $M$.
    Since $M$ is assumed to be finitely generated, we can prove the theorem by induction on $r$. (See in the following proof how the hypothesis that $D$ is a PID is applied.)

    \textbf{Step 1. Proof for the case where $r=1$}\newline\noindent
    When $r=1$, $M$ is generated by a single element $x_1\in M$, so $M=Dx_1$.
    Then a nonzero $D$-submodule $N$ of $M$ is generated by scalar multiples of $x_1$; because $D$ is a PID, it can easily be explained that $N$ is generated by one element of $M$.
    Thus, $N=D(cx_1)$ for some $c\in D\setminus\{0\}$.
    By \cref{integral domain is isomorphic to the rank 1 free module over itself}, $N\approx M$, so $N$ is a free $D$-module of rank 1.

    \textbf{Step 2. Proof by induction}\newline\noindent
    Suppose that the theorem is valid for all free $D$-modules of rank less than $r$.
    Consider the $D$-module homomorphism $\pi_i: N\rightarrow Dx_i$ for $i=1, \cdots, r$, where $\pi_i(\sum_{j=1}^r c_jx_j)=c_ix_i$ for $c_j\in D$.
    Since $N$ is nonzero, $\pi_i(N)$ is nonzero for some index $i$; without loss of generality, assume $i=1$.
    \begin{enumerate}
        \item[(\romannumeral 1)]
        {
            By \cref{domain-kernel-free_image}, we have $N=\ker\pi_1\oplus\mc{F}$, where $\mc{F}$ is a $D$-submodule of $N$ which is isomorphic to $\range\pi_1=Dx_1\approx D$.
            Hence, $\mc{F}$ is a free $D$-module of rank 1.
        }
        \item[(\romannumeral 2)]
        {
            $\ker\pi_1$ is a $D$-submodule of $Dx_2\oplus\cdots\oplus Dx_r\approx D^{r-1}$.By the induction hypothesis, $\ker\pi_1$ is a free $D$-submodule of $N$ and $\rank_D(\ker\pi_1)\leq r-1$.
        }
    \end{enumerate}
    Therefore, $N$ is a free $D$-submodule of $M$ and $\rank_D(N)\leq\rank_D(M)$.
\end{proof}
\begin{cor}
    If $D$ is a PID, then the submodules of a finitely generated $D$-module are also finitely generated.
\end{cor}
\begin{proof}
    Note that an object is a homomorphic image of a free object.
    \ifinclude
    Let $M$ be a $D$-module generated by a finite set $S$, and consider the free $D$-module $\mc{F}_D(S)$ generated by $S$.
    If $\jmath: S\hookrightarrow M$ is the inclusion map, there is a unique $D$-module homomorphism $\tilde\jmath: \mc{F}_D(S)\rightarrow M$ extending $\jmath$, and $\tilde\jmath$ is clearly surjective.
    By \cref{submodules of a finite-rank free module over a pid are free}, $D$-submodule $\tilde\jmath^{-1}(N)$ of $\mc{F}_D(S)$ is a free $D$-module of finite rank, hence its image $N=\tilde\jmath(\tilde\jmath^{-1}(N))$ is finitely generated.
    \else
    \fi
\end{proof}
We introduce some application of the preceeding theory inlinear algebra over PID.s.
\begin{exmp}[Dimension theorem]
    When $V$ and $W$ are finite dimensional vector spaces over a field $F$ and $L: V\rightarrow W$ is an $F$-linear map, then $\dim_F V=\dim_F\ker L+\dim_F\range L$.
    As $V$ and $W$ are free $F$-modules of finite ranks, we assume as follows: Let $D$ be a PID and let $M, N$ be free $D$-modules of finite ranks, and let $\phi: M\rightarrow N$ be a $D$-linear map.
    We will show that
    \begin{align*}
        \rank_D(M)=\rank_D(\ker\phi)+\rank_D(\range\phi).
    \end{align*}

    Since $W$ is a finitely generated free $D$-module and $D$ is a PID, $\range\phi$ is a free $D$-submodule of $W$.
    Hence, we can apply \cref{domain-kernel-free_image} to write $M=\ker\phi\oplus\mc{F}$, where $\mc{F}\approx\range\phi$.
    Note that $\ker\phi$ is a free $D$-submodule of $V$, for $V$ is a finitely generated free $D$-module.
    Therefore, we obtain that $\rank_D(M)=\rank_D(\ker\phi)+\rank_D(\range\phi)$.
\end{exmp}
\begin{exmp}[Coincidence of the rank and the \textit{corresponding} dimension]
    Let $D$ be a PID and let $\mc{F}_D=\bigoplus_{i\in I} D$ and $\mc{F}_Q=\bigoplus_{i\in I}Q$, where $I$ is a nonempty finite set.
    Identify $\mc{F}_D$ as a subset of $\mc{F}_Q$ and let $S$ be a nonempty subset of $\mc{F}_D$.

    Observe that $\sum_{x\in S} Dx$ is a $D$-submodule of the finitely generated free $D$-module $\mc{F}_D$, so $\sum_{x\in S} Dx$ is a free $D$-module.
    Hence, a $D$-basis of $\sum_{x\in S} Dx$ is a $Q$-basis of $\sum_{x\in S} Qx$, implying that
    \begin{align*}
        \rank_D\left(\sum_{x\in S}Dx\right)=\dim_Q\left(\sum_{x\in S}Qx\right).
    \end{align*}

    Using the above coincidence, one can prove the rank theorem for finite dimensional matrices over PID.s.
    Let $A$ be an $m\times n$ matrix over $D$.
    Then $\rank_D C_D(A)=\dim_Q C_Q(A)$ and $\rank_D C(A^\star)=\dim_Q C_Q(A^\star)$(where the subscripts in the column spaces denotes the collection of scalars), where $\star=T$ or $\star=H$.
    Because $\dim_Q C_Q(A)=\dim_Q C_Q(A^\star)$, we obtain the rank theorem for finite dimensional matrices over PID.s.
\end{exmp}

Before proving another structure theorem, we introduce a lemma which characterizes a finitely generated free module over a PID.
\begin{lem}
    If $D$ is a PID, then a finitely generated \textit{torsion-free} $D$-module is \textit{free}.
\end{lem}
\begin{proof}
    Let $S$ be a finite set generating a torsion-free module $M$ over $D$, and \color{teal}let $T=\{x_1, \cdots, x_s\}$ be a maximal $D$-linearly independent subset of $S$\color{black}.
    Then $\genone{T}=\bigoplus_{i=1}^s Dx_i$ is a free $D$-module of rank $s$ with a $D$-basis $T$.
    \begin{center}
        Want to show: For each $x\in S$, there is a nonzero element $m_x\in D$ such that $m_xx\in\genone{T}$.
    \end{center}
    (The above statement is clearly valid if $x\in T$.
    If $x\in S\setminus T$, because $T\sqcup\{x\}$ is $D$-linearly dependent, $a_1x_1+\cdots+a_sx_s+bx=0$ implies $b\neq 0$.)

    Since $D$ is an integral domain and $S$ is finite, the product $m$ of $m_x$ for $x\in S$ is nonzero, and $mx\in\genone{T}$ for all $x\in S$.
    Now, consider the map $\phi: M\rightarrow\genone{T}$ defined by $\phi(x)=mx$ for $x\in M$.
    (Because $M$ is generated by $S$, the map $\phi$ is a well-defined $D$-module homomorphism.)
    \textit{Because $M$ is torsion-free}, $\ker\phi=0$.
    Therefore, by the first isomorphism theorem, $M\approx\phi(M)=\genone{T}$, so $M$ is free.
\end{proof}
\begin{rmk}
    It is clear that free modules over integral domains are torsion-free (and we already proved it).
    The preceeding proposition states that a finitely generated module over a PID is free if it is torsion-free.
    \begin{center}
        Conclusion: For finitely generated modules over PID.s, being free and being torsion-free coincide.
    \end{center}
\end{rmk}

\begin{cor}[Structure theorem for finitely generated modules over PID.s]\label{decomposition of f.g. modules over pids}
    Let $D$ be a PID and $M$ be a finitely generated $D$-module.
    Then there is a free $D$-submodule $\mc{F}$ of $M$ such that
    \begin{align*}
        M=M_\tor\oplus\mc{F}.
    \end{align*}
    Furthermore, if $T_1\oplus\mc{F}_1=M=T_2\oplus\mc{F}_2$ for some torsion $D$-submodules $T_1, T_2$ and free $D$-submodules $\mc{F}_1, \mc{F}_2$ of $M$, then $T_1=T_2$ and $\mc{F}_1\approx\mc{F}_2$.
    We call $\mc{F}$ the free part of $M$ and its rank the free rank of $M$.
\end{cor}
\begin{proof}
    Since $M/M_\tor$ is torsion-free and finitely generated, $M/M_\tor$ is a free $D$-module, so we naturally consider the natural projection $\pi: M\twoheadrightarrow M/M_\tor$.
    By \cref{domain-kernel-free_image}, we have $M=M_\tor\oplus\mc{F}$ for some free $D$-submodule $\mc{F}$ of $M$ such that $\mc{F}\approx M/M_\tor$.

    We now justify the uniqueness part.
    Suppose that $T_1\oplus\mc{F}_1=M=T_2\oplus\mc{F}_2$ for some torsion $D$-submodules $T_1, T_2$ and free $D$-submodules $\mc{F}_1, \mc{F}_2$ of $M$.
    Because $M_\tor=(T_i\oplus\mc{F}_i)_\tor\textbf{\color{brown}=\color{black}}(T_i)_\tor\oplus(\mc{F}_i)_\tor=T_i$ for $i=1, 2$, we have $T_1=T_2=M_\tor$.
    Also, because $M/M_\tor=M/M_i\approx\mc{F}_i$, we have $\mc{F}_1\approx\mc{F}_2$.
\end{proof}
\begin{rmk}
    The above structure theorem states the following:
    \begin{enumerate}
        \item[(a)]
        {
            (Existence)
            If $M$ is a finitely generated module over a PID, then $M$ is the direct sum of a torsion submodule and a free submodule.
        }
        \item[(b)]
        {
            (Uniqueness)
            Furthermore, if there are two such expressions $T_1\oplus\mc{F}_1=M=T_2\oplus\mc{F}_2$ ($T_i$ is a torsion submodule of $M$ and $\mc{F}_i$ is a free submodule of $M$ for $i=1, 2$), then the torsion submodules $T_1, T_2$ are the torsion part of $M$ and the free submodules $\mc{F}_1, \mc{F}_2$ are isomorphic as modules.
            In short, $T_1=T_2=M_\tor$ and $\mc{F}_1\approx\mc{F}_2\approx M/M_\tor$.
        }
    \end{enumerate}
    Later in this chapter (after studying some linear algebra over $\bb{Z}$), we will study the decomposition of the torsion part of a finitely generated modules over PID.s.
\end{rmk}

We end this section with an obvious observation.
\begin{exmp}
    Let $R$ be a ring with the nonzero identity and let $M$ be an $R$-module.
    Suppose that $N$ is an $R$-submodule of $M$, and assume that both $N$ and $M/N$ are finitely generated.
    Then $N=\genone{x_1, \cdots, x_n}$ and $\ol M=M/N=\genone{\ol{y_1}, \cdots, \ol{y_j}}$ for some $x_1, \cdots, x_n, y_1, \cdots, y_j\in M$, and one can easily verify that $M=\genone{x_1, \cdots, x_n, y_1, \cdots, y_j}$, i.e., $M$ is finitely generated.
\end{exmp}