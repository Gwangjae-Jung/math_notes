\section{Ranks of free modules}

\begin{center}
    Idea: To start with $Q^n$ to study $D^n$.
\end{center}
\begin{exmp}
    We will explain that the $\bb{Z}$-modules $\bb{Z}^3$ and $\bb{Z}^2$ are not isomorphic.
    For this, it suffices to show that no pair of two elements of $\bb{Z}^3$ generate $\bb{Z}^3$.
    Assume that there is a pair of $x, y\in\bb{Z}^3$ generating $\bb{Z}^3$.
    Then, in particular, $e_1, e_2, e_3$ can be written as a $\bb{Z}$-linear combination of $\{x, y\}$.
    Hence, $\{x, y\}\subset\bb{Z}^3\subset\bb{Q}^3$ generated the $\bb{Q}$-vector space $\bb{Q}^3$, implying that $\dim_\bb{Q} \bb{Q}^3\leq 2$, a contradiction.
    Therefore, $\bb{Z}^2\not\approx\bb{Z}^3$.
\end{exmp}

Using the above idea, we first establish an essential theorem.
\begin{thm}\label{free D-modules and Q-vector spaces}
    Let $I$ be a nonempty set and let $\mc{F}_D=\bigoplus_{i\in I} D$ and $\mc{F}_Q=\bigoplus_{i\in I} Q$.
    And identify $\mc{F}_D$ as a subset of $\mc{F}_Q$.
    And let $S$ be a subset of $\mc{F}_D$.
    \begin{enumerate}
        \item[(a)]
        {
            If $S$ generates the (free) $D$-module $\mc{F}_D$, then $S$ generates the $Q$-vector space $\mc{F}_Q$.
        }
        \item[(b)]
        {
            $S$ is a $D$-linearly independent subset of $\mc{F}_D$ if and only if $S$ is a $Q$-linearly independent subset of $\mc{F}_Q$.
        }
        \item[(c)]
        {
            Hence, if $\mc{B}$ is a $D$-basis of $\mc{F}_D$, then $\mc{B}$ is a $Q$-basis of $\mc{F}_Q$.
        }
    \end{enumerate}

    If $M$ is a free $D$-module such that $M\approx\bigoplus_{i\in A} D$ for some nonempty set $A$ and $\mc{B}$ is a $D$-basis of $M$, then $\mc{B}$ and $A$ are in bijection.
\end{thm}
\begin{proof}
    \begin{enumerate}
        \item[(a)]
        {
            If $S$ generates $\mc{F}_D$, then in particular, $e_s$ is a $D$-linear combination of $S$ for each $s\in S$.
            Since $\{e_s: s\in S\}$ is a $Q$-basis of $\mc{F}_Q$, it follows that $S$ generates $\mc{F}_Q$.
        }
        \item[(b)]
        {
            It is clear that $S$ is $D$-linearly independent if it is $Q$-linearly independent.
            Suppose conversely that $S$ is $D$-linearly independent but $S$ may be $Q$-linearly dependent.
            By reducing fractions, we can find a nontrivial $D$-linear combination of $0\in\mc{F}_D$ by elements in $S$, which contradicts the assumption.
        }
    \end{enumerate}
    (c) follows from (a) and (b).

    Since $M$ is a free $D$-module, there is a set $A$ such that $M\approx\bigoplus_{a\in A} D$.
    Assume that $A$ is nonempty, and let $\phi: M\xrightarrow{\approx}\bigoplus_{a\in A}D$ be a $D$-module isomorphism.
    Then $\phi(\mc{B})$ is a $D$-basis of $\bigoplus_{a\in A} D$; by (c) of \cref{free D-modules and Q-vector spaces}, we have $|\mc{B}|=|\phi(\mc{B})|=|A|$.
\end{proof}
\begin{cor}
    Suppose that $S$ and $T$ are nonempty sets.
    Then $S$ and $T$ are in bijection if and only if the free $D$-modules generated by $S$ and $T$ are isomorphic.
\end{cor}
\begin{proof}
    It is already proved that the free $D$-modules generated by $S$ and $T$ are isomorphic if $S$ and $T$ are in bijection.
    Assume that the free $D$-modules generated by $S$ and $T$ are isomorphic.
    Then $\bigoplus_{s\in S} D\approx \bigoplus_{t\in T} D$(,so an isomorphic image of a $D$-basis of $\bigoplus_{s\in S} D$ is a $D$-basis of $\bigoplus_{t\in T} D$).
    Hence, $S$ and $T$ are in bijection.
\end{proof}
\begin{rmk}
    In accordance with the above corollary, we can define the rank of a free $D$-module, whose counterpart in linear algebra over fields is the dimension of a vector space over a field; when $M$ is a free $D$-module and $M\approx\bigoplus_{i\in I} D$ and both $\mc{B}$ and $\mc{C}$ are $D$-bases of $M$, then $\mc{B}$, $I$, and $\mc{C}$ are in bijections.
    \begin{defi}
        Given a free $D$-module $M$, if $\mc{B}$ is a $D$-basis of $M$, then $|\mc{B}|$ is called the rank of $M$, and we write $\rank_D(M)=|\mc{B}|$.
    \end{defi}
    Hence, to state (c) of \cref{free D-modules and Q-vector spaces}, we can write as follows: $\rank_D(\mc{F}_D)=\dim_Q(\mc{F}_Q)$.
\end{rmk}
\begin{cor}
    Suppose that $S$ and $T$ are nonempty sets.
    Then $S$ and $T$ are in bijection if and only if the free groups generated by $S$ and $T$ are isomorphic.
\end{cor}
\begin{proof}
    Again, the free groups generated by $S$ and $T$ are isomorphic if $S$ and $T$ are in bijection.
    Assuming conversely and remarking that $\bb{Z}$ is an integral domain, we find that
    \begin{align*}
        \mc{F}_\bb{Z}(S)\approx\frac{\mc{F}_\textsf{gp}(S)}{[\mc{F}_\textsf{gp}(S), \mc{F}_\textsf{gp}(S)]}\approx\frac{\mc{F}_\textsf{gp}(T)}{[\mc{F}_\textsf{gp}(T), \mc{F}_\textsf{gp}(T)]}\approx\mc{F}_\bb{Z}(T).
    \end{align*}
    By the preceeding corollary, $S$ and $T$ are in bijection.
\end{proof}