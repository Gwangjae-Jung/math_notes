\section{Examples of submodules}

\begin{exmp}[Abelian groups and $\bb{Z}$-modules]
    Given an abelian group $A$, the \textit{natural} action of $\bb{Z}$ on $A$ gives a $\bb{Z}$-scalar multiplication on $A$.
    With this action, the abelian group $A$ can be considered a $\bb{Z}$-module.
    The converse understanding is clear, by forgetting $\bb{Z}$-scalar multiplications.
    Therefore, we can identify an abelian group as a $\bb{Z}$-module and vice versa.

    In particular, suppose that $A$ is an abelian group and let $m$ be a positive integer such that $mx=0\in A$ for all $x\in A$.
    Then $A$ can also be considered $\bb{Z}/m\bb{Z}$-module.
    In particular, if $m$ is a prime number $p$, then $A$ can be considered an $\bb{F}_p$-vector space.
\end{exmp}

\begin{exmp}[Vector spaces over fields as modules over polynomial rings]
    Let $F$ be a field and $V$ be an $F$-vector space and let $T$ be a linear operator on $V$.
    Given a polynomial $f(t)\in F[t]$, define the action of $f(t)$ on the element $v$ of $V$ as follows:
    \begin{align*}
        f(t)\cdot v:=f(T)(v).
    \end{align*}
    Together with action, $V$ can be considered an $F[t]$-module, whose scalar multiplication extends the scalar multiplication of the $F$-vector space $V$.

    Now we discuss when an $F$-subspace of $V$ is an $F[t]$-submodule of $V$.
    It is clear that an $F[t]$-submodule of $V$ is an $F$-subspace of $V$.
    So let $W$ be an $F$-subspace of $V$ and try to find a condition under which $W$ is an $F[t]$-submodule of $V$.
    If $W$ is an $F[t]$-submodule of $V$, then $W$ has to be closed under $F[t]$-linear combinations.
    Hence, in particular, $W$ has to be $T$-invariant.
    Conversely, if $W$ is an $F$-subspace ov $V$ which is $T$-invariant, it can easily be explained that the $F[t]$-scalar multiplication defined above behaves well on $W$, i.e., $f(t)\cdot w\in W$ whenever $f(t)\in F[t]$ and $w\in W$.
    Therefore, an $F$-subspace $W$ of $V$ is an $F[t]$-submodule of $V$ if and only if $W$ is $T$-invariant.
\end{exmp}

We now study quotient modules.
For this, assume that $M$ is an $R$-module and $N$ be an $R$-submodule of $N$.
For the quotient $M/N$ to be an $R$-module, $M/N$ has to be an abelian group, which is automatically achieved. \color{brown}(How?) \color{black}
And it is easy to observe that the \textit{natural} $R$-scalar multiplication on $M/N$ behaves well.
Therefore, $M/N$ is an $R$-module whenever $N$ is an $R$-submodule of $M$.

\begin{exmp}
    Let $F$ be a field and $V$ be a finite dimensional vector space over $F$.
    Supppose that $W$ is an $F$-subspace of $V$.
    Considering the $F$-linear map $\pi_N: M\rightarrow M/N$, by the dimension theorem, we have $\textsf{dim}M = \textsf{dim}\ker(\pi_N) + \textsf{dim}\range(\pi_N)=\textsf{dim}N+\textsf{dim}(M/N)$.
\end{exmp}