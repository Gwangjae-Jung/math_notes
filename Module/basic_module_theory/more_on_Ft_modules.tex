\section{More on modules over polynomial rings over fields}

In this section, we study some basic theory regarding finite dimensional $F$-vector spaces which are understood as $F[t]$-modules, where $F$ is a field.
In this section, we assume that $F$ is a field and $V$ is a finite dimensional $F$-vector space, and we understand $V$ as an $F[t]$-module with a given $F$-linear operator $T$ on $V$.

Before studying some properties of $F$-vector spaces as $F[t]$-modules, we first prove that a linear operator on an \textit{algebraically closed} fields are triangularizable.
\begin{thm}[Triangularization]
    Let $E$ be an \textit{algebraically closed} field and $V$ be a finite dimensional $E$-vector space.
    If $T$ is an $E$-linear operator on $V$, then there is an $E$-basis of $V$ with respect to which the matrix representation of $T$ is upper triangular, i.e., $[T]^\mc{B}_\mc{B}$ is upper triangular.
\end{thm}
\begin{proof}
    We prove the theorem by induction on $\dim_E V$.
    When $\dim_E V=1$, the result is clear.
    Assume the theorem holds for all $E$-vector spaces with $\dim_E V<n$, and assume $\dim_E V=n$.
    \textit{Because $E$ is algebraically closed}, the characteristic polynomial $\phi_T(t)$ of $T$ has a root $\lambda_1$ in $E$; let $v_1$ be a nonzero eigenvector of $T$ belonging to $\lambda_1$, and let $\mc{B}_1$ be an $E$-basis of $V$ containing $v_1$.
    Then
    \begin{align*}
        [T]^{\mc{B}_1}_{\mc{B}_1}=\left(\begin{array}{c|ccc}
            \lambda_1 & *    & \cdots & *\\\hline
            0         & *    & \cdots & *\\
            \vdots    &\vdots& \ddots &\vdots\\
            0         & *    & \cdots & *
        \end{array}\right),
    \end{align*}
    and the lower-right $(n-1)\times(n-1)$ matrix is triangularizable by induction hypothesis.
    Therefore, there is an $E$-basis with respect to which the matrix representation of $T$ is upper triangular.
\end{proof}

\subsection{Annihilator ideals}

\begin{thm}[Annihilator ideal]
    The annihiltor ideal $I_T$ of $F[t]$ with regard to $T$ is defined as
    \begin{align*}
        I_T:=\{f(t)\in F[t]: f(T)=O\}.
    \end{align*}
\end{thm}
\begin{rmk}
    \begin{enumerate}
        \item[(1)]
        {
            Let $n=\dim_F V$.
            Then $L(V, V)$ is a $n^2$-dimensional $F$-vector space, so
            \begin{align*}
                \{\id{V}, T, T^2, \cdots, T^{n^2}\}
            \end{align*}
            is $F$-linearly dependent.
            This implies that there is a nonzero polynomial $f(t)\in F[t]$ such that $f(T)=O$, so $I_T$ is nonzero.
        }
        \item[(2)]
        {
            One can easily check that $I_T$ is an ideal of $F[t]$: If $f(t), g(t)\in I_T$ and $r(t)\in F[t]$, then $f(t)-g(t), f(t)r(t)=r(t)f(t)\in I_T$.
            Also, since $F[t]$ is a Euclidean domain, $F[t]$ is a PID., so $I_T$ can be generated by a \color{brown}unique \color{black} monic polynomial $m_T(t)\in F[t]$, which is called the minimal polynomial of $T$.
            \color{brown}One can easily check that the $F$-dimension of $\{f(T): f(t)\in F[t]\}\leq L(V, V)$ is the degree of the minimal polynomial of $T$. \color{black}
        }
    \end{enumerate}
\end{rmk}

\begin{obs}
    Let $f(t)$ be a polynomial over $F$ and $S, T$ be an $F$-linear operator on $V$.
    \begin{enumerate}
        \item[(a)]
        {
            If $\lambda$ is an eigenvalue of $T$ and $v$ is an eigenvector of $T$ belonging to $\lambda$, then $v$ is an eigenvector of $f(T)$ belonging to $f(\lambda)$.
        }
        \item[(b)]
        {
            If $U$ is an automorphism of $V$, then $f(UTU^{-1})=U\cdot f(T)\cdot U^{-1}$.
            Hence, $f(T)\sim f(S)$ if $T\sim S$.
        }
    \end{enumerate}
    In particular, from (b), it follows that $I_T=I_S$ (or equivalently, $m_T(t)=m_S(t)$) if $T\sim S$.
    In other words, annihiltor ideals are invariant under similarity transforms.
\end{obs}

\begin{prop}[Cayley-Hamilton theorem]
    Suppose that $E$ is an \textit{algebraically closed} field.
    If $T$ is an $E$-linear operator on $V$, then $\phi_T(t)\in I_T$, where $\phi_T(t)$ is the characteristic polynomial of $T$.
    Hence, $m_T(t)|\phi_T(t)$.
\end{prop}
\begin{proof}
    \textit{Because $E$ is algebraically closed}, there is an $E$-basis $\mc{B}$ of $V$ with respect to which the matrix representation of $T$ is upper triangular.
    Letting $n=\dim_F V$ and writing $\phi_T(t)=(t-\lambda_1)\cdots(t-\lambda_n)$, the matrix representation of $T-\lambda_i\id{V}$ with respect to $\mc{B}$ is upper triangular with 0 on its $(i, i)$-entry.
    Hence, $\phi_T(T)=O$, so $\phi_T(t)\in I_T$.
\end{proof}

\begin{prob}
    Suppose that $B\in\mc{M}_{m, m}(F)$, $C\in\mc{M}_{n, n}(F)$, and $D\in\mc{M}_{m, n}(F)$.
    And let $A=\left(\begin{array}{c|c}
        B&D\\   \hline
        O&C
    \end{array}\right)$.
    Prove the following statements:
    \begin{enumerate}
        \item[(a)]
        {
            $\phi_A(t)=\phi_B(t)\phi_C(t)$.
        }
        \item[(b)]
        {
            $m_A(t)$ is a common multiple of $m_B(t)$ and $m_C(t)$.
        }
        \item[(c)]
        {
            If $D=O$, then $m_A(t)$ is the monic least common multiple of $m_B(t)$ and $m_C(t)$.
        }
    \end{enumerate}
\end{prob}
\begin{sol}
    (a) is clear by the definition of characteristic polynomial.
    Because
    \begin{align*}
        f(A)=\begin{pmatrix}
            f(B)&\ast\\O&f(C)
        \end{pmatrix}    
    \end{align*}
    whenever $f(t)$ is a polynomial over $F$, $m_A(t)$ is necessarily a multiple of $m_B(t)$ and $m_C(t)$.
    In particular, if $D=O$, then $f(A)=\begin{pmatrix}
        f(B)&O\\O&f(C)
    \end{pmatrix}$, so the monic least common multiple of $m_B(t)$ and $m_C(t)$ annihilates $A$.
    Therefore, if $D=O$, then $m_A(t)$ is the monic least common multiple of $m_B(t)$ and $m_C(t)$.
\end{sol}

\begin{prob}
    If $T$ is unipotent (in other words, $T-\id{V}$ is nilpotent) and there is a positive integer $m$ such that $T^m$ is the identity map on $V$, then $T$ is the identity map on $V$.
\end{prob}
\begin{sol}
    By hypotheses, the minimal polynomial of $T$ divides $(t-1)^r$ for some positive integer $r$ and $t^m-1=(t-1)(t^{m-1}+\cdots+t+1)$, hence $m_T(t)=t-1$ and $T=\id{V}$.
\end{sol}

\begin{prop}
    The monic irreducible divisors of $\phi_T(t)$ and $m_T(t)$ are the same.
\end{prop}
\begin{proof}
    It suffices to prove that the monic irreducible divisors of $\phi_T(t)$ are the monic irreducible divisors of $m_T(t)$.
    There is no problem when we assume that $T\in\mc{M}_{n, n}(F)$.
    Let $p(t)\in F[t]$ be an irreducible divisor of $m_T(t)$, and suppose that $\alpha$ is a root of $p(t)$ in the splitting field $K$ of $p(t)$ over $F$.
    Then $\alpha$ is an eigenvalue of $T\in\mc{M}_{n, n}(K)$, so there is a nonzero vector $v\in K^n$ such that $Tv=\alpha v$, and we have $0=m_T(T)v=m_T(\alpha)v$.
    Therefore, $m_T(\alpha)=0$ and $p(t)$ divides $m_T(t)$, since $p(t)$ is the irreducible polynomial of $\alpha$ over $F$.
\end{proof}

The following proposition is valid when the base field is $\bb{R}$, whose algebraic closure is $\bb{C}$.
\begin{prop}
    Suppose that $A\in\mc{M}_{n, n}(\bb{R})$.
    Then the minimal polynomimal of $A$ when $A$ is considered a matrix over $\bb{R}$ or over $\bb{C}$ coincide.
\end{prop}
\begin{proof}
    Let $m(t)\in \bb{R}[t]$ denote the minimal polynomial of $A$ when $A$ is considered a matrix over $\bb{R}$; let $m_*(t)\in \bb{C}[t]$ denote the minimal polynomial of $A$ when $A$ is considered a matrix over $\bb{C}$.
    It suffices to show that $m_*(t)\in\bb{R}[t]$; if it is proved that $m_*(t)\in\bb{R}[t]$, then
    \begin{enumerate}
        \item[(1)]
        {
            because $A\in\mc{M}_{n, n}(\bb{C})$ and $m(A)=O$, we have $m_*(t)|m(t)$, and
        }
        \item[(2)]
        {
            because $m_*(t)\in\bb{R}[t]$ and $m_*(A)=O$, we have $m(t)|m_*(t)$
        }
    \end{enumerate}
    so that $m(t)=m_*(t)$.
    Because $A$ is a matrix over $\bb{R}$, we have $\ol A=A$, thus $\ol{m_*}(A)=\ol{m_*}(\ol{A})=\ol{m_*(A)}=O$, so $m_*(t)|\ol{m_*(t)}$.
    Because $\deg m_*(t)=\deg m(t)$, $m_*(t)\in\bb{R}[t]$, as desired.
\end{proof}

\subsection{Subspaces which are invariant under linear operators}

\begin{exmp}
    Justifing the following statements is left as an exercise.
    \begin{enumerate}
        \item[(a)]
        {
            If $\lambda\in F$ is an eigenvalue of $T$, then $E_\lambda^T$, the eigenspace of $T$ belonging to $\lambda$ is $T$-invariant.
        }
        \item[(b)]
        {
            If $U, W\leq V$ are $T$-invariant, then both $U\cap W$ and $U+W$ are $T$-invariant subspaces of $V$.
        }
        \item[(c)]
        {
            Given $v\in V$, the $F$-subspace $F[t]v=\{f(T)v: f(t)\in F[t]\}$ is the $T$-invariant subspace of $V$ generated by $\{v, Tv, T^2v, \cdots\}$.
            (In fact, $F[t]v$ is the cyclic $F[t]$-submodule of $V$ generated by $v\in V$.)
        }
        \item[(d)]
        {
            If $W$ is a $T$-invariant subspace of $V$ and $T$ is invertible, then $T|_W: W\rightarrow W$ is also invertible.
            Hence, $W$ is $T^{-1}$-invariant and $T^{-1}|_W=(T|_W)^{-1}$.
        }
    \end{enumerate}    
\end{exmp}
The following example is separated from the preceeding example, due to its importance in further theory.
\begin{exmp}
    Suppose $f(t)\in F[t]$.
    \begin{enumerate}
        \item[(a)]
        {
            Then $\ker f(T)$ and $\range f(T)$ are $T$-invariant.
            Hence, in particular, $\ker T$ and $\range T$ are $T$-invariant.
        }
    \end{enumerate}
    Assume that $W$ is a $T$-invariant subspace of $V$.
    \begin{enumerate}
        \item[(b)]
        {
            $W$ is $f(T)$-invariant, so $f(T|_W)=f(T)|_W$.
        }
    \end{enumerate}
    In particular, assume that $W=\ker f(T)$.
    \begin{enumerate}
        \item[(c)]
        {
            $W$ is $T$-invariant, and $f(t)$ is a multiple of $m_{T|_W}(t)$, for $f(t)$ annihilates $W$, i.e., $f(T)w=0$ for all $w\in W$.
        }
    \end{enumerate}
\end{exmp}

\begin{obs}
    Let $W$ be a $T$-invariant subspace of $V$ and let $\mc{B}$ be an $F$-basis of $V$ which extends an $F$-basis $\mc{C}$ of $W$.
    Then the matrix representation of $T$ with respect to $\mc{B}$ is block-diagonal:
    \begin{align*}
        [T]^\mc{B}_\mc{C}=\begin{pmatrix}
            [T|_W]^\mc{C}_\mc{C}&*\\O&*
        \end{pmatrix}.
    \end{align*}
    Thus, the characteristic polynomial of $T_W$ plays an essential role when determining the characteristic polynomial of $T$; when the first quadrant of $[T]^\mc{B}_\mc{B}$ (or equivalently, the (1, 2)-block of $[T]^\mc{B}_\mc{B}$) is the zero matrix, the minimal polynomial of $T|_W$ plays an essential role when determining the minimal polynomial of $T$.
    For example, suppose that $U, W$ are $T$-invariant subspaces of $V$ such that $V=U\oplus W$.
    If $\mc{C}$ and $\mc{D}$ are $F$-bases of $U$ and $W$, respectively, then $\mc{B}:=\mc{C}\sqcup\mc{D}$ is an $F$-basis of $V$ and
    \begin{align*}
        [T]^\mc{B}_\mc{B}=\begin{pmatrix}
            [T|_U]^\mc{C}_\mc{C}    &   O   \\
            O   &   [T|_W]^\mc{D}_\mc{D}
        \end{pmatrix},
    \end{align*}
    so $\phi_T(t)=\phi_{T|_U}(t)\phi_{T|_W}(t)$ and $m_T(t)=\textsf{lcm}\{\phi_{T|_U}(t), \phi_{T|_W}(t)\}$.
    In fact, this result naturally extends to the case where there are finitely many $T$-invariant subspaces of $V$ whose direct sum is $V$: Letting $U_i\,(i=1, \cdots, k)$ be $T$-invariant subspaces of $V$ whose direct sum is $V$ and $\mc{B}_i$ be an $F$-basis of $U_i$ for each $i$, then
    \begin{enumerate}
        \item[(1)]
        {
            $\mc{B}=\bigsqcup_{i=1}^k \mc{B}_i$ is an $F$-basis of $V$ and
            \begin{align*}
                [T]^\mc{B}_\mc{B}=\diag\left([T_1]^{\mc{B}_1}_{\mc{B}_1}, \cdots, [T_k]^{\mc{B}_k}_{\mc{B}_k}\right),
            \end{align*}
            where $T_i=T|_{U_i}$ for each $i$, and
        }
        \item[(2)]
        {
            $\phi_T(t)=\phi_{T_1}(t)\cdots\phi_{T_k}(t)$ and $m_T(t)=\textsf{lcm}\{m_{T_1}(t), \cdots, m_{T_k}(t)\}$.
        }
    \end{enumerate}
\end{obs}

\subsection{Subspaces which are cyclic with respect to linear operators}

Again, suppose that  $V$ is a finite dimensional vector space over the field $F$ and write $n=\dim_F V$.
We first observe the following definitions of $T$-cylcic (sub)spaces.
\begin{defi}\label{old cylcic spaces in vector spaces}
    Let $V$ be an $n$-dimensional vector space over the field $F$ ($n<\infty$) and $T$ be an $F$-linear operator on $V$.
    \begin{enumerate}
        \item[(a)]
        {
            ($T$-cyclic space)
            $V$ is called a $T$-cyclic space if there is a vector $v\in V$ such that $V=F[t]v$.
        }
        \item[(b)]
        {
            ($T$-cyclic subspace)
            Suppose that $W$ is a $T$-invariant subspace of $V$.
            Then $W$ is called a $T$-cyclic subspace of $V$ if $W$ is a $T|_W$-cyclic space.
        }
    \end{enumerate}
\end{defi}
\begin{obs}
    \begin{enumerate}
        \item[(a)]
        {
            By definition, it is clear that a $T$-cyclic space is a cyclic $F[t]$-module and vice versa.
            In fact, an $F[t]$-submodule $W$ of $V$ which is cyclic (with respect to $T$) is a $T$-cyclic subspace.
        
            Assume that $W$ is a cyclic $F[t]$-submodule of $V$.
            Then $W=F[t]v$ for some $v\in W$, so $W$ is $T$-invariant, and this implies
            \begin{align}\label{obs in defining cyclic subspaces}
                F[t]v=\{f(T)v: f(t)\in F[t]\}=\{f(T|_W)v: f(t)\in F[t]\}
            \end{align}
            so $W$ is $T|_W$-cyclic.
            Even it is assumed that $W$ is a $T$-cyclic subspace of $V$, \cref{obs in defining cyclic subspaces} is valid, so $W$ is a cyclic $F[t]$-submodule of $V$.
        
            Therefore, a $T$-cyclic (sub)space can be considered a cyclic $F[t]$-(sub)module with respect to $T$, and vice versa.        
        }
        \item[(b)]
        {
            Cyclicity of a vector space is determined by the linear operator.
            If $V$ is a vector space over a field $F$ and $W$ is a \textit{finite dimensional} $F$-subspace of $V$, then there is a linear operator $T$ of $V$ for which $W$ is $T$-cyclic, i.e., a cyclic $F[t]$-submodule of $V$.
            Since $W$ is finite dimensional, there is an $F$-basis $\{v_1, \cdots, v_n\}$ of $W$, and there is a basis $\mc{B}$ of $V$ extending $\{v_1, \cdots, v_n\}$.
            Defining the linear operator $T$ of $V$ by
            \begin{align*}
                Tu=\left\{\begin{array}{cc}
                        u   &   \textit{(if $u\in\mc{B}\setminus\{v_1, \cdots, v_n\}$)}\\
                    v_{i+1} &   \textit{(if $u=v_i$ for some $i=1, \cdots, n$)}
                \end{array}\right.,
            \end{align*}
            where $v_{n+1}=v_1$, then $W$ is a cyclic $F[t]$-submodule of $V$, for $W=F[t]v_1$.
        }
    \end{enumerate}
\end{obs}

In this subsection, we discover some basic properties regarding cyclic $F[t]$-submodules of $V$; after studying cyclic decomposition theorem, we will discover further properties.
\begin{exmp}
    Suppose that $V=F[t]v$ is $T$-cyclic and $g(t)\in F[t]$.
    If $g(T)v=0$, then $g(t)\in\ann_{F[t]}(v)=\ann_{F[t]} V$, so $g(T)$ is the zero ($F$-linear) operator on $V$.
    It can be alternatively explained as follows: For any vector $x\in V$, there is a polynomial $a(t)\in F[t]$ such that $x=a(T)v$, so $g(T)x=g(T)a(T)v=a(T)g(T)v=0$.
\end{exmp}

\begin{lem}\label{T-cyclic basic}
    Suppose that $V$ is a $T$-cyclic space generated by $v\in V$, and let $d=\deg m_T(t)$.
    \begin{enumerate}
        \item[(a)]
        {
            $\mc{B}:=\{v, Tv, \cdots, T^{d-1}v\}$ is an $F$-basis of $V$.
            Hence, $\deg m_T(t)=\dim_F V$.
        }
        \item[(b)]
        {
            $\phi_T(t)=m_T(t)$.
        }
    \end{enumerate}
    (The following converse of (b) will be proved in \cref{char_poly and min_poly are the same implies cyclicity}: If $\phi_T(t)=m_T(t)$, then $V$ is $T$-cyclic.)
\end{lem}
\begin{proof}
    It suffices to prove that $\mc{B}$ is an $F$-basis of $V$, which follows easily from the minimality of $\deg m_T(t)$.
    \color{brown}Checking details are left as an exercise.\color{black}
\end{proof}

By considering the natural $F$-basis for a $T$-cyclic space as given above, we can naturally introduce the companion matrix of a polynomial.
In \cref{T-cyclic basic}, if $m_T(t)=t^n+a_{n-1}t^{n-1}+\cdots+a_1t+a_0$, then the matrix representation of $T$ with respect to $\mc{B}$ is given as follows (this basis is called a $T$-cyclic basis of $V$):
\begin{align*}
    [T]^{\mc{B}}_{\mc{B}}=\begin{pmatrix}
        0   &       &       &       &       &       &-a_0       \\
        1   &   0   &       &       &       &       &-a_1       \\
            &   1   &   0   &       &       &       &-a_2       \\
            &       &\ddots &\ddots &       &       &\vdots     \\
            &       &       &\ddots &\ddots &       &\vdots     \\
            &       &       &       &   1   &   0   &-a_{n-2}   \\
            &       &       &       &       &   1   &-a_{n-1}
    \end{pmatrix}.
\end{align*}
Remark that we proved $\phi_T(t)=m_T(t)$ when $V$ is $T$-cyclic.
Thus, we naturally consider the converse case; whether we can find a matrix over $F$ whose minimal polynomial and characteristic polynomial are $\psi(t)$, where a nonconstant monic polynomial $\psi(t)$ over $F$ is given.
\begin{defi}[Companion matrix]
    Given a nonconstant monic polynomial $\psi(t)=t^n+a_{n-1}^{n-1}+\cdots+a_1t+a_0$ over $F$, define the companion matrix $C(\psi)$ of $\psi(t)$ by
    \begin{align*}
        C(\psi)=\begin{pmatrix}
            0   &       &       &       &       &       &-a_0       \\
            1   &   0   &       &       &       &       &-a_1       \\
                &   1   &   0   &       &       &       &-a_2       \\
                &       &\ddots &\ddots &       &       &\vdots     \\
                &       &       &\ddots &\ddots &       &\vdots     \\
                &       &       &       &   1   &   0   &-a_{n-2}   \\
                &       &       &       &       &   1   &-a_{n-1}
        \end{pmatrix}.
    \end{align*}
\end{defi}
To state the result of the preceeding observation, $[T]^\mc{B}_\mc{B}=C(m_T)$ if $V=F[t]v$ is $T$-cyclic and $\mc{B}$ is a $T$-cyclic basis of $V$.
In fact, this identity holds whenever $\psi(t)$ is a nonconstant monic polynomial over $F$.
\begin{prop}
    Suppose that a nonconstant monic polynomial $\psi(t)$ over $F$ is given, and let $C=C(\psi)$ be the companion matrix of $\psi(t)$.
    Let $n=\deg\psi(t)$.
    \begin{enumerate}
        \item[(a)]
        {
            $C^ke_1=e_{1+k}$ whenever $0\leq k\leq n-1$, so $F^n=F[t]e_1$ is $C$-cyclic.
        }
        \item[(b)]
        {
            $m_C(t)=\psi(t)=\phi_C(t)$.
        }
    \end{enumerate}
\end{prop}
\begin{proof}
    (a) follows easily from the given idenitity.
    Thus, $\phi_C(t)=m_C(t)$.
    To prove (b), it suffices to show that $\psi(t)$ annihilates $e_1$, which easily follows.
\end{proof}
\begin{rmk}
    Suppose that there is an $F$-basis $\mc{B}=\{v_i\}_{i=1}^n$ of $V$ with respect to which $T$ has the companion matrix of a monic polynomial $\psi(t)\in F[t]$ of degree $n$ as the matrix representation, i.e., $[T]^\mc{B}_\mc{B}=C(\psi)$.
    Then $v_{1+k}=T^kv_1$ for $0\leq k\leq n-1$, so $V$ is a $T$-cyclic space generated by $v_1$, and $m_T(t)=\psi(t)=\phi_T(t)$.
    As a conclusion, if there is an $F$-basis with respect to which the matrix representation of the given linear operator on $V$ is the companion matrix of $\psi(t)\in F[t]$, then $V$ is $T$-cyclic and $\phi_T(t)=\psi(t)=m_T(t)$.
\end{rmk}

\begin{nota}
    When $W$ is a $T$-cyclic subspace of $V$ generated by $w\in W$, the minimal polynomial of $T|_W$ will be denoted by $m_w(t)$ and will also be called the minimal polynomial of $w$.
    (Such abuse of notation is allowed, since $W$ is generated by $w$ and $F[t]$ is commutative.)
\end{nota}
\begin{obs}
    Given a nonzero vector $w$ of $V$, let $W=F[t]w$.
    And let $d=\deg(m_w(t))=\dim_F W$.
    \begin{enumerate}
        \item[(a)]
        {
            Because $W$ is a cyclic $F[t]$-submodule of $V$, $\mc{C}:=\{w, Tw, \cdots, T^{d-1}w\}$ is an $F$-basis of $V$ and $m_w(t)=\phi_{T|_W}(t)$.
        }
        \item[(b)]
        {
            As observed earlier, $[T|_W]^\mc{C}_\mc{C}=C(m_w)$.
        }
        \item[(c)]
        {
            Clearly, $m_w(t)|m_T(t)$.
        }
    \end{enumerate}
    In fact, (c) can be generalized as follows:
    \begin{enumerate}
        \item[(d)]
        {
            If $\{v_1, \cdots, v_n\}$ is an $F$-basis of $V$, then $m_T(v)=\textsf{lcm}\{m_{v_1}(t), \cdots, m_{v_n}(t)\}$.
        }
    \end{enumerate}
    (d) can be justified as follows, letting $l(t)=\textsf{lcm}\{m_{v_1}(t), \cdots, m_{v_n}(t)\}$, it is clear that $l(t)|m_T(t)$ since each $m_{v_i}(t)$ divides $m_T(t)$; because $l(t)$ annihilates $F[t]v_i$ for all $i=1, \cdots, n$, $l(t)$ is a multiple of $m_T(t)$, proving that $m_T(t)=l(t)$.

\end{obs}

The following proposition will be essential when we apply the cyclic decomposition theorem on finite dimensional $F$-vector spaces.
\begin{prop}
    Suppose that $V=F[t]v$ is $T$-cyclic, and write $m_T(t)=d(t)e(t)$ for some monic polynomials with $\deg(d(t))\geq 1$.
    And let $w=e(T)v$.
    \begin{enumerate}
        \item[(a)]
        {
            $\ker d(T)=F[t]w$.
        }
        \item[(b)]
        {
            $m_w(t)=d(t)$.
        }
        \item[(c)]
        {
            Hence, $\dim_F\ker d(T)=\dim_F F[t]w=\deg d(t)$.
        }
    \end{enumerate}
\end{prop}
\begin{proof}
    \hangindent=0.65cm
    \noindent(a)
    Clearly, $F[t]w\subset \ker d(T)$.
    To show the converse inclusion, assume that $y\in\ker d(T)$ and write $y=f(T)v$ for some $f(t)\in F[t]$.
    Then $d(t)f(t)$ annihilates $v$, so $m_T(t)|d(t)f(t)$ and $e(t)|f(t)$.
    Thus, $y\in F[t]$, as desired.

    \noindent(b)
    Note that $d(t)$ annihilates $v$ so $m_w(t)|d(t)$.
    If $m_w(t)$ is a propoer divisor of $d(t)$, then $m_w(t)e(t)$ annihilates $v$ but is a proper divisor of $m_T(t)$, a contradiction.

    \noindent(c)
    It follows from (a) and (b).
\end{proof}