\section{Cyclic decomposition}\label{Cyclic decomposition}

Before studying cyclic decompostion, we first observe the following result which is valid for modules over any rings with the ninzero identities.
\begin{obs}\label{cyclic submodules are isomorphic to the quotients of the ring by an ideal}
    Let $R$ be a ring with the nonzero identity and $M$ be an $R$-module.
    Given an element $x\in M$, define a map $\rho_x: R\rightarrow M$ by $\rho_x(r)=rx$ for $r\in R$.
    \begin{enumerate}
        \item[(a)]
        {
            $\rho_x$ is an $R$-module homomorphism with $\ker\rho_x=\ann_R(x)$ and $\range\rho_x=Rx$.
            Hence, $R/\ann_R(x)$ is a cyclic $R$-module which is isomorphic to $Rx$.
        }
        \item[(b)]
        {
            Conversely, any cyclic $R$-submodule of $M$ is isomorphic to $R/\ann_R(v)$ for some $v\in M$.
            In fact, if $N$ is the cyclic $R$-submodule of $M$ generated by $v\in M$, by considering the $R$-module homomorphism $\rho_v$, we can deduce that $R/\ann_R(v)\approx N$.
        }
    \end{enumerate}
    In particular, if $D$ is a PID. and if we write $\ann_D(x)=(a)$ for a given $x\in M$, then $Dx\approx D/\ann_D(x)=D/(a)$.
\end{obs}

In the preceeding section, we found that a finitely generated torsion $D$-module is an internal direct sum of $D$-submodules whose annihilator ideals are of the form $(p^f)$ for some $p\in\mc{P}$ and a positive integer $f$.
The cyclic decomposition states a decomposition of such $D$-modules.
Hence, in this section, we assume further that $\ann_D(M)=(p^f)$ for some $p\in\mc{P}$ and a positive integer $f$.

Before studying cyclic decomposition theorem, we first observe the following basic lemma.
\begin{lem}
    Suppose that $M$ is a finitely generated torsion $D$-module, where $D$ is a PID., and assume that $\ann_D(M)=(p^f)$ for some $p\in\mc{P}$ and a positive integer $f$.
    \begin{enumerate}
        \item[(a)]
        {
            If $x\in M$ is nonzero, then there is a positive integer $r$ with $r\leq f$ such that $\ann_D(x)=(p^r)$.
        }
        \item[(b)]
        {
            If $M=Dx_1+\cdots+Dx_h$ and $\ann_D(x_i)=({p_i}^{r_i})$ for each $i=1, \cdots, h$, then $f=\max\{r_1, \cdots, r_h\}$.
        }
    \end{enumerate}
\end{lem}
\begin{proof}
    (a) easily follows, since $\ann_D(M)\nmal\ann_D(x)$ so $\ann_D(x)=(p^r)$ for some integer $0\leq r\leq f$.
    
    We now prove (b).
    For simplicity, let $m=\max\{r_1, \cdots, r_h\}$.
    Then $p^m$ annihilates every $D$-linear combination of $\{x_1, \cdots, x_h\}$, so $p^m\in\ann_D(M)=(p^f)$ and $f\leq m$.
    Conversely, since $\ann_D(x_i)\nmal\ann_D(M)$, we have $r_i\leq f$ for each $i=1, \cdots, h$, so $m\leq f$.
\end{proof}

\begin{thm}[Cyclic decomposition theorem (Form \Romannumeral{1})]\label{CDT 1}
    Let $D$ be a PID. and $M$ be a finitely generated torsion $D$-module, and write $\ann_D(M)=(p^f)$ for some $p\in\mc{P}$ and $f\geq 1$.
    Then there exist
    \begin{itemize}
        \item
        {
            a positive integer $h$
        }
        \item
        {
            $x_1, \cdots, x_h\in M$
        }
        \item
        {
            positive integers $r_1, \cdots, r_h$
        }
    \end{itemize}
    satisfying the following properties:
    \begin{enumerate}
        \item[(\romannumeral 1)]
        {
            $M=Dx_1\oplus\cdots\oplus Dx_h$.
        }
        \item[(\romannumeral 2)]
        {
            For each $j=1, \cdots, h$, $\ann_D(x_j)=(p^{r_j})$ with $f=r_1\geq\cdots\geq r_h\geq 1$.
        }
    \end{enumerate}
    Moreover, such integers $h$ and $r_1, \cdots, r_h$ are unique, i.e., they are independent of the choice of $x_1, \cdots, x_h$.
\end{thm}
\begin{proof}
    We prove the existence part by induction on $n$, where $n$ is the minimum number of the elements of a subset of $M$ which generates $M$.
    If $M$ is cyclic, the result is clear, so we may assume that $M$ is not cyclic.
    Suppose that the result holds for any torsion $D$-modules generated by less than $n$-elements.
    And let $\{x_1, y_2, \cdots, y_n\}$ be a subset of $M$ generating $M$ and assume $\ann_D(x_1)=(p^f)$.
    Now consider the quotient $\ol M=M/Dx_1$.

    \textbf{Step 1. Finding appropriate $x_2, \cdots, x_h\in M$}\newline\indent
    Note that $\ol M$ is a torsion $D$-module and $\ol M\neq 0$, since $M$ is not cyclic.
    Because $\ol M$ is generated by $\{\ol{y_2}, \cdots, \ol{y_n}\}$, by induction hypothesis, there is a positive integer $h\geq 2$ and there are elements $\ol{\theta_2}, \ol{\theta_h}\in\ol M$ such that
    \begin{align*}
        \ol M = D\ol{\theta_2} \oplus \cdots \oplus D\ol{\theta_h}.
    \end{align*}
    By \cref{a tool for proving the CDT existence part}, for each $j=2, \cdots, h$, there is an element $x_j\in M$ such that
    \begin{align*}
        \ol{x_j}=\ol\theta_j\quad\textsf{and}\quad\ann_D(x_j)=\ann_D(\ol{\theta_j}).
    \end{align*}

    \textbf{Step 2. Proving the exsitence part}\newline\indent
    In this step, we will prove the following assertion:
    \begin{align*}
        M=Dx_1\oplus\cdots\oplus Dx_h.
    \end{align*}
    \begin{enumerate}
        \item[(1)]
        {
            Given $x\in M$, there are scalars $a_2, \cdots, a_h\in D$ such that $\ol x=a_2\ol{x_2}+\cdots+a_h\ol{x_h}$ (and such $a_j$ is uniquely determined for each $j=2, \cdots, h$), hence $x=a_1x_1+a_2x_2+\cdots+a_hx_h$ for some $a_1\in D$.
        }
        \item[(2)]
        {
            To show that the internal sum of $Dx_i$'s is direct, assume that $a_1x_1+a_2x_2+\cdots+a_hx_h=0$ for some scalars $a_1, a_2, \cdots, a_h\in D$. (Note that $a_i x_i$ is a typical element of $Dx_i$ for each $i=1, \cdots, h$.)
            Then $a_2\ol{x_2}+\cdots+a_h\ol{x_h}=\ol 0$, so $a_j\ol{x_j}=0$ for each $j=2, \cdots, h$.
            Since $\ann_D(x_j)=\ann_D(\ol{\theta_j})$ and $\ol{x_j}=\ol{\theta_j}$, we have $a_j x_j=0$ ($j=2, \cdots, h$).
            It follows that $a_1x_1=0$, so the internal sum of $Dx_i$'s is direct.
        }
    \end{enumerate}

    The proof of the uniqueness part will be given in the end of this section.
\end{proof}
\begin{lem}\label{a tool for proving the CDT existence part}
    In the proof of \cref{CDT 1}, suppose that $\ol\theta\in\ol M\,(\theta\in M)$.
    Then there is an element $y\in M$ such that $\ol y=\xi$ and $\ann_D(y)=\ann_D(\xi)$.
\end{lem}
\begin{proof}
    \textit{Before reading this proof, note that the proof is quite technical.}

    Write $\ann_D(\ol\theta)=(p^r)$, where it is clear that $r\leq f$.
    Choose any $z\in M$ such that $\ol z=\ol\theta$ and write $\ann_D(z)=(p^s)$ (clearly, $r\leq s\leq f$).
    Because $p^r\ol\theta=\ol 0$, $p^r z=bx_1$ for some $b\in D$; because $p^s z=0$, $p^{s-r}b x_1=0$, so $p^f|p^{s-r}b$.
    Therefore, there is $c\in D$ such that $p^{s-r}b=cp^f$ and $b=cp^{f-s+r}$, so $p^r(z-cp^{f-s}x_1)=0$.

    Now define $y=z-cp^{f-s}x_1$.
    As observed in the preceeding paragraph, $p^r y=0$, so $(p^r)\nmal\ann_D(y)$.
    Conversely, (writing $\ann_D(y)=(\alpha)$) because $\alpha y=0$ and $\alpha\ol y=\alpha\ol z$, $\alpha\in\ann_D(\ol\theta)=(p^r)$, proving that $\ann_D(y)=\ann_D(\ol\theta)$.
\end{proof}

Combining the primary decomposition theorem, we obtain the following theorem.
\begin{thm}[Cyclic decomposition theorem (Form \Romannumeral{1}-(\romannumeral 1))]\label{CDT 1 with the PDT}
    Let $D$ be a PID. and $M$ be a finitely generated $D$-module, and write
    \begin{align*}
        \ann_D(N)=(m)\quad\textsf{and}\quad m={p_1}^{f_1}\cdots{p_k}^{f_k},
    \end{align*}
    where each $p_i$ belongs to $\mc{P}$ and $f_i\geq 1$.
    Then there are
    \begin{itemize}
        \item
        {
            positive integers $h_1, \cdots, h_k$
        }
        \item
        {
            sets of module elements $\{x_{ij}\in M\}_{j=1}^{h_j}$ for each $i=1, \cdots, k$
        }
        \item
        {
            sets of positive integers $\{r_{ij}\}_{j=1}^{h_j}$ for each $i=1, \cdots, k$
        }
    \end{itemize}
    satisfying the following properties:
    \begin{enumerate}
        \item[(1)]
        {
            $M=\bigoplus_{i=1}^k\left(\bigoplus_{j=1}^{h_j} Dx_{ij}\right)$.
        }
        \item[(2)]
        {
            $\ann_D(x_{ij})=(p_i^{r_{ij}})$ for each $i$ and $j$.
        }
        \item[(3)]
        {
            $f_i = r_{i1} \geq\cdots\geq r_{ih_i}\geq 1$ for each $i$.
        }
    \end{enumerate}
    Moreover, $\{h_i\}_i$ and $\{r_{ij}\}_{i, j}$ are uniquely determined and are independent of the choice of $\{x_{ij}\}_{i, j}$.
\end{thm}
\begin{proof}
    Clear.
\end{proof}

Combining the primary decomposition theorem together with \cref{cyclic submodules are isomorphic to the quotients of the ring by an ideal}, we obtain the following theorem.

\begin{thm}[Cyclic decomposition theorem (Form \Romannumeral{1}-(\romannumeral 2))]\label{CDT 1-2}
    Let $D$ be a PID. and $M$ be a finitely generated $D$-module, and write
    \begin{align*}
        \ann_D(N)=(m)\quad\textsf{and}\quad m={p_1}^{f_1}\cdots{p_k}^{f_k},
    \end{align*}
    where each $p_i$ belongs to $\mc{P}$ and $f_i\geq 1$.
    Then there are
    \begin{itemize}
        \item
        {
            positive integers $h_1, \cdots, h_k$
        }
        \item
        {
            sets of positive integers $\{r_{ij}\}_{j=1}^{h_j}$ for each $i=1, \cdots, k$
        }
    \end{itemize}
    satisfying the following properties:
    \begin{enumerate}
        \item[(1)]
        {
            $M\approx\bigoplus_{i=1}^k\left(\bigoplus_{j=1}^{h_j} D/({p_i}^{r_{ij}})\right)$.
        }
        \item[(2)]
        {
            $f_i = r_{i1} \geq\cdots\geq r_{ih_i}\geq 1$ for each $i$.
        }
    \end{enumerate}
    Moreover, $\{h_i\}_i$ and $\{r_{ij}\}_{i, j}$ are uniquely determined.
\end{thm}
\begin{proof}
    Clear.
\end{proof}

We call the data which consists of
\begin{eqnarray*}
    m={p_1}^{f_1}\cdots{p_k}^{f_k}&\textsf{(the annihilator ideal of $M$)},\\
    \{h_i\in\bb{Z}^{>0}: 1\leq i\leq k\}&\textsf{(the number of cyclic summands in each primary summand)},\\
    \{r_{ij}\in\bb{Z}^{>0}: 1\leq i\leq k, 1\leq j\leq h_i\}&\textsf{(the annihilator ideal of each cyclic summand)}
\end{eqnarray*}
the invariants of $M$.
Clearly, the finitely generated $D$-module $M$ is completely determined (up to isomorphism) by the invariants.
And the uniqueness part of the cyclic decomposition theorem implies that any two finitely generated $D$-modules with distinct invariants are not isomorphic.
This establishes the collection of the isomorphism types of finitely generated $D$-modules and the collection of invariants.

We are now concerened in proving the uniqueness part.
For this, we introduce the following form of the cyclic decomposition theorem.
\begin{thm}[Cyclic decomposition theorem (Form \Romannumeral{2})]\label{CDT 2}
    Let $D$ be a PID. and $M$ be a finitely generated $D$-module, and write $\ann_D(M)=(m)$.
    Then there are nonzero and nonunit elements $d_1, \cdots, d_r\in D$ satisfying the following properties:
    \begin{enumerate}
        \item[(1)]
        {
            $M \approx D/(d_1) \oplus \cdots \oplus D/(d_r)$.
        }
        \item[(2)]
        {
            $d_1|\cdots|d_r=m$.
        }
    \end{enumerate}
    Moreover, the positive integer $r$ and the elements $d_1, \cdots, d_r\in D$ are determined uniquely (up to unit).
\end{thm}
\begin{proof}
    Consider the isomorphism given in \cref{CDT 1-2} and apply the Chinese remainder theorem.
    (For example, if $m=p_1p_2^3p_3^4$ and
    \begin{align*}
        M \approx D/(p_1) \oplus D/(p_2^2) \oplus D/(p_2^2) \oplus D/(p_2^3) \oplus D/(p_3) \oplus D/(p_3^4),
    \end{align*}
    enumerate the generators of ideals as follows
    \begin{eqnarray*}
                &       &p_1\\
        p_2^2   &p_2^2  &p_2^3\\
                &p_3    &p_3^4    
    \end{eqnarray*}
    to obtain $d_1=p_2^2$, $d_2=p_2^2p_3$, and $d_3=p_1p_2^3p_3^4=m$.
    In fact,
    \begin{eqnarray*}
        M
        &\approx& D/(p_1) \oplus D/(p_2^2) \oplus D/(p_2^2) \oplus D/(p_2^3) \oplus D/(p_3) \oplus D/(p_3^4)
        \\
        &\approx& D/(p_2^2)\oplus\left(D/(p_2^2)\oplus D/(p_3)\right)\oplus\left(D/(p_1)\oplus D/(p_2^3)\oplus D/(p_3^4)\right),
    \end{eqnarray*}
    and the Chinese remainder theorem implies that $D\approx D/(d_1)\oplus D/(d_2)\oplus D/(d_3)$.)
\end{proof}

Some more preparations are required.
In the following propositions, assume that $D$ is a PID. and $p\in\mc{P}$.
\begin{obs}\label{inheritance of a direct decomposition}
    Let $N$ be a $D$-module and suppose that $N_i\leq_D N$ for $i=1, \cdots, r$ and
    \begin{align*}
        N=N_1\oplus\cdots\oplus N_r.
    \end{align*}
    We wish to show that a direct decomposition of $N$ naturally inherits to the $D$-submodule of $N$ annihilated by $p$; in other words, we wish to show that
    \begin{align*}
        \Ann_N(p)=\Ann_{N_1}(p)\oplus\cdots\oplus\Ann_{N_r}(p).
    \end{align*}
    
    The internal sum of $\Ann_{N_i}(p)$ for all $i$ is direct, since each $\Ann_{N_i}(p)$ is a $D$-submodule of $N_i$.
    Also, it is clear that $\Ann_{N_1}(p)+\cdots+\Ann_{N_r}(p)\subset\Ann_N(p)$.
    Thus, it remains to prove the converse inclusion; for this, write $x\in\Ann_N(p)$ as $x=x_1+\cdots+x_r$ with $x_i\in N_i$ for each $i$.
    Then $px_1+\cdots+px_r=0$, implying $x_i\in\Ann_{N_i}(p)$ so that $\Ann_N(p)\subset\Ann_{N_1}(p)+\cdots+\Ann_{N_r}(p)$.
\end{obs}    
\begin{obs}\label{the structure of each Ann(direct summand)}
    If $M$ is a $D$-module, by \cref{CDT 2}, there are nonzero and nonunit elements $d_1, \cdots, d_r\in D$ such that $M\approx D/(d_1)\oplus\cdots\oplus D/(d_r)$.
    In this observation, we study the structure of each $M$-submodule of the direct summand annihilated by $p$.
    In other words, we study the structure of each $\Ann_{D/(d_i)}(p)$.

    Given a nonzero element $d\in D$, consider the cyclic $D$-module $N=D/(d)$.
    \begin{enumerate}
        \item[(a)]
        {
            Suppose $\ol x\in\Ann_N(p)$ for some $x\in D$, where the overline notation is used to denote the quotient by $(d)$.
            It is equivalent to $px\in (d)$, or equivalently, $d|px$.
            \begin{enumerate}
                \item[(\romannumeral 1)]
                {
                    Assume that $p$ divides $d$ and write $d=pb$ for some $b\in D$.
                    Then $\Ann_N(p)=bD/(pb)$, and $\Ann_N(p)$ can be considered a $D/(p)$-vector space, for $\Ann_N(p)$ is annihilated by $p$.
                }
                \item[(\romannumeral 2)]
                {
                    Assume that $p$ does not divide $d$.
                    Then $d|x$, so $\ol x=\ol 0$ and $\Ann_N(p)=0$.
                }
            \end{enumerate}
        }
        \item[(b)]
        {
            Under the situation of $p|d$, we wish to find the $D/(p)$-dimension of $\Ann_N(p)$.
            Construct a map $\phi: D\rightarrow bD/(pb)=\Ann_N(p)$ by $\phi(a)=\ol{ba}$ for $a\in D$.
            Because $\phi$ is a surjective $D$-module homomorphism with $\ker\phi=(p)$, we have $\Ann_N(p)=bD/(pb)\approx D/(p)$.
            Therefore, if $p$ divides $d$, then $\Ann_N(p)$ is a 1-dimensional $D/(p)$-vector space; if $p$ does not divide $d$, then $\Ann_N(p)=0$.
        }
    \end{enumerate}
\end{obs}
\begin{obs}
    Given a nonzero element $b\in D$, consider the (cyclic) $D$-module $N=D/(pb)$ and its $D$-submodule $pD/(pb)$.
    Construct a map $\psi: D\rightarrow pD/(pb)$ defined by $\psi(a)=\ol{pa}$, which is clearly a $D$-module homomorphism.
    Since $\psi$ is surjective and $\ker\psi=(b)\nmal D$, we have $pD/(pb)\approx D/(b)$.
    Then $pD/(pb)\approx D/(b)$.
\end{obs}

Now we prove the uniqueness part of the cyclic decomposition theorem.
In particular, we prove the uniqueness part of \cref{CDT 2}; this proves the uniqueness part of \cref{CDT 1-2}.
\begin{proof}[Proof of the uniqueness part of \cref{CDT 2}]
    We prove the uniqueness part by induction on the number of irreducible divisors of $m$, counting multiplicity.
    Suppose that there are two isomorphic types of $M$:
    \begin{eqnarray*}
        M
        &\approx& D/(d_1)\oplus\cdots\oplus D/(d_r)\\
        &\approx& D/(c_1)\oplus\cdots\oplus D/(c_s)
    \end{eqnarray*}
    with $d_1|\cdots|d_r=m$ and $c_1|\cdots|c_s=m$, and all $d_i$'s and $c_j$'s being nonzero and nonunit.
    Choose an irreducible element $p\in\mc{P}$ such that $p|d_1$.

    \textbf{Step 1. Deducing $r=s$}\newline\indent
    By \cref{inheritance of a direct decomposition}, we have
    \begin{align*}
        \Ann_M(p)\approx\Ann_{D/(d_1)\oplus\cdots\oplus D/(d_r)}(p)=\Ann_{D/(d_1)}(p)\oplus\cdots\oplus\Ann_{D/(d_r)}(p).
    \end{align*}
    Since $p|d_i$ for all $i$, writing $\ol D=D/(p)$, we have $r=\dim_{\ol D}(\Ann_M(p))$, thus $r$ is the number of indices $j$ for which $p|c_j$.
    So $r\leq s$; by symmetry, we have $s\leq r$.
    Thus, $r=s$ and $p|c_j$ for all $j$.

    \textbf{Step 2. Deducing $c_i\sim_\times d_i$ for all $i$}\newline\indent
    If $m$ has (up to unit) one irreducible divisor, then $c_1\sim_\times m\sim_\times d_1$, so the uniqueness is proved.
    To proceed the proof by induction, we consider the $D$-submodule $pM$ of $M$.
    Writing $d_i=pd_i'$ and $c_i=dc_i'$ for each $i$, we have
    \begin{align*}
        pM\approx pD/(pd_1')\oplus\cdots\oplus pD/(pd_r')\approx D/(d_1')\oplus\cdots\oplus D/(d_r')
    \end{align*}
    and
    \begin{align*}
        pM/\approx D/(c_1')\oplus\cdots\oplus D/(c_r').
    \end{align*}
    Then $d_r'=c_r'$ has a fewer irreducible divisors than $d_r=m=c_r$.
    Hence, by induction hypothesis, $d_i'\sim_\times c_i'$ for all $i$, thus $d_i\sim_\times c_i$ for all $i$.
\end{proof}