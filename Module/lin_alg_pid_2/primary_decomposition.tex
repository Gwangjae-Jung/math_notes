\section{Primary decomposition}

We start our argument with the following easy, but important discovery.
\begin{obs}
    If $M$ is a finitely generated \textit{torsion} module over the PID. $D$, then $\ann_D(M)\neq 0$.
    In fact, if $M=\genone{x_1, \cdots, x_n}$ and $\ann_D(x_i)=(m_i)$ for some nonzero element $m_i\in D$ for each integer $1\leq i\leq n$, then
    \begin{align*}
        \ann_D(M)=(\textsf{lcm}\{m_1, \cdots, m_n\}).
    \end{align*}
\end{obs}
\begin{proof}
    Suppose that $M$ is generated by $\{x_1, \cdots, x_n\}\subset M$.
    Since $M$ is a \textit{torsion} module, for each integer $1\leq i\leq n$, there is a nonzero element $r_i\in D$ such that $r_i x_i=0$.
    Since $D$ is integral, the product $r$ of $r_i$'s is nonzero, thus $\ann_D(M)\neq 0$, for $r\in\ann_D(M)$.

    We now justify the second assertion.
    Letting $\ann_D(M)=(a)$ for some (nonzero) element $a\in D$ and $m=\textsf{lcm}\{m_1, \cdots, m_n\}$, $m|a$, because $a$ annihilates each $x_i$ so that $m_i|a$.
    Conversely, $a|m$, because $m$ annihilates any $D$-linear combination of $\{x_1, \cdots, x_n\}$.
\end{proof}

\begin{exmp}
    We introduce some typical examples of finitely generated torsion modules over PID.s.
    Further theory on such objects will be discussed in \Cref{applications of structure theorems}.
    \begin{enumerate}
        \item[(a)]
        {
            (Finite abelian groups)
            Suppose that $A$ is an abelian group of finite order $n$.
            Considering $A$ as a $\bb{Z}$-module, $n\in\ann_\bb{Z}(A)$, so we may write $\ann_\bb{Z}(A)=(m)$ for some nonzero integer $m$.
            Clearly, $m$ divides $n$.
        }
        \item[(b)]
        {
            ($F$-vector spaces as $F[t]$-modules)
            As we have done, we only consider finite dimensional vector spaces only; let $V$ be a finite dimensional vector space over $F$ and let $T$ be an $F$-linear operator on $V$.
            Then the annihilator ideal $I_T$ of $T$ in $F[t]$ is nonzero and is generated by a unique monic polynomial over $F$, called the minimal polynomial of $T$.
            In fact, by the definition of the annihilator ideal of $T$, it is clear that $\ann_{F[t]}(V)=I_T$, so $\ann_{F[t](V)}=(m_T(t))$.
        }
    \end{enumerate}
\end{exmp}

\begin{nota}
    Given a finitely generated torsion $D$-module $M$, let $\mc{P}$ denote a complete set of representatives of irreducible elements in $D$ modulo units.
    For example, if $D=\bb{Z}$, we set $\mc{P}$ be the set of all positive prime numbers; if $D=F[t]$, we set $\mc{P}$ be the set of all monic irreducible polynomials over $F$.

    Also, given $p\in\mc{P}$, we call the following set
    \begin{align*}
        M(p)=\{x\in M: p^i x=0\textsf{ for some positive integer }i\}=\bigcup_{i=1}^\infty \Ann_M(p^i)
    \end{align*}
    the $p$-torsion \color{brown}submodule \color{black} of $M$.
\end{nota}

We now start preparation of proving the primary decomposition theorem.
\begin{prop}\label{the smallest annihilated submodule deterimines if the p-torsion submodule is nonzero}
    $M(p)\neq 0$ if and only if $\Ann_M(p)\neq 0$.
\end{prop}
\begin{proof}
    It suffices to prove that $\Ann_M(p)\neq 0$ if $M(p)\neq 0$.
    Assume that $\Ann_M(p)=0$ and let $k$ be \textit{the smallest} positive integer such that $\Ann_M(p^k)\neq 0$ (such $k$ exists, since $M(p)\neq 0$).
    If $x$ is a \textit{nonzero} element of $\Ann_M(p^k)$, then $px\in\Ann_M(p^{k-1})$, so $px=0$ by the minimality of $k$.
    Then $x\in\Ann_M(p)$, a contradiction.
    Therefore, $\Ann_M(p)\neq 0$.
\end{proof}
\begin{prop}\label{a half of the weak Cauchy's theorem}
    Let $M$ be a finitely generated torsion $D$-module and $p$ be an element of $\mc{P}$, and write $\ann_D(M)=(m)$.
    If $p|m$, then $\Ann_M(p)\neq 0$, so $M(p)\neq 0$.
    Hence, if $M(p)=0$, then $p\nmid m$.
\end{prop}
\begin{proof}
    Suppose that $p$ is an element of $\mc{P}$ dividing $m$.
    We may write $m=pk$ for some $k\in D$, and there is an element $x\in M$ such that $mx=0$ but $kx\neq 0$; otherwise, since $k$ annihilates $M$, we have $(k)=\ann_D(M)=(p)$, a contradiction.
    Then $kx\in\Ann_M(p)\leq M(p)$, proving the lemma.
\end{proof}
\begin{rmk}
    The converse of \cref{a half of the weak Cauchy's theorem} will be proved in this section after proving the primary decomposition theorem.
\end{rmk}

\begin{thm}[Primary decomposition theorem]\label{primary decomposition theorem}
    Let $D$ be a PID and $M$ be a nonzero finitely generated torsion $D$-module.
    Write $\ann_D(M)=(m)$ and let $m={p_1}^{f_1}\cdots {p_k}^{f_k}$ be the factorization of $m$ into irreducible factors, where $p_i\in\mc{P}$ and $f_i\geq 1$ for all $i=1, \cdots, k$.
    \begin{enumerate}
        \item[(a)]
        {
            $\Ann_M(m)=M=\Ann_M({p_1}^{f_1})\oplus\cdots\oplus\Ann_M({p_k}^{f_k})$.
        }
        \item[(b)]
        {
            $\ann_D(\Ann_M({p_i}^{f_i}))=({p_i}^{f_i})$ for all $i=1, \cdots, k$.
        }
    \end{enumerate}
\end{thm}
\begin{lem}[Primary decomposition theorem]\label{primary decomposition theorem_lemma}
    In \cref{primary decomposition theorem}, write $m=ab$ for some $a, b\in D$, where $(a, b)=D$.
    \begin{enumerate}
        \item[(a)]
        {
            $\Ann_M(m)=M=\Ann_M(a)\oplus\Ann_M(b)$.
        }
        \item[(b)]
        {
            $\ann_D(\Ann_M(a))=(a)$ and $\ann_D(\Ann_M(b))=(b)$.
        }
    \end{enumerate}
\end{lem}
\begin{proof}[Proof of \cref{primary decomposition theorem_lemma}]
    Since $a$ and $b$ are relatively prime, there are elements $s, t\in D$ such that $as+bt=1$.
    Hence, for any $x\in M$, $x=1x=b(tx)+a(sx)\in\Ann_M(a)+\Ann_M(b)$.
    Furthermore, if $x\in\Ann_M(a)\cap\Ann_M(b)$, then $ax=bx=0$, so $x=1x=s(ax)+t(bx)=0$, so $M=\Ann_M(a)\oplus\Ann_M(b)$, proving (a).

    To prove (b), note that it is clear that $a\in\ann_D(\Ann_M(a))$ and $b\in\ann_D(\Ann_M(b))$.
    Thus, letting $\ann_D(\Ann_M(a))=(m_a)$ and $\ann_D(\Ann_M(b))=(m_b)$ for some $m_a, m_b\in D$, we find that $m_a|a$ and $m_b|b$, so $m_a$ and $m_b$ are relatively prime.
    Thus,
    \begin{center}
        if one can show that $ab=m\sim_\times m_a m_b$, then it follows that $(a)=(m_a)$ and $(b)=(m_b)$.
    \end{center}
    From (a), we have $\ann_D(M)=(\textsf{lcm}\{m_a, m_b\})=(m_a m_b)$, so $m\sim_times m_a m_b$, as desired.
\end{proof}
\begin{proof}[Proof of \cref{primary decomposition theorem}]
    Use the results of \cref{primary decomposition theorem_lemma} inductively.
\end{proof}
\begin{cor}
    Use the notations in \cref{primary decomposition theorem}.
    \begin{enumerate}
        \item[(a)]
        {
            If $p\in\mc{P}$ and $p\nmid m$, then $M(p)=0$.
        }
        \item[(b)]
        {
            $\Ann_M({p_i}^{f_i})=M(p_i)$ and $M=\bigoplus_{p\in\mc{P}} M(p)$.
        }
        \item[(c)]
        {
            If $f_i\leq e_i$ for all $i=1, \cdots, k$, then $\Ann_M({p_i}^{f_i})=\Ann_M({p_i}^{e_i})$.
        }
    \end{enumerate}
\end{cor}
\begin{proof}
    To prove (a), assume $x\in M(p)$.
    For each $i=1, \cdots, k$, there is a unique element $x_i\in\Ann_M({p_i}^{f_i})$ such that $x=x_1+\cdots+x_k$.
    Since $p^l x=0$ for some positive integer $l$, $p^lx_i=0$ for each $i$; because $\textsf{gcd}\{p^l, {p_i}^{f_i}\}\sim_\times 1$ for each $i$, we have $x_i=0$ for each $i$, so $x=0$.

    To prove (b), it suffices to prove that $M(p_i)\leq\Ann_M({p_i}^{f_i})$ for all $i=1, \cdots, k$.
    Suppose that $x\in M(p_i)$ and write $x=x_1+\cdots+x_k$ as in the proof of (a).
    If $l$ is a positive integer such that $p_i^l x=0$, we have $p_i^l x_j=0$ for all $j=1, \cdots, k$.
    In particular, if $j\neq i$, then $x_j=0$, for $p_i$ and $p_j$ are relatively prime.
    Hence, $x=x_i\in\Ann_M({p_i}^{f_i})$ and $M(p_i)=\Ann_M({p_i}^{f_i})$ for each $i$.
    And we have $M=\bigoplus_{i=1}^k\Ann_M({p_i}^{f_i})=\bigoplus_{p\in\mc{P}} M(p)$ by (a).

    (c) naturally follows from (b), since $M(p_i)=\Ann_M({p_i}^{f_i})\leq\Ann_M({p_i}^{e_i})\leq M(p_i)$.
\end{proof}
\begin{rmk}[The weak Cauchy's theorem]
    (a) of the preceeding corollary and \cref{a half of the weak Cauchy's theorem} directly yields the following equivalence, which is called the weak Cauchy's theorem:
    \begin{center}
        Suppose $p\in\mc{P}$.
        Then $p$ divides $m$ if and only if the $p$-torsion submodule of $M$ is nonzero.
    \end{center}
    In short, the $p$-torsion submodule is nonzero if and only if $p$ divides $m$.
    Moreover, by considering \cref{the smallest annihilated submodule deterimines if the p-torsion submodule is nonzero}, we can establish the following equivalence:
    \begin{enumerate}
        \item[(\romannumeral 1)]
        {
            $p$ divides $m$.
        }
        \item[(\romannumeral 2)]
        {
            The $p$-torsion submodule of $M$ is nonzero.
        }
        \item[(\romannumeral 3)]
        {
            The $D$-submodule of $M$ annihilated by $p$ is nonzero.
        }
    \end{enumerate}
\end{rmk}

In fact, \cref{primary decomposition theorem} can be generalized to the following proposition, where there is a product of pairwise relatively prime elements which is divisible by $m$.
\begin{prop}
    Let $D$ be a PID and $M$ be a nonzero finitely generated torsion $D$-module.
    Write $\ann_D(M)=(m)$ and suppose that $m|a_1\cdots a_k$ for some nonzero and nonunit pairwise relatively prime elements $a_1, \cdots, a_k\in D$.
    \begin{enumerate}
        \item[(a)]
        {
            $\Ann_M(m)=M=\Ann_M(a_1)\oplus\cdots\oplus\Ann_M(a_k)$.
        }
        \item[(b)]
        {
            Writing $\ann_D(\Ann_M(a_i))=(m_i)$ for each $i=1, \cdots, k$, we have $m_i|a_i$ for each $i$.
        }
    \end{enumerate}
\end{prop}
\begin{proof}
    In fact, it suffices to prove the proposition for $k=2$; the general case follows from (b) by induction on $k$.
    
    Assume $k=2$, and let $s, t\in D$ be elements such that $sa_1+ta_2=1$.
    If $x\in M$, then $x=1x=ta_2x+sa_1x\in\Ann_M(a_1)+\Ann_M(a_2)$, so $M=\Ann_M(a_1)+\Ann_M(a_2)$.
    To prove that the sum is direct, assume that $y\in\Ann_M(a_1)\cap\Ann_M(a_2)$.
    Then $y=1y=sa_1y+ta_2y=0+0=0$, so the internal sum is direct.

    To show that $m_i|a_i$, it suffices to justify that $a_i\in (m_i)=\ann_D(\Ann_M(a_i))$, which is obvious.
\end{proof}