\section{Applications of structure theorems}\label{applications of structure theorems}

As assumed starting from the preceeding chapter, we assume that $D$ is a PID. and $M$ is a finitely generated $D$-module (by here, $M$ need not be a \textit{torsion} module).
By the structure theorem, there is a decomposition of $M$ into a torsion $D$-submodule and a free $D$-submodule.
Moreover, if $M=T_1\oplus\mc{F}_1=T_2\oplus\mc{F}_2$ are such decompositions ($T_i$ is a torsion $D$-submodule of $M$ and $\mc{F}_i$ is a free $D$-submodule of $M$ for $i=1, 2$), then $T_1=M_\tor=T_2$ and $\mc{F}_1\approx M/M_\tor\approx \mc{F}_2$.
Since the free part of $M$ has an obvious structure (being isomorphic to $R^k$ for some nonnegative integer $k$), the structure of $M_\tor$ needs to be investigated to fully understand the structure of $M$.
Thus, from now on, we assume that $M$ is a finitely generated torsion $D$-module.

The first tool required to understand the structure of $M$ is the primary decomposition theorem, which states that a finitely generated torsion module over a PID. can be understood as the internal direct sum of the $p$-torsion submodules for $p\in\mc{P}$; to be precise, if $\ann_M(D)=(m)$ with the factorization $m={p_1}^{f_1}\cdots{p_k}^{f_k}$, then $M=\bigoplus_{i=k}\Ann_M({p_i}^{f_i})=\bigoplus_{p\in\mc{P}} M(p)$.
And the second tool (or together with the first tool) is the cyclic decomposition theorem, which determines the isomorphic type of the given torsion module.
In fact, isomorphic torsion $D$-modules share the same invariants and non-isomorphic torsion $D$-modules have distinct invariants.

\subsection{Finitely generated abelian groups}

By the structure theorem, given a finitely generated abelian group $A$, we have $A=A_\tor\oplus\mc{F}$ for some free $\bb{Z}$-submodule of $A$.
Thus, we may assume that $A$ is a torsion $\bb{Z}$-module.
\begin{exmp}
    Let $A$ be an abelian group of order $72=2^3\cdot 3^2$.
    Letting $\ann_\bb{Z}(A)=(m)$ for some positive integer $m$, since $m|n$, $m=2^a\cdot 3^b$, where $a$ and $b$ are integers such that $1\leq a\leq 3$ and $1\leq b\leq 2$.\footnote{Note that $a, b$ cannot be zero, because there are elements of $A$ of order 2 and 3, respectively.}
    By the primary decomposition theorem, $A=A(2)\oplus A(3)$ and $\ann_\bb{Z}(A(2))=(2^a)$ and $\ann_\bb{Z}(A(3))=(3^b)$.
    Using the cyclic decomposition theorem, the isomorphic types of $A(2)$ and $A(3)$ are given as follows:
    \begin{eqnarray*}
        A(2) & : &
            Z_2 \oplus Z_2 \oplus Z_2\,(a=1),\;
            Z_2 \oplus Z_4\,(a=2),\;
            Z_8\,(a=3)\\
        A(3) & : &
            Z_3 \oplus Z_3\,(b=1),\;
            Z_9\,(b=2).
    \end{eqnarray*}
    Hence, there are 6 isomorphic types for $A$.

    In fact, if $A$ is an abelian group of order ${p_1}^{f_1}\cdots{p_k}^{f_k}$ with $p_1, \cdots, p_k$ being pairwise distinct positive prime numbers and $f_i\geq 1$ for all $i$, then $\ann_\bb{Z}(A)=(m)$ with $m={p_1}^{e_1}\cdots{p_k}^{e_k}$ for some $1\leq e_i\leq f_i$ for each $i$ and $A=\bigoplus_{i=1}^k A(p_i)$ with $\ann_\bb{Z}(A(p_i))=({p_i}^{e_i})$ for each $i$.
    Because there are $\pi(f_i)$ isomorphic types for each $A(p_i)$, there are $\pi_(f_1)\cdots\pi(f_k)$ isomorphic types for $A$.
\end{exmp}

\begin{prop}
    Let $A$ be a finite abelian group with $\ann_\bb{Z}(A)=(m)$ for some positive integer $m$.
    Then $A$ is cyclic if and only if $|A|=m$.
\end{prop}
\begin{proof}
    Note that it is alwasy true that $m$ divides the order of $A$.
    If $A$ is cyclic, then there is an element $x$ of $A$ whose order is $|A|$, so $m=|A|$.
    Conversely, if $m=|A|$, then there is an element $y$ of $A$ annihiated by $m$ but not by any proper divisor of $m$ (otherwise, $\ann_\bb{Z}(A)$ would properly contain $(m)$, a contradiction).
    Then $\genone{y}$ is a subgroup of $A$ of order $|A|$, so $A$ is cyclic.
\end{proof}

We now introduce an application of the above proposition in the field theory.
\begin{prop}
    Let $F$ be a field and $G$ be a finite subgroup of the multiplicative group $F^\times$.
    Then $G$ is a cyclic group.
\end{prop}
\begin{proof}
    Let $n$ be the order of $G$.
    Since $G$ is abelian, by the cyclic decomposition theorem, there are positive integers $d_1, \cdots, d_k$ with $d_1|\cdots|d_k$ such that $G\approx Z_{d_1}\times\cdots\times Z_{d_k}$ and $\ann_\bb{Z}(G)=(d_k)$.
    Hence, every element of $G$ is a root of the polynomial $x^{n_k}-1$, which has at most $n_k$-distinct roots.
    Therefore, $k=1$ and $G$ is cyclic.
\end{proof}

\subsection{Vector spaces over fields}

Let $V$ be an $n$-dimensional vector space over a field $F$ and let $T$ be an $F$-linear operator on $V$.
Write
\begin{align*}
    \phi_T(t)=p_1(t)^{e_1}\cdots p_k(t)^{e_k}\quad\textsf{and}\quad m_T(t)=p_1(t)^{f_1}\cdots p_k(t)^{f_k},
\end{align*}
where $p_1(t), \cdots, p_k(t)$ are pairwise distinct monic irreducible polynomial over $F$ and $1\leq f_i\leq e_i$ for each $i=1, \cdots, k$.

\subsubsection*{Applying the primary decomposition theorem on $V$}
By the primary decomposition theorem, $V=\bigoplus_{i=1}^k\Ann_V((p_i(t)^{f_i}))=\bigoplus_{i=1}^k\ker(p_i(T)^{f_i})$.
For notational convinience, let $W_i=\Ann_V(p_i(t)^{f_i})=\ker(p_i(T)^{f_i})$ and $T_i=T|_{W_i}$ for each $i=1, \cdots, k$.
Then, whenever $\mc{B}_i$ is an $F$-basis of $W_i$ and $\mc{B}=\bigsqcup_{i=1}^k\mc{B}_i$, then
\begin{align*}
    [T]^\mc{B}_\mc{B}=\begin{pmatrix}
        [T_1]^{\mc{B}_1}_{\mc{B}_1}&&&\\
        &[T_2]^{\mc{B}_2}_{\mc{B}_2}&&\\
        &&\ddots&\\
        &&&[T_k]^{\mc{B}_k}_{\mc{B}_k}
    \end{pmatrix}.
\end{align*}
Hence, $m_T(t)=m_{T_1}(t)\cdots m_{T_k}(t)$ and $\phi_T(t)=\phi_{T_1}(t)\cdots\phi_{T_k}(t)$.
The following results can also be deduced.
\begin{enumerate}
    \item[(\romannumeral 1)]
    {
        $\ann_{F[t]}(\Ann_V(p_i(t)^{f_i}))=(p_i(t)^{f_i})$, so $m_{T_i}(t)=p_i(t)^{f_i}$.
        Also, $\phi_{T_i}=p_i(t)^{e_i}$, so $\dim_F(W_i)=\deg(p_i(t)^{e_i})=e_i\deg(p_i(t))$.
    }
    \item[(\romannumeral 2)]
    {
        Whenever $g_i\geq f_i$, we have $\Ann_V(p_i(t)^{f_i})=V(p_i(t))=\Ann_V(p_i(t)^{g_i})$.
        Thus, in particular, $W_i=\ker(p_i(T)^{e_i})$.
    }
\end{enumerate}
Before considering the cylcic decomposition, we deduce a criterion to determine if a given linear operator is diagonalizable.
\begin{cor}[Diagonalizability]
    $T$ is diagonalizable over $F$ if and only if the minimal polynomial of $T$ splits completely over $F$.
\end{cor}
\begin{proof}
    If $T$ is diagonalizable over $F$ and has eigenvalues $\lambda_1, \cdots, \lambda_k$, then $(t-\lambda_1)\cdots(t-\lambda_k)$ is the minimal polynomial of $T$.
    Conversely, if the minimal polynomial of $T$ splits completely over $F$, then the primary decomposition of $V$ with respect to $T$ is the eigenspace decomposition of $V$, i.e., $V$ is diagonalizable.
\end{proof}

\subsubsection*{Applying the cyclic decomposition theorem on each $W_i$}
After determining the primary decomposition of $V$ with respect to a given linear operator $T$, for each $W_i$, we can apply the first form of cyclic decomposition: For each primary summand $W_i$,
\begin{enumerate}
    \item[(\romannumeral 1)]
    {
        there are $T$-invariant subspaces $U_{i 1}, \cdots, U_{i h_i}$ of $W$ such that $W_i = U_{i 1}\oplus\cdots\oplus U_{i h_i}$.
        }
    \item[(\romannumeral 2)]
    {
        Writing $\phi_{T|_{U_{i j}}}(t)=m_{T|_{U_{i j}}}(t)=p(t)^{r_{i j}}$ for each $j=1, \cdots, h_i$,  there are unique integer $h_i$ and $r_{i 1}, \cdots, r_{i h_i}$ such that $f_i=r_{i 1}\geq\cdots\geq r_{i h_i}\geq 1$.
    }
\end{enumerate}
As illustrated in the preceeding section, the invariants
\begin{align*}
    \{m_T(t), h_i, r_{ij}: 1\leq i\leq k, 1\leq j\leq h_i\}
\end{align*}
determines the isomorphic types of $V$.
In particular, since $\dim_F W_i=\sum_{j=1}^{h_i}\dim_F U_{ij}=\sum_{j=1}^{h_i}(r_{ij}\deg p_i(t))$, the invariants determines the isomorphic types of $V$.
In fact, the invariants determines the similarity class of the operator $T$.
\begin{prop}
    The invariants determine the similarity class of $T$.
\end{prop}
\begin{proof}
    As explained above, from the invariants we can restore the dimension of $V$.
    Considering a matrix representation of $T$ restricted to the $T$-cyclic subspace with the minimal polynomial $p_i(t)^{r_{ij}}$ for each $i$ and $j$, there is an $F$-basis $\mc{B}_{i j}$ of the $T$-cyclic subspace such that
    \begin{align*}
        [T|_{U_{i j}}]^{\mc{B}_{i j}}_{\mc{B}_{i j}}=C(p_i^{r_{i j}}).
    \end{align*}
    This determines a matrix representation of $T$ restricted to each primary summand up to similarity transform, hence determines a similarity class of $T$.
\end{proof}
\begin{cor}
    If it is given that the characteristic polynomial of $T$ is $p_1(t)^{e_1}\cdots p_k(t)^{e_k}$, there are $\pi(e_1)\cdots\pi(e_k)$-distinct similarity classes of $T$.
\end{cor}

In \cref{T-cyclic basic}, it is proved that if $V$ is $T$-cyclic then $\phi_T(t)=m_T(t)$.
The statement and the proof of the converse is given as follows.
\begin{prop}\label{char_poly and min_poly are the same implies cyclicity}
    Suppose that the characteristic polynomial and the minimal polynomial of $T$ are the same.
    Then $V$ is $T$-cyclic, i.e., $V$ is a cyclic $F[t]$-module.
\end{prop}
\begin{proof}
    Using the notation of in \cref{CDT 1 with the PDT}, we have $e_i=f_i$ and $h_i=1$ for all $i=1, \cdots, k$.
    Therefore, by the Chinese remainder theorem
    \begin{align*}
        V\approx \frac{F[t]}{(p_1(t)^{f_1})}\oplus\cdots\oplus \frac{F[t]}{(p_k(t)^{f_k})}\approx \frac{F[t]}{(m_T(t))},
    \end{align*}
    so $V$ is a cyclic $F[t]$-submodule.
\end{proof}

\begin{prob}
    Suppose that $E/F$ be a field extension and $A, B\in\mc{M}_{n, n}(F)$.
    Show that $A\sim B$ over $F$ if $A\sim B$ over $E$.
\end{prob}
\begin{sol}
    It suffices to show that the invariants computed over $E$ and those computed over $F$ are equal; then it is naturally deduced that the invariants of $A$ and $B$ computed over $F$ are equal so $A$ and $B$ are similar over $F$.
    Let $M$ be the rational canonical form of $A$ computed over $F$, which is unique up to similarity transform.
    Since $M$ can be considered a rational canonical form of $A$ computed over $E$, by the uniqueness of the invariants, $M$ is similar to the rational canonical form of $A$ computed over $E$.
    This implies that the invariants of $A$ computed over $E$ and $F$ are equal.
\end{sol}
\begin{rmk}
    In fact, in the above solution, we proved a more general result: the invariants computed over a field and those computed over an extended field are identical.
    Transitivity of similarity is due to this general result and the uniqueness of cyclic decomposition.
    By the general result, it can easily be deduced that the minimal polynomial of a linear operator over a base field and the minimal polynomial computed over a subfield are equal.
\end{rmk}

\subsection{Jordan canonical form}
Assume that $F$ is algebraically closed.
Then we can write $m_T(t)=(t-\lambda_1)^{f_1}\cdots(t-\lambda_k)^{f_k}$ and $\phi_T(t)=(t-\lambda_1)^{e_1}\cdots(t-\lambda_k)^{e_k}$.
Letting $U_{i j}$ be the $(i, j)$-th $T$-cyclic subspace, define $N_{i j}=(T-\lambda_1\id{V})|_{U_{i j}}$.
Then there is a $T$-cyclic basis $\mc{B}_{i j}$ such that
\begin{align*}
    [N_{i j}]^{\mc{B}_{i j}}_{\mc{B}_{i j}}=\begin{pmatrix}
        0   &       &       &       &   \\
        1   &   0   &       &       &   \\
            &   1   &   0   &       &   \\
            &       &\ddots &\ddots &   \\
            &       &       &   1   &   0
    \end{pmatrix}\in\mc{M}_{r_{i j}, r_{i j}}(F).
\end{align*}
(By reversing $\mc{B}_{i j}$, we can obtain the transposed nilpotent matrix.)
Hence,
\begin{align*}
    [T_{U|_{i j}}]^{\mc{B}_{i j}}_{\mc{B}_{i j}}=\begin{pmatrix}
        \lambda_i   &           &           &       &   \\
            1       &\lambda_i  &           &       &   \\
                    &     1     &\lambda_i  &       &   \\
                    &           &\ddots     &\ddots &   \\
                    &           &           &   1   &\lambda_i
    \end{pmatrix}\in\mc{M}_{r_{i j}, r_{i j}}(F),
\end{align*}
and gathering all the (reversed) bases, we obtain the Jordan canonical form of $T$.

\begin{obs}
    Let $\mc{C}$ be the $F$-basis of $V$ obtained by reversing an $F$-basis $\mc{B}$ of $V$.
    If $[T]^\mc{B}_\mc{B}=(a_{i, j})$, where $1\leq i\leq n$ and $1\leq j\leq n$, then
    \begin{align*}
        [T]^\mc{C}_\mc{C}=\begin{pmatrix}
            a_{n, n}    &   a_{n, n-1}      &   \cdots  &   a_{n, 1}    \\
            a_{n-1, n}  &   a_{n-1, n-1}    &   \cdots  &   a_{n-1, 1}  \\
            \vdots      &   \vdots          &   \ddots  &   \vdots      \\
            a_{1, n}    &   a_{1, n-1}      &   \cdots  &   a_{1, 1}
        \end{pmatrix}.
    \end{align*}
    In particular, applying this idea to a Jordan canonical form, we find that a Jordan canonical form is similar to its transposition over $F$; implying that $A\sim A^T$ over $F$.
\end{obs}

\begin{prob}[Exercise 12.2.15, \textit{Abstract Algebra}, third edition]
    Determine up to similarity all 2$\times$2 matrices of order 4 over $\bb{Q}$ and over $\bb{C}$, respectively.
\end{prob}
\begin{sol}
    Since $A$ satisfies the polynomial $t^4-1=(t-1)(t+1)(t^2+1)$ over $\bb{Q}$, the minimal polynomial of $A$ over $\bb{Q}$ is among the following polynomials over $\bb{Q}$ (note that the degree of the minimal polynomial of $A$ over $\bb{Q}$ is at most 2):
    \begin{align*}
        t-1,\quad t+1,\quad (t+1)(t-1),\quad t^2+1
    \end{align*}
    The first three cases induce the rational canonical forms $\begin{pmatrix}1&0\\0&1\end{pmatrix},\,\begin{pmatrix}-1&0\\0&-1\end{pmatrix},\,\begin{pmatrix}1&0\\0&-1\end{pmatrix}$ over $\bb{Q}$, respectively, which are not of order 4.
    The last case induces the rational canonical form $\begin{pmatrix}0&-1\\1&0\end{pmatrix}$ over $\bb{Q}$ of order 4.

    When $A$ is understood as a matrix over $\bb{C}$, then the minimal polynomial of $A$ over $\bb{C}$ divides $(t-1)(t+1)(t-i)(t+i)$.
    Among such cases, the only choices of the minimal polynomial $m(t)$ of $A$ over $\bb{C}$ for which $A$ is of order 4 are given as follows:
    \begin{eqnarray*}
        \textsf{(\romannumeral 1)}&\quad&m(t)=t\pm i,\, A\sim\begin{pmatrix}\pm i&0\\0&\pm i\end{pmatrix}\quad\textsf{(2 distinct similarity classes)}\\
        \textsf{(\romannumeral 2)}&\quad&m(t)=(t\pm i)(t\pm 1),\, A\sim\begin{pmatrix}\pm i&0\\0&\pm 1\end{pmatrix}\quad\textsf{(4 distinct similarity classes)}
    \end{eqnarray*}
\end{sol}

Read also \textit{The conjugacy classes of $GL(2, \bb{F}_q)$}, written by Harold Cooper.