\section{Some problems in module theory}

\begin{prob}
    Let $p$ be a positive prime number and let $n$ be a positive integer.
    And let $F: \bb{F}_{p^n}\rightarrow\bb{F}_{p^n}$ be the Frobenius map.
    \begin{enumerate}
        \item[(a)]
        {
            Note that $F\in\aut{\bb{F}_{p^n}/\bb{F}_p}$.
            Show that the order of $F$ is $n$.
        }
        \item[(b)]
        {
            Understand $\bb{F}_{p^n}$ as an $n$-dimensional vector space over $\bb{F}_p$.
            Find the rational canonical form and the Jordan canonical form of $F$.
        }
    \end{enumerate}
\end{prob}
\begin{sol}
    \begin{enumerate}
        \item[(a)]
        {
            If $\alpha$ is a primitive element of $\bb{F}_{p^n}/\bb{F}_p$, then the order of $\alpha\in\bb{F}_{p^n}^\times$ is $n$.
        }
        \item[(b)]
        {
            By (a), $F$ is satisfied by the polynomial $f(t)=t^n-1\in\bb{F}_p[t]$.
            We wish to justify that $f(t)$ is the minimal polynomial of $F$ over $\bb{F}_p$.
            Suppose that $f(t)$ is not the minimal polynomial of $F$ over $\bb{F}_p$.
            Then, there is a nonconstant polynomial $g(t)$ of degree $k<n$ over $\bb{F}_p$ which satisfies $F$.
            In this case, we have
            \begin{align*}
                \sum_{i=0}^k a_ix^{p^i}=g(F)(x)=O\alpha=0\quad\textsf{($a_i\in\bb{F}_p$ for $i=0, 1, \cdots, k$)}
            \end{align*}
            for all $x\in\bb{F}_{p^n}$.
            So, every element of $\bb{F}_{p^n}$ is a root of a polynomial of degree $p^k<p^n$, a contradiction.

            Therefore, $f(t)$ is the minimal polynomial (and the characteristic polynomial) of $F$ over $\bb{F}_p$.
            Hence, the rational canonical form of $F$ over $\bb{F}_p$ is given by
            \begin{align*}
                \left(\begin{array}{ccccc|c}
                        0   &   0   &   0   &   \cdots  &   0   &   1   \\\hline
                        1   &   0   &   0   &   \cdots  &   0   &   0   \\
                        0   &   1   &   0   &   \cdots  &   0   &   0   \\
                    \vdots  &\vdots &\vdots &   \ddots  &\vdots & \vdots\\
                        0   &   0   &   0   &   \cdots  &   0   &   0   \\
                        0   &   0   &   0   &   \cdots  &   1   &   0   \\
                \end{array}\right).
            \end{align*}

            To find the Jordan canonical form of $F$ over $\bb{F}_p$, assume that $n$ is not divisible by $p$.
            Then $f(t)$ is separable, so $F$ is diagonalizable, with the eigenvalues being all $n$-th roots of unity.
            Suppose $n=p^a m$, where $a$ and $m$ are positive integers and $p$ does not divide $m$.
            Then $f(t)=(t^m-1)^{p^a}$, so all distinct roots of $f(t)$ are given as $\zeta_m^i$ for $i=0, 1, \cdots, m-1$, where $\zeta_m$ is a primitive $m$-th root of unity.
            In this case, the Jordan canonical form of $F$ is $\diag(J_0, J_1, \cdots, J_{m-1})$, where
            \begin{align*}
                J_i=
                \left(\begin{array}{ccccccc}
                    \zeta_m^i   &   1   &   0    &   \cdots  &   0   &   0   \\
                        0   &\zeta_m^i  &   1    &   \cdots  &   0   &   0   \\
                        0   &   0   &\zeta_m^i   &   \cdots  &   0   &   0   \\
                    \vdots  &\vdots &\vdots      &   \ddots  &\ddots &\vdots \\
                        0   &   0   &   0    &   \cdots  &\zeta_m^i  &   1   \\
                        0   &   0   &   0    &   \cdots      &   0   &\zeta_m^i
                \end{array}\right)
                \in\mc{M}_{p^a, p^a}(\bb{F}_p)
            \end{align*}
            for each $i=0, 1, \cdots, m-1$.
        }
    \end{enumerate}
\end{sol}