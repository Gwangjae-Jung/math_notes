\section{Measures}

\begin{defi}[Measure]
    Let $(X, \mc{M})$ be a measurable space.
    A function $\mu: \mc{M}\rightarrow[0, \infty]$ with $\mu(\varnothing)=0$ is called a measure on $\mc{M}$ if $\mu$ is countably additive, i.e., if $\{E_n\}_{n\in\bb{N}}\subset\mc{M}$ is pairwise disjoint, then $\mu\left(\bigsqcup_{n=1}^\infty E_n\right)=\sum_{n=1}^\infty \mu(E_n)$.
    If $\mu$ is a measure on $\mc{M}$, the tuple $(X, \mc{M}, \mu)$ is called a measure space.
\end{defi}
\begin{rmk}
    The monotonicity and the countable subadditivity of measures, as illustrated in \cref{properties of measures}, are due to the countable additivity of measures.
    Whichever type of measure we consider (outer measures, premeasures, etc.) is assumed to satisfy the empty set condition, monotonicity, and an appropriate (sub)additivity; here, additivity implies monotonicity, and monotonicity implies an appropriate subadditivity.
\end{rmk}
\begin{defi}[Some particular measures]
    Let $(X, \mc{M})$ be a measurable space, and let $\mu: \mc{M}\rightarrow[0, \infty]$ be a function such that $\mu(\varnothing)=0$.
    $\mu$ is called a finitely additive measure, if $\mu$ is finitely additive but not necessarily countably additive.
    
    Assume $(X, \mc{M}, \mu)$ is a measure space.
    \begin{enumerate}
        \item[(a)]
        {
            $\mu$ is called a finite measure, if $\mu(X)<\infty$ (so that $\mu(E)<\infty$ for all $E\in\mc{M}$).
        }
        \item[(b)]
        {
            $\mu$ is called a $\sigma$-finite measure, if there is a countable collection $\{E_n\}_{n\in\bb{N}}\subset\mc{M}$ covering $X$ such that $\mu(E_n)<\infty$ for each $n\in\bb{N}$.
            Furthermore, if $E=\bigcup_n E_n$ with $E_n\in\mc{M}$ for each $n\in\bb{N}$ and $\mu(E_n)<\infty$ for each $n$, then $E$ is said to be $\sigma$-finite for $\mu$.
        }
        \item[(c)]{
            $\mu$ is called a semifinite measure, if for each $E\in\mc{M}$ with $\mu(E)=\infty$ there is $F\in\mc{M}$ with $F\subset E$ such that $0<\mu(F)<\infty$.
        }
    \end{enumerate}
\end{defi}

\begin{rmk}
    \begin{enumerate}
        \item[(a)]
        {
            A finite measure is $\sigma$-finite, and a $\sigma$-finite measure is semifinite. \color{brown}(Why?)\color{black}
        }
        \item[(b)]
        {
            A semifinite measure $\mu: \mc{M}\rightarrow[0, \infty]$ requires every member $E\in\mc{M}$ of infinite measure for $\mu$ to contain a set $F\in\mc{M}$ such that $0<\mu(F)<\infty$.
            In fact, when finding such $F$, we may restrict the lower bound for the measure of $F$.
            To be precise, if $\mu$ is semifinite, $E$ is a measurable set of infinite measure, and $C$ is any positive real number, then there is a measurable set $F$ such that
            \begin{align*}
                F\subset E\quad\textsf{and}\quad C<\mu(F)<\infty.
            \end{align*}
            See \cref{semifiniteness strengthened to}.
        }
    \end{enumerate}
\end{rmk}

\begin{exmp}
    \begin{enumerate}
        \item[(a)]
        {
            Let $(X, \mc{M})$ be a measurable space and $f: X\rightarrow [0, \infty]$ be a function.
            The function $\mu_f: \mc{M}\rightarrow[0, \infty]$ defined by $\mu_f(A)=\sum_{x\in A} f(x)$ is a measure on $\mc{M}$.
        }
        \item[(b)]
        {
            Let $X$ be an infinite set and $\mc{M}=P(X)$.
            The function $\mu: \mc{M}\rightarrow[0, \infty]$ defined by $\mu(E)=0$ if $E$ is finite and $\mu(E)=\infty$ if $E$ is infinite is a finitely additive measure on $\mc{M}$, but it is not a measure on $\mc{M}$.
        }
    \end{enumerate}
\end{exmp}

The following are basic properties of measures:
\begin{prop}\label{properties of measures}
    Let $(X, \mc{M}, \mu)$ be a measure space.
    \begin{enumerate}
        \item[(a)]
        {
            (Monotonicity) If $E, F\in\mc{M}$ and $E\subset F$, then $\mu(E)\leq\mu(F)$.
        }
        \item[(b)]
        {
            (Subadditivity) If $E_n\in\mc{M}$ for each $n\in\bb{N}$, then $\mu\left(\bigcup_n E_n\right)\leq\sum_n\mu(E_n)$.
        }
        \item[(c)]
        {
            (Continuity from below) If $(E_n)_{n\in\bb{N}}\subset\mc{M}$ is an ascending chain, then $\mu\left(\bigcup_n E_n\right)=\lim_{n\rightarrow\infty}\mu(E_n)$.
        }
        \item[(d)]
        {
            (Continuity from above) If $(E_n)_{n\in\bb{N}}\subset\mc{M}$ is a descending chain, then $\mu\left(\bigcap_n E_n\right)=\lim_{n\rightarrow\infty}\mu(E_n)$, provided that $\mu(E_1)<\infty$.
        }
    \end{enumerate}
\end{prop}
\begin{proof}
    It is easy to check (a) and (b), so we will prove (c) and (d).

    \hangindent=0.65cm
    \noindent(c)
    Define $F_n:=E_n\setminus E_{n-1}$ for each $n\in\bb{N}$, where $E_0:=\varnothing$.
    Then $\bigcup_n E_n=\bigsqcup_n F_n$, thus 
    \begin{align*}
        \mu\left(\bigcup_n E_n\right)=\mu\left(\bigsqcup_n F_n\right)=\sum_n\mu(F_n)=\sum_n(\mu(E_n)-\mu(E_{n-1}))=\lim_{n\rightarrow\infty}\mu(E_n).
    \end{align*}

    \noindent(d)
    Define $A_n=E_1\setminus E_n$ for each $n\in\bb{N}$.
    Then $(A_n)_{n\in\bb{N}}$ is an ascending chain in $\mc{M}$, so
    \begin{align*}
        \mu\left(\bigcup_n A_n\right)=\lim_{n\rightarrow\infty}\mu(A_n)=\mu(E_1)-\lim_{n\rightarrow\infty}\mu(E_n).
    \end{align*}
    (Note that $(\mu(E_n))_{n\in\bb{N}}$ is convergent, since the sequence is decreasing and is bounded below.)
    Because $E_1\setminus\bigcup_n A_n=\bigcap_n(E_1\setminus A_n)=\bigcap_n E_n$, we have the desired identity.
\end{proof}

\begin{defi}
    Let $(X, \mc{M}, \mu)$ be a measure space.
    \begin{enumerate}
        \item[(a)]
        {
            ($\mu$-Null set)
            A set $E\in\mc{M}$ such that $\mu(E)=0$ is called a $\mu$-null set.
            If a statement about points $x\in X$ is true except for $x$ in some $\mu$-null set, then the statement is said to be true $\mu$-almost everywhere on $X$.
        }
        \item[(b)]
        {
            (Complete measure)
            A measure whose domain contains all subsets of $\mu$-null sets is said to be complete.
        }
    \end{enumerate}
\end{defi}
\begin{nota}
    Given a measure space $(X, \mc{M}, \mu)$, let $\mc{N}_\mu$ be the collection of all $\mu$-null sets in $\mc{M}$, i.e.,
    \begin{eqnarray*}
        \mc{N}_\mu:=\{E\in\mc{M}: \mu(E)=0\}.
    \end{eqnarray*}
    One can easily find that $\mc{N}_\mu$ is closed under countable unions.
\end{nota}

Completeness of a measure can obviate annoying technical problems, and it can be achieved by enlarging the domain of $\mu$ as follows.
\begin{thm}[Completion of a measure]\label{completion of a measure}
    Let $(X, \mc{M}, \mu)$ be a measure space.
    Define
    \begin{align*}
        \ol{\mc{M}}=\{E\cup F:\textsf{$E\in\mc{M}$ and $F$ is a subset of a $\mu$-null set}\}.
    \end{align*}
    \begin{enumerate}
        \item[(a)]
        {
            $\ol{\mc{M}}$ is a $\sigma$-algebra on $X$ containing $\mc{M}$.
        }
        \item[(b)]
        {
            There is a unique extension $\ol{\mu}$ of $\mu$ to a complete measure on $\ol{\mc{M}}$.
        }
    \end{enumerate}
\end{thm}
\begin{rmk}
    The following proof seems quite similar to adjoining the fractional parts to the integer parts.
\end{rmk}
\begin{proof}
    We first show that $\ol{\mc{M}}$ is a $\sigma$-algebra on $X$.
    It is clear that $\ol{\mc{M}}$ is closed under countable unions, since $M$ and $\mc{N}_\mu$ are closed under countable unions.
    Thus, it remains to show that $\ol{\mc{M}}$ is closed under set complements.
    Suppose $E\in\mc{M}$ and $F$ is a subset of $N\in\mc{N}_\mu$, and write $N=F\sqcup G$ (clearly, $G=N\setminus F$), $A=E\cup N\in\mc{M}$.
    Then $E\cup F=A\setminus G$ and one can show that
    \begin{align*}
        X\setminus (E\cup F)=(X\setminus A)\cup G\in\ol{\mc{M}},
    \end{align*}
    which proves that $\ol{\mc{M}}$ is closed under set complements.
    
    We now construct a measure $\ol{\mu}$ on $\ol{\mc{M}}$ which extends $\mu$.
    Given $E\cup F\in\ol{\mc{M}}$ with $E$ and $F$ being assumed as earlier, one may suggest that $\ol{\mu}(E\cup F)$ be $\mu(E)$.
    In fact, this definition is unambiguous; if $E_1\cup F_1=E_2\cup F_2$ with $E_i$ and $F_i$ being assumed correspondingly, we have $E_1\subset E_2\cup F_2\subset E_2\cup N_2$ so $\mu(E_1)\leq\mu(E_2)$, and $\mu(E_2)\leq\mu(E_1)$ by symmetry.
    It remains to check
    \begin{enumerate}
        \item[(i)]
        {
            if the constructed map $\ol{\mu}$ is a complete measure on $\ol{\mc{M}}$, and
        }
        \item[(ii)]
        {
            if such extension is unique.
        }
    \end{enumerate}
    It is clear that $\ol{\mu}$ is a measure on $\ol{\mc{M}}$ and is complete.
    It remains to prove the uniqueness part.
    Suppose $\widetilde{\mu}: \ol{\mc{M}}\rightarrow[0, \infty]$ is an extension of $\mu$ to a complete measure on $\ol{\mc{M}}$.
    Then $\ol{\mu}$ and $\wt{\mu}$ coincide on $M$.
    By monotonicity, we can find that $\widetilde{\mu}(E\cup F)=\mu(E)$, which proves the uniqueness.
\end{proof}
\begin{rmk}
    \begin{enumerate}
        \item[(a)]
        {
            In the preceeding theorem, $\ol{\mu}$ is called the completion of $\mu$, and $\ol{\mc{M}}$ is called the completion of $\mc{M}$ with respect to $\mu$ (or simply called the $\mu$-completion of $\mc{M}$).
            Indeed, a completion of a $\sigma$-algebra is determined by a measure on the $\sigma$-algebra.
        }
        \item[(b)]
        {
            One should not be confused that the extension of $\mu$ to a complete measure is unique.
            The preceeding theorem states that the extension of $\mu$ to a complete measure ``over $\ol{\mc{M}}$'' is unique.
        }
    \end{enumerate}
\end{rmk}

We move to the list of problems after introducing one remark, which will be helpful in solving some problems in this section.
\begin{obs}[Maximal measurable subset of finite measure]\label{maximal measurable subset of finite measure}
    Given a measure space $(X, \mc{M}, \mu)$, let $E$ be a measurable set such that $s=\sup\left(\mu(R_E)\right)$ is finite, where
    \begin{align*}
        R_E=\{
            A\subset E:
            \textsf{$A$ is measurable and $\mu(A)<\infty$}
        \}.
    \end{align*}
    Our goal is to prove that there is a member $A$ of $R_E$ whose measure is $s$.

    Let $(A_n)_{n\in\bb{N}}\subset R_E$ be a sequence such that $\mu(A_n)\rightarrow s$ as $n\rightarrow\infty$. \color{brown}(Why does such sequence exist?) \color{black}
    Let $B_n=\bigcup_{i=1}^n A_i$ for each $n$.
    Because $A_n\subset B_n\in R_E$ for each $n$ and $(B_n)_n$ is ascending, we have $\mu\left(\bigcup_n A_n\right)=\mu\left(\bigcup_n B_n\right)=\lim\mu(B_n)=s<\infty$.
    Therefore, the union of $A_n$ for all $n$ belongs to $R$, and its measure is $s$.
\end{obs}

\subsection*{Problems}

\begin{prob}[Exercise 1.8]
    Suppose $(X, \mc{M}, \mu)$ is a measure space and $(E_n)_{n\in\bb{N}}\subset M$.
    Show that $\mu\left(\liminf_{n\rightarrow\infty}E_n\right)\leq\liminf_{n\rightarrow\infty}\mu(E_n)$.
    Also, show that $\mu\left(\limsup_{n\rightarrow\infty}E_n\right)\geq\limsup_{n\rightarrow\infty}\mu(E_n)$ if $\mu\left(\bigcup_n E_n\right)<\infty$.
\end{prob}
\begin{sol}
    Remark that
    \begin{align*}
        \limsup_{n\rightarrow\infty} A_n=\bigcap_{n=1}^\infty \bigcup_{k\geq n} A_k
        \quad\textsf{and}\quad
        \liminf_{n\rightarrow\infty} A_n=\bigcup_{n=1}^\infty \bigcap_{k\geq n} A_k
    \end{align*}
    whenever $A_n$ is a subset of a set $X$ for each $n\in\bb{N}$.
    Since $\left(\bigcup_{k\geq n} A_k\right)_{n\in\bb{N}}$ is ascending, we have
    \begin{align*}
        \mu\left(\liminf_{n\rightarrow\infty}E_n\right)=\lim_{n\rightarrow\infty}\mu\left(\bigcap_{k\geq n} E_k\right)\leq\lim_{n\rightarrow\infty}\inf_{k\geq n}\mu(E_k)=\liminf_{n\rightarrow\infty}\mu(E_n).
    \end{align*}
    Similarly, $\left(\bigcup_{k\geq n} A_k\right)_{n\in\bb{N}}$ is descending and $\bigcup_{k\geq 1} E_k$ is finite for $\mu$, we have
    \begin{align*}
        \mu\left(\limsup_{n\rightarrow\infty}E_n\right)=\lim_{n\rightarrow\infty}\mu\left(\bigcup_{k\geq n} E_k\right)\geq\lim_{n\rightarrow\infty}\sup_{k\geq n}\mu(E_k)=\limsup_{n\rightarrow\infty}\mu(E_n).
    \end{align*}
\end{sol}

\begin{prob}[Exercise 1.11]
    Let $\mu$ be a finitely additive measure on a measurable space $(X, \mc{M})$.
    Show that $\mu$ is a measure if and only if $\mu$ is continuous from below.
    Show also that $\mu$ is a measure if and only if $\mu$ is continuous from above, provided that $\mu(X)<\infty$.
\end{prob}
\begin{sol}
    It suffices to show in each case that continuity implies countable additivity.
    Let $\{E_n\}_{n\in\bb{N}}$ be a set of pairwise disjoint subsets of $X$ belonging to $\mc{M}$.
    \begin{enumerate}
        \item[(a)]
        {    
            Define $F_n:=\bigcup_{i=1}^n E_i$ for each $n\in\bb{N}$.
            Clearly, $(F_n)_n$ is an ascending chain in $\mc{M}$, hence
            \begin{align*}
                \mu\left(\bigsqcup_n E_n\right)=\mu\left(\bigcup_n F_n\right)=\lim_{n\rightarrow\infty}\mu(F_n)=\lim_{n\rightarrow\infty}\sum_{i=1}^n\mu(E_i)=\sum_n\mu(E_n).
            \end{align*}
        }
        \item[(b)]
        {
            In part (a), define further $G_n:=X\setminus F_n$ for each $n\in\bb{N}$.
            Because $\mu(G_1)<\infty$ and $(G_n)_n$ is a descending chain in $\mc{M}$, we have
            \begin{align*}
                \mu\left(\bigcap_n G_n\right)=\lim_{n\rightarrow\infty}\mu(G_n)=\mu(X)-\sum_n\mu(E_n).
            \end{align*}
            Because
            \begin{align*}
                \mu\left(\bigcap_n G_n\right)=\mu(X)-\mu\left(\bigcup_n F_n\right)=\mu(X)-\mu\left(\bigsqcup_n E_n\right),
            \end{align*}
            we find that $\mu$ is a measure.
        }
    \end{enumerate}
\end{sol}

\begin{prob}[Exercise 1.12]
    Let $(X, M, \mu)$ be a finite measure.
    Assume $E,\,F,\,G\in\mc{M}$.
    \begin{enumerate}
        \item[(a)]
        {
            Show that $\mu(E)=\mu(F)=\mu(E\cap F)$, if $\mu(E\triangle F)=0$.
        }
        \item[(b)]
        {
            Define a relation $\sim$ on $\mc{M}$ by $E\sim F$ if and only if $\mu(E\triangle F)=0$.
            Show that $\sim$ denotes an equivalence relation on $\mc{M}$.
        }
        \item[(c)]
        {
            Define a map $\rho: \mc{M}\times\mc{M}\rightarrow[0, \infty]$ by $\rho(E, F)=\mu(E\triangle F)$.
            Show that $\rho(E, G)\leq\rho(E, F)+\rho(F, G)$, and deduce that $\rho$ induces a metric on the space $\mc{M}/\sim$ of the equivalence classes.
            To be precise, the map $\ol{\rho}: (\mc{M}/\sim)\times(\mc{M}/\sim)\rightarrow[0,\infty]$ defined by $\ol{\rho}(\ol{E}, \ol{F})=\rho(E, F)$ for all $\ol E,\,\ol F\in\mc{M}/\sim$ is a well defined metric on $\mc{M}/\sim$.
        }
    \end{enumerate}
\end{prob}
\begin{sol}
    \begin{enumerate}
        \item[(a)]
        {
            If $\mu(E\triangle F)=0$, then is $\mu(E\setminus F)=\mu(F\setminus E)=0$, so $\mu(E)=\mu(F)=\mu(E\cap F)$.
        }
        \item[(b)]
        {
            It is clear that $\sim$ is symmetric and reflexive.
            To show transitivity, assume $E\sim F$ and $F\sim G$.
            Note that $E\triangle G\subset(E\triangle F)\cup(F\triangle G)$ so $\mu(E\triangle G)\leq\mu(E\triangle F)+\mu(F\triangle G)=0$.
        }
        \item[(c)]
        {
            The triangular inequality with regard to $\rho$ is explained in part (b).
            We now justify that $\ol\rho$ is a metric on $\mc{M}/\sim$.
            Suppose $E\sim A$ and $F\sim B$.
            Because $\rho(E, F)\leq\rho(E, A)+\rho(A, B)+\rho(B, F)=\rho(A, B)$ and $\rho(A, B)\leq\rho(E, F)$ by symmetry, the map $\ol{\rho}$ is well-defined.
            Therefore, $\ol{\rho}$ is a metric on $\mc{M}/\sim$, which follows directly from the definition of $\ol{\rho}$.
        }
    \end{enumerate}
\end{sol}

\begin{prob}[Exercise 1.14]\label{semifiniteness strengthened to}
    Let $(X, \mc{M}, \mu)$ be a semifinite measure space, and assume $E\in\mc{M}$ is of infinite measure for $\mu$.
    Show that for any $C>0$ there is a subset $F\in\mc{M}$ of $E$ such that $C<\mu(F)<\infty$.
\end{prob}
\begin{sol}
    Suppose that there is a member $E\in\mc{M}$ such that $\mu(E)=\infty$ and any measurable subset $A$ of $E$ satisfies $\mu(A)=\infty$ or $\mu(A)\leq\alpha$ for some real $\alpha>0$.
    Then, the supremum $s$ of $R_E$ is finite, where $R_E$ is defined as in \pageref{maximal measurable subset of finite measure}.
    Hence, there is a member $A$ of $R_E$ such that $\mu(A)=s$.
    Because $\mu(E\setminus A)=\infty$, by semifiniteness, there is a member $B\subset E\setminus A$ belonging to $\mc{M}$ such that $0<\mu(B)<\infty$.
    Then, $A\sqcup B\in R_E$ but $\mu(A\sqcup B)>s$, a contradiction.
\end{sol}

\begin{prob}[Exercise 1.15]
    Given a measure $\mu$ on a measurable space $(X, \mc{M})$, define a map $\mu_0: \mc{M}\rightarrow[0, \infty]$ by
    \begin{align*}
        \mu_0(E):=\sup\{\mu(F):F\in\mc{M}\textsf{ is a subset of }E\textsf{ of finite measure for }\mu\}.
    \end{align*}
    \begin{enumerate}
        \item[(a)]
        {
            Show that $\mu_0$ is a semifinite measure on $\mc{M}$.
            $\mu_0$ is called the semifinite part of $\mu$.
        }
        \item[(b)]
        {    
            Show that if $\mu$ is semifinite, then $\mu=\mu_0$.
        }
        \item[(c)]
        {
            Show that there is a measure $\nu$ on $\mc{M}$ which assumes only the values $0$ and $\infty$ such that $\mu=\mu_0+\nu$.
            (In general, such $\nu$ need not be unique.)
        }
    \end{enumerate}
\end{prob}
\begin{sol}
    Remark that $\mu_0\leq\mu$, and $\mu_0(E)=\mu(E)$ if $E\in\mc{M}$ and $\mu(E)<\infty$.
    \begin{enumerate}
        \item[(a)]
        {
            We first show that the map $\mu_0$ is a measure on $\mc{M}$.
            Since $\mu_0(\varnothing)=0$, it suffice to show that $\mu_0$ is countably additive.
            Suppose a collection $\{A_n\}_{n\in\bb{N}}\subset\mc{M}$ is pairwise disjoint, and let $B$ be the union of $A_n$ for all $n$.
            Given a measurable subset $E$ of $B$, we have $\mu(E)=\sum_n\mu(E\cap A_n)\leq\sum_n\mu_0(A_n)$, so $\mu_0(B)\leq\sum_n\mu_0(A_n)$.
            Conversely, given a measurable subset $F_n$ of $A_n$ of finite measure for each $n$, because every finite union of $F_n$'s is a measurable subset of $B$ of finite measure,
            \begin{align*}
                \sum_{n=1}^N\mu(F_n)=\mu\left(\bigsqcup_{n=1}^N F_n\right)\leq\mu_0(B)
                \quad\textsf{and}\quad
                \sum_{n=1}^N\mu_0(A_n)\leq\mu_0(B),
            \end{align*}
            so $\sum_n\mu_0(A_n)\leq\mu_0(B)$.
            Hence, $\mu_0$ is a measure on $\mc{M}$.

            Semifiniteness of $\mu_0$ is straightforward; if $\mu_0(E)=\infty$ for some $E\in M$, there is a measurable subset $F$ of $E$ with nonzero finite measure, and $\mu_0(F)=\mu(F)<\infty$.
        }
        \item[(b)]
        {
            It follows from \cref{semifiniteness strengthened to} that $\mu(E)=\mu_0(E)$ whenever $E\in\mc{M}$ and $\mu(E)=\infty$. \color{brown}(How?)\color{black}
        }
        \item[(c)]
        {
            Assume such measure $\nu$ exists and $E\in\mc{M}$.
            If $\mu_0(E)=\mu(E)<\infty$, then $\nu(E)=0$; if $\mu_0(E)<\infty$ bt $\mu(E)=\infty$, then $\nu(E)=\infty$.
            Also, if $E$ is $\sigma$-finite, then $\nu(E)=0$ by $\sigma$-additivity.
            Hence, to solve this problem, we need a set map from $\mc{M}$ onto $\{0, \infty\}$ satisfying the above properties.

            Define a set map $\nu: \mc{M}\rightarrow\{0, \infty\}$ by
            \begin{align*}
                \nu(E)=\left\{\begin{array}{cc}
                        0   &   \textsf{(if $E$ is $\sigma$-finite for $\mu$)}\\
                    \infty  &   \textsf{(otherwise)}
                \end{array}\right..
            \end{align*}
            To check that $\nu$ satisfies the above properties, one must verify that $\mu_0(E)=\infty$ when $\mu(E)=\infty$ and $E$ is $\sigma$-finite, where $E\in\mc{M}$.
            \color{brown}(This is left as an exercise.) \color{black}
            $\nu$ clearly satisfies the empty set axiom.
            Let $\{A_n\}_{n\in\bb{N}}$ be a countable collection of pairwise disjoint measurable sets.
            If $A_n$ is $\sigma$-finite for all $n$, then $\bigsqcup_n A_n$ is also $\sigma$-finite, hence $\nu(\bigsqcup_n A_n)=0=\sum_n\nu(A_n)$.
            If $A_k$ is not $\sigma$-finite for some $k\in\bb{N}$, then $\bigsqcup_n A_n$ is not $\sigma$-finite, so $\nu(\bigsqcup_n A_n)=\infty=\sum_n\nu(A_n)$.

            So far, we have proved that $\nu$ is a measure.
            To check if $\mu=\mu_0+\nu$, we only need to check $\mu(E)=\mu_0(E)+\nu(E)$ for all measurable sets $E$ with $\mu_0(E)<\infty$; if $\mu(E)<\infty$, $E$ is $\sigma$-finite for $\mu$; otherwise, $E$ is not $\sigma$-finite for $\mu$.
        }
    \end{enumerate}
\end{sol}
