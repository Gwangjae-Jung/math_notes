\section{$\sigma$-algebras}

We start this note with the definition of algebras and $\sigma$-algebras on a nonempty set.
\begin{defi}[Algebra on a set]
    Let $X$ be a nonempty set.
    A nonempty collection $\mc{A}$ of subsets of $X$ is called an algebra on $X$ if
    \begin{enumerate}
        \item[(a)]
        {
            $\mc{A}$ is closed under set complements
        }
        \item[(b)]
        {
            $\mc{A}$ is closed under arbitrary finite unions.
        }
    \end{enumerate}
    Furthermore, if $\mc{A}$ is closed under arbitrary `countable' unions, then $\mc{A}$ is called a $\sigma$-algebra on $X$.
    If $\mc{M}$ is a $\sigma$-algebra on the set $X$, then the tuple $(X, \mc{M})$ is also called a measurable space.
    Also, any set in $\mc{M}$ is called a measurable set.
\end{defi}
\begin{rmk}[A practical tip]
    Suppose $\mc{A}$ is a nonempty collection of subsets of $X$ which is closed under set complements in $X$.
    To show $\mc{A}$ is a $\sigma$-algebra on $X$, it suffices to show that $\mc{A}$ is closed under arbitrary countable \textit{disjoint} unions; if $\{A_n\}_n$ is a countable collection of members of $\mc{A}$, then the union of $A_n$'s is the \textit{disjoint} union of $F_n$'s where
    \begin{align*}
        F_1:=A_1\quad\textsf{and}\quad F_n:=A_n\setminus\bigcup_{k=1}^{n-1}A_k\textsf{ for }n\geq 2.
    \end{align*}
\end{rmk}
\begin{rmk}[Algebras of subsets]
    Let $\mc{A}$ be an algebra on a set $X$.
    Then, $\mc{A}$ is a $\sigma$-algebra on $X$ if and only if $\mc{A}$ is closed under arbitrary countable unions of sets in an ascending chain in $\mc{A}$.
    
    To justify the assertion, it suffices to show that $\mc{A}$ is closed under arbitrary countable unions of members in $\mc{A}$, under the assumption that $\mc{A}$ is closed under arbitrary countable unions of sets in an ascending chain in $\mc{A}$.
    If $\{E_n\}_{n=1}^\infty\subset\mc{A}$ and $F_k=\bigcup_{n=1}^k E_n$, then $F_1\subset F_2\subset\cdots$ and $\bigcup_n E_n=\bigcup F_n\in\mc{A}$, by assumption.
\end{rmk}

\begin{exmp}[Restriction of a $\sigma$-algebra]
    Let $\mc{M}$ be a $\sigma$-algebra on a nonempty set $X$.
    For a nonempty subset $A$ of $X$, let
    \begin{align*}
        \mc{M}|_A=\{A\cap E: E\in\mc{M}\}.
    \end{align*}
    Then $\mc{M}|_A$ is a $\sigma$-algebra on $A$, which is called the $\sigma$-algebra on $A$ inherited (restricted) from $X$ to $A$.
    (The same argument holds for algebras on $X$, too.)
\end{exmp}

As there were bases for topology or algebraic structures, we can consider a basis of a $\sigma$-algebra.
\begin{defi}[Generated $\sigma$-algebra]
    Suppose $X$ is a nonempty set and $E$ is a collection of subsets of $X$.
    The collection $\mc{M}(E)$ of the smallest $\sigma$-algebra containing $E$ is called the $\sigma$-algebra on $X$ generated by $E$.
\end{defi}
\begin{rmk}
    Let $X$ be a nonempty set and $E$ be a collection of subsets of $X$.
    \begin{enumerate}
        \item[(a)]
        {
            $\mc{M}(E)$ is the intersection of all $\sigma$-algebras on $X$ containing $E$.
        }
        \item[(b)]
        {
            If $E\subset\mc{M}(F)$, then $\mc{M}(E)\subset\mc{M}(F)$.
        }
    \end{enumerate}
    Because the proof is straightforward and easy, it is left as an exercise.
\end{rmk}

In particular, when $X$ is a topological space, the $\sigma$-algebra on $X$ generated by the topology on $X$ is called the Borel $\sigma$-algebra on $X$.

\begin{exmp}[The Borel $\sigma$-algebra on the extended real system $\ol{\bb{R}}$]
    We define some basic operations over $\ol{\bb{R}}=\bb{R}\sqcup\{\infty, -\infty\}$ as follows: For all $a\in\bb{R}$ and $c\in\bb{R}\setminus\{0\}$, we let
    \begin{align*}
        \infty+a=\infty,\quad c\cdot\infty=\infty,\quad 0\cdot\infty=0
    \end{align*}
    and we do not define $\infty-\infty$.
    We can impose a metric on $\bb{R}$ by the following metric:
    \begin{align*}
        d: \ol{\bb{R}}\times\ol{\bb{R}}\rightarrow[0, \infty),\quad(a, b)\mapsto|\arctan(a)-\arctan(b)|.
    \end{align*}
    In fact, the metric topology on $\ol{\bb{R}}$ induced by $d$ is generated as a basis by the following intervals:
    \begin{center}
        $(a, b)$, $[-\infty, b)$, and $(a, \infty]$ with $-\infty\leq a<b\leq\infty$.
    \end{center}
    The Borel $\sigma$-algebra on $\ol{\bb{R}}$ is generated by the intervals of the last two types.
    Also, it is equivalent to define
    \begin{align*}
        \borel{\ol{\bb{R}}}=\{
            E\subset\ol{\bb{R}}
            :
            E\cap\bb{R}\in\borel{\bb{R}}
        \}.
    \end{align*}
\end{exmp}

\subsection*{Problems}

\begin{prob}[Exercise 1.3]
    Let $\mc{M}$ be an infinite $\sigma$-algebra on $X$.
    Show that $\mc{M}$ contains infinitely many nonempty and pairwise disjoint sets, and $\card{\mc{M}}\geq\card{\bb{R}}$
\end{prob}
\begin{sol}
    We first show the existence of a nonempty set $E\in\mc{M}$ such that the restriction of $\mc{M}$ to $X\setminus E$ is still infinite.\footnote{It is left as an exercise to show that the restriction $\mc{M}_E:=\{E\cap M:M\in\mc{M}\}$ of $\mc{M}$ to $E\in\mc{M}$ is a $\sigma$-algebra on $E$.}
    This implies, by induction, the existence of infinitely many nonempty and pairwise disjoint sets in $\mc{M}$.
    Suppose $\mc{M}$ has no such $E$.
    In other words, suppose that the restriction of $\mc{M}$ to $X\setminus E$ is finite whenever $E\in\mc{M}$ is nonempty and $E\neq X$.
    Because the restriction of $\mc{M}$ to $X\setminus E$ is also finite, $\mc{M}$ is finite, which contradicts the assumption that $\mc{M}$ is infinite.
    
    Let $\{E_n\}_{n\in\bb{N}}\in\mc{M}$ be an infinite collection of nonempty and pairwise disjoint members in $\mc{M}$.
    Define a function $f: \mc{M}\rightarrow\bb{R}$ as follows:
    \begin{align*}
        f(A)=\sum\{2^{-n}
        :
        A\cap E_n\neq\varnothing\}.
    \end{align*}
    The function $f$ is onto $[0, 1]$, so $\card{\mc{M}}\geq\card{[0, 1]}\geq\card{\bb{R}}$.
\end{sol}
