\section{Premeasures}

We introduce premeasures in this section which has some nice extension properties when generating a complete measure, where the generation is done as in Carath\'{e}odory extension, i.e., an outer measure from an appropriate basic function, and then a complete measure from the outer measure.
We also introduce further properties when extending a premeasure $\mu_0$ to a complete measure $\mu$, when $X$ is $\sigma$-finite for $\mu_0$.

\begin{defi}[Premeasures]
    Let $X$ be a nonempty set and $\mc{A}$ be an algebra on $X$.
    The function $\mu_0: \mc{A}\rightarrow[0, \infty]$ with $\mu_0(\varnothing)=0$ is called a premeasure on $X$ if $\mu_0$ is countably additive, i.e., whenever $\{A_n\}_n$ is a countable collection of pairwise disjoint members of $\mc{A}$ and $\bigsqcup_n A_n\in\mc{A}$, we have
    \begin{align*}
        \mu_0\left(\bigsqcup_n A_n\right)=\sum_n\mu_0(A_n).
    \end{align*}
\end{defi}
\begin{rmk}
    As it was true for measures, the countable additivity of premeasures implies that premeasures are monotonic and appropriately countably subadditive.
\end{rmk}

As the first step, we induce an outer measure $\mu^*$ on $X$ from the premeasure $\mu_0$.
\begin{prop}[Extension of a premeasure]\label{extension of a premeasure}
    Let $\mc{A}$ be an algebra on $X$, $\mu_0$ be a premeasure on $\mc{A}$, and $\mu^*$ be the outer measure on $X$ induced by $\mu_0$ as in \cref{induced outer measure} on \cpageref{induced outer measure}.
    \begin{enumerate}
        \item[(a)]
        {
            $\mu^*$ extends $\mu_0$, i.e., $\mu^*\vline_\mc{A}=\mu_0$.
        }
        \item[(b)]
        {
            Every set in $\mc{A}$ is $\mu^*$-measurable.
            Hence, every member of the $\sigma$-algebra generated by $\mc{A}$ is $\mu^*$-measurable.
        }
    \end{enumerate}
    In short, the restriction of $\mu^*$ to $\mu^*$-measurable sets is a complete measure which extends $\mu_0$.
\end{prop}
\begin{proof}
    \begin{enumerate}
        \item[(a)]
        {
            To show that $\mu^*$ extends $\mu_0$, it suffices to show that $\mu^*(E)\geq\mu_0(E)$ for all $E\in\mc{A}$; for this, one need to show that $\sum_n\mu_0(A_n)\geq\mu_0(E)$ whenever $\{A_n\}_n$ is a countable covering of $E$ by members of $\mc{A}$.
            Let $\{F_n\}_n$ be the usual partition for $\bigcup_n A_n$.
            Since each $F_n$ belongs to $\mc{A}$, $B_n:=E\cap F_n$ is a member of $\mc{A}$ and the union of $B_n$ for all $n$ is $E$.
            Thus, by the $\sigma$-additivity of $\mu_0$,
            \begin{align*}
                \mu_0(E)=\mu_0\left(\bigsqcup_n B_n\right)=\sum_n\mu_0(B_n)\leq\sum_n\mu_0(A_n),
            \end{align*}
            from which it is followed that $\mu_0(E)\leq\mu^*(E)$.
        }
        \item[(b)]
        {
            We will show that $\mu^*(E)\geq\mu^*(E\cap A)+\mu^*(E\setminus A)$ for all $E\subset X$ and $A\in\mc{A}$.
            By definition, given $\epsilon>0$, we can find a countable covering $\{C_n\}_n$ of $E$ by members of $\mc{A}$ such that
            \begin{align*}
                \mu^*(E)\leq\sum_n\mu_0(C_n)<\mu^*(E)+\epsilon.
            \end{align*}
            Because $\mu_0(C_n)=\mu_0(C_n\cap A)+\mu_0(C_n\setminus A)$ for all $n$,
            \begin{eqnarray*}
                \mu^*(E)+\epsilon&>&\sum_n(\mu_0(C_n\cap A)+\mu_0(C_n\setminus A))\\
                &=&\sum_n\mu_0(C_n\cap A)+\sum_n\mu_0(C_n\setminus A)\geq\mu^*(E\cap A)+\mu^*(E\setminus A).
            \end{eqnarray*}
            Since $\epsilon$ is arbitrary, $A$ is $\mu^*$-measurable.
            The latter result follows, because the collection of $\mu^*$-measurable sets is a $\sigma$-algebra on $X$.
        }
    \end{enumerate}
    It follows easily that the restriction of $\mu^*$ to $\mu^*$-measurable sets extends $\mu_0$.
\end{proof}

We now restrict the outer measure $\mu^*$ to $\genone{\mc{A}}$, the $\sigma$-algebra on $X$ generated by the algebra $\mc{A}$ on $X$.
As mentioned in (c) of the following theorem, when $X$ is $\sigma$-finite for the premeasure $\mu_0$, then the extension of $\mu_0$ to a complete measure is unique.
Moreover, in proving (c), we can observe how $\sigma$-finiteness can be enjoyed in some circumstances.
\begin{thm}
    Let $\mc{A}$ be an algebra on $X$, $\mu_0$ be a premeasure on $\mc{A}$.
    \begin{enumerate}
        \item[(a)]
        {
            There is a measure $\mu$ on $\genone{\mc{A}}$ whose restriction to $\mc{A}$ is the premeasure $\mu_0$; namely, $\mu:=\mu^*\vline_\genone{\mc{A}}$, where $\mu^*$ is the outer measure on $X$ induced by $\mu_0$.\footnote{Indeed, $\mu_0$ extends to the restriction of $\mu^*$ to the collection of all $\mu^*$-measurable sets.}
        }
        \item[(b)]
        {
            If $\nu$ is another measure on $\genone{\mc{A}}$ extending $\mu_0$, then $\nu\leq\mu$, with equality for all $E\in\genone{\mc{A}}$ such that $\mu(E)<\infty$.
            }
        \item[(c)]
        {
            In particular, if $\mu_0$ is a $\sigma$-finite premeasure, then $\mu$ is the unique extension of $\mu_0$ to a measure on $\genone{\mc{A}}$.
        }
    \end{enumerate}
\end{thm}
\begin{proof}
    \begin{enumerate}
        \item[(a)]
        {
            This follows directly from the Carath\'eodory extension theorem and \cref{extension of a premeasure}.
        }
        \item[(b)]
        {
            Suppose $E\in{\mc{A}}$ and $\{A_n\}_n$ be a countable covering of $E$ by members of $\mc{A}$.
            Then $\nu(E)\leq\nu\left(\bigcup_n A_n\right)\leq\sum_n\nu(A_n)=\sum_n\mu_0(A_n)$, so $\nu(E)\leq\mu(E)$.

            Let $A=\bigcup_n A_n$, where each $A_n$ belongs to $\mc{A}$.
            Since every finite union of $A_n$'s is a member of $\mc{A}$,
            \begin{align*}
                \nu(A)=\lim_{n\rightarrow\infty}\nu\left(\bigcup_{k=1}^n A_k\right)=\lim_{n\rightarrow\infty}\mu\left(\bigcup_{k=1}^n A_k\right)=\mu(A)
            \end{align*}
            Therefore, $\mu$ and $\nu$ coincide at every countable union of members of $\mc{A}$.
            
            If $E$ is a member of $\genone{\mc{A}}$ such that $\mu(E)<\infty$, there is a countable covering $\{B_n\}_n$ of $E$ by members of $\mc{A}$ such that
            \begin{align*}
                \mu(E)\leq\sum_n\mu_0(B_n)<\mu(E)+\epsilon.
            \end{align*}
            With $B=\bigcup_n B_n$, we have $\mu(B\setminus E)<\epsilon$ \color{brown}(why?)\color{black}, and
            \begin{align*}
                \mu(E)\leq\mu(B)=\nu(B)=\nu(E)+\nu(B\setminus E)\leq\nu(E)+\mu(B\setminus E)<\nu(E)+\epsilon.
            \end{align*}
            Thus, $\nu\leq\mu$ with equality at every member of $\genone{\mc{A}}$ of finite measure for $\mu$.
        }
        \item[(c)]
        {
            Finally, assume that $\mu_0$ is a $\sigma$-finite premeasure on $\mc{A}$.
            Then there is a countable collection $\{A_n\}_n\subset\genone{A}$ such that $X=\bigcup_{n=1}^\infty A_n$ and $\mu(A_n)<\infty$ for each $n\in\bb{N}$.
            By partitioning, we may assume that $\{A_n\}_n$ is pairwise disjoint.
            If $E\in\genone{\mc{A}}$, we have
            \begin{align*}
                \mu(E)=\sum_n\mu(E\cap A_n)=\sum_n\nu(E\cap A_n)=\nu(E),
            \end{align*}
            proving that $\mu$ is the unique extension of the premeasure $\mu_0$ to a measure on $\genone{\mc{A}}$.
        }
    \end{enumerate}
    This completes the proof.
\end{proof}

Our last goal of this section is to prove the following:
\begin{quotation}
    Given a \textit{$\sigma$-finite} measure space $(X, \mc{M}, \mu)$, the completion of $\mu$ and the Carath\'{e}odory extension of $\mu$ coincide.
    In other words, the domain of the Carath\'{e}odory extension of $\mu$ is the completion of $\mc{M}$ with regard to $\mu$.
\end{quotation}

\begin{lem}
    Let $\mc{A}$ be an algebra on $X$, $\mu_0$ be a premeasure on $\mc{A}$, and $\mu^*$ be the outer measure induced by $\mu_0$.
    Let $\mc{A}_\sigma$ denote the collection of all countable unions of the members of $\mc{A}$, and let $\mc{A}_{\sigma\delta}$ denote the collection of all finite intersections of the members of $\mc{A}_\sigma$.
    \begin{enumerate}
        \item[(a)]
        {
            For any $E\subset X$ and a real $\epsilon>0$, there is a member $A\in\mc{A}_\sigma$ such that $E\subset A$ and $\mu^*(A)\leq\mu^*(E)+\epsilon$.
        }
        \item[(b)]
        {
            Suppose $E\subset X$ and $\mu^*(E)<\infty$.
            Then $E$ is $\mu^*$-measurable if and only if there is a member $B\in\mc{A}_{\sigma\delta}$ such that $E\subset B$ and $\mu^*(B\setminus E)=0$.
        }
        \item[(c)]
        {
            If $\mu_0$ is \textit{$\sigma$-finite}, the restriction $\mu^*(E)<\infty$ in (b) is superfluous.
        }
    \end{enumerate}
\end{lem}
\begin{proof}
    \begin{enumerate}
        \item[(a)]
        {
            Let $\{A_n\}_n$ be a countable covering of $E$ by members of $\mc{A}$ such that $\sum_n\mu_0(A_n)\leq\mu^*(E)+\epsilon$.
            Then $E\subset\bigcup_n A_n\in\mc{A}_\sigma$ and $\mu^*(\bigcup_n A_n)\leq\sum_n\mu^*(A_n)=\sum_n\mu_0(A_n)\leq\mu^*(E)+\epsilon$.
        }
        \item[(b)]
        {
            Note that every member of $\genone{\mc{A}}$ is $\mu^*$-measurable.

            Assume first that $\mu^*(E)<\infty$ and $E$ is $\mu^*$-measurable.
            For each $n\in\bb{N}$, let $A_n$ be a member of $\mc{A}_\sigma$ such that $E\subset A_n$ and $\mu^*(A_n)<\mu^*(E)+n^{-1}$; and let $B=\bigcap_n A_n$.
            Clearly, $B\in\mc{A}_{\sigma\delta}$, $E\subset B$, and $\mu^*(B)=\mu^*(E)$.
            Because $E$ is $\mu^*$-measurable, $\mu^*(B)=\mu^*(E)+\mu^*(B\setminus E)$, so $\mu^*(B\setminus E)=0$.

            Assume conversely that there is a member $B\in\mc{A}_{\sigma\delta}$ such that $E\subset B$ and $\mu^*(B\setminus E)=0$.
            Whenever $A\subset X$, we have $\mu^*(A)=\mu^*(A\cap B)+\mu^*(A\setminus B)\geq\mu^*(A\cap E)+\mu^*(A\setminus B)$.
            Because $\mu^*(A\setminus E)\leq\mu^*(A\setminus B)+\mu^*(A\cap(B\setminus E))=\mu^*(A\setminus B)$, we have $\mu^*(A)\geq\mu^*(A\cap E)+\mu^*(A\setminus E)$.
            Thus, $E$ is $\mu^*$-measurable.
        }
        \item[(c)]
        {
            Let $\{X_n\}_n$ be a countable covering of $X$ by members of $\mc{A}$ such that $\mu_0(X_n)<\infty$ for all $n$, and define $E_n=E\cap X_n$; for each $n$, $\mu_0(E_n)<\infty$.
            Now assume $E$ is $\mu^*$-measurable.
            Then so is each $E_n$ (because $X_n$ is also $\mu^*$-measurable), so there is a member $B_n\in\mc{A}_{\sigma\delta}$ such that $E_n\subset B_n$ and $\mu^*(B_n\setminus E_n)=0$.
            The union $B$ of $B_n$ for all $n$ satisfies $E\subset B$ and $\mu^*(B\setminus E)=0$.
            Conversely, if such $B$ for $E$ exists, let $B_n=B\cap X_n$; $E_n=E\cap X_n\subset B_n$ and $\mu^*(B_n\setminus E_n)=0$.
            Hence, each $E_n$ is $\mu^*$-measurable, so $E=\bigcup_n E_n$ is also $\mu^*$-measurable.
        }
    \end{enumerate}
    This completes the proof of lemma.
\end{proof}

\begin{thm}
    If $(X, \mc{M}, \mu)$ is a \textit{$\sigma$-finite} measure space, then the Carath\'{e}odory extension of $\mu$ is the completion of $\mu$.
\end{thm}
\begin{proof}
    Note that the domain of the Carath\'{e}odory extension $\ol\mu$ of $\mu$ is the collection $\mc{M}^*$ of $\mu^*$-measurable sets in $X$, where $\mu^*$ is the outer measure induced by $\mu$.

    We first show that $\mc{M}^*$ and the completion $\ol{\mc{M}}$ of $\mc{M}$ with regard to $\mu$ coincide.
    Assume first that $E\subset X$ is $\mu^*$-measurable.
    Then there is a member $B\in\mc{M}_{\sigma\delta}=\mc{M}$ containing $E$ such that $\mu^*(B\setminus E)=0$.
    For each $j\in\bb{N}$, let $\{K_j(n)\}_n$ be a countable covering of $B\setminus E$ by members of $\mc{M}$ such that $\sum_n\mu(K_j(n))<1/j$.
    Letting $K_j=\bigcup_n K_j(n)\in\mc{M}$, we have $\mu(K_j)\leq\sum_n\mu(K_j(n))<1/j$; $\mu(K)=0$ if $K=\bigcap_{j=1}^\infty K_j\in\mc{M}$.
    Moreover, $B\setminus E$ is a subset of the $\mu$-null set $K$.
    Therefore, $B\supset E=B\setminus(B\setminus E)\supset B\setminus K$, i.e., $E\in\ol{\mc{M}}$ and $\mc{M}^*\leq\ol{\mc{M}}$.
    \color{brown}The converse inclusion is easy to verify\color{black}, so we may conclude that $\ol{\mc{M}}=\mc{M}^*$.
    

    Note that both $\ol\mu$ and the completion of $\mu$ are extensions of $\mu$ to a complete measure on $\ol{\mc{M}}$.
    By the uniqueness of such an extension, the completion of $\mu$ and $\ol\mu$ coincide.
\end{proof}