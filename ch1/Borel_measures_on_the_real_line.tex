\section{Borel measures on the real line}

\begin{nota}
    In this section, an interval of the form $(a, b]$ with $-\infty\leq a\leq b\leq\infty$ is called an o-c interval, and the collection of o-c intervals is denoted by $\mc{A}$.
\end{nota}
\begin{rmk}
    $\mc{A}$ is an algebra on $\bb{R}$, and the $\sigma$-algebra on $\bb{R}$ generated by $\mc{A}$ is the Borel $\sigma$-algebra $\mc{B}_\bb{R}$ on $\bb{R}$.
\end{rmk}

\begin{obs}
    Let $\mu$ be a finite Borel measure on $\bb{R}$ and let $F: \bb{R}\rightarrow\bb{R}$ be the function defined by $F(x)=\mu((-\infty, x])$ for all $x\in\bb{R}$.
    Then $F$ is increasing and right continuous.
    Moreover, whenever $a, b$ are real numbers with $a<b$, then $\mu((a, b])=F(b)-F(a)$.
\end{obs}
\begin{sketch}
    Turning around the above observation; starting from an increasing and right continuous function $F$, we will construct a Borel measure $\mu_F$.
\end{sketch}

\begin{lem}
    Let $F: \bb{R}\rightarrow\bb{R}$ be an increasing and right continuous function.
    If $(a_j, b_j]$ for $j=1, \cdots, n$ are pairwise disjoint o-c intervals, let
    \begin{align*}
        \mu_0\left(\bigsqcup_{j=1}^n (a_j, b_j]\right)=\sum_{j=1}^n(F(b_j)-F(a_j)),
    \end{align*}
    and $\mu_0(\varnothing)=0$.
    Then $\mu_0$ is a premeasure on $\mc{A}$.
\end{lem}
\begin{proof}
    \textbf{Step 1: Checking if $\mu_0$ is well-defined.}\newline\noindent
    Let $\{I_i\}_{i=1}^m$ and $\{J_j\}_{j=1}^n$ be collections of pairwise disjoint o-c intervals with the same union $I$.
    If $I=(a, b]$ for some $a, b\in\bb{R}$, we can easily check that $\mu_0$ coincide for those collections, i.e., $\mu_0$ is well-defined for a single o-c interval.
    In general, because each $I_i$ or $J_j$ is an o-c interval and so is $I_i\cap J_j$, we have
    \begin{align*}
        \sum_i\mu_0(I_i)=\sum_{i,\,j}\mu_0(I_i\cap J_j)=\sum_j\mu_0(J_j).
    \end{align*}
    Therefore, the function $\mu_0$ is well-defined.
    Moreover, by construction, $\mu_0$ is finitely additive.

    \textbf{Step 2: Showing that $\mu_0$ is a premeasure on $\mc{A}$.}\newline\noindent
    Since $\mu_0$ satisfies the empty set condition, it remains to show that $\mu_0$ is appropriately $\sigma$-additive.
    Let $\{I_n\}_{n=1}^\infty$ be a countable collection of pairwise disjoint o-c intervals with the union $I$ in $\mc{A}$.
    Because $I\in\mc{A}$ so that $I$ can be partitioned into finitely many pairwise disjoint o-c interval and $\mu_0$ is finitely additive, without loss of generality, we may assume $I=(a, b]$ for some $a, b\in\bb{R}$.
    Then
    \begin{align*}
        \mu_0(I)=\mu_0\left(\bigsqcup_{n=1}^N I_n\right)+\mu_0\left(I\setminus\bigsqcup_{n=1}^n I_n\right)\geq\mu_0\left(\bigsqcup_{n=1}^n I_n\right)=\sum_{n=1}^n\mu_0(I_n),
    \end{align*}
    so $\mu_0(\bigsqcup_n I_n)\geq\sum_n\mu_0(I_n)$.
    It now remains to show the converse inequality.
    \begin{enumerate}
        \item[(\romannumeral 1)]
        {
            First, assume that $I$ is bounded.
            By the right continuity of $F$, given $\epsilon>0$, there is $\delta>0$ and $\delta_n>0$ for each $n\in\bb{N}$ such that $F(a)\leq F(a+\delta)<F(a)+\epsilon$ and $F(b_n)\leq F(b_n+\delta_n)<F(b_n)+\epsilon\cdot 2^{-n}$.
            Then, the open intervals $(a_n, b_n+\delta_n)$ ($n\in\bb{N}$) cover the compact interval $[a+\delta, b]$, so finitely many intervals $(a_n, b_n+\delta_n)$ cover $[a+\delta, b]$; (after renumbering, if necessary) let such open intervals be written by $(a_i, b_i+\delta_i)$ for $i=1, \cdots, N$, where
            \begin{center}
                $b_j+\delta_j\in(a_{j+1}, b_{j+1}+\delta_{j+1})$ for $j=1, \cdots, N-1$.
            \end{center}
            (Also, we may discard an interval which is contained in another interval.)
            Then the following string of inequalities holds:
            \begin{eqnarray*}
                \mu_0(I)
                &=&F(b)-F(a)\leq F(b)-F(a+\delta)+\epsilon\\
                &\leq&F(b_N+\delta_N)-F(a_1)+\epsilon\\
                &=&F(b_N+\delta_N)-F(a_N)+\sum_{j=1}^{N-1}(F(a_{j+1})-F(a_j))+\epsilon\\
                &\leq&F(b_N+\delta_N)-F(a_N)+\sum_{j=1}^{N-1}(F(b_j+\delta_j)-F(a_j))+\epsilon\\
                &<&\sum_{j=1}^N(F(b_j)-F(a_j)+\epsilon\cdot 2^{-j})+\epsilon\\
                &<&\sum_n\mu_0(I_n)+2\epsilon.
            \end{eqnarray*}
        }
        \item[(\romannumeral 2)]
        {
            We now assume $I$ is unbounded, i.e., $I=(-\infty, b]$ or $I=(a, \infty)$.
            For the case $I=(-\infty, b]$, given $M<\infty$, some finitely many intervals $(a_j, b_j+\delta_j)$ cover $[-M, b]$.
            The same reasoning proves $F(b)-F(-M)\leq\sum_n\mu_0(I_n)+2\epsilon$ for all $M>0$.
            When $I=(a, \infty)$, for all $M>a$, we have $F(M)-F(a)\leq\sum_n\mu_0(I_n)+2\epsilon$.
        }
    \end{enumerate}
    Therefore, $\mu_0$ is appropriately $\sigma$-additive, so $\mu_0$ is a premeasure on $\mc{A}$.  
\end{proof}

\begin{thm}[Uniqueness of a Borel measure on $\bb{R}$]
    Let $F: \bb{R}\rightarrow\bb{R}$ be an increasing and right continuous function.
    \begin{enumerate}
        \item[(a)]
        {
            There is a unique Borel measure $\mu_F$ on $\bb{R}$ such that $\mu_F((a, b])=F(b)-F(a)$ for all $a, b\in\bb{R}$ with $a<b$, i.e., there is a unique extension of $\mu_0$ in the preceeding lemma to a Borel measure on $\bb{R}$.
        }
        \item[(b)]
        {
            If $G$ is another such function, we have $\mu_F=\mu_G$ if and only if $F-G$ is constant.
        }
        \item[(c)]
        {
            Conversely, if $\mu$ is a Borel measure on $\bb{R}$ which is finite on all bounded Borel sets and we define
            \begin{eqnarray*}
                H(x)=\left\{\begin{matrix}
                    \mu((0, x])     &   (x>0)\\
                        0           &   (x=0)\\
                    -\mu((x, 0])    &   (x<0)
                \end{matrix}\right.,
            \end{eqnarray*}
            then $H$ is an increasing and right continuous function on $\bb{R}$, and $\mu=\mu_H$.
        }
    \end{enumerate}
\end{thm}
\begin{proof}
    \hangindent=0.65cm
    \noindent(a)
    A desired Borel measure on $\bb{R}$ should extend the premeasure $\mu_0$ on $\mc{A}$ in the preceeding lemma; because $\bb{R}$ is $\sigma$-finite for $\mu_0$, a Borel measure extending $\mu_0$ is unique.

    \noindent(b)
    Clear.
    
    \noindent(c)
    Given a Borel measure $\mu$ on $\bb{R}$ which is finite on all bounded Borel sets, the suggested function $H$ is well-defined, increasing, and right continuous.
    Because $\mu$ and $\mu_F$ coincide on $\mc{A}$ and $\mu$ is a unique extension of $\mu_0$, we have $\mu=\mu_F$.
\end{proof}
\begin{rmk}
    Suppose $F: \bb{R}\rightarrow\bb{R}$ is an increasing and right continuous function.
    Because $\bb{R}$ is $\sigma$-finite for $\mu_F$, the Carath\'{e}odory extension of $\mu_F$ is the completion of $\mu_F$ and the domain contains $\mc{B}_\bb{R}$.
    We shall usually denote this complete measure also by $\mu_F$ and call this the Lebesgue-Stieltjes measure associated to $F$.
\end{rmk}

In the rest of this section, we fix a (complete) Lebesgue-Stieltjes measure $\mu$ on $\bb{R}$ associated to an increasing and right continuous function $F$, and we denote $\mc{M}_\mu$ the domain of $\mu$, i.e., 
\begin{eqnarray*}
    \mc{M}_\mu
    &=&\textsf{(Carath\'eodory style)}\,\{
    E\subset\bb{R}
    :
    \textsf{$E$ is $(\mu_F)^*$-measurable}
    \}\\
    &=&\textsf{($\mu_F$-completion style)}\,\{
    E\cup F
    :
    \textsf{$E\in\mc{B}_\bb{R}$ and $F$ is a subset of a $\mu_F$-null set}
    \}.
\end{eqnarray*}
Then, for any $E\in\mc{M}_\mu$, we have
\begin{eqnarray*}
    \mu(E)
    &=&\inf\left\{
        \sum_n\mu_F(A_n)
        :
        E\subset\bigcup_n A_n,\textsf{ where }A_n\in\mc{B}_\bb{R}
        \right\}\\
    &=&\inf\left\{
        \sum_n(F(b_n)-F(a_n))
        :
        E\subset\bigcup_n(a_n, b_n]
        \right\}\\
    &=&\inf\left\{
        \sum_n\mu((a_n, b_n])
        :
        E\subset\bigcup_n(a_n, b_n]
        \right\},
\end{eqnarray*}
which is easy to check.
(Note that the Lebesgue-Stieltjes measure $\mu$ is extended from the measure $\mu_F$, not directly from $\mu_0$. \color{brown}It is left as an exercise to explain how the above equality could hold.\color{black})

We first check the o-c intervals in the above equation can be replaced by open intervals.
\begin{prop}
    For any $E\in\mc{M}_\mu$,
    \begin{center}
        $\ds{\mu(E)=\inf\left\{
            \sum_{n=1}^\infty\mu_F((a_n, b_n))
            :
            E\subset\bigcup_{n=1}^\infty(a_n, b_n)
        \right\}}$.\footnote{In this proposition we try to distinguish $\mu$ and $\mu_F$, which is, in fact, not necessary; all $\mu_F$'s in this proposition (including the proof) can be replaced by $\mu$.}
    \end{center}
\end{prop}
\begin{proof}
    Let the value on the right hand side be denoted by $\nu(E)$.
    For each $n$, the open interval $(a_n, b_n)$ is a union of pairwise disjoint o-c intervals $(c_n(j), c_n(j+1)]$ for $j\in\bb{N}$, where $c_n(1)=a_n$, $(c_n(j))_j$ is increasing, and $c_n(j)\nearrow b_n$ as $j\rightarrow\infty$.
    We then have
    \begin{align*}
        \sum_n\mu_F((a_n, b_n))=\sum_{n, j}\mu_F((c_n(j), c_n(j+1)])\geq\mu(E)
    \end{align*}
    so that $\mu(E)\leq\nu(E)$.
    To show the converse inequality, suppose $\epsilon>0$ is given and let $((a_n, b_n])_{n=1}^\infty$ be a countable covering of $E$ by o-c intervals such that
    \begin{align*}
        \mu(E)\leq\sum_{n=1}^\infty\mu_F((a_n, b_n])<\mu(E)+\epsilon.
    \end{align*}
    For each $n$, let $\beta_n>0$ be a real number greater than $b_n$ such that $F(\beta_n)<F(b_n)+\epsilon\cdot 2^{-n}$.
    Because $((a_n, \beta_n))_{n=1}^\infty$ covers $E$, we have
    \begin{align*}
        \sum_n\mu_F((a_n, \beta_n))\leq\sum_n\mu_F((a_n, b_n])+\epsilon\leq\mu(E)+2\epsilon,
        \quad\textsf{so}\quad
        \nu(E)\leq\mu(E)+2\epsilon.
    \end{align*}
    Therefore, $\nu(E)\leq\mu(E)$, proving the identity.    
\end{proof}

The above proposition introduces a method of computing $\mu(E)$, where $E\in\mc{M}_\mu$.
Remarking that an open subset of $\bb{R}$ is a countable union of open intervals in $\bb{R}$, such computation reduces to the method given in the following proposition.
\begin{prop}
    If $E\in\mc{M}_\mu$, then
    \begin{eqnarray*}
        \mu(E)
        &=&\inf\{
            \mu(U)
            :
            \textsf{$U\supset E$ and $U$ is open in $\bb{R}$}
            \}\\
        &=&\sup\{
            \mu(K)
            :
            \textsf{$K\subset E$ and $K$ is compact}\}.
    \end{eqnarray*}
\end{prop}
\begin{proof}
    The first identity is straightforward, because every open subset of $\bb{R}$ is a countable union of open intervals in $\bb{R}$.
    To prove the second identity, suppose first $E$ is bounded and not closed; if $E$ is closed, then the equality is clear, since $E$ is compact.
    Considering $\ol{E}\setminus E$, we can find an open subset $U\subset\bb{R}$ containing $\ol{E}\setminus E$ such that $\mu(U)<\mu(\ol{E}\setminus E)+\epsilon$.
    Setting $K=\ol{E}\setminus U$, we find that $K$ is compact and is contained in $E$.
    Because $E\setminus K=E\cap U$, we have
    \begin{eqnarray*}
        \mu(K)&=&\mu(E)-\mu(E\cap U)=\mu(E)-(\mu(U)-\mu(U\setminus E))\\
        &\geq&\mu(E)-\mu(U)+\mu(\ol{E}\setminus E)\geq\mu(E)-\epsilon,
    \end{eqnarray*}
    proving the second equality for bounded $E$ in $\mc{M}_\mu$.
    If $E$ is unbounded, let $E_j=E\cap(j, j+1]$, which is bounded for each $j\in\bb{Z}$.
    For each $j\in\bb{Z}$, let $K_j$ be a compact subspace of $E_j$ such that $\mu(E_j)-\epsilon\cdot 3^{-|j|}<\mu(K_j)\leq\mu(E_j)$, and define $H_n$ be the union of $K_j$'s for $|j|\leq n$.
    Note that each $H_n$ is a compact subspace of $E$ and $\mu(H_n)\geq\mu\left(\bigcup_{j=-n}^n E_j\right)-2\epsilon$.
    Because $\mu(E)=\lim\mu\left(\bigcup_{j=-n}^n E_j\right)$, we have $\mu(H_n)\geq\mu(E)-3\epsilon$ for large $n$'s.
\end{proof}

We already checked that the Lebesgue-Stieltjes measure $\mu$ associated to $F$ is the Carath\'{e}odory extension of the measure $\mu_F$ on $\mc{B}_\bb{R}$ and the completion of $\mu_F$, which also explains the domain of the Lebesgue-Stieltjes measure $\mu$.
The following proposition states another way of expressing the domain of $\mu$; all Borel sets or all sets in $\mc{M}_\mu$ are of a reasonably simple form modulo sets of measure zero.
\begin{prop}
    Suppose $E$ is a subset of $\bb{R}$.
    Then, the following statements are equivalent:
    \begin{enumerate}
        \item[(a)]
        {
            $E\in\mc{M}_\mu$, i.e, $E$ belongs to the domian of the (complete) Lebesgue-Stieltjes measure $\mu$.
        }
        \item[(b)]
        {
            $E=V\setminus N_1$, where $V$ is a $G_\delta$ set and $\mu(N_1)=0$.
        }
        \item[(c)]
        {
            $E=H\cup N_2$, where $H$ is an $F_\sigma$ set and $\mu(N_2)=0$.
        }
    \end{enumerate}
\end{prop}
\begin{proof}
    Since $\mu$ is complete on $\mc{M}_\mu$, (b) and (c) each imply (a).
    We will show that (a) implies (b) and (c).
    First, assume $E\in\mc{M}_\mu$ and $\mu(E)<\infty$.
    Using the preceeding proposition, for each $n\in\bb{N}$, let $U_n$ and $V_n$ be open and compact subsets of $\bb{R}$ such that
    \begin{align*}
        E\subset U_n,\quad \mu(U_n)<\mu(E)+\dfrac{1}{n},\\
        E\supset K_n,\quad \mu(K_n)>\mu(E)-\dfrac{1}{n}.
    \end{align*}
    Letting $V=\bigcap_{n=1}^\infty U_n$ and $H=\bigcup_{n=1}^\infty K_n$, we find that $V$ is a $G_\delta$ set and $H$ is an $F_\sigma$ set and that $\mu(V)=\mu(E)$ and $\mu(H)=\mu(E)$.
    Because $E\subset V$ and $E\supset H$, $\mu(V\setminus E)=\mu(E\setminus H)=0$.
    When $\mu(E)=\infty$, we may use the $\sigma$-finiteness of $\bb{R}$ relative to $\mu$.
    For each $j\in\bb{Z}$, let $E_j=E\cap[j, j+1)$ and let $\{U_{j, k}\}_{k\in\bb{N}}$ be a countable collection of open sets in $\bb{R}$ such that
    \begin{align*}
        E_j\subset U_{j, k},\quad \mu(E_j)\leq\mu(U_{j, k})<\mu(E_j)+\dfrac{1}{2^{|j|}}\dfrac{1}{2^k}.
    \end{align*}
    Define $V_k=\bigcup_{j\in\bb{Z}} U_{j, k}$; then $V_k$ is an open subset of $\bb{R}$ containing $E$ and $\mu(E)\leq\mu(V_k)<\mu(E)+3\cdot 2^{-k}$, so $V=\bigcap_{k\in\bb{N}} V_k$ is an open subset of $\bb{R}$ containing $E$ such that $\mu(E)=\mu(V)$.
    Hence, $E=V\setminus(V\setminus E)$ is a desired form of identity.
    To show that (a) implies (c), note that $X\setminus E\in\mc{M}_\mu$ if $E\in\mc{M}_\mu$.
    By (b), $X\setminus E=V\setminus N_1$ for some $G_\delta$ set $V$ and a $\mu$-null set $N_1$, so $E=(X\setminus V)\cup N_1$ and (c) is deduced.
\end{proof}

There is an approximation theorem for sets in $\mc{M}_\mu$ of finite measures, given as the following proposition.
The proposition is not even an equivalence theorem, but it will play an essential role in proving the density of $C^0$ space in $L^1$ space (in the $L^1$ metric topology).
\begin{prop}
    Suppose $E\in\mc{M}_\mu$ and $\mu(E)<\infty$.
    Given $\epsilon>0$, there is a set $F$ which is a finite union of open intervals such that $\mu(E\triangle F)<\epsilon$.
\end{prop}
\begin{proof}
    Let $U$ be an open set in $\bb{R}$ containing $E$ such that $\mu(E)\leq\mu(U)<\mu(E)+\epsilon$, and let $((a_n, b_n))_n$ be the countable collection of pairwise disjoint open intervals in $\bb{R}$ whose union is $U$.
    Because $\mu(U)$ is finite, there is an integer $N$ such that $\sum_{n>N}\mu((a_n, b_n))<\epsilon$.
    Letting $F=\bigsqcup_{n=1}^N (a_n, b_n)$, we have
    \begin{align*}
        \mu(E\setminus F)\leq\mu(U\setminus F)<\epsilon,\quad \mu(F\setminus E)\leq\mu(U\setminus E)<\epsilon.
    \end{align*}
    Hence, $\mu(E\triangle F)<2\epsilon$, as desired.
\end{proof}

\begin{nota}
    When $\mu$ is the Lebesgue-Stieltjes measure on $\bb{R}$ associated to the identity map on $\bb{R}$, $\mu$ is called the Lebesgue measure, and denoted by $m$.
    Furthermore, the domain $\mc{M}_m$ of the Lebesgue measure $m$ is denoted by $\mc{L}$.
    Sometimes, we also call the restriction of $m$ to $\mc{B}_\bb{R}$ the Lebesgue measure, too.
\end{nota}

Among the most significant properties of Lebesgue measure are its invariance under translations and simple behavior under dilations.
\begin{prop}
    If $E\in\mc{L}$ and $s, r\in\bb{R}$, then $E+s,\,rE\in\mc{L}$, and $m(E+s)=m(E)$ and $m(rE)=|r|m(E)$.
\end{prop}
\begin{proof}
    Almost clear.
\end{proof}

\begin{exmp}[Topological magnitude and measure need not be consistent]
    Let $\{r_i\}$ be an enumeration of the rational numbers in $[0, 1]$, and given $\epsilon>0$, let $I_i$ be the open interval centered at $r_i$ of length $2^{-i}\epsilon$; then set $U=(0, 1)\cap\bigcup_{i=1}^\infty I_i$.
    The set $U$ is open in $\bb{R}$ and dense in $[0, 1]$, but $m(U)\leq\epsilon$.
    The set $K=[0, 1]\setminus I$ is closed in $\bb{R}$ and nowhere dense in $\bb{R}$, i.e., the closure of $K$ in $\bb{R}$ has no interior point.
    Thus, $K$ is \textit{topologically small}, but $\mu(K)\geq 1-\epsilon$.
\end{exmp}

\begin{exmp}[Cantor set]
    Let $C$ denote the Cantor set.
    The following statements are basic properties regarding the Cantor set.
    \begin{enumerate}
        \item[(a)]
        {
            $C$ is compact, nowhere dense in $\bb{R}$, and totally disconnected.\footnote{A nonempty subset $E$ of a topological space $X$ is said to be totally disconnected if the only connected subspaces of $E$ are the singletons.}
            Moreover, $C$ has no isolated points.
        }
        \item[(b)]
        {
            $m(C)=0$.
        }
        \item[(c)]
        {
            $\card{C}=\card{\bb{R}}$.
        }
        Let $f: C\rightarrow[0, 1]$ be the monotonically increasing function defined as follows:
        \begin{quotation}
            If $0.a_1a_2a_3\cdots$ is the base-3 expansion of $x\in C$ with $a_i=0$ or $a_i=2$ for all $i\in\bb{N}$, let $0.b_1b_2b_3\cdots$ with $b_i={a_i}/{2}$ be the base-2 expansion of $f(x)$.
            (In fact, this definition contains a gap.)
        \end{quotation}
        It is clear that whenever $x, y\in C$ and $x<y$ we have $f(x)<f(y)$, unless both $x$ and $y$ are the endpoints of one interval which is deleted from $[0, 1]$ to form $C$; in this case, we have $f(x)=f(y)$.
        By declaring $f$ to be constant on each interval deleted from $[0, 1]$ to form $C$, we can extend $f$ to $[0, 1]$.
        Because $f$ is monotonically increaing on $[0, 1]$ and is onto $[0, 1]$, $f$ is surprisingly a continuou function.
        Such $f$ is called the Cantor function.
    \end{enumerate}
\end{exmp}

\subsection*{Problems}

\begin{prob}[Exercise 1.30]
    Suppose $E\in\mc{L}$ and $m(E)>0$.
    Show that for any real $0<\alpha<1$ there is an open interval $I$ such that $m(E\cap I)>\alpha m(I)$.
\end{prob}
\begin{sol}
    Assume first that the statement is valid whenever $E\in\mc{L}$ and $m(E)<\infty$.
    If $E\in\mc{L}$ and $m(E)=\infty$, there is an integer $j>0$ such that $F=E\cap(-j, j)$ is of nonzero (finite) measure.
    Finding an open interval $I$ for $F$, we have $m(E\cap I)\geq m(F\cap I)>\alpha m(I)$.

    It remains to prove the statement for $m(E)<\infty$.
    Let $U$ be an open subset of $\bb{R}$ containing $E$ such that $m(E)\leq m(U)<(1+\rho)m(E)$ for some $\rho>0$.
    Since $U$ is open in $\bb{R}$, there is a (unique) countable collection $\{(a_n, b_n)\}_{n=1}^\infty$ of pairwise disjoint open intervals in $\bb{R}$ with the union $U$.
    Letting $E_j=E\cap (a_n, b_n)$ for each $n\in\bb{N}$, we have
    \begin{align*}
        \sum_{n=1}^\infty m(E_n) \leq\sum_{n=1}^\infty m((a_n, b_n)) < (1+\rho)\sum_{n=1}^\infty m(E_n)
    \end{align*}
    Hence, for some integer $j\in\bb{N}$, we have $m((a_j, b_j))<(1+\rho)m(E\cap (a_j, b_j))$.
    This proves the statement when letting $\rho=\alpha^{-1}-1$.
\end{sol}

\begin{prob}[Exercise 1.31]
    Suppose $E\in\mc{L}$ and $m(E)>0$.
    Show that $E-E$ contains an open interval centered at 0.
\end{prob}
\begin{sol}
    Let $\alpha$ be a real number such that $0<\alpha<1$.
    Then there is an open interval $I$ in $\bb{R}$ such that $m(E\cap I)>\alpha m(I)$.
    Let $E_0=E\cap I$ and assume that $E_0-E_0$ contains no open interval centered at 0.
    Then, whenever $\epsilon>0$, there is a positive real number $a<\epsilon$ for which $E_0$ and $a+E_0$ are disjoint. \color{brown}(Why?) \color{black}
    For such $0<a<\epsilon$, from the inclusion $E_0\sqcup(a+E_0)\subset I\cup(a+I)$, we have $2m(E_0)\leq a+m(I)$, i.e., $2\alpha m(I)<m(I)+a$.
    Here arises a contradiction, when $1/2<\alpha<1$.
    Therefore, $E_0-E_0$ contains an open interval centered at 0, proving the statement.
\end{sol}

\begin{prob}
    Let $A$ be the subset of $[0, 1]$ which consists of all numbers which do not have the digit 4 appearing in their decimal expansion.
    Find $m(A)$.
\end{prob}
\begin{sol}
    Oberve that $A\subset[0, 0.4)\sqcup[0.5, 1]$.
    Proceeding as constructing the Cantor set, we have $m(A)\leq 0.9^n$ for all positive integer $n$, implying that $m(A)=0$.
\end{sol}