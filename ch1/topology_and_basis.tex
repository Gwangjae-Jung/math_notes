\section{Topology and basis}

\begin{defi}[Topology on a set]
    Let $X$ be a nonempty set.
    A collection $\mc{T}$ is called a topology on $X$ if
    \begin{enumerate}
        \item[(a)] Both $X$ and the empty set belongs to $\mc{T}$,
        \item[(b)] $\mc{T}$ is closed under arbitrary unions,
        \item[(c)] $\mc{T}$ is closed under arbitrary finite intersections.
    \end{enumerate}
\end{defi}

\begin{defi}[Subbasis and basis of a topology]
    A collection $\mc{B}$ is called a subbasis of the topology on $X$ if $\mc{B}$ covers $X$.
    A collection $\mc{B}$ is called a basis of the topology on $X$ if
    \begin{enumerate}
        \item[(a)] $\mc{B}$ covers $X$, i.e., $\mc{B}$ is a subbasis on $X$,
        \item[(b)] given $B_1, B_2\in\mc{B}$, there is another member $B_3\in\mc{B}$ contained in $B_1\cap B_2$.
    \end{enumerate}
    The topology $\langle\mc{B}\rangle$ on $X$ generated by the basis $\mc{B}$ is the following collection of subsets of $X$:
    \begin{center}
        $\genone{\mc{B}}=\left\{U\subset X:\textsf{Given }x\in U, \textsf{there is a basis element }B\in\mc{B}\textsf{ such that }x\in B\subset U\right\}$.
    \end{center}
    In accordance with the definition of the first-countability, we will say that $\genone{\mc{B}}$ consists of all subsets of $X$ based on $\mc{B}$.\footnote{The readers are supposed to be familiar with such definition of the topology on $X$ generated by the basis $\mc{B}$. Remark how you defined an open subset of a metric space when you first studied mathematical analysis.}
\end{defi}

\begin{prop}
    Let $X$ be a set and suppose $\Gamma$ is a subset of $\mc{P}(X)$.
    \begin{enumerate}
        \item[(a)]
            The topology generated by $\Gamma$ is the collection $\mc{C}$ of all unions of members in $\Gamma$.
        \item[(b)]
            The topology generated by $\Gamma$ is the intersection $I$ of all topologies on $X$ containing $\Gamma$. (Hence, the topology on $X$ generated by $\Gamma$ is the smallest topology on $X$ containing $\Gamma$.)
    \end{enumerate}
\end{prop}
\begin{proof}
    \begin{enumerate}
        \item[(a)]
            By definition, it is clear that $\mc{C}$ is contained in $\genone{\Gamma}$.
            To show the converse inclusion, suppose $U\in\genone{\Gamma}$.
            By definition, for each $x\in U$, there is a basis member $B_x\in\Gamma$ such that $x\in B_x\subset U$, hence $U$ is the union of $B_x$'s with $x\in U$.
        \item[(b)]
            Because $\genone{\Gamma}$ is a topology on $X$ containing $\Gamma$, $I$ is contained in $\genone{\Gamma}$.
            Conversely, because each topology on $X$ containing $\Gamma$ also includes $\genone{\Gamma}$, so $I$ contains $\genone{\Gamma}$.
    \end{enumerate}
\end{proof}

\begin{rmk}
    \begin{align*}
        \textsf{(subbasis)}\xrightarrow{\textsf{finite intersections}}
        \textsf{(basis)}\xrightarrow{\textsf{arbitrary unions}}
        \textsf{(topology)}
    \end{align*}
\end{rmk}