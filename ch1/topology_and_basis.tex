\section{Topology and basis}

\begin{defi}[Topology on a set]
    Let $X$ be a nonempty set.
    A collection $\mc{T}$ is called a topology on $X$ if
    \begin{enumerate}
        \item[(a)]
        {
            both $X$ and the empty set belongs to $\mc{T}$,
        }
        \item[(b)]
        {
            $\mc{T}$ is closed under arbitrary unions,
        }
        \item[(c)]
        {
            $\mc{T}$ is closed under arbitrary finite intersections.
        }
    \end{enumerate}
\end{defi}

\begin{defi}[Subbasis and basis of a topology]
    A collection $\mc{B}$ is called a subbasis of the topology on $X$ if $\mc{B}$ covers $X$.
    A collection $\mc{B}$ is called a basis of the topology on $X$ if
    \begin{enumerate}
        \item[(a)]
        {
            $\mc{B}$ covers $X$, i.e., $\mc{B}$ is a subbasis on $X$,
        }
        \item[(b)]
        {
            given $B_1, B_2\in\mc{B}$, there is another member $B_3\in\mc{B}$ contained in $B_1\cap B_2$.
        }
    \end{enumerate}
    The topology $\langle\mc{B}\rangle$ on $X$ generated by the basis $\mc{B}$ is the following collection of subsets of $X$:
    \begin{align*}
        \genone{\mc{B}}=\left\{
            U\subset X:
                \begin{array}{c}
                    \textsf{Given $x\in U$, there is a basis member}\\
                    \textsf{$B\in\mc{B}$ such that $x\in B\subset U$}
                \end{array}\right\}.
    \end{align*}
    (Remark how an open subset of a metric space is defined in the course of mathematical analysis.)
    In accordance with the definition of the first-countability, we will say that $\genone{\mc{B}}$ consists of all subsets of $X$ based on $\mc{B}$.
\end{defi}

\begin{obs}
    Let $X$ be a set and suppose $\mc{B}$ is a basis of $X$.
    \begin{enumerate}
        \item[(a)]
        {
            The topology generated by $\mc{B}$ is the collection $\mc{C}$ of all unions of members in $\mc{B}$.
        }
        \item[(b)]
        {
            The topology generated by $\mc{B}$ is the intersection $\mc{I}$ of all topologies on $X$ containing $\mc{B}$. (Hence, the topology on $X$ generated by $\mc{B}$ is the smallest topology on $X$ containing $\mc{B}$.)
        }
    \end{enumerate}
\end{obs}
\begin{proof}
    We first prove (a).
    By definition, it is clear that $\mc{C}$ is contained in $\genone{\mc{B}}$.
    To show the converse inclusion, suppose $U\in\genone{\mc{B}}$.
    By definition, for each $x\in U$, there is a basis member $B_x\in\mc{B}$ such that $x\in B_x\subset U$, hence $U$ is the union of $B_x$ for $x\in U$.

    We now prove (b).
    Because $\genone{\mc{B}}$ is a topology on $X$ containing $\mc{B}$, $\mc{I}$ is contained in $\genone{\mc{B}}$.
    Conversely, by (a), every topology on $X$ containing $\mc{B}$ also includes $\genone{\mc{B}}$.
    Thus, $\mc{I}$ contains $\genone{\mc{B}}$.
\end{proof}

\begin{lem}[Containment criterion]
    Let $\mc{B}, \mc{B}'$ be a basis of the topology $\mc{T}, \mc{T}'$ of $X$, respectively.
    Then the followings are equivalent:
    \begin{enumerate}
        \item[(a)]
        {
            $\mc{T}$ is finer than $\mc{T}'$.
        }
        \item[(b)]
        {
            For each point $x\in X$ and a basis member $B'\in\mc{B}'$ containing $x$, there is a basis member $B\in\mc{B}$ such that $x\in B\subset B'$.
        }
    \end{enumerate}
\end{lem}
\begin{proof}
    Assume (a) and let $x$ be a point of $X$ and $B'$ be a basis member containing $x$.
    Then $B'\in\mc{T}$, and there is a basis member $B\in\mc{B}$ such that $x\in B\subset B'$.

    Assume (b) and let $U'$ be a member of $\mc{T}'$.
    For each point $p\in U'$, there is a basis member $B_p\in\mc{B}$ such that $p\in B_p\subset U'$.
    (There was a leap in the argument: Letting $B_p'$ be a basis member of $\mc{B}'$ such that $p\in B_p'\subset U'$, by the assumption we have a basis member $B_p\in\mc{B}$ such that $x\in B_p\subset B_p'$.)
    Therefore, $U'\in\mc{T}$.
\end{proof}

\begin{rmk}
    \begin{enumerate}
        \item[(a)]
        {
            $
            \textsf{(subbasis)}\xrightarrow{\textsf{finite intersections}}
            \textsf{(basis)}\xrightarrow{\textsf{arbitrary unions}}
            \textsf{(topology)}
            $
        }
        \item[(b)]
        {
            Suppose that $\mc{B}$ is a basis of a topology $\mc{T}$ on a set $X$.
            Then the topology generated by $\mc{B}$ as a `subbasis' is $\mc{T}$.
            If $\mc{B}'$ is the collection of all finite intersections of the members of $\mc{B}$, then $\mc{B}'\supset\mc{B}$, so the topology $\mc{T}'$ generated by $\mc{B}'$ as a basis (i.e., by $\mc{B}$ as a subbasis) is finer than $\mc{T}$.
            To show that $\mc{T}$ is finer than $\mc{T}'$, it suffices to prove for each $x\in X$ and a basis member $B'\in\mc{B}'$ containing $x$ there is a basis member $B\in\mc{B}$ such that $x\in B\subset B'$.
            Becasue $\mc{B}$ is a basis, if $B'=B_1\cap\cdots\cap B_n$ for some $B_1, \cdots, B_n\in\mc{B}$ then there is a basis member $B_0\in\mc{B}$ such that $x\in B_0\subset B_1\cap\cdots\cap B_n=B'$.
        }
    \end{enumerate}
\end{rmk}