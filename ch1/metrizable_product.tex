\section{Metrizable product spaces}

Let $X_n$ be a metrizable space for each $n\in\bb{N}$.

\begin{nota}
    Given a metric $d_n$ inducing the topology on $X_n$ for each $n$,
    define the following metric on the product space $X:=\prod_{n=1}^\infty X_n$ as follows:
    \begin{align*}
        \ol{\rho}: X\times X\rightarrow\bb{R},\,(x, y)\mapsto\sup_{n\in\bb{N}}\left\{\ol{d_n}(x_n, y_n)\right\},\\
        D: X\times X\rightarrow\bb{R},\,(x, y)\mapsto\sup_{n\in\bb{N}}\left\{\frac{\ol{d_n}(x_n, y_n)}{n}\right\}.
    \end{align*}
\end{nota}

The former metric is called the uniform metric on $X$ and the latter metric is called the $D$-metric on $X$.
Among them, the definition of the former metric generalizes to arbitrary product spaces: If $\{X_\alpha\}_{\alpha\in I}$ is an indexed family of metric spaces and $X=\prod_{\alpha\in I} X_\alpha$, then we may define the uniform metric $\rho: X\times X\rightarrow \bb{R}$ by $\rho(x, y)=\sup_{\alpha\in I}\{\ol{d}_\alpha(x_\alpha, y_\alpha)\}$ for $(x, y)\in X\times X$.

\begin{thm}
    Let $(X_n, d_n)$ be a metric space for each $n\in\bb{N}$ and write $X=\prod_{n\in\bb{N}} X_n$.
    Then $D$-metric on $X$ induces the product topology on $X$.
\end{thm}
\begin{proof}
    We first show that the topology $\mc{T}_D$ induced by the $D$-metric is finer than the product topology.
    Given a point $x\in X$ and a basis member $B=\prod_{n\in\bb{N}}B_n$ of the product topology with
    \begin{align*}
        \ds{\{n\in\bb{N}:B_n\neq X_n\}=\{n_1, \cdots, n_k\}},
    \end{align*}
    let us find a basis member of $\mc{T}_D$ which contains $x$ and contained in $B$.
    For each $i=1, \cdots, k$, let $r_i$ be a real number such that $0<r_i<1$ and $B_{d_{n_i}}(x_{n_i}, r_i)\subset B_{n_i}$.
    Set
    \begin{align*}
        \epsilon=\min_{1\leq i\leq k}\left\{\frac{r_i}{n_i}\right\}
    \end{align*}
    and suppose $y\in B_D(x, \epsilon)$.
    Then $\ol{d_n}(x_n, y_n)/n<\epsilon$ for all $n\in\bb{N}$.
    In particular, $\ol{d_{n_i}}(x_{n_i}, y_{n_i})<n_i\epsilon\leq r_i$, so $y_{n_i}\in B_{d_{n_i}}(x_{n_i}, r_i)$ for each all $i$.
    Hence, $x\in B_D(x, \epsilon)\subset B$, as desired.
    \ifinclude\else
    For each $n$, let $C_n=B_{d_n}(x_n, r_n)$ be a neighborhood of $x_n$ contained in $B_n$, where $C_n=X_n$ if and only if $B_n=X_n$ and $0<r_n<1$ for the other $n$'s, and write $C=\prod_{n\in\bb{N}}C_n$.
    And let $\ds{\epsilon=\min_{1\leq i\leq k}\left\{\frac{r_{n_i}}{n_i}\right\}}$.
    Consider $G=B_D(x, \epsilon)$.
    If a point $y$ of $X$ is in $G$, then
    \begin{align*}
        \sup_{n\in\bb{N}}\left\{\frac{\ol{d_n}(x_n, y_n)}{n}\right\} < \epsilon,\quad\ol{d_n}(x_n, y_n) < r_n,
    \end{align*}
    so $y\in C\subset B$, i.e., $x\in G\subset B$.
    \fi
    
    Conversely, suppose a point $x$ of $X$ and a basis element $B_D(p, r)\in \mc{T}_D$ containing $x$ are given.
    By choosing a small real number $0<\epsilon<1$, we can achieve $x\in B_D(x, \epsilon)\subset B_D(p, r)$.
    \color{teal}(Such leap is helpful in proving many other propositions regarding metric spaces.) \color{black}
    Let $N$ be a positive integer such that $1/N < \epsilon$.
    Let $B_n=B_{d_n}(x_n, \epsilon)$ for each positive integer $n<N$ and let $B_n=X_n$ otherwise.
    Then $x\in\prod_{n\in\bb{N}}B_n\subset B_D(x, \epsilon)\subset B_D(p, r)$, as desired.
\end{proof}

\begin{prob}
    Let $(X, \alpha)$ and $(Y, \beta)$ be metric spaces.
    Show that every isometry from $X$ into $Y$ is an embedding.
\end{prob}
\begin{sol}
    If $f: X\rightarrow Y$ is an isometry, then $f$ is clearly injective and continuous.
    Because an open ball of radius $r>0$ in $X$ is mapped onto an open ball of radius $r$ in $Y$ under $f$, $f$ is an open map, so $f$ is an embedding.
\end{sol}

We now introduce the structure of an open ball in $\bb{R}^\bb{N}$ which equips the uniform metric $\ol{\rho}$.
\begin{prop}
    Let $\ol{\rho}$ be the uniform metric on $\bb{R}^\bb{N}$.
    Given $x=(x_1, x_2, \cdots)\in\bb{R}^\bb{N}$ and a real number $0<\epsilon<1$, define
    \begin{align*}
        U(x, \epsilon):=\prod_{n=1}^\infty (x_n-\epsilon, x_n+\epsilon).
    \end{align*}
    \begin{enumerate}
        \item[(a)]
        {
            $U(x, \epsilon)\neq B_{\ol{\rho}}(x, \epsilon)$ and $U(x, \epsilon)$ is not open in the uniform topology.
        }
        \item[(b)]
        {
            $B_{\ol{\rho}}(x, \epsilon)=\bigcup_{0<r<\epsilon} U(x, r)$.
        }
    \end{enumerate}
\end{prop}
\begin{proof}
    In proving (a), note that the point $(x_n+2^{-n}\epsilon)_{n\in\bb{N}}$ is in $U(x, \epsilon)$ but not in $B_{\ol{\rho}}(x, \epsilon)$.
    Furthermore, no neighborhood of this point entirely lies in $U(x, \epsilon)$.
    
    We now prove (b).
    For each real number $0<r<\epsilon$, we have $U(x, r)\subset B_{\ol{\rho}}(x, \epsilon)$.
    Conversely, if $y\in B_{\ol{\rho}(x, \epsilon)}$, then $\ol{\rho}(x, y)<\epsilon$, so $\sup\{\ol{d_n}(x_n, y_n): n\in\bb{N}\}=\delta$ for some $0\leq\delta<\epsilon$.
    Hence, $B_{\ol{\rho}}(x, \epsilon)\subset U(x, r)$ for some real number $r$ such that $\delta<r<\epsilon$.
\end{proof}