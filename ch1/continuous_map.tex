f\section{Continuous maps}

In this section, we assume $X, Y, Z$ are topological spaces.

\begin{defi}[Continuity]
    The map $f: X\rightarrow Y$ is said to be continuous if $f^{-1}(U)$ is open in $X$ whenever $U$ is open in $Y$.
    If the map $f$ is bijective and its inverse is also continuous, then $f$ is called a homeomorphism.
    In addition, if $f$ is injective and continuous, and the induced map $\widetilde{f}: X\rightarrow f(X)$ defined by $\widetilde{f}(a)=f(a)$ for all $a\in X$ is a homeomorphism, then $f$ is called an embedding of $X$ into $Y$.
    (It will be explained that such restriction of continuous maps are always continuous.)
\end{defi}

\begin{rmk}
    \begin{enumerate}
        \item[(a)]
        {
            The procedure for checking continuity can be reduced to the members of a basis or a subbasis generating the topology on the codomain $Y$. \color{brown}(Why?)\color{black}
        }
        \item[(b)]
        {
            A homeomorphism naturally induces a bijection between the topologies on the domain and the codomain of the homeomorphism. Also, a bijective continuous map is a homeomorphism if and only if the map is an open map.
        }
    \end{enumerate}
\end{rmk}

\begin{thm}
    Let $X, Y$ be topological spaces and $f: X\rightarrow Y$ be a map.
    Then the followings are equivalent:
    \begin{enumerate}
        \item[(a)]
        {
            $f$ is a continuous map.
        }
        \item[(b)]
        {
            For any closed subset $B$ of $Y$, $f^{-1}(B)$ is closed in $X$.
        }
        \item[(c)]
        {
            For each $x\in X$ and a neighborhood $V$ of $f(x)$ in $Y$, there is a neighborhood $U$ of $x$ in $X$ such that $f(U)\subset V$.
        }
        \item[(d)]
        {
            Whenever $U\subset X$, $f(\ol U)\subset\ol{f(U)}$.
        }
    \end{enumerate}
\end{thm}
\begin{proof}
    \hangindent=0.65cm
    (a)$\Leftrightarrow$(b):
        This follows directly by considering set complements.

    \noindent(a)$\Rightarrow$(c):
        If $x\in X$ and $V$ is a neighborhood of $f(x)$ in $Y$, then $f^{-1}(V)$ is a neighborhood of $x$ in $X$ whose image under $f$ is $V$.
    
    \noindent(c)$\Rightarrow$(a):
        Let $V$ be an open subset of $Y$ and let $U=f^{-1}(V)\subset X$.
        By assumption, for each $x\in X$, there is a neighborhood $A_x$ of $x$ in $X$ such that $f(A_x)\subset V$.
        Then $A_x\subset U$, so $U$ is open in $X$.
    
    \noindent(a)$\Rightarrow$(d):
        Suppose $U\subset X$.
        We will show that if $x\in\ol U$ then $f(x)\in\ol{f(U)}$.
        If $V$ is a neighborhood of $f(x)$ in $Y$, then $f^{-1}(V)$ is a neighborhood of $x$ in $X$, so $f^{-1}(V)$ intersects $U$.
        Thus, $V\cap f(U)=f(f^{-1}(V)\cap U)$ is nonempty, so $f(x)\in\ol{f(U)}$, as desired.

    \noindent(d)$\Rightarrow$(b):
        Let $B$ be a closed subset of $Y$ and let $A=f^{-1}(B)$.
        By assumption, we have $B=f(A)\subset f(\ol A)\subset \ol{f(A)}=\ol B=B$, so $B=f(\ol A)$ and $\ol A\subset f^{-1}(B)=A$, proving that $A=\ol A$ is closed.
\end{proof}

If given maps $f: X\rightarrow Y$ and $g: Y\rightarrow Z$ are continuous, some naturally induced maps are also continuous, as indicated below. \color{brown}(Checking continuity is left as an exercise.)\color{black}

\begin{prop}
    Suppose $f: X\rightarrow Y$ and $g: Y\rightarrow Z$ are continuous.
    \begin{enumerate}
        \item[(a)]
        {
            Every constant map is continuous.
        }
        \item[(b)]
        {
            $g\circ f$ is continuous.
        }
        \item[(c)]
        {
            Every restriction of a continuous map on a subspace is continuous. Also, if $Y$ is a subspace of $Z$, then $\widetilde{f}: X\rightarrow Z$ defined by $\widetilde{f}(x)=f(x)$ for all $x\in X$ is continuous.
        }
    \end{enumerate}
\end{prop}

\begin{prob}
    Suppose that $f: X\rightarrow Y$ is continuous.
    If $x$ is a limit point of the subset $A$ of $X$, is it necessarily true that $f(x)$ is a limit point of $f(A)$?
\end{prob}
\begin{sol}
    It is not necessarily true; consider constant maps.
\end{sol}

\begin{thm}
    Let $A$ be a set; let $\{X_\alpha\}_{\alpha\in I}$ be an indexed family of spaces; and let $\{f_\alpha: A\rightarrow X_\alpha\}_{\alpha\in I}$ be an indexed family of functions.
    \begin{enumerate}
        \item[(a)]
        {
            There is a unique coarsest topology $\mc{T}$ on $A$ relative to which each of the function $f_\alpha$ is continuous. In fact, the topology $\mc{T}$ is generated as a subbasis by the following collection:
            \begin{align*}
                \left\{f_\alpha^{-1}(U_\alpha)
                :
                \alpha\in I\textsf{ and }U_\alpha\textsf{ is open in }X_\alpha\right\}.
            \end{align*}
        }
        \item[(b)]
        {
            A map $g: Y\rightarrow A$ is continuous relative to $\mc{T}$ if and only if each composition $f_\alpha\circ g$ is continuous.
        }
    \end{enumerate}
\end{thm}
\begin{proof}
    (a) is almost clear; such topology necessarily contains the given collection.
    In proving (b), it suffices to prove if part.
    Suppose $g_\alpha:=f_\alpha\circ g$ is continuous for each $\alpha\in I$.
    Given an open subset $U$ of $A$, we have $g_\alpha^{-1}(A)=g^{-1}(f_\alpha^{-1}(A))$, completing the proof.
\end{proof}
\begin{rmk}
    The product topology on a product space satisfies the above properties; in fact, the product topology is the topology on $A$ with
    \begin{center}
        $\ds{A=\prod_{\alpha\in I}X_\alpha}$\quad and\quad$f_\alpha=\pi_\alpha$ for each $\alpha\in I$.
    \end{center}
\end{rmk}

\begin{prob}
    Let $(X, d)$ be a metric space.
    \begin{enumerate}
        \item[(a)]
        {
            Show that $d: X\times X\rightarrow\bb{R}$ is continuous.
        }
        \item[(b)]
        {
            Let $X'$ be a space having the same underlying set as $X$, and define the map $d': X'\times X'\rightarrow\bb{R}$ by $d'(a, b)=d(a, b)$ for all $(a, b)\in X'\times X'$.
            Show that the topology on $X'$ is finer than the topology on $X$, if $d'$ is continuous.
            Deduce that the metric topology on $X$ induced by the metric $d$ is the coarsest topology on $X$ relative to which $d$ is continuous.
        }
    \end{enumerate}
\end{prob}
\begin{sol}
    \begin{enumerate}
        \item[(a)]
        {
            Choose a point $(a, b)\in X\times X$ and write $r=d(a, b)$.
            We will show that given a positive real number $\epsilon$ there is a neighborhood of $(a, b)$ whose image under $d$ is contained in $(r-\epsilon, r+\epsilon)$.
            Let $V=B_d(a, \delta)\times B_d(b, \delta)$ with $\delta>0$, which is a neighborhood of $(a, b)$ in $X\times X$.
            Whenver $(p, q)\in V$, we have
            \begin{align*}
                d(p, q)\leq d(p, a)+d(a, b)+d(b, q)
                \quad\textsf{and}\quad
                d(a, b)\leq d(a, p)+d(p, q)_d(q, b,)
            \end{align*}
            from which we obtain $r-2\delta<d(p, q)<r+2\delta$.
            Hence, by choosing $0<\delta<\epsilon/3$ we have $d(V)\subset(r-\epsilon, r+\epsilon)$, as desired.
            Therefore, $d$ is a continuous map.
        }
        \item[(b)]
        {
            Choose $a\in X'$, and define the map $\epsilon_a: X'\rightarrow X'\times X'$ by $\epsilon_a(x)=(x, a)$ for $x\in X'$.
            Then $\epsilon_a$ is continuous, so $k:=d'\circ\epsilon_a$ is continuous.
            Hence, $k^{-1}((-\infty, r))=B_{d'}(a, r)=B_d(a, r)$ is an open subset of $X'$ for all $r\in\bb{R}$ with $r>0$.
            Therefore, the topology on $X'$ contains $\{B_d(a, r): a\in X,\, r>0\}$, so the topology on $X'$ is finer than the topology on $X$.
            The last assertion easily follows.
        }
    \end{enumerate}
\end{sol}