\section{Topologies on function spaces}

Let $X_\alpha$ be a topological space for each $\alpha\in I$, and write $X=\prod_{\alpha\in I} X_\alpha$.

\begin{defi}[Box topology]
    The topology on $X$ generated by the following collection as a basis is called the box topology on $X$:
    \begin{align*}
        \left\{\prod_{\alpha\in I} U_\alpha: \textsf{For each $\alpha\in I$, $U_\alpha$ is an open subset of $X_\alpha$}\right\}.
    \end{align*}
\end{defi}

Indeed, the box topology on $X$ is finer than the product topology on $X$.
Also, it is easy to verify that the box topology on $X$ can be generated as a basis by the following collection, given that $\mc{B}_\alpha$ is a basis of the topology on $X$ for each $\alpha\in I$:
\begin{align*}
    \left\{\prod_{\alpha\in I} B_\alpha: \textsf{For each $\alpha\in I$, $B_\alpha\in\mc{B}_\alpha$}\right\}.
\end{align*}

\begin{rmk}
    For $X$ equipped with the product topology, we have observed that a sequence $(x_n)_{n\in\bb{N}}\subset X$ is convergent if and only if $(\pi_\alpha(x_n))_{n\in\bb{N}}\subset X_\alpha$ is convergent for each $\alpha\in I$.
    For $X$ equipped with the box topology, however, if part is generally not true (only if part is true).
\end{rmk}

We now assume that $(Y, d)$ is a metric space and $I$ is an index set.
Even $Y$ is not metrizable, we could have considered the product and box topology on $Y^I$, but the following topology is valid when a metric on $Y$ is given.
In other words, the following topology on $Y^I$ (generally) depends on the choice of the metric on $Y$.
\begin{defi}
    \begin{enumerate}
        \item[(a)]
        {
            (Uniform metric)
            Define the map $\ol\rho: Y^I\times Y^I\rightarrow [0,\infty)$ by
            \begin{align*}
                (f, g)\mapsto\sup\{\ol{d}(f(x), g(x)): x\in I\}
                \quad
                \textsf{for all $f, g\in Y^I$.}
            \end{align*}
            The map $\ol\rho$ is a metric on $Y^I$, which is called the uniform metric.
        }
        \item[(b)]
        {
            (Uniform topology)
            Let $I$ be a nonempty set and $(Y, d)$ be a metric space.
            The topology on $Y^I$ generaged by the following collection as a basis is called the uniform topology on $Y^I$:
            \begin{align*}
                \left\{
                    B_{\ol\rho}(f, \epsilon):
                    \textsf{$f\in Y^I$ and $\epsilon>0$}
                \right\}
            \end{align*}
        }
    \end{enumerate}
\end{defi}

\begin{prop}
    Let $I$ be a nonempty set and $(Y, d)$ be a metric space.
    Then, we have the following inclusions of the topologies on $Y^I$:
    \begin{align*}
        \textsf{(the product topology)}\subset\textsf{(the uniform topology)}\subset\textsf{(the box topology)}.
    \end{align*}
    In particular, if $I=\bb{N}$ and $(Y, d)$ is the Euclidean metric space $(\bb{R}, d)$, then the above three topologies do not coincide.
\end{prop}
\begin{proof}
    \textbf{Step 1. Showing the inclusions.}\newline\noindent
    Given a basis member $\prod_{x\in I} U_x$ of the product topology on $Y^I$, we have $U_x=Y$ for all but finitely many values of $x$; namely, $x_1, \cdots, x_k$.
    Suppose $f\in \prod_{x\in I} U_x$.
    Because $f(x_i)\in U_{x_i}$ for each $i=1, \cdots, k$, there is a positive real number $0<r_i<1$ such that $B_d(f(x_i), r_i)\subset U_{x_i}$.
    Letting $r=\min\{r_1, \cdots, r_k\}$, we have $f\in B_{\ol\rho}(f, r)\subset \prod_{x\in I} U_x$.

    Given a basis member $B_{\ol\rho}(f, \epsilon)$ of the uniform topology on $Y^I$ (with $0<\epsilon<1$) with a point $g$ in the member, there is a positive real number $0<r<1$ such that $g\in B_{\ol\rho}(g, r)\subset B_{\ol\rho}(f, \epsilon)$.
    In fact, we also have $g\in \prod_{x\in I} B_d(g(x), r/2)\subset B_{\ol\rho}(g, r)$, proving the assertion.

    \textbf{Step 2. Investigating the given particular case.}\newline\noindent
    Observe, when $(Y, d)$ is the Euclidean metric space $(\bb{R}, d)$ and $I=\bb{N}$, that $B_{\ol\rho}(0, 1/2)$ cannot be an intersection of finitely many subsets of the form $\pi_n^{-1}(\bb{R})\, (n\in\bb{N})$, and that $\prod_{n\in\bb{N}}(1/n, 1/n)$ cannot be open in the uniform topology on $\bb{R}^\bb{N}$.
\end{proof}