\section{Product $\sigma$-algebras}

When we studied general topology, we learned how to impose a topology on the set defined as the Cartesian product of sets.
There is a corresponding counterpart in the theory of measure, called the product $\sigma$-algebra.
\begin{defi}[Product $\sigma$-algebra]
    Given measurable spaces $(X_\alpha, \mc{M}_\alpha)$ with $\alpha\in I$, the product $\sigma$-algebra on the product $X=\prod_{\alpha\in I}X_\alpha$ is defined as the $\sigma$-algebra on $X$ generated by the following collection:
    \begin{align*}
        \{
            \pi_\alpha^{-1}(E_\alpha)
            :
            \textsf{$E_\alpha\in\mc{M}_\alpha$ for $\alpha\in I$}
        \}.
    \end{align*}
    And the product $\sigma$-algebra on $X$ is denoted by $\bigotimes_{\alpha\in I}\mc{M}_\alpha$.
\end{defi}
\begin{rmk}
    Given a collection of maps $\{f_\alpha: A\rightarrow X_\alpha\}_{\alpha\in I}$ where each $X_\alpha$ is a topological space, we have imposed $A$ the smallest topology relative to which each $f_\alpha$ is continuous; the topology generated by $\{f_\alpha^{-1}(U_\alpha): \textsf{$\alpha\in I$ and $U_\alpha$ is open in $X_\alpha$}\}$, and the topology on $A=\prod_{\alpha\in I} X_\alpha$ with $f_\alpha=\pi_\alpha$ for each $\alpha\in I$ is the product topology on $\prod_{\alpha\in I} X_\alpha$.
    Similar argument is appliable in the theory of $\sigma$-algebras; given a collection of maps $\{f_\alpha: A\rightarrow X_\alpha\}_{\alpha\in I}$ where each $(X_\alpha, \mc{M}_\alpha)$ is a measurable space, we impose $A$ the smallest $\sigma$-algebra relative to which each $f_\alpha$ is measurable; the $\sigma$-algebra generated by
    \begin{align*}
        \{f_\alpha^{-1}(E_\alpha): \textsf{$\alpha\in I$ and $E_\alpha\in\mc{M}_\alpha$}\},
    \end{align*}
    and the topology on $A=\prod_{\alpha\in I} X_\alpha$ with $f_\alpha=\pi_\alpha$ for each $\alpha\in I$ is the product $\sigma$-algebra on $\prod_{\alpha\in I} X_\alpha$.
\end{rmk}

We now introduce some propositions regarding product $\sigma$-algebras.
The first proposition states that the generators of a product $\sigma$-algebra reduce to the products of members in $\sigma$-algebras, when there are countably many $\sigma$-algebras.
(Remark the counterpart in the topology.)
\begin{prop}
    If $I$ is countable, then $\bigotimes_{\alpha\in I}\mc{M}_\alpha$ is the $\sigma$-algebra on $X$ generated by $\{\prod_{\alpha\in I} E_\alpha : \textsf{$E_\alpha\in\mc{M}_\alpha$ for $\alpha\in I$}\}$.
\end{prop}
\begin{proof}
    The countability of $\sigma$-algebras asserts that both $\{\prod_{\alpha\in I} E_\alpha : \textsf{$E_\alpha\in\mc{M}_\alpha$ for $\alpha\in I$}\}$ and $\{\pi_\alpha^{-1}(E_\alpha): \textsf{$E_\alpha\in\mc{M}_\alpha$ for $\alpha\in I$}\}$ generates the same $\sigma$-algebra on $\prod_{\alpha\in I} X_\alpha$.
\end{proof}

Remark that a product topology is generated as a subbais by all preimages of open sets under canonical projections and as a subbasis by all preimages of basis members under canonical projections.
Likewisely, a product $\sigma$-algebra is generated by all preimages under canonical projections of all sets generating $\sigma$-algebras.
(Again, remark the counterpart in the topology.)
\begin{prop}[Reduction to generators]
    Suppose $\mc{M}_\alpha$ is generated by $\mc{F}_\alpha$ for each $\alpha\in I$.
    Then $\bigotimes_\alpha\mc{M}_\alpha$ is generated by $\mc{F}:=\{\pi_\alpha^{-1}(E_\alpha):\textsf{$E_\alpha\in\mc{F}_\alpha$ for all values of $\alpha$}\}$.
    Furthermore, if $I$ is countable, then $\bigotimes_\alpha\mc{M}_\alpha$ is generated by $\{\prod_\alpha E_\alpha:\textsf{$E_\alpha\in\mc{F}_\alpha$ for all values of $\alpha$}\}$.
\end{prop}
\begin{proof}
    Since it is clear that $\genone{\mc{F}}\leq\bigotimes_\alpha\mc{M}_\alpha$, it is required to show the converse inclusion.
    What we want to achieve is the following:
    \begin{center}
        A generator $\pi_\alpha^{-1}(E_\alpha)$ of $\bigotimes_\alpha\mc{M}_\alpha$, where $\alpha\in I$ and $E_\alpha\in\mc{M}_\alpha$, belongs to $\genone{\mc{F}}$, i.e., $\pi_\alpha^{-1}(E_\alpha)\in\genone{\mc{F}}$.
    \end{center}
    For this, we seek to prove that $E_\alpha\in\mc{M}_\alpha$ belongs to the following collection for any $\alpha\in I$:
    \begin{align}\label{strategy for non-constructive objects}
        \{U\subset X_\alpha
        :
        \pi_\alpha^{-1}(U)\in\genone{\mc F}
        \}.
    \end{align}
    The above collection contains $\mc{F}_\alpha$ and is a $\sigma$-algebra on $X_\alpha$, thus it contains $\mc{M}_\alpha$.
    Hence, whenever $\alpha\in I$ and $E_\alpha\in\mc{M}_\alpha$, we have $\pi_\alpha^{-1}(E_\alpha)\in\genone{\mc F}$, proving the first statement.

    Letting the second collection in the statement be denoted by $\mc H$, all we need to prove is $\genone{\mc H}$ contains the product $\sigma$-algebra, which is as clear as the preceeding proposition.
\end{proof}
\begin{rmk}
    Given a subbasis of a topology, every member of the topology is constructive; collecting every finite intersection of the members of the subbasis gives a basis, and collecting all arbitrary unions of the members of the basis gives the topology.
    However, there is no corresponding proposition regarding forming $\sigma$-algebra from generators.

    For such non-constructive objects, we may detour in proving some properties and set a collection as in \cref{strategy for non-constructive objects}.
    To review, \color{magenta}what we wanted to show was that every generator for the product $\sigma$-algebra to be in $\genone{\mc F}$\color{black}, i.e., whenever $\alpha\in I$ and $E\in\mc{M}_\alpha$, we wanted \color{blue}$\pi_\alpha^{-1}(E_\alpha)\in\genone{\mc F}$\color{black}.
    Hence, we set an appropriate collection \color{magenta}which can be compared with $\mc{F}_\alpha$ \color{black} as
    \begin{align*}
        \{U\subset X_\alpha: \color{blue}\pi_\alpha^{-1}(U)\in\genone{\mc{F}}\color{black}\},
    \end{align*}
    and then proved that the above collection contains $\mc{M}_\alpha$.

    The lesson from this remark is that sometimes it is helpful to observe the statement to be proved and set an appropriate test set.
    Such strategy would be adopted in some propositions throughout this document.
\end{rmk}

\begin{prop}
    Let $X_i$ be metric spaces for $i=1, 2, \cdots, n$, and let $X=\prod_i X_i$, equipped with the product metric.
    Then $\bigotimes_i \mc{B}_{X_i}$ is contained in $\mc{B}_X$.
    When each $X_i$ is separable, then $\bigotimes_i \mc{B}_{X_i}=\mc{B}_X$.
\end{prop}
\begin{proof}
    Note that $\bigotimes_i \mc{B}_{X_i}=\genone{\pi_i^{-1}(E_i):\textsf{$E_i$ is open in $X_i$ for $1\leq i\leq n$}}$ and each $\pi_i^{-1}(E_i)$ is open in $X$, so $\bigotimes_i \mc{B}_{X_i}\subset\mc{B}_X$.
    Now, assume each $X_i$ contains a countable dense subset $D_i$, and consider the following collection:
    \begin{align*}
        \left\{
        \prod_{i=1}^n B_{X_i}(x_i, n^{-1})
        :
        \textsf{$x_i\in D_i$ and $n\in\bb{N}$}
        \right\}.
    \end{align*}
    The above collection is countable and generates the metric topology on $X$.
    Hence, every set open in $X$ is a countable union of members in the above collection.
    Therefore, $\mc{B}_X$ is contained in $\bigotimes_{i=1}^n \mc{B}_{X_i}$ and $\mc{B}_X=\bigotimes_i\mc{B}_{X_i}$.
\end{proof}
